\section*{Resultater}
\label{Resultater}
%
Ud fra det udarbejdede \textit{affinity diagram} udvælges de to grønne, overordnede kategorier, der er relevante i forhold til dette mini projekt, som er: \textit{Robottens udseende}, illustreret på \autoref{fig:RsUdseende}, samt \textit{Robottens væremåde}, illustreret på \autoref{fig:RsVaeremaade}.
%
\begin{figure}[H]
\centering
\includegraphics[width = 0.9\textwidth]{Figure/RsUdseende} 
\caption{Udsnit af det samlede \textit{affinity diagram}, som vedrører robottens udseende.}
\label{fig:RsUdseende}
\end{figure}
\noindent
%
Da dette miniprojekt bygger på data indsamlede i forbindelse med semesterprojektet er det ikke alle kategorier, der er angivet med pink labels, som har relevans for miniprojektet. I henhold til miniprojektet er det kun to af de pink labels, som kan anses for at være objektive attributter: \textit{R's højde} samt \textit{Hvor menneskelig skal R være?}, hvor R er en forkortelse for robot. I begge tilfælde er det muligt at ændre robottens fysiske karakteristika i forhold til højde og hvor menneskelig robotten perciperes. 
%
\begin{figure}[H]
\centering
\includegraphics[width = 0.9\textwidth]{Figure/RsVaeremaade} 
\caption{Udsnit af det samlede \textit{affinity diagram}, som vedrører robottens væremåde.}
\label{fig:RsVaeremaade}
\end{figure}
\noindent
%
Ud af de seks pink kategorier, der fremgår på \autoref{fig:RsVaeremaade}, vil følgende kategorier inkluderes i miniprojektsammenhæng: \textit{R's bevægelser}, \textit{R's hastighed} samt \textit{Ser R som menneskelig}. Sidstnævnte kategori vedrører den samme attribut, som blev valgt fra \autoref{fig:RsUdseende}, hvorfor de to vil betragtes som værende én attribut som kan evalueres på flere skalaer. Selvom den pink label: \textit{R skal ikke være anmassende} er præferencebetonet vil det stadig blive inddraget i miniprojektet, da det er muligt at udvikle en skala som evaluerer en objektiv attribut, som henvender sig til hvorvidt robotten er anmassende.  

\subsection*{Liste af attributter samt udvikling af skala}
\label{ListeAttributterSkala}
%
Baseret på kategorien: \textit{R's udseende}, kan følgende attributter udledes: 



Baseret på kategorien: \textit{R's væremåde}, kan følgende objektive attributter udledes: \blankline
%
\begin{itemize}
  \item Hastighed
  \item Menneskelig
  \item Bevægelser
  \item Henvendelse\blankline
\end{itemize}
%
Hastigheden gengives skalaen afbilledet på \autoref{fig:Hastighed} 
%
\begin{figure}[H]
\centering
\includegraphics[width =\textwidth]{Figure/Hastighed} 
\caption{Bipolær skala til at evaluere robottens hastighed.}
\label{fig:Hastighed}
\end{figure}
\noindent
%
Hvorvidt robotten perciperes som værende menneskelig eller ej gengives på skalaen afbilledet på \autoref{fig:Menneskelig}
%
\begin{figure}[H]
\centering
\includegraphics[width =\textwidth]{Figure/Menneskelig} 
\caption{Bipolær skala til at evaluere robottens menneskelighed.}
\label{fig:Menneskelig}
\end{figure}
\noindent
%






