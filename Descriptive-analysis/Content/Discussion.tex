\section*{Diskussion}
\label{Diskussion}
%
Der er både fordele og ulemper ved at rekruttere testepersoner i en lufthavn og spørge dem hvad deres oplevelse med robotten er. Fordelen er at testpersonerne er potentielle brugere, som vil komme til at interagere med robotten i fremtiden. I det henseende vil det være muligt at designe dels robottens bevægelser, dens henvendelse samt dens udseende udfra et brugerorienteret synspunkt. Det tillader at robotten designes ud fra brugerens oplevelse fremfor eksperter, som formentlig vil fokusere på andre aspekter af robotten. 

Ulempen ved at have naive testpersoner er, at der ingen kontrol er over, om testpersonerne har samme forståelse for de ord, der bliver brugt til at beskrive oplevelsen. Der er derfor ikke opnået en fælles forståelse for definitionen af de attributter, som oplevelsen med robotten beskrives ud fra. Dog er der i ord eliciteringen udarbejdet et \textit{affinity diagram}, som netop har til formål at kategorisere og gruppere testpersonernes problemer, behov og synspunkter, hvorfor der opnåes en form for konsensus om hvad ordene betyder. Selvom det ikke bygger på testpersonernes indbyrdes forståelse men på projektgruppens, antages det at der er opnået en fælles forståelse for definitionen af de valgte attributter.

Den manglende fælles forståelse ville være undgået, hvis der var brugt et ekspert panel, som i fællesskab kommer frem til definitionen af de anvendte attributter. \blankline   
%
 
 
 
 
 

Ydermere vil testpersonen normaltvist ved en udledelse af vigtige attributter ved en robot blive bedt om at fortælle alt, hvad de synes om den pågældende robot. I denne undersøgelse er testpersonerne blevet stillet specifikke spørgsmål, hvorfor der kan være vigtige ord om robotten, der er blevet udeladt.  
