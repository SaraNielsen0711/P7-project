\section*{Testdesign}
\label{Testdesign}
%
Følgende afsnit vil belyse de forskellige aspekter af feltundersøgelsen fra hvilke spørgsmål der stilles i interviewet for at få testpersonerne til at beskrive deres oplevelse med robotten, rollefordelingen samt hvilke testpersoner, der tilstræbes at rekruttere til hvor testen afvikles og med hvilket udstyr samt fremgangsmåden.  

\subsection*{Testperson}
\label{Testpersoner}
%
I forbindelse med miniprojektet omkring \textit{Descriptive Analysis} vælges det er inddrage naive testpersoner, som derfor ikke får træning i at beskrive deres oplevelse ud fra specifikke termer eller parametre. De naive testpersoner vil derfor være dem der "finder" ordene, som beskriver stimuli - interaktionen med robotten.

Testpersonerne, de rejsende, bliver rekrutteret direkte i Aalborg Lufthavn i området efter sikkerhedskontrollen. Det er som udgangspunkt robotten, der står får rekrutteringen ved at henvende sig til de rejsende med spørgsmålet: \textit{Jeg kommer fra Aalborg Universitet. Må jeg hjælpe dig med at finde rundt i Aalborg Lufthavn?}, har den rejsende lyst til at deltage vil robotten føre dem igennem nogle foruddefineret brugsscenarier, såsom at finde toiletfaciliteter, information om gate og lignende. Fordelen ved at få robotten til at rekruttere testpersoner er, at testpersonerne får et upåvirket førstehåndsindtryk af robotten, som det vil være tilfældet første gang de oplever robotten i lufthavnen. \blankline 
%
Det tilstræbes, at foretage underøgelsen på både kvinder og mænd, gerne med forskellige aldre og rejseformål; forretningsrejse eller ferierejse, hvis muligt vil det ydermere bestræbes at inddrage førstegangs rejsende. Derudover er det et krav, at den rejsende er dansktalende for at undgå, at vigtige pointer går tabt i oversættelsen, når data efterfølgende skal behandles. Antallet af testpersoner er ikke forudbestemt, da undersøgelsen i stedet afsluttes, når der opnåes mætning, hvorved der ikke indsamles ny viden. Dette vurderes af de tilstede værende gruppemedlemmer. Det forventes dog, at udføre undersøgelsen på minimum fem testpersoner.


\subsection*{Interview}
\label{Interview}
%
Efter testpersonen har angivet hvad robotten skal hjælpe med i lufthavnen vil robotten bede testpersonen om at følge efter robotten, istedet for at robotten følger testpersonen hen til det angivne sted følges testpersonen til et interview. Igennem interviewet vil testpersonerne blive stilt nogle spørgsmål, der vedrører hvordan de oplevede interaktionen med robotten. Ud fra de spørgsmål er det muligt at finde de ord naive testpersoner bruger når de beskriver stimuli, robotten. Følgende interviewspørgsmål er vejledende og vil blive tilpasset undervejs i interviewet.  
%
\begin{itemize}
\item Førstehåndsindtryk af robotten - fra rekrutteringen
\item Måden hvorpå robotten henvender sig
\item Hvad testpersonen synes om robotten
\item Hvad testpersonerne tror andre rejsende tænker om interaktionen 
\item Robottens relevans
\item Robottens pålidelighed
\item Normal oplevelse i en lufthavn uden hjælp fra en robot\blankline 
\end{itemize}
\noindent
%
\begin{itemize}
\item Hvad synes du om..
	\begin{itemize}
		\item Robottens hastighed?
		\item Robottens højde?
		\item Robottens afstand til dig?
		\item Robottens generelle bevægelse?
		\item Robottens udseende?
		\item Den retning robotten henvendte sig til dig fra?
	\end{itemize}
\item Hvor gammel er du?
\item Hvor ofte flyver du?
\item Spørgsmål og/eller kommentarer til undersøgelsen 
\item Spørgsmål og/eller kommentarer til vores projekt
\item Afrunding og ønsk testpersonerne god rejse
\end{itemize}
% 

\subsection*{Rollefordeling}
\label{Rollefordeling}
%
For at udføre feltundersøgelsen er det nødvendigt at definere nogle roller, som relaterer sig til specifikke dele af undersøgelsen. Der vil i alt blive defineret fire roller, som vil rotere mellem projektgruppen.
%
\subsubsection*{Robot styre}
Da testpersonerne skal interagere med en \textit{Double}-robot, som ikke kan forudprogrammeres er det nødvendigt, at have én til at styre robotten. For at de rejsende får et indtryk af at robotten er autonom og har en form for social intelligens er det favorabelt, at den rejsende ikke registrerer, at det er en person, som styrer robotten. For at efterkomme det vil personen, som styrer robotten være placeret så de rejsende ikke direkte kan se hvordan robotten styres, dog skal personen stadig have mulighed for at overvære og høre interaktionen mellem robot og den rejsende, så robotten kan styres derefter.

Der vil derfor ikke være nogle foruddefineret stimuli, som præsenteres i en bestemt rækkefølge da det ikke er muligt at programmere robotten til at bevæge sig på en bestemt måde. Det er derfor robot styrens opgave, at sørger for at robotten både ændre sin højde, afstanden til testpersonen, hvordan robotten henvender sig og generelt bevæger sig.   

\subsubsection*{Testleder}
Testlederen har flere opgaver, først og fremmest at få mundtlig samtykke fra testpersonerne til at optage interviewet og derefter starte lydoptagelsen. Derudover skal testlederen sørger for, at stille spørgsmålene angivet i \fullref{Interview}. Da samtalen mellem testleder og testperson varierer udarbejdes der ikke specifikke instruktioner. Dog sørger testlederen for at introducere testpersonen til hvem projektgruppen er og at der vil blive taget noter undervejs. 

Når robotten ikke interagerer med en testperson er det testlederens opgave, at holde øje med at robotten ikke kører ind i noget eller at andre rejsende ikke går ind i den.    

\subsubsection*{Observatør}
Der vil i alt være tre observatører til stede, hvor minimum én af dem vil observere interaktionen mellem testperson og robot før robotten leder testpersonen over til testlederen. Da det tilstræbes at interaktionen mellem testperson og robot er så naturlig som muligt, vil observatøren holde afstand til dem. De to andre observatører sidder ved et bord i nærheden og noterer deres observationer. Formålet med at have tre observatører er for at være sikker på, at der indsamles så meget data som muligt og fordi det forventes at observatørerne bemærker forskellige ting. Derudover vil der ikke optages video, hvorfor det er vigtigt at have gode notater. Derudover vil det tilstræbes at én observatør fokuserer på testpersonernes mimik, én anden observatør fokuserer på kropsholdning og den sidste observatør fokuserer på decideret pointer koblet til en robot aktion. Er det muligt for observatørerne, at notere flere ting så gør de bare det. Når testlederen starter lydoptagelsen vil observatørerne ligeledes starte deres tidstagning så observationerne kan tidsstemples. Observatørens noter vil blive nedskrevet på notespapir. 


     
