\section*{Ord elicitering}
\label{OrdElicitering}

Ord eliciteringen har til formål at finde de ord der er blevet brugt til at beskrive de perciperede dimensioner af robotten. 
Det er valgt at gøre dette med \textit{affinity diagram}, hvor data grupperes efter deres udtalelser fremfor forudbestemte kategorier. Hver gruppering får tildelt en label, som gengiver hvad testpersonerne siger og mener, \parencite[s. 159]{Book:BuildingAnAffinity}. Fordelen ved at analysere det indsamlede kvalitative data ved hjælp af \textit{affinity diagram} er, at hver eneste udtalelse fra testpersonerne skal behandles, \parencite[s. 25]{PDF:ConsolidationIdeationAffinity}. \blankline
%
Første del af ord eliciteringen består af at transskribere alle de foretaget interviews. Dette gøres for at sikre at det er testpersonernes ord og direkte udtalelser der bruges ved den næste del af ord eliciteringen. Det er en fordel at nedskrive interviewsene da det gør det nemmere at danne affinity notes, som skal gøres i næste del. \blankline
%
Anden del af ord eliciteringen er fortolkningssessionen, hvor de transskriberede interviews gennemgås slavist og udtalelser skrives på \textit{affinity notes}. Her er der fokus på at få nedskrevet noterne med de ord som testpersonerne brugte. \blankline
%
Når alle \textit{affinity notes} er nedskrevet begynder udviklingen af selve diagrammet. Denne udvikling består af i alt fire dele hvor notesene i hver del sorteres i kategorier, som derefter vil få tildelt en label, der afspejler den måde hvorpå testpersonerne omtaler et bestemt problem, behov eller synspunkt på, \parencite[s. 159]{Book:BuildingAnAffinity}. Ud fra et \textit{affinity diagram} er det muligt at udlede hvilke krav potentielle brugere har til et produkt, hvor det i projektsammenhæng vil relatere sig til hvilke parametre danske rejsende tilskriver interaktionen med en social robot. 

\textit{Affinity notes} bliver nedskrevet på gule \textit{sticky notes} og der startes med at gruppere disse notes. Hver gruppering af \textit{affinity notes} vil derefter få tildelt en label, som nedskrives på blå \textit{sticky notes}. De blå labels nedskrives i førsteperson, som hvis det var brugeren selv, der fortalt hvad det specifikke problem, behov eller synspunkt vedrører, \parencite[s. 160]{Book:BuildingAnAffinity}. Derefter vil de blå labels blive kategoriseret efter sammenhæng og få tildelt en label, som nedskrives på pink \textit{sticky notes}, som igen skal afspejle brugerens måde at beskrive problemet, behovet eller synspunktet, \parencite[s. 160]{Book:BuildingAnAffinity}. Det sidste og højeste niveau i et \textit{affinity diagram} gengives med grønne labels, som bliver dannet ud fra en kategorisering af pink labels og som kan nedskrives ente i førsteperson, som afspejler brugeren eller med mere generelle termer, \parencite[s. 160]{Book:BuildingAnAffinity}. \blankline
%
Ud fra det udarbejdede \textit{affinity diagram} udvælges de to grønne, overordnede kategorier, der er relevante i forhold til dette mini projekt, som er ``robottens udseende`` og ``robottens væremåde''. 

(beskriv her hvad vi så gjorde derefter.)