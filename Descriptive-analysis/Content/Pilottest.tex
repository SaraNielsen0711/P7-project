\section*{Ord elicitering}
\label{OrdElicitering}

Ord eliciteringen har til formål at finde de ord der er blevet brugt til at beskrive de perciperede dimensioner af robotten. 
Det er valgt at gøre dette med \textit{affinity diagram}, hvor data grupperes efter deres udtalelser fremfor forudbestemte kategorier. Hver gruppering får tildelt en label, som gengiver hvad testpersonerne siger og mener, \parencite[s. 159]{Book:BuildingAnAffinity}. Fordelen ved at analysere det indsamlede kvalitative data ved hjælp af \textit{affinity diagram} er, at hver eneste udtalelse fra testpersonerne skal behandles, \parencite[s. 25]{PDF:ConsolidationIdeationAffinity}.

Første del af ord eliciteringen består af at transskription af alle de foretaget interviews. Dette gøres for at sikre at det er testpersonernes ord og direkte udtalelser der bruges ved den næste del af ord eliciteringen. 

Anden del er fortolkningssessionen, hvor de transskriberede interviews gennemgås slavist og udtalelser skrives på affinity notes. 
