\chapter*{Attributter, der tilskrives social robot i lufthavn}
%
Der er foretaget en feltundersøgelse i Aalborg lufthavn, der har til formål at udlede de attributter, der, ifølge danske rejsende, har indflydelse på interaktionen med en social robot. Formålet ved at udlede disse attributter er for efterfølgende at kunne gengive dem på skalaer, som kan bruges til at evaluere oplevelsen af et stimulus, hvor de fysiske karakteristikas kan ændres. 

Undersøgelsen er opdelt i to dele, hvor første del handler om at den rejsende skal interagere med en robot. Det tilstræbes at interaktionen er så naturlig som muligt, hvorfor der ikke vil være en testleder, der sørger for at beskrive undersøgelsen, men kun testpersonen overfor robotten. Anden del af testen starter når robotten opfordre testpersonen til at følge efter den, hvor testpersonen følges hen til testlederen, som afvikler et interview. Den indsamlede data vil blive analyseret ved at udvikle et \textit{affinity diagram}. Resultaterne fra analysen er en liste af attributter, der bruges til at beskrive robotten samt grupperinger af disse attributter i forhold til hvad de beskriver. Disse attributter vil derefter blive gengivet på skalaer, som efterfølgende vil kunne bruges til at evaluere oplevelsen.  