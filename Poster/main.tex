\documentclass[paperwidth=160cm,paperheight=100cm,landscape,fontscale=0.3010]{baposter}
%Dortes fontscale: 0.2941
\usepackage{lipsum} 
%841mm x 1189mm
\usepackage[font=small,labelfont=bf,hypcap=false]{caption} % Required for specifying captions to tables and figures
%\usepackage{cite}
\usepackage{booktabs} % Horizontal rules in tables
\usepackage{relsize} % Used for making text smaller in some places
%\usepackage[urlcolor  = blue]{hyperref}
%\graphicspath{{figures/}} % Directory in which figures are stored
\usepackage{multicol}
%\usepackage[style=ieee]{biblatex}
%\addbibresource{bib.bib}
\usepackage[utf8]{inputenc} 
%\usepackage{hyperref}
\usepackage{placeins}
\usepackage{epstopdf}
\usepackage{subcaption}
\usepackage[]{graphicx}
\usepackage{natbib}
\usepackage{times}

\bibliographystyle{abbrvnat}

\selectcolormodel{RGB} %<-- Add colour model defintion
\definecolor{bordercol}{RGB}{255,255,255}%{33,26,82} % Border color of content boxes
\definecolor{headercol1}{RGB}{255,255,255}%{33,26,82} % Background color for the header in the content boxes (left side)
%\definecolor{headercol2}{RGB}{5,2,82} % Background color for the header in the content boxes (right side)
\definecolor{headerfontcol}{RGB}{0,0,0}%{255,255,255} % Text color for the header text in the content boxes
\definecolor{boxcolor}{RGB}{255,255,255} % Background color for the content in the content boxes

\begin{document}
\graphicspath{{Pictures/}}
\background{ % Set the background to an image (background.pdf)

}

\begin{poster}{
grid=false,
headerheight=0.17\textheight,
borderColor=bordercol, % Border color of content boxes
headerColorOne=headercol1, % Background color for the header in the content boxes (left side)
%headerColorTwo=headercol2, % Background color for the header in the content boxes (right side)
headerFontColor=headerfontcol, % Text color for the header text in the content boxes
boxColorOne=boxcolor, % Background color for the content in the content boxes
headershape=rectangle, % Specify the rounded corner in the content box headers
headershade=plain,
headerfont=\Large\sf\bf, % Font modifiers for the text in the content box headers
textborder=rectangle,
background=user,
headerborder=open, % Change to closed for a line under the content box headers
boxshade=plain,
eyecatcher=true
}
%
%----------------------------------------------------------------------------------------
%	TITLE AND AUTHOR NAME
%----------------------------------------------------------------------------------------
%
%\vspace{2em}
% Eye Catcher Images to go left of your title.
{
\includegraphics[height=0.13\textheight]{aau_logo_new.eps}
} %will not show if put eyecatcher=false
% Title
{\vspace{2pt}
Subjective Experience of Interacting with a Social Robot at a Danish Airport}
% Author
{
\vspace{3pt}
\normalsize{\textbf{Andreas Kornmaaler Hansen, Emil Bonnerup, Juliane Nilsson, Lucca Julie Nellemann \& Sara Nielsen}\\
Psychology Engineering - 17gr782 - Fall 2017 - School of Information and Communication Technology\\ Aalborg University, Aalborg, Denmark\\ }
$\{$akha14, ebonne14, jnils12, ljne14, snie14$\}$@student.aau.dk\\
}
{
\includegraphics[height=0.13\textheight]{aau_logo_new.eps}
}

%----------------------------------------------------------------------------------------
%	INTRODUCTION
%----------------------------------------------------------------------------------------
\headerbox{Introduction}
{name=introduction,column=0,row=0, span=1}
{\parskip 5pt   
This study originates from a social robot research project at Aalborg University with the aim of developing and implementing robots in a variety of contexts. This raises questions on how social robots should behave and which variables in a social robot is important. When important variables are eliciteted scales can be developed from these variables which can be use to test a social robot. The study consists of two tests, one where variables are elicitated and one where the scales are used to evaluate the robot, so possible correlation can be detected.  
}

\headerbox{Methods}
{name=method,span=1,column=0,below=introduction, span=1}
{\parskip 5pt 
To investigate which variables are important when interacting with social robots and to check for correlation on scales designed based on these variabels two tests are set up in Aalborg Airport (AAL). Both tests was conducted on Danish Travellers who interacted with a \textit{Double} robot shown on figure 1. In the first test subjects was asked to participate in a semi-structured interview about their first impressions after the interaction and in the second test subjects were asked to rate their interactions on the developed scales. The \textit{Double} robot was remotely controlled via a computer and a present controller. On the screen a developed wireframe to help with wayfinding in AAL was presented. 
\vspace{-10pt}  

\begin{center}
\includegraphics[width=0.30\linewidth]{ModificeretDoubleFront.eps}
\includegraphics[width=0.30\linewidth]{ModificeretDoubleSide.eps}

\textbf{Figure~1. }\textit{Double}'s front and profile.
\end{center}
\vspace{-10pt}  

The subjects were recruited by the robot, which provides a more ecological and undisturbed interaction between robot and subject. The robot approached potential subjects after the security check and asked to help travellers with wayfinding. If travellers wanted help, they were presented with four wayfinding options: Food, Shopping, Toilets or Gate information. After the interaction the robot led subjects to an interviewer. 

\textbf{Data Processing}
From the first test the interviews and observations were coded into affinity notes and an affinity diagram was made. This affinity diagram is pivotal in eliciting the variables that affect HRI, and thereafter in creating the scales to be used for further work. 
567 affinity notes were sorted into 10 green categories with individual subcategories. 

From the second test \textbf{Beskriv kort den databehandling}
}


\headerbox{Results - Elicitation of words}
{name=results1,span=1,column=1,aligned=introduction, span=1}
{\parskip 5pt 

}



\headerbox{Results - Scale Testing}
{name=results2,span=1,column=2,aligned=introduction}
{\parskip 5pt


}


\headerbox{Conclusion}
{name=conclusion,span=1,column=3,aligned=introduction}
{\parskip 5pt

}


\headerbox{Acknowledgements}
{name=akn,span=1,column=3,below=conclusion}
{\parskip 5pt

}


\headerbox{References}
{name=references,column=3,below=akn}
{
\renewcommand{\section}[2]{}%
%\parskip 5pt
 %   \bibliography{bib.bib}
%\tiny
\footnotesize
%Dorte har skrevet kilder ind på følgende måde
%Abdala C (1996) JASA \textbf{100}(6):3726-3740.
%Bonfils P et al. (1991) \textit{Arch Otolaryngol Head Neck Surg} \textbf{117}(10):1167–1171.
%Brown DK et al. (2000) \textit{Hearing Res}. \textbf{145}(1-2):17-24.
}


%-------------------------------------------------------------------------

\end{poster}
\end{document}