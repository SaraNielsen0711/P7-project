\documentclass[paperwidth=160cm,paperheight=100cm,landscape,fontscale=0.2941]{baposter}
\usepackage{lipsum} 
%841mm x 1189mm
\usepackage[font=small,labelfont=bf,hypcap=false]{caption} % Required for specifying captions to tables and figures
%\usepackage{cite}
\usepackage{booktabs} % Horizontal rules in tables
\usepackage{relsize} % Used for making text smaller in some places
%\usepackage[urlcolor  = blue]{hyperref}
%\graphicspath{{figures/}} % Directory in which figures are stored
\usepackage{multicol}
%\usepackage[style=ieee]{biblatex}
%\addbibresource{bib.bib}
\usepackage[utf8]{inputenc} 
%\usepackage{hyperref}
\usepackage{placeins}
\usepackage{epstopdf}
\usepackage{subcaption}
\usepackage[]{graphicx}
\usepackage{natbib}
\usepackage{times}

\bibliographystyle{abbrvnat}

\selectcolormodel{RGB} %<-- Add colour model defintion
\definecolor{bordercol}{RGB}{255,255,255}%{33,26,82} % Border color of content boxes
\definecolor{headercol1}{RGB}{255,255,255}%{33,26,82} % Background color for the header in the content boxes (left side)
%\definecolor{headercol2}{RGB}{5,2,82} % Background color for the header in the content boxes (right side)
\definecolor{headerfontcol}{RGB}{0,0,0}%{255,255,255} % Text color for the header text in the content boxes
\definecolor{boxcolor}{RGB}{255,255,255} % Background color for the content in the content boxes

\begin{document}
\graphicspath{{Pictures/}}
\background{ % Set the background to an image (background.pdf)

}

\begin{poster}{
grid=false,
headerheight=0.17\textheight,
borderColor=bordercol, % Border color of content boxes
headerColorOne=headercol1, % Background color for the header in the content boxes (left side)
%headerColorTwo=headercol2, % Background color for the header in the content boxes (right side)
headerFontColor=headerfontcol, % Text color for the header text in the content boxes
boxColorOne=boxcolor, % Background color for the content in the content boxes
headershape=rectangle, % Specify the rounded corner in the content box headers
headershade=plain,
headerfont=\Large\sf\bf, % Font modifiers for the text in the content box headers
textborder=rectangle,
background=user,
headerborder=open, % Change to closed for a line under the content box headers
boxshade=plain,
eyecatcher=true
}
%
%----------------------------------------------------------------------------------------
%	TITLE AND AUTHOR NAME
%----------------------------------------------------------------------------------------
%
%\vspace{2em}
% Eye Catcher Images to go left of your title.
{
\includegraphics[height=0.13\textheight]{aau_logo_new.eps}
} %will not show if put eyecatcher=false
% Title
{\vspace{2pt}
Subjective Experience of Interacting with a Social Robot at a Danish Airport}
% Author
{
\vspace{3pt}
\normalsize{\textbf{Andreas Kornmaaler Hansen, Emil Bonnerup, Juliane Nilsson, Lucca Julie Nellemann \& Sara Nielsen}\\
Psychology Engineering - 17gr782 - Fall 2017 - School of Information and Communication Technology\\ Aalborg University, Aalborg, Denmark\\ }
$\{$akha12, ebonne14, jnils12, ljne14, snie14$\}$@student.aau.dk\\
}
{
\includegraphics[height=0.13\textheight]{aau_logo_new.eps}
}

%----------------------------------------------------------------------------------------
%	INTRODUCTION
%----------------------------------------------------------------------------------------
%\begin{multicols}{4}
\headerbox{Introduction}
{name=introduction,column=0,row=0, span=1}
{\parskip 5pt   
This study originates from a social robot research project at Aalborg University with the aim of developing and implementing robots in a variety of contexts. This raises questions on how social robots should behave and which variables in a social robot is important. When important variables are eliciteted scales can be developed from these variables which can be use to test a social robot. The study consists of two tests, one where variables are elicitated and one where the scales are used to evaluate the robot. 
}
%\end{multicols}

\headerbox{Method - Elicitation of words}
{name=method1,span=1,column=0,below=introduction, span=1}
{\parskip 5pt 
The first test was conducted on Danish travellers who interacted with a social robot in a natural setting. The test was conducted at Aalborg Airport (AAL) after the travellers went through the security check and before they reached the shopping and dining area at the airport. The travellers who interacted with the robot were asked to participate in a semi-structured interview about their first impressions while being observed during both the interaction and the interview. 

\subsection{Materials}
For the test a \textit{Double} robot from Double Robotics Inc with an iPad Air 2 was used. It was decided to change the head mount so that the iPad is angled upwards. The modified \textit{Double} robot is shown on \autoref{fig:ModificeretDoubleFront} and \autoref{fig:ModificeretDoubleSideClose}. The \textit{Double} robot was controlled via a computer and on the screen a developed wireframe to help with wayfinding in AAL was presented. 

\subsection{Subject Recruitment}
30 subjects from the age of 8 to 62 years (M=37.9, SD=17.1) distributed among 16 females and 14 males participated. The subjects were recruited by the robot itself, which was remotely controlled by a present controller. The robot recruitment was chosen because it provide a more ecological and undisturbed interaction between robot and subject. The robot approached potential subjects in the area after the security check. The wireframe on the iPad asked the subjects if it might help them find their way around AAL and presented a "Yes/No" option. If "No" was pressed, the robot wished the traveller a pleasent journey. If "Yes" was pressed, the subjects were presented with four wayfinding options: Food, Shopping, Toilets, or Gate information. The subjects were then kindly asked to follow the robot after they had chosen their preferred option. The robot then led the subjects to an interviewer who shortly informed them of the study and received verbal consent to record audio during the semi-structured interview. In total 18 interviews were conducted of which 11 were on single travellers and seven were on a group of travellers.
%
\begin{figure}[H]
	\centering
	\begin{minipage}{.25\textwidth}
		\centering
		\includegraphics[width=\linewidth, angle =-90]{Figure/ModificeretDoubleFront}
		\caption{\textit{Double}'s front.}
		\label{fig:ModificeretDoubleFront}
	\end{minipage}%
	\begin{minipage}{.25\textwidth}
		\centering
		\includegraphics[width=\linewidth, angle =-90]{Figure/ModificeretDoubleSideClose}
		\caption{\textit{Double}'s profile.}
		\label{fig:ModificeretDoubleSideClose}
	\end{minipage}
\end{figure}
\noindent
%

\subsection{Semi-structured Interview}
The interview was a two part semi-structured interview. The first part consisted of probing the subjects for their first impression and experience of interacting with the robot in regard to their thoughts about the robot itself and what they think other travellers might think about the interaction. In addition to the aforementioned conversation topics the subjects were asked about their opinion regarding the robots approach, relevance, and reliability. The last topic of conversation in the first part of the interview regards the subjects experiences at an airport where robot help might have been useful.

%The following are only guidance to the conversation topics and not specific questions:\\ 
%
%\begin{itemize}
%\item First impression of the robot. SKREVET IND
%\item How the robot approached the subject. SKREVET IND
%\item What the subject think about the robot. SKREVET IND
%\item What the subject think other travellers think of their interaction with the robot. SKREVET IN
%\item Robot relevance. SKREVET IND
%\item Robot reliability. SKREVET IND
%\item Experiences at an airport where robot help might have been useful.\\
%\end{itemize}
%\noindent
%
The second part consisted of asking specific questions relating to the robots physical characteristics such as speed, height, distance, movements, appearance, and approach. These questions were asked because the variables are known to affect the experience of HRI \cite{PDF:HowMayIServeYou}.\\

The two parts were conducted in continuation of each other and the questions in the second part were only asked if the subjects had not previously answered them spontaneously.

\subsection{Roles}
Five researchers were present during the test at AAL. One controlled the \textit{Double} robot, one conducted the interview, and the remaining three observed the travellers as they interacted with or walked past the robot. 

\subsection{Data Processing}
The interviews were transcribed and coded along with observations into affinity notes. The purpose was to create an affinity diagram, which brings insights and issues into a hierarchical diagram based on subjects' statements and behaviour \cite{Wendell2005}. In the end this affinity diagram will be pivotal in eliciting the variables that affect HRI, and thereafter in creating the scales to be used for further work. 
567 affinity notes were sorted into 10 green categories with individual subcategories. To gain more insight it was decided to mix the spontaneous answers from the conversation topics with the answers from the specific questions. These were not differentiated in the affinity diagram.
}

\headerbox{Results - Elicitation of words}
{name=results1,span=1,column=1,alignes=introduction, span=1}
{\parskip 5pt 

}


\headerbox{Method - Scale Testing}
{name=method2,span=1,column=1,below=results1}
{\parskip 5pt 


}

\headerbox{Results - Scale Testing}
{name=results2,span=1,column=2,aligned=introduction}
{\parskip 5pt


}


\headerbox{Conclusion}
{name=conclusion,span=1,column=3,aligned=introduction}
{\parskip 5pt

}


\headerbox{Acknowledgements}
{name=akn,span=1,column=3,below=conclusion}
{\parskip 5pt

}


\headerbox{References}
{name=references,column=3,below=akn}
{
\renewcommand{\section}[2]{}%
%\parskip 5pt
 %   \bibliography{bib.bib}
%\tiny
\footnotesize
%Dorte har skrevet kilder ind på følgende måde
%Abdala C (1996) JASA \textbf{100}(6):3726-3740.
%Bonfils P et al. (1991) \textit{Arch Otolaryngol Head Neck Surg} \textbf{117}(10):1167–1171.
%Brown DK et al. (2000) \textit{Hearing Res}. \textbf{145}(1-2):17-24.
}


%-------------------------------------------------------------------------

\end{poster}
\end{document}