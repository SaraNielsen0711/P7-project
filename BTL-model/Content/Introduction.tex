\section*{Introduction}
\label{Introduktion}

The purpose of this article is to analyse data obtained from \textcite{Ellermeier2004} using the probabilistic choice model Bradley-Terry-Luce (BTL). More specifically to check for transitivity in the data which includes Weak Stochastic Transitivity (WST), Mean Stochastic Transitivity (MST), and Strong Stochastic Transitivity (SST). The dataset is analysed using the matlab function fOptiPt.m developed by Florian Wickelmaier and Christian Schmid, \parencite{Wickelmaier2004}. See \autoref{tab:data} for the absolute frequencies used in a cumulative preference matrix. The scores denote how many participants voted the sound in question as being the most unpleasant.

\begin{table}[H]
\centering
\begin{tabular}{@{}llllllllllll@{}}
\toprule
Sounds     & No. & 1  & 2  & 3  & 4  & 5  & 6  & 7  & 8  & 9  & 10 \\ \midrule
Truck      & 1   & 0  & 9  & 16 & 45 & 56 & 5  & 29 & 6  & 24 & 33 \\
Brake      & 2   & 51 & 0  & 34 & 58 & 58 & 13 & 46 & 30 & 39 & 50 \\
Train      & 3   & 44 & 26 & 0  & 55 & 57 & 9  & 48 & 37 & 38 & 55 \\
Water      & 4   & 15 & 2  & 5  & 0  & 38 & 2  & 17 & 6  & 6  & 20 \\
Boat       & 5   & 4  & 2  & 3  & 22 & 0  & 3  & 6  & 3  & 3  & 12 \\
Jackhammer & 6   & 55 & 47 & 51 & 58 & 57 & 0  & 58 & 53 & 55 & 57 \\
Mower      & 7   & 31 & 14 & 12 & 43 & 54 & 2  & 0  & 16 & 17 & 41 \\
Crash      & 8   & 54 & 30 & 23 & 54 & 57 & 7  & 44 & 0  & 40 & 52 \\
Mixer      & 9   & 36 & 21 & 22 & 54 & 57 & 5  & 43 & 20 & 0  & 43 \\
Vent       & 10  & 27 & 10 & 5  & 40 & 48 & 3  & 19 & 8  & 17 & 0  \\ \bottomrule
\end{tabular}
\caption{The cumulative preference matrix of the unpleasant sounds from \textcite{Ellermeier2004}. The absolute frequency denotes how many participants judged the sound as being the most unpleasant}
\label{tab:data}
\end{table} 

The probabilistic method used is called the BTL model. It is a model which checks for transitivity in data containing pairwise comparisons and is able to predict the outcome of the comparison. See \autoref{eq:BTL}.

\begin{equation}
P_{ab} =\frac{v(a)}{v(a)+v(b)} 
\label{eq:BTL}
\end{equation}
%
$P_{ab}$ is the probability of sound \textit{a} being rated more unpleasant than sound \textit{b}. \textit{v(a)} and \textit{v(b)} denotes the scale values of the sounds \textit{a} and \textit{b}.
%




In order to get the scale values and parameter estimates\fxnote{er ikke sikker}, one has to maximise the likelihood of the data seen in \autoref{tab:data}. This is done given the model: 

\begin{equation}
L(D|\Theta_{model}) = \prod_{i<j} p_{ij} ^{n_{ij}}\cdot(1- p_{ij})^{n-n_{ij}}
\end{equation}

where n is the number of comparisons, which is equal to the amount of test subjects in this case (\textit{n=60}) and $n_{ij}$ is the frequency of how many times the sound in row \textit{i} was rated as being more unpleasant over the sound in column \textit{j}. $P_{ij}$ is the preference probability and is estimated using \autoref{eq:BTL}. 
