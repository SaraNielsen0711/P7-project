% This file was converted to LaTeX by Writer2LaTeX ver. 1.2
% see http://writer2latex.sourceforge.net for more info
\documentclass[a4paper]{article}
\usepackage[latin1]{inputenc}
\usepackage[T1]{fontenc}
\usepackage[english]{babel}
\usepackage{amsmath}
\usepackage{amssymb,amsfonts,textcomp}
\usepackage{color}
\usepackage[top=2.54cm,bottom=2.54cm,left=2.5cm,right=2.5cm,nohead,nofoot]{geometry}
\usepackage{array}
\usepackage{hhline}
\usepackage{hyperref}
\hypersetup{colorlinks=true, linkcolor=blue, citecolor=blue, filecolor=blue, urlcolor=blue}
% Footnote rule
%\setlength{\skip\footins}{0.119cm}
%\renewcommand\footnoterule{\vspace*{-0.018cm}\setlength\leftskip{0pt}\setlength\rightskip{0pt plus 1fil}\noindent\textcolor{black}{\rule{0.25\columnwidth}{0.018cm}}\vspace*{0.101cm}}
\title{Subjective Experience of Interacting with a Social Robot at a Danish Airport}
\begin{document}

%\clearpage\clearpage\setcounter{page}{1}{\centering\color{black}
{\centering 
\subsection*{Subjective Experience of Interacting with a Social Robot at a Danish Airport}}

{\centering
\textit{Andreas Kornmaaler Hansen*, Emil Bonnerup**, Juliane Nilsson, \\
Lucca Julie Nelleman and Sara Nielsen}
\par}

{\centering
\textbf{Group 782}
\par}


\bigskip

\paragraph{Introduction}
%This sample abstract provides instructions and the basic guide-lines for preparing an abstract for the Student Conference in Scientific Communication (SEMCON) to be held in Aalborg, Denmark. The abstracts must contain no more than 400 words (excl. references) and no more than 1 page. The title of the abstract is followed by the list of authors in alphabetic order. Below is given the affiliation of the authors in terms of the group number. The name of the presenter of the poster must be marked with {\textquotedbl}*{\textquotedbl} and the name of the person giving the oral presentation with {\textquotedbl}**{\textquotedbl}.
An ecological field study and a scaling test were conducted to assess the experience of social robots at Aalborg Airport. The purpose of the field study was to develop scales based on Danish travellers' own words and observational data. The scale test had the purpose of using and testing the scales developed from the field study. The idea behind the scales is that the scales will help robot designers to design their robot for different use cases and target groups. 

\paragraph{Methods and Proposals}
%The text was checked for spelling-errors. Abbreviations and acronyms were always defined the first time they were used in the text. The following style was used for references and citations [1]. Multiple references were numbered consecutively in square brackets [1,2]. The sentence punctuation followed the brackets [2]. Avoid the use of {\textquotedbl}Ref. [2]{\textquotedbl} or {\textquotedbl}reference [2]{\textquotedbl}, except at the beginning of a sentence. The list of references was formatted as follows: List of authors. Title of paper with first word capitalized, name of journal, year, volume number, page numbers separated by a hyphen [1]. For references to books use the style as in [2].
%
In both tests the travellers were recruited by a remote controlled robot from Double Robotics, Inc., which had an iPad with an interface asking if it may help the travellers with wayfinding at AAL. When the subjects had chosen the desired location they were kindly asked to follow the robot. In the first test the robot led them to a semi-structured interview about their experience with the robot. The behaviour of the subjects was observed throughout the interaction with the robot and the interview. In the second test the physical parameters height, distance to subject, and angle of approach were altered throughout the study in order to be able to test the scales. When subjects started following the robot they were stopped and asked to evaluate the robot and the interaction with it on the 24 scales develop from the previous study. 

\paragraph{Results - Elicitation of Parameters}
%The abstracts should be submitted in a pdf file adhering exactly to this format. Please send it to Tatiana K. Madsen, \href{mailto:tatiana@es.aau.dk}{tatiana@es.aau.dk}. State clearly in the subject [SEMCON] and the group number. All abstracts should be in accordance with the instructions.
The observations and the 30 subjects' statements were interpreted and coded using an affinity diagram. 567 affinity notes were sorted by a bottom-up procedure into ten categories which roughly revolved around appearance, trust, behaviour, approach, problems with touch screen, avoidance of interaction, personal interest, positivity towards the robot, usefulness, and tech-experience. Variables were formulated as scale questions with labels for each category. A Visual Analogue Scale (VAS) with closed endpoints was used for the scale presentation. In total 24 scales were developed, 11 scales were unipolar and 13 were bipolar. All of the bipolar scales had a midpoint and on two of them the label ``fine'' was added.  

\paragraph{Results - Scale Testing}
In total 43 subjects participated and answered the scales. The variance for each scale was very different, because of the varied use of midpoints and labels. The scale responses were analysed with Principal Component Analysis (PCA) with groupings relating to the robot's height, distance from subject, and direction of approach. Within the group \textit{robot height} four pairs of positive correlation and five negative correlations were found. For the group \textit{distance} there were four positive and six negative correlations, and for the group \textit{direction} three positive and five negative correlations were found. 

\paragraph{Discussion}
%An online book of abstracts will be prepared containing all abstracts for SEMCON including a specific time schedule for the conference. This was organized by the conference secretariat. The abstract book allows a better communication between participants during the conference. Please be aware of the tight schedule for the SEMCON, which does not allow for computer-projector problems. For each session, the presenters should upload their presentations to the designated computer in the presentation room 15 min before the session start. In case of technical problems, it is recommended to also send your presentation to Tatiana Madsen. Only presentations in Powerpoint (.ppt, pptx) or PDF (.pdf) formats are accepted.
The results presented in this study probably needs further validation given that they were collected on a small sample size. Even though correlation is found it might be useful to test the reliability of the scale items. Overall the subjects who participated were very fond of technology which might have biased their scale responses. Further, the mid label ``fine '' should be reconsidered due to the small variation in the scale responses. Perhaps the label "fine" is to broad and does not represent a fitting mid-point. 

\paragraph{References}

\begin{description}
%\item{[1]} Haugland MK, Hoffer JA. Sensory nerve signals obtained from cuff electrodes during functional electrical stimulation of nearby muscles. IEEE Trans. Rehab. Eng. 1994; 2: 37-40.
%
%\item{[2]} Jacobs IS, Bean CP. Fine particles, thin films and exchange anisotropy. In: Rado GT, Suhl H (eds.), Magnetism (vol. 3). New York NY: Academic Press, 1963: 271-350.
%
{\footnotesize \item{[1]} Halpern D, Katz JE. Close but not stuck: Understanding social distance in human-robot interaction through a computer mediation approach. 2013; 1: 17-34.
%
\item{[2]} Pacchierotti E, Christensen HI, Jensfelt P. Human-robot embodied interaction in hallway settings: a pilot user study. IEEE. 2005: 164-171.
%
\item{[3]} Dautenhahn K, Walters M, Woods S, Koay KL, Nehaniv CL, Sisbot EA, Alami R, Sim{\'e}on T. How may I serve you? A robot companion approaching a seated person in a helping context. 2006: 172-179.
%
\item{[4]} De Graaf MMA, Allouch SB. Exploring influencing variables for the acceptance of social robots. Robotics and Autonomous Systems. Elsevier. 2013; 61: 1476-1486.
%
\item{[5]} Kim Y, Mutlu B. How social distance shapes human-robot interaction. Int. J. Human-Computer Studies, 2014; 72: 783-795.
}
\end{description}

\end{document}