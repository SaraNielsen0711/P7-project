% This file was converted to LaTeX by Writer2LaTeX ver. 1.2
% see http://writer2latex.sourceforge.net for more info
\documentclass[a4paper]{article}
\usepackage[latin1]{inputenc}
\usepackage[T1]{fontenc}
\usepackage[english]{babel}
\usepackage{amsmath}
\usepackage{amssymb,amsfonts,textcomp}
\usepackage{color}
\usepackage[top=2.54cm,bottom=2.54cm,left=2.54cm,right=2.54cm,nohead,nofoot]{geometry}
\usepackage{array}
\usepackage{hhline}
\usepackage{hyperref}
\hypersetup{colorlinks=true, linkcolor=blue, citecolor=blue, filecolor=blue, urlcolor=blue}
% Footnote rule
%\setlength{\skip\footins}{0.119cm}
%\renewcommand\footnoterule{\vspace*{-0.018cm}\setlength\leftskip{0pt}\setlength\rightskip{0pt plus 1fil}\noindent\textcolor{black}{\rule{0.25\columnwidth}{0.018cm}}\vspace*{0.101cm}}
\title{Subjective Experience of Interacting with a Social Robot at a Danish Airport}
\begin{document}

%\clearpage\clearpage\setcounter{page}{1}{\centering\color{black}
{\centering 
\subsection*{Subjective Experience of Interacting with a Social Robot at a Danish Airport}}

{\centering
\textit{Andreas Kornmaaler Hansen*, Emil Bonnerup**, Juliane Nilsson, \\
Lucca Julie Nelleman and Sara Nielsen}
\par}

{\centering
\textbf{Group 782}
\par}


\bigskip

\paragraph{Introduction}
%This sample abstract provides instructions and the basic guide-lines for preparing an abstract for the Student Conference in Scientific Communication (SEMCON) to be held in Aalborg, Denmark. The abstracts must contain no more than 400 words (excl. references) and no more than 1 page. The title of the abstract is followed by the list of authors in alphabetic order. Below is given the affiliation of the authors in terms of the group number. The name of the presenter of the poster must be marked with {\textquotedbl}*{\textquotedbl} and the name of the person giving the oral presentation with {\textquotedbl}**{\textquotedbl}.
This paper investigates the subjective experience of interaction with a social robot at Aalborg Airport (AAL) by conducting an ecological field study and a scaling experiment. The purpose of the field study was to develop scales based on Danish travellers' own words and observational data. The idea behind the scales was that the scales will help robot designers to better design their robot for different use cases and target groups. The paper describes two parts of the study. The first part revolves around the development of the scales and the second part describes the testing of the scales.

\paragraph{Methods and Proposals - Elicitation of parameters}
%The text was checked for spelling-errors. Abbreviations and acronyms were always defined the first time they were used in the text. The following style was used for references and citations [1]. Multiple references were numbered consecutively in square brackets [1,2]. The sentence punctuation followed the brackets [2]. Avoid the use of {\textquotedbl}Ref. [2]{\textquotedbl} or {\textquotedbl}reference [2]{\textquotedbl}, except at the beginning of a sentence. The list of references was formatted as follows: List of authors. Title of paper with first word capitalized, name of journal, year, volume number, page numbers separated by a hyphen [1]. For references to books use the style as in [2].
In the first part, travellers were recruited by a remote controlled robot from Double Robotics, Inc., which had an iPad with an interface asking if it may help the travellers with wayfinding at AAL. When the subjects had chosen the desired location they were kindly asked to follow the robot, which led them to a semi-structured interview about their experience with the robot. The behaviour of the subjects was observed throughout the interaction with the robot and the interview. 

\paragraph{Results - Elicitation of parameters}
%The abstracts should be submitted in a pdf file adhering exactly to this format. Please send it to Tatiana K. Madsen, \href{mailto:tatiana@es.aau.dk}{tatiana@es.aau.dk}. State clearly in the subject [SEMCON] and the group number. All abstracts should be in accordance with the instructions.
The observations and the subjects' statements were interpreted and coded using an affinity diagram. 567 affinity notes were sorted by a bottom-up procedure into ten categories which roughly revolved around appearance, trust, behaviour, approach, problems with touch screen, avoidance of interaction, personal interest, positivity towards the robot, usability for people, and tech-experience. Variables were formulated as scale questions for each of the ten categories which was used in the second part of the study where the scales were tested. A Visual Analogue Scale (VAS) was used for the scale presentation.

\paragraph{Methods and Proposals - Scale Testing}
The second part of the study was also conducted in AAL and participants were again recruited by the robot. The physical parameters height, distance to subject, and angle of approach were altered throughout the study in order to be able to test the scales. When subjects started following the robot they were stopped and asked to evaluate the robot and the interaction with it on the 24 scales develop from the previous study. 

\paragraph{Results - Scale Testing}
In total 43 subject participated and answered the scales. The scale responses were analysed with Principal Component Analysis (PCA). 

\paragraph{Discussion}
%An online book of abstracts will be prepared containing all abstracts for SEMCON including a specific time schedule for the conference. This was organized by the conference secretariat. The abstract book allows a better communication between participants during the conference. Please be aware of the tight schedule for the SEMCON, which does not allow for computer-projector problems. For each session, the presenters should upload their presentations to the designated computer in the presentation room 15 min before the session start. In case of technical problems, it is recommended to also send your presentation to Tatiana Madsen. Only presentations in Powerpoint (.ppt, pptx) or PDF (.pdf) formats are accepted.


\paragraph{References}

\begin{description}
\item{[1]} Haugland MK, Hoffer JA. Sensory nerve signals obtained from cuff electrodes during functional electrical stimulation of nearby muscles. IEEE Trans. Rehab. Eng. 1994; 2: 37-40.

%\item{[2]} Jacobs IS, Bean CP. Fine particles, thin films and exchange anisotropy. In: Rado GT, Suhl H (eds.), Magnetism (vol. 3). New York NY: Academic Press, 1963: 271-350.
\end{description}

\end{document}