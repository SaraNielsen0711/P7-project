\label{Abstract}
% As a general rule, do not put math, special symbols or citations
% in the abstract
This paper investigates the subjective experience of interacting with a social robot at Aalborg Airport (AAL) by conducting an ecological field study.  The purpose was to develop scales based on Danish travellers' own words and observational data. The idea behind the scales was that the scales will help robot designers to better design their robot for different use cases and target groups. The paper describes two parts of the study. The first part revolves around the development of the scales and the second part describes the testing of the scales.
%{\color{red} Her tilføjes en kort beskrivelse af at der er udført to test og hvorfor. }

In the first part, travellers were recruited by a remote controlled robot from Double Robotics, Inc., which had an iPad with an interface asking if it may help the travellers with wayfinding at AAL. When the subjects had chosen the desired location they were kindly asked to follow the robot, which led them to a semi-structured interview about their experience with the robot. The behaviour of the subjects was observed throughout the interaction with the robot and the interview. 

The observations and the subjects' statements were interpreted and coded using an affinity diagram. 567 affinity notes were sorted by a bottom-up procedure into ten categories which roughly revolved around appearance, trust, behaviour, approach, problems with touch screen, avoidance of interaction, personal interest, positivity towards the robot, usability for people, and tech-experience.  Variables were formulated as scale questions for each of the ten categories which was used in the second part of the study where the scales were tested. It was decided to use a Visual Analogue Scale (VAS) for scale presentation.

The second part of the study was also conducted in AAL and participants were again recruited by the robot. The physical parameters height, distance to subject, and angle of approach were altered throughout the study in order to be able to test the scales. The participants answered the scales on a PC. 

{\color{red} Mangler resultater samt en kort opsummering.}

