\section*{Læsevejledning}
\label{Laesevejledning}
Arbejdsbladene bør læses kronologisk, da nogle afsnit antager, at læseren har kendskab til tidligere afsnit i rapporten. Derudover er rapporten struktureret således, at resultater og viden løbende diskuteres og konkluderes.

\subsection*{Henvisninger}
Alle henvisninger i rapporten fungerer som hyperlinks, som illustreret i eksemplet her: \autoref{}.

\subsection*{Kildehenvisninger}
Kildehenvisninger angives enten som en del af teksten eller i parentes. Et eksempel på de to kildehenvisningsmetoder: \textcite[s. 13]{PDF:RobotShiftFromIPtoSR} eller \parencite[s. 13]{PDF:RobotShiftFromIPtoSR}. Såfremt der refereres til en bestemt del af kilden, angives dette med sidetal, eksempelvis; s. 13 for én bestemt side eller ss. 1-3 for flere sider.
%
\subsection*{Afsnitshenvisning}
Afsnitshenvisninger angives med et afsnitsnummer efterfulgt af et afsnitsnavn. Et eksempel på en afsnitshenvisning: \fullref{KarakteriseringAfSocialRobot}. Samme gør sig gældende for kapitler.
%
\subsection*{Figurhenvisning}
Henvisninger til figurer angives med et decimaltal, som først gengiver kapitlets nummer efterfulgt af figurnummeret i det pågældende kapitel. Et eksempel på en figurhenvisning: \autoref{fig:CategorizationOfRobots}, der svarer til figur 1 i kapitel 2. 
%
\subsection*{Henvisninger til elektronisk bilag}
Henvisninger til det elektroniske bilag angives med en afsnitshenvisning, hvor stien til det pågældende bilag fremgår i det pågældende afsnit. 



% \subsubsection*{Kildehenvisninger}
% Kildehenvisninger angives enten som en del af teksten eller i parentes. Et eksempel på de to kildehenvisningsmetoder: \citet[p. 1]{DiscoveringStatisticsUsingR}, eller \citep[p. 1]{DiscoveringStatisticsUsingR}. Såfremt der refereres til en bestemt del af kilden angives dette med sidetal, eksempelvis; p. 1 eller pp. 1-3.
% Figurer der ikke er kreeret af gruppen selv, vil have et link eller en henvisning i figurteksten.
% %
% \subsection*{Afsnitshenvisning}
% Afsnitshenvisninger angives enten med et afsnitsnummer efterfulgt af et afsnitsnavn;\\
% \fullref{cha:DiskoLineOgCombine}\\
% eller bare kapitel- eller afsnitsnummer;\\
% \autoref{cha:DiskoLineOgCombine}.\\
% Det samme gør sig gældende for kapitler.
% %
% \subsection*{Figurhenvisning}
% Henvisninger til figurer angives med et decimaltal, som først gengiver kapitlets nummer efterfulgt af figurnummeret i det pågældende kapitel. Et eksempel på en figurhenvisning: \autoref{fig:KortGroenland}.
% %
% \subsection*{Bilagshenvisninger}
% Henvisninger til bilag angives med et bogstav. Et eksempel på en bilagshenvisning: \autoref{app:Analytics}
% %
% \subsection*{Decimalseparator}
% Der benyttes "," (komma), som decimalseparator.

% \noindent Rapporten indeholder en analyse af Disko Lines nuværende salgsportal, www.diskoline.dk. Denne analyse er udarbejdet i løbet af februar og marts, men siden da har Combine foretaget mindre ændringer på hjemmesiden, som ikke beskrives yderligere i rapporten. Enkelte aspekter af analysen vil derfor muligvis ikke stemme overens med den nuværende version af hjemmesiden. 