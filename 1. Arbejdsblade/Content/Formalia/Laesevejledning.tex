\section*{Læsevejledning}
\label{Laesevejledning}
% SKAL OMSKRIVES SÅ DET PASSER MED VORES PROJEKT
Rapporten bør læses kronologisk, da nogle afsnit antager, at læseren har kendskab til tidligere afsnit i rapporten. Derudover er rapporten struktureret således, at resultater og viden løbende diskuteres og konkluderes.
%
\subsection*{Kildehenvisninger}
Kildehenvisninger angives enten som en del af teksten eller i parentes. Et eksempel på de to kildehenvisningsmetoder: \textcite[s. 13]{PDF:RobotShiftFromIPtoSR} eller \parencite[s. 13]{PDF:RobotShiftFromIPtoSR}. Såfremt der refereres til en bestemt del af kilden, angives dette med sidetal, eksempeltvist; s. 13 for én bestemt side eller ss. 1-3 for flere sider.
%
\subsection*{Afsnitshenvisning}
Afsnitshenvisninger angives med et afsnitsnummer efterfulgt af et afsnitsnavn. Et eksempel på en afsnitshenvisning: \fullref{KarakteriseringAfSocialRobot}. Samme gør sig gældende for kapitler.
%
\subsection*{Figurhenvisning}
Henvisninger til figurer angives med et decimaltal, som først gengiver kapitlets nummer efterfulgt af figurnummeret i det pågældende kapitel. Et eksempel på en figurhenvisning: \autoref{fig:CategorizationOfRobots}, der svarer til figur 1 i kapitel 2. 
%
\subsection*{Henvisninger til elektronisk bilag}
Henvisninger til det elektroniske bilag angives med en afsnitshenvisning, hvor stien til det pågældende bilag fremgår i det pågældende afsnit. 