\section*{Læsevejledning}
\label{Laesevejledning}
Arbejdsbladene er skrevet kronologisk, efterhånden som projektet er skredet frem. De kan med fordel læses kronologisk, da nogle afsnit antager, at læseren har kendskab til tidligere afsnit i arbejdsbladene. Der vil dog også være henvisninger til tidligere omtalte emner, når relevant.

\subsection*{Henvisninger}
Alle henvisninger i rapporten fungerer som hyperlinks, som illustreret i eksemplet her: \autoref{fig:CategorizationOfRobots}.
%
\subsection*{Kildehenvisninger}
Kildehenvisninger angives enten som en del af teksten eller i parentes. Et eksempel på de to kildehenvisningsmetoder: \textcite[s. 13]{PDF:RobotShiftFromIPtoSR} eller \parencite[s. 13]{PDF:RobotShiftFromIPtoSR}. Såfremt der refereres til en bestemt del af kilden, angives dette med sidetal, eksempelvis; s. 13 for én bestemt side eller ss. 1-3 for flere sider.
%
\subsection*{Afsnitshenvisning}
Afsnitshenvisninger angives med et afsnitsnummer efterfulgt af et afsnitsnavn. Et eksempel på en afsnitshenvisning: \fullref{KarakteriseringAfSocialRobot}. Samme gør sig gældende for kapitler.
%
\subsection*{Figurhenvisning}
Henvisninger til figurer angives med et decimaltal, som først gengiver kapitlets nummer efterfulgt af figurnummeret i det pågældende kapitel. Et eksempel på en figurhenvisning: \autoref{fig:CategorizationOfRobots}, der svarer til figur 1 i kapitel 2. 
%
\subsection*{Henvisninger til elektronisk bilag}
Henvisninger til det elektroniske bilag angives med en afsnitshenvisning, hvor stien til det pågældende bilag fremgår i det pågældende afsnit. 

\subsection*{Decimalseparator}
Der benyttes "." (punktum), som decimalseparator.