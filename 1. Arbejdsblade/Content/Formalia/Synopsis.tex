%\fxwrite{Synopsis}
I dette studie undersøges danske rejsendes subjektive oplevelse af at blive betjent af en social robot i Aalborg Lufthavn (AAL), gennem to økologiske feltstudier. I begge tests blev de rejsende tilnærmet af en Double robot, som tilbød at hjælpe dem med at finde vej. Efter en kort interaktion, beder robotten den rejsende følge efter sig. I første test kørte robotten de rejsende hen til et semi-struktureret interview om deres oplevelse. På baggrund af observationer og interviews fra denne test, blev 567 affinity notes kategoriseret i et affinity diagram med 10 kategorier, hvorfra vigtige parametre i forhold til brugeroplevelsen af sociale robotter udledes. På baggrund af disse parametre udvikles 24 skalaer, som efterfølgende er blevet anvendt i lufthavnen. I anden test ledte robotten de rejsende hen til en forsøgsleder, hvor de på de udviklede skalaer vurderede deres oplevelse af interaktionn med robotten. Resultaterne fra anden test blev analyseret med Principal Component Analysis, som viste både positive og negative korrelationer mellem flere af skalaerne.