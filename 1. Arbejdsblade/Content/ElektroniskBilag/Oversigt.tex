\chapter{Oversigt}
\label{ElektroniskBilagOversigt}
%
Følgende er en oversigt over det elektroniske bilag, som er i den vedhæftede ZIP-fil: (/ElektroniskBilag).

\section{Transskriberede data}
\label{ElektroniskBilagTranskriberede}
%
Det transskriberede data er transskriberet ud fra lydoptagelserne foretaget i feltundersøgelsen, og forefindes i dokumentet: (/ElektroniskBilag/TransskriberedeData). Første side i dokumentet indeholder en oversigt over hvor transskriberingen til hver testperson findes. 

\section{Affinity notes}
\label{ElektroniskBilagAffinityNotes}
%
De udarbejde \textit{affinity notes} fra feltundersøgelsen forefindes: (/ElektroniskBilag/AffinityNotes), hvor \textit{affinity notes} til hvert interview samt observationerne er skrevet ind. 

\section{Program til VAS}
\label{ElektroniskBilagProgram}
%
Det udarbejdede program til præsentation af skalaer forefindes i: (/ElektroniskBilag/VASProgram)

\section{Skala Oversigt}
\label{ElektroniskBilagSkalaOversigt}
%
En oversigt over hvilke skala spørgsmål (\textit{Scale question}), der er hører til skala nummer (\textit{SQ\#}) samt tilhørende labels (\textit{Skala labels}), der præsenteres på hver af de syv sider (\textit{Side}), forefindes i: (/ElektroniskBilag/SkalaOversigt.pdf)

\section{Rådata i Excel}
\label{ElektroniskBilagExcel}
%
Rådata, der består af testpersonernes respons på de 23 skalaer samt demografi, forefindes i: (/ElektroniskBilag/RaaData).

\section{Hisogram og normalfordelings plot}
\label{ElektroniskBilagHistNormal}
%
Til hver skala spørgsmål, samt til besvarelserne ift teknologi, er der opstillet et histogram for besvarelserne hvor funktionen for normalfordelingen er plottet ovenpå.  
Skaleringen af akserne er ens for alle plots, med undtagelse af plottet for SQ5, SQ7 og SQ11. Alle plots findes i: (/ElektroniskBilag/HistogramNormalfordeling)

\section{MATLAB Scripts til PCA}
\label{ElektroniskBilagMatLabPCA}
%
MATLAB scriptet der er anvendt i forbindelse med PCA samt script der anvendes til import af data i MATLAB forefindes i: (/ElektroniskBilag/MATLABScripts).

\section{3D Bi-plots}
\label{ElektroniskBilag3D}
%
Der er vedlagt tre 3D \textit{Bi}-plots for henholdvist: Det samlede data, højde og indgangsvinkel, som forefindes i: (/ElektroniskBilag/3DBiplots/), hvor det samlede data fremgår af: Biplot3D.fig, højde fremgår af: RHeight-3D.fig, og afstand fremgår af: Direction-3D.fi

\section{Tendenser}
\label{ElektroniskBilagTendenser}
%
For undersøgelsen af tendenser mellem testpersonernes besvarelser og robottens højde, forefindes samtlige grafer i: (/ElektroniskBilag/Tendenser/Højde-skala.xlsx)\blankline
%
For undersøgelsen af tendenser mellem testpersonernes besvarelser og robottens afstand, forefindes samtlige grafer i: (/ElektroniskBilag/Tendenser/Afstand-skala.xlsx). Afstanden angives i Excel-dokumentet som 1 (tæt på), 2 (tilpas) og 3 (langt fra).\blankline
%
For undersøgelsen af tendenser mellem testpersonernes besvarelser og robottens indgangsvinkel, forefindes samtlige grafer i: (/ElektroniskBilag/Tendenser/Indgangsvinkel-skala.xlsx). Indgangsvinklen angives i Excel-dokumentet som 1 (venstre), 2 (forfra), 3 (højre) og 4 (kommer selv).\blankline
%
For undersøgelsen af tendenser mellem testpersonernes besvarelse og alder, forefindes samtlige grafer i: (/ElektroniskBilag/Tendenser/Alder-skala.xlsx)\blankline
%
For undersøgelsen af tendenser mellem testpersonernes besvarelser og højdeforskellen mellem testpersonernes højde og robottenshøjde, forefindes samtlige grafer i: \\
(/ElektroniskBilag/Tendenser/Højdeforskelle-skala.xlsx)\blankline
%
For undersøgelsen af tendenser mellem testpersonernes besvarelser og hvor glade testpersonerne er for teknologi, forefindes samtlige grafer i:\\
(/ElektroniskBilag/Tendenser/GladForTeknologi-skala.xlsx)

\section{Korrelationsgrafer}
\label{ElektroniskBilagKorrelationsgrafer}
%
Billederne: (/ElektroniskBilag/KorrelationsGrafer/SQ5+SQ7.pdf), dette er blot et eksempel på en af henvisningerne til en specifik graf. Henvises der til en specifik graf, der ikke er SQ5+SQ7, vil nummeret efter SQ ændres til de specifikke skalaer.\blankline
%
Samtlige grafer for sammenligning af korrelerede parametre fundet for højde forefindes i: (/ElektroniskBilag/KorrelationsGrafer/Korrelation-Højde.xlsx)\blankline
%
Samtlige grafer for sammenligning af korrelerede parametre fundet for afstand forefindes i: (/ElektroniskBilag/KorrelationsGrafer/Korrelation-Afstand.xlsx)\blankline
%
Samtlige grafer for sammenligning af korrelerede parametre fundet for indgangsvinkel forefindes i: (/ElektroniskBilag/KorrelationsGrafer/Korrelation-Indgangsvinkel.xlsx)
