\chapter{Feltundersøgelse}
\label{Feltundersoegelse}
%
Formålet med feltundersøgelsen er, at få danske rejsende til at sætte deres egne ord på oplevelsen af interaktionen med en social robot. Baseret på den kvalitative respons vil det forsøges at udlede hvilke parametre, der er essentielle for at designe en brugervenlig interaktion med en social robot. I det henseende er det ikke nødvendigvis vigtigt at undersøge, om en given interaktion fungerer, men nærmere hvordan den skal være for at fungere. Da konteksten hvori HRI foregår har stor betydning for hvilke parametre, der har indflydelse på oplevelsen, vil feltundersøgelsen blive udført i en dansk lufthavn. Det vælges at feltundersøgelsen udføres i Aalborg Lufthavn, da kontakt og aftaler med Københavns Lufthavn er tidskrævende og ikke nødvendigvis kan efterkommes inden for projektperioden. Ydermere er det nemmere at transportere robotten til Aalborg Lufthavn, hvor det allerede er aftalt at udføre undersøgelsen. Tilmed er Aalborg Lufthavn mindre end Københavns Lufthavn, hvorfor der vil være mere ro til at udføre undersøgelsen.\blankline
%
Feltundersøgelser, hvor potentielle brugere mødes ude i den virkelige verden samt i den rette kontekst, kan ofte give en større indsigt i hvordan interaktionen med produktet, i det her tilfælde en social robot, faktisk foregår. Endvidere kan det afsløre nogle af de problemer eller holdninger til interaktionen med en social robot, som ikke nødvendigvis vil blive udtryk i en laboratorieundersøgelse. I denne sammenhæng vælges det at de rejsende skal interagere med robotten under specifikke brugsscenarier, hvorefter de vil blive bedt om at genfortælle deres oplevelse, hvilket uddybes i følgende afsnit.   
