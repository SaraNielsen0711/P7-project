\section{Sammenligning af relaterede studier og scenariet i AAL}
\label{ParametreTidligereStudier}
%
Ud fra resultaterne fra affinity diagrammet fås de forskellige beskrevne parametre. Sammenholdes dette med den beskrevne litteratur i \fullref{InteraktionSocialeRobotter} kan det ses, at nogle af parametrene går igen. Dette gør sig gældende for følgende parametre:

%
\begin{multicols}{2}
	\begin{itemize}
		\item Kendskab til teknologi/robotter
		\item Hvor anmasende robotten er
		\item Hvor irriterende robotten er
		\item Hvor nemt/svært det var at bruge robotten
		\item Fornøjelse ved brug af robotten
		\item Brugbarheden af robotten
		\item Hvor sjov robotten er
		\item Højden af robotten
		\item Hastigheden af robotten
		\item Hvor rolige bevægelser robotten har
		\item Hvor menneskelig robotten perciperes
		\item Tryghed ved brug af robotten
		\item Hvor meget man stoler på robotten følger en på vej
		\item Skærmens reaktion
		\item Afstand til brugeren
	\end{itemize}
\end{multicols}
\noindent
%
Alle disse parametre har i højere eller mindre grad inflydelse på, hvordan interaktionen med robotten opleves og kan derfor tages i betragtning under designet af en social robot. \blankline
% MAngler at skrive alle de nye parametre ind og skrive mere til de parametre vi ikke finder
Udover de parametre, som går igen i projektets undersøgelse og litteraturen, så nævner testpersonerne i AAL andre aspekter, som ikke går igen i den gennemgåede litteratur. Under de forskellige kategorier bliver der, udover de nævnte parametre, lagt vægt på følgende:
%
\begin{multicols}{2}
	\begin{itemize}
		\item Om robotten er imødekommende
		\item Om robotten stod i vejen
		\item Om robottens hjælp perciperes som personlig
		\item Om robottens henvendelse overrasker
		\item Om robotten forskrækker
		\item Om robotten er elegant
		\item Om robotten er sød
		\item Om robotten er sej
		\item Om robotten er spændende
	\end{itemize}
\end{multicols}
\noindent
%
Disse parametre er ikke nødvendigvis nye, men kan ligge implicit i nogle af de parametre, der overlapper med litteraturen, uden at være nævnt med sammme ord. Disse parametre er desuden mere overordnet for hele oplevelsen af interaktion med robotten frem for specifikke parametre, der kan justeres på af en designer. Parametrene kommer til udtryk, når testen køres i AAL og har måske ikke betydning andre steder, der ikke minder om en lufthavnssituation. Dette gør sig blandt andet gældende for brugernes holdning til om robottens henvendelse overrasker eller forskrækker, men også om hvorvidt robotten stod i vejen. I en lufthavn kan det forestilles, at rejsende har travlt eller er i stressede situationer, hvor interaktionen eller mødet med en robot godt kan blive for meget, hvis brugeren godt ved hvor vedkommende er på vej hen. Det kan derfor være et vigtigt parameter at robotten ikke forskrækker og i hvert fald ikke står i vejen, da dette kan tilføje yderligere stress til situationen. Parametrene kan på den måde godt være gældende kun for en lufthavnssituation, men ikke desto mindre er det parametre, der kan måles, når brugeren interagere med robotten, for at få en idé om helhedsindtrykket af robotten og hvor en social robot eventuelt kan forbedres. \blankline
%
Ifølge litteraturen er der også betydende parametre ved sociale robotter, som ikke er dukket op ved undersøgelsen i AAL. Dette gælder blandt andet intelligens, som ikke er noget der er blevet nævnt af rejsende i AAL. Derudover nævner testpersonerne heller ikke noget om \textit{companionship}, \textit{social influence}, \textit{image}, \textit{social distance}, \textit{anxiety}, hvilken indgangsvinkel robotten kommer fra eller farven på robotten. Selvom testpersonerne blev spurgt ind til hvad de forestillede sig andre tænke om dem, når de interagerede med robotten, jævnfør \textit{social influence}, så havde de som regel ikke noget svar til dette. Hvor der i andre kulturer muligvis lægges meget vægt på hvad andre tænker om én, tyder det altså på at danske rejsende ikke interesserer sig så meget for hvad andre tænker i denne henseende. \blankline
%
På trods af at der i vesten kan være en frygt for robotter, som beskrevet i \fullref{InteraktionSocialeRobotterGenerelleTendenser}, så ses det på testpersonerne at \textit{anxiety} ikke er et parameter af betydning. Dette kan skyldes den brugte Double robot, der ikke fylder så meget i rummet og derudover oftest blev omtalt som værende sød, sej eller elegant. \blankline
%
Det ses at indgangsvinkel heller ikke bliver nævnt som et betydende parametre. Dette står i modstrid til undersøgelsen af \textcite{PDF:HowMayIServeYou}, der konkluderer, at indgangsvinklen er af betydning. Det kan have haft betydning, at testpersonerne i AAL har bevæget sig frit på en åben plads i lufthavnen. Oftest er det blevet observeret, at rejsende ignorerer robotten, medmindre den står med front direkte mod én, hvorfor det ikke bemærkes, hvis robotten prøver at henvende sig fra den ene eller anden side af. 


%Mange undersøgelser laves bare med spørgeskemaer, likert-skalaer, korte interaktioner med robotten, videoklip med robotten. 

%\noindent Inddrag Personality of social robots perceived through the appearances.

%\noindent Kom ind på hvordan det måles, hvad har andre gjort i forhold til det (overvej om det skal være en sektion for sig selv).\blankline


%Out of the five widely used personality dimensions, namely the extroversion, agreeableness, conscientiousness, neuroticism, and openness [5], the most important dimensions for social interactions are those that concern individual differences in social behavior, namely extroversion and agreeableness or their common rotations, ‘friendliness’ and ‘dominance’ [6]. Fra Personality of social robots perceived trough the appearance side. 272.