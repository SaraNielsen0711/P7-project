\section{Sammenligning mellem fundne parametre og relaterede studier}
\label{ParametreTidligereStudier}
%
Ud fra resultaterne præsenteret i \fullref{ParametreDatabehandlingAffinityDiagram} udledes de forskellige beskrevne parametre. Sammenholdes de fundne parametre med den beskrevne litteratur belyst i \fullref{InteraktionSocialeRobotter}, er nogle af parametrene gennemgående for HRI. Det gør sig gældende for følgende parametre:
%
\begin{multicols}{2}
	\begin{itemize}
		\item Kendskab til teknologi/robotter
		\item Hvor anmassende robotten er
		\item Hvor irriterende robotten er
		\item Hvor nemt/svært det var at bruge robotten
		\item Fornøjelse ved brug af robotten
		\item Brugbarheden af robotten
		\item Hvor sjov robotten er
		\item Højden af robotten
		\item Hastigheden af robotten
		\item Hvor rolige bevægelser robotten har
		\item Hvor menneskelig robotten perciperes
		\item Tryg ved robotten
		\item Hvor meget brugeren stoler på at robotten følger en på vej
		\item Skærmens reaktion
		\item Afstand til brugeren
	\end{itemize}
\end{multicols}
\noindent
%
Alle disse parametre har i højere eller mindre grad indflydelse på, hvordan interaktionen med robotten opleves og bør derfor tages i betragtning under designet af en social robot. Baseret på testpersonernes udsagn, jævnfør \textit{affinity notes}, er der en tendens til at testpersonerne antropomorfiserer robotten, da de ofte tildeler robotten menneskelig karakteristikas blandt andet i forhold til dens bevægelser.\blankline
%
Udover de parametre, som går igen i projektets undersøgelse og litteraturen, så nævner testpersonerne i AAL andre aspekter, som adskiller sig fra den gennemgåede litteratur. Under de forskellige kategorier bliver der, udover de nævnte parametre, lagt vægt på følgende:
%
\begin{multicols}{2}
	\begin{itemize}
		\item Om robotten er imødekommende
		\item Om robotten stod i vejen
		\item Om robottens hjælp perciperes som personlig
		\item Om robottens henvendelse overrasker
		\item Om robotten forskrækker
		\item Om robotten er elegant
		\item Om robotten er sød
		\item Om robotten er sej
		\item Om robotten er spændende
	\end{itemize}
\end{multicols}
\noindent
%
Disse parametre er ikke nødvendigvis nye, men kan ligge implicit i nogle af de parametre, der overlapper eller forefindes i litteraturen, uden at være nævnt med sammme ord. Parametrene kommer til udtryk, når testen afvikles i AAL og har måske ikke betydning andre steder, der ikke minder om en lufthavnssituation. Dette gør sig blandt andet gældende for brugernes holdning til om robottens henvendelse overrasker eller forskrækker en, men også om hvorvidt robotten stod i vejen. I en lufthavn kan det forestilles, at rejsende har travlt eller er i stressede situationer, hvor interaktionen eller mødet med en robot kan virke overvældende, hvis brugeren godt ved hvor de er på vej hen. Det kan derfor være et vigtigt parameter at robotten ikke forskrækker og i hvert fald ikke står i vejen, da dette kan tilføje yderligere stress til situationen. Det er derfor muligt at parametrene kun gør sig gældende for en lufthavnssituation, men ikke desto mindre er det parametre, der kan måles, når brugeren interagerer med robotten, for at få en idé om helhedsindtrykket af robotten og hvor der kan foretages forbedringer. \blankline
%
Ifølge litteraturen er der også betydende parametre ved sociale robotter, som ikke er forefindes i undersøgelsen foretaget i AAL. Dette er blandt andet \textit{image}, der vedrører brugerens overbevisning om at interaktionen med robotten øger ens status set med andres øjne, \parencite[s. 1478]{PDF:ExploringInfluencingVariable}, og i nogen grad \textit{social influence}, hvor der kun er én testperson, der har givet udtryk for, hvad andre kunne tænke om situationen. Det kan muligvis skyldes at danske rejsende ikke interesserer sig specielt meget for hvad andre tænker om dem i den pågældende situation, hvilket kan adskille sig fra andre kulturer. 

Ydermere nævner testpersonerne ikke direkte noget om \textit{social distance}, hvilket kan skyldes, at robotten har overholdt de sociale normer, og at testpersonerne har følt, at de har været i kontrol over situationen. Ud fra teorien omkring \textit{power distance} tyder det på, at testpersonerne har haft en følelse af at de var den dominante partner, hvor robotten var den underordnede. Da interaktionen har været baseret på at robotten skulle hjælpe den rejsende med at finde rundt i lufthavnen, hvilket kan betragtes som et samarbejde mellem robot og testperson, er det muligt at de rejsende har opført sig mere venligt, intimt og involveret end hvad der måske havde været tilfældet i en anden situation. Dette hænger sammen med \textit{task distance}, \parencite[s. 784]{PDF:HowSocialDistanceShapesHRI}. 

Indgangsvinkel er ikke nævnt som et betydende parameter blandt de danske rejsende. Dette står i modstrid til undersøgelsen foretaget af \textcite{PDF:HowMayIServeYou}, der konkluderer, at indgangsvinklen har betydning for HRI. Det kan have haft betydning, at testpersonerne i AAL har bevæget sig frit i test området efter sikkerhedskontrollen i lufthavnen. Det er blevet observeret, at rejsende ignorerer robotten, medmindre den står med front direkte mod én, hvorfor det ikke bemærkes, hvis robotten prøver at henvende sig fra den ene eller anden side. 

En parameter, som ikke direkte er blevet udledt baseret på de danske rejsendes udtalelser er \textit{companionship}, fortolkes der derimod på testpersonernes udtalelser tyder det på at denne parameter faktisk har en betydningen for HRI. Dette er dog primært rette mod at robotten kan hjælpe ældre og børn, der rejser alene, hvor det bliver nævnt at børn kan have nemmere ved at danne et forhold til robotter end voksne. Det vælges ikke at udvikle en decideret skala hvorpå det kan evalueres hvorvidt robotten betragtes som en følgesvend. Dog vurderes det, at dette parameter formentlig bliver dækket af skalaspørgsmålet: \textit{Jeg oplever robottens hjælp som personlig}. \blankline
%
Selvom der fra litteraturen forekommer parametre, som har væsentlig betydning for HRI og som ikke er udledt i denne undersøgelse, vælges det kun at designe skalaer baseret på de fundne parametre. 
