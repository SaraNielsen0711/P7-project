\section{Sammenligning af relaterede studier og scenariet i AAL}
\label{ParametreTidligereStudier}
%
Ud fra resultaterne fra affinity diagrammet fås de forskellige beskrevne parametre. Sammenholdes dette med den beskrevne litteratur i \fullref{InteraktionSocialeRobotter} kan det ses, at nogle af parametrene går igen. Dette gør sig blandt gældende for parametre som hvor tæt robotten skal komme på brugeren, hvor menneskelig robotten skal se ud, højden på robotten, hvor levende robotten perciperes, robottens hastighed og bevægelser, hvor tryg brugeren er ved brug af robotten, hvor meget vedkommende stoler på robotten og hvor brugbar robotten er. Alle disse parametre har i højere eller mindre grad inflydelse på, hvordan interaktionen med robotten opleves og kan derfor tages i betragtning under designet af en social robot. \blankline
%
Udover de parametre, som går igen i projektets undersøgelse og litteraturen, så nævner testpersonerne i AAL andre aspekter, som ikke går igen i den gennemgåede litteratur. Under henvendelsen af robotten bliver der nemlig, udover de nævnte parametre, lagt vægt på, om robotten er imødekommende, intimiderende, stod i vejen, selv henvendte sig eller om brugeren blev overrasket over robottens henvendelse. Når udseendet diskuteres, er det parametre som at robotten ser mærkelig ud eller om robotten er elegant der bliver lagt vægt på. Under væremåde er det hvorvidt robotten er anmasende, irriterende eller har rolige og behagelige bevægelser der bliver nævnt. Når tillid bliver diskuteret er det eneste nye parametre om hvorvidt brugeren bliver forskrækket af robotten. Disse parametre er nødvendigvis mere overordnet for hele oplevelsen af interaktion med robotten frem for specifikke parametre, der kan justeres på af en designer. Ydermere kommer disse parametre til udtryk, når testen køres i AAL og har måske ikke betydning andre steder, der ikke minder om en lufthavnssituation. Dette gør sig blandt andet gældende for brugernes holdning til om robotten selv skal henvende sig eller det er dem, der skal henvende sig til robotten, men også om hvorvidt robotten er anmasende. I en lufthavn kan det forestilles, at rejse har travlt eller er i stressede situationer, hvor interaktionen med en robot godt kan blive for meget, hvis brugeren godt ved hvor vedkommende er på vej hen. Det kan derfor være et vigtigt parameter at robotten ikke selv henvender sig og i hvert fald ikke er anmasende, da dette kan tilføje yderliger stress til situationen. Parametrene kan på den måde godt være gældende kun for en lufthavnssituation, men ikke desto mindre er det parametre, der kan måles, når brugeren interagere med robotten, for at få en idé om helhedsindtrykket af robotten og hvor en social robot eventuelt kan forbedres. \blankline
%
Ifølge litteraturen er der også betydende parametre ved sociale robotter, som ikke er dukket op ved undersøgelsen i AAL. Dette gælder blandt andet intelligens, som ikke er noget der er blevet nævnt af rejsende i AAL. Derudover nævner testpersonerne heller ikke noget om \textit{companionship}, \textit{social influence}, \textit{image}, \textit{social distance}, \textit{anxiety}, hvilken indgangsvinkel robotten kommer fra eller farven på robotten. Selvom testpersonerne blev spurgt ind til hvad de forestillede sig andre tænke om dem, når de interagerede med robotten, jævnfør \textit{social influence}, så havde de som regel ikke noget svar til dette. Hvor der i andre kulturer muligvis lægges meget vægt på hvad andre tænker om én, tyder det altså på at danske rejsende ikke interesserer sig så meget for hvad andre tænker i denne henseende. 

På trods af at der i vesten hersker en frygt for robotter, som beskrevet i \fullref{InteraktionSocialeRobotterGenerelleTendenser}, så ses det på testpersonerne at \textit{anxiety} ikke er et parameter af betydning. Dette kan skyldes den brugte Double robot, der ikke fylder så meget i rummet og derudover oftest blev omtalt som værende sød, sej eller elegant. 

Det ses at indgangsvinkel heller ikke bliver nævnt som et betydende parametre. Dette står i modstrid til undersøgelsen af \textcite{PDF:HowMayIServeYou}, der konkluderer, at indgangsvinklen er af betydning. Det kan have haft betydning, at testpersonerne i AAL har bevæget sig frit på en åben plads i lufthavnen. Oftest er det blevet observeret, at rejsende ignorerer robotten, medmindre den står med front direkte mod én, hvorfor det ikke bemærkes, hvis robotten prøver at henvende sig fra den ene eller anden side af. 

%SKAL HANDLE OM HVORVIDT VORES FUNDE PARAMETRE STEMMEROVERNS MED DET ANDRE KILDER HAR FUNDET\blankline
%
%Denne sektion hører muligvis ind, når resultaterne skal sammenlignes med litteraturen og vi beslutter hvordan skalaerne skal opbygges!\blankline


%Mange undersøgelser laves bare med spørgeskemaer, likert-skalaer, korte interaktioner med robotten, videoklip med robotten. 

%\noindent Inddrag Personality of social robots perceived through the appearances.

%\noindent Kom ind på hvordan det måles, hvad har andre gjort i forhold til det (overvej om det skal være en sektion for sig selv).\blankline


%Out of the five widely used personality dimensions, namely the extroversion, agreeableness, conscientiousness, neuroticism, and openness [5], the most important dimensions for social interactions are those that concern individual differences in social behavior, namely extroversion and agreeableness or their common rotations, ‘friendliness’ and ‘dominance’ [6]. Fra Personality of social robots perceived trough the appearance side. 272.