\subsection{Databehandlingsmetode: Affinity diagram}
\label{ParametreMetodeovervejelserAffinityDiagram}
%
Baseret på den foregående metodeovervejelser til hvordan feltundersøgelsen skal afvikles er det klart, at den indsamlede data vil være kvalitativt. Da formålet med undersøgelsen er at udlede hvilke parametre danske rejsende tilskriver interaktionen med en social robot vil databehandlingen forsøge at opnå dette. Det kan blandt andet gøres med \textit{affinity diagram}, hvor data grupperes efter nøgle problemer fremfor forudbestemte kategorier. Hver gruppering får tildelt en label, som gengiver hvad testpersonerne gør eller tænker, \parencite[s. 159]{Book:BuildingAnAffinity}. 

Dette afsnit præcisere processen lige fra hvordan det indsamlede data fortolkes og gengives på \textit{affinity notes} til hvordan det endelige \textit{affinity diagram} bygges. \blankline
%
Følgende bygger på hvordan data fortolkes og gengives på \textit{affinity notes}, hvilket fremgår af \textcite[ss. 101-122]{Book:CIInterpretationSession}. Som udgangspunkt vil det tage lige så lang tid at fortolke data som det tog at indsamle data, \parencite[s. 102]{Book:CIInterpretationSession}. Det er favorabelt at gruppen, som fortolker data, består af forskellige profiler eller stakeholders, \parencite[s. 104]{Book:CIInterpretationSession}, i denne sammenhæng vil gruppen dog bestå af projektgruppens medlemmer. Derudover bør der minimum være to der fortolker data, fortrinsvist fire og makismalt seks deltagere, ellers bør der foretages flere fortolknings sessioner, \parencite[s. 104]{Book:CIInterpretationSession}. \blankline
%
Inden sessionen starter skal der uddeles nogle roller til dem, der deltager i fortolkningen. Den første rolle er intervieweren, svarende til testlederen, som sørger for at præsentere brugeren, svarende til testpersonen, for derefter at gennemgå interviewet ud fra de noter, der er taget, \parencite[ss. 106-107]{Book:CIInterpretationSession}. Er det mere end to dage siden interviewet fandt sted skal intervieweren sørger for at høre lydoptagelsen. Derudover deltager intervieweren i diskussionen, tilbyder indsigt, fortolkninger samt design idéer og sørger for pointer fanges og noteres, \parencite[s. 107]{Book:CIInterpretationSession}. Envidere er det intervieweren, som har det sidste ord i forhold til brugeren og brugeroplevelsen for på den måde at sikre at fortolkningen afspejler virkeligheden så godt som muligt, \parencite[s. 107]{Book:CIInterpretationSession}.

Den anden rolle handler om at tage noter, som udgør \textit{affinity notes}, dette dækker over at notere observationer, problemer, spørgsmål, huller, indsigter og design idéer, \parencite[s. 107]{Book:CIInterpretationSession}. Huller relaterer til at der mangler information, \parencite[s. 162]{Book:BuildingAnAffinity}. Derudover tilføjer personen, der tager noter, hvis der undervejs i fortolknings sessionen opstår ny demografisk viden til brugerprofilen, \parencite[s. 107]{Book:CIInterpretationSession}. For at påtage sig rollen kræver det, ifølge \textcite[s. 107]{Book:CIInterpretationSession}, at personen er i stand til at lytte, skrive og processere information samtidigt og personens evne til det vil styre hvor hurtig fortolknings sessionen foregår. Det er vigtigt at de andre deltagere ved hvem, der tager noter så de kan henvende sig til den person, hvis der opstår spørgsmål. I tillæg må personen, som tager noter, stille opklarende spørgsmål i forhold til hvordan en \textit{affinity note} bedst formuleres. I den forbindelse pointerer \textcite[s. 108]{Book:CIInterpretationSession}, at det er ineffektivt at diskutere hvor god en \textit{affinity note} er, hvorfor det er bedre at tilføje en ekstra \textit{affinity note}. 

De resterende deltagere vil påtage sig rollen som det generelle fortolknings team, hvis opgave består i at lytte til intervieweren, stille opklarende spørgsmål, tilbyder indsigt, fortolkninger og design idéer, \parencite[s. 108]{Book:CIInterpretationSession}. Derudover har de også et ansvar for at sikre, at vigtige pointer bliver noteret, gennemgå \textit{affinity notes} for at sikre at de er nøjagtige og samtidig sørger for at diskussionen holdes til emnet, \parencite[s. 108]{Book:CIInterpretationSession}. I tillæg skal de ydermere sørger for ikke at sammenligne dette interview med andre interviews, som endnu ikke er fortolket, \parencite[s. 108]{Book:CIInterpretationSession}. 

Èn af de resterende deltagere vil udover at have rollen som én af de generelle fortolknings deltagere også have rollen som moderator, hvis primære opgave er at holde diskussionen på sporet, \parencite[s. 108]{Book:CIInterpretationSession}. I tillæg er moderatorens opgave at sørger for at alle deltager i diskussionen, sikre at personen, som tager noter kan følge med og ikke kun noterer de inputs personen kan lide, samt sikre at intervieweren holder sig til noterne i kronologisk orden og ikke springer over noget for at besvare et spørgsmål, \parencite[ss. 108-109]{Book:CIInterpretationSession}. Med andre ord er det moderatorens opgave, at sørge for at fortolknings sessionen foregår efter reglerne. 

Fælles for alle deltagerer er, at de i fællesskab skal sørger for at samtalen holdes på sporet og derfor kun til hvad den specifikke testperson kommenterer. Oplever en af deltagerne at samtalen kører ud på et sidespor kan personen signalere til de andre, at nu er samtalen kørt af sporet. Dette betegnes som et \textit{rat hole} og kan eksempelvis være et flag eller anden illustration af et hul eller med teksten \textit{rat hole}, som smides på bordet, \parencite[s. 109]{Book:CIInterpretationSession}.\blankline
%   
Da brugeren, testpersonerne i denne sammenhæng, får oplyst at deres deltagelse behandles fortroligt er det favorabelt at anvende en kode og et nummer, som forbinder brugeren med data, \parencite[s. 111]{Book:CIInterpretationSession}. Koden angiver eksempelvis testperson, TP, hvor nummeret angiver hvilket nummer testpersonen er. Istedet for at testpersonen kaldes ved navn, vil den første testperson blive kaldt for TP01. Er der indsamlet demografisk data, såsom alder, køn, beskæftligelse, erfaring med robotter og lignede kan det være en idé at udarbejde en brugerprofil, som adskilles fra det resterende data, da det ikke anbefales at notere demografiske observationer på \textit{affinity notes}, \parencite[s. 109]{Book:CIInterpretationSession}. At udvikle brugerprofiler er det første step i fortolknings sessionen. 

Det næste step i sessionen er, at intervieweren præsenterer interviewet i kronologisk orden og uden at springe noget over eller sammenfatter noterne. Ifølge \textcite[s. 113]{Book:CIInterpretationSession} bør \textit{affinity notes} nedskrives i en teksteditor, som projiceres på en stor skærm eller væg så alle kan se hvad der noteres. Undervejs stiller de resterende deltagerer spørgsmål omkring hvad der foregår men aldrig omkring hvad der efterfølgende sker, \parencite[s. 114]{Book:CIInterpretationSession}.

Personen som står for at notere \textit{affinity notes} skal løbende sørger for, at de resterende deltagerer læser de noterede noter for at sikre at de er korrekte. Derudover er det vigtigt at hver \textit{affinity note} kun indeholer én tanke eller pointe samt en referencen til den specifikke testperson, \parencite[s. 115]{Book:CIInterpretationSession}. I tilfælde af at intervieweren ikke kan svare tilfredstillende på et spørgsmål er det personen, som står for at notere, der sørger for at notere det som en \textit{affinity note} med et \textit{Q}, så deltagerne ved at det er noget de skal være opmærksomme på senere, \parencite[s. 115]{Book:CIInterpretationSession}. Ønskes det at citerer testpersonerne gøres det ved at sætte \textit{" "} rundt om citatet. \textcite[s. 116]{Book:CIInterpretationSession} opstiller en tabel over \textit{Dos} og \textit{Don'ts} i forhold til at notere \textit{affinity notes}. Nogle af de ting, der, ifølge \textcite[s. 116]{Book:CIInterpretationSession}, bør opfyldes er:\blankline
%
\begin{itemize}
  \item Vær klar og tydelig omkring pronomener, det er ikke tilstrækkeligt at skrive \textit{han} eller \textit{hun}, da det efter et par dage vil være glemt hvem de var, derfor foreslåes det at notere noget, som identificere den person.
  \item Sørg for at sproget afspejler testpersonerne, fremfor det sprog der anvendes af projektgruppen.
  \item  Husk altid at testpersonerne er fortrolige, hvorfor deres navn ikke bør fremgå nogen steder. 
  \item Sørg for at der kun er én note på én \textit{affinity note}.
  \item Noterne skal være uafhængige af hinanden. 
  \item Sørg for at demografisk data nedskrives andetsteds og ikke på \textit{affinity notes}. 
  \item Hvis der er uenighed om et specifikt ord, så brug skråsteg og notér mulighederne.\blankline
\end{itemize}
%
Derudover kommenterer \textcite[s. 116]{Book:CIInterpretationSession},  at som tommelfingerregel bør der produceres 50 til 100 notater per to timer. For at gøre det nemmere for deltagerne at relatere sig til testpersonernes situation er det favorabelt, at de præsenteres for det fysiske miljø, eksempelvis med tegninger eller fotografier, \parencite[s. 119]{Book:CIInterpretationSession}. Da deltagerne i fortolknings sessionen i denne sammenhæng består af projektgruppen, som alle har været med ude i Aalborg Lufthavn, hvor feltundersøgelsen foretages, antages det at dette ikke er nødvendigt da de i forvejen er kendt med miljøet. 

Det sidste step i fortolknings sessionen handler om at indsamle indsigter. Ifølge \textcite[s. 119]{Book:CIInterpretationSession} skal indsigter repræsentere mønstre, situationer, behov og hverken løsninger eller design idéer. Indsigterne bygger på de deltagenes oplevelse af interviewet.     



    






  


Brug den her kilde: \textcite[ss. 24-28]{PDF:ConsolidationIdeationAffinity}. 

Brug særligt den her kilde, hvor hele kapitel 8 handler om at lave affinity diagrammer samt hele processen bag det: \textcite[ss. 159-179]{Book:BuildingAnAffinity}. Kapitel 5 handler om hvordan interpretation session skal foregå - altså hvordan vi skal lave vores gule post-its, \textcite[ss. 101-122]{Book:CIInterpretationSession}.
