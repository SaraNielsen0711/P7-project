\section{Databehandlingsmetode: Affinity diagram}
\label{ParametreMetodeovervejelserAffinityDiagram}
%
Baseret på den foregående metodeovervejelser til hvordan feltundersøgelsen skal afvikles er det klart, at den indsamlede data er kvalitativt. Da formålet med undersøgelsen er at udlede hvilke parametre danske rejsende tilskriver interaktionen med en social robot vil databehandlingen forsøge at opnå dette. Det kan blandt andet gøres med et \textit{affinity diagram}, hvor data grupperes efter problemer, behov og synspunkter som testpersonerne giver udtryk for, fremfor forudbestemte kategorier. Hver gruppering får tildelt en label, som gengiver hvad testpersonerne gør eller tænker, \parencite[s. 159]{Book:BuildingAnAffinity}. Fordelen ved at analysere det indsamlede kvalitative data ved hjælp af \textit{affinity diagram} er, at hver eneste observation og pointe fra testpersonerne skal behandles, \parencite[s. 25]{PDF:ConsolidationIdeationAffinity}.

Dette afsnit præcisere processen lige fra hvordan det indsamlede data fortolkes og gengives på \textit{affinity notes} til hvordan det endelige \textit{affinity diagram} bygges.

\subsection{Fortolkningssession}
\label{ParametreFortolkningssession}
%
Følgende bygger på hvordan data fortolkes og gengives på \textit{affinity notes}, hvilket fremgår af \textcite[ss. 101-122]{Book:CIInterpretationSession}. Som udgangspunkt vil det tage lige så lang tid at fortolke data som det tog at indsamle data, \parencite[s. 102]{Book:CIInterpretationSession}. Det er favorabelt at gruppen, som fortolker data, består af forskellige profiler eller stakeholders, \parencite[s. 104]{Book:CIInterpretationSession}, i denne sammenhæng vil gruppen dog bestå af projektgruppens medlemmer. Derudover bør der minimum være to der fortolker data, fortrinsvist fire og makismalt seks deltagere, ellers bør der foretages flere fortolkningssessioner, \parencite[s. 104]{Book:CIInterpretationSession}. \blankline
%
Inden sessionen starter skal der uddeles nogle roller til dem, der deltager i fortolkningen. Den første rolle er intervieweren, svarende til testlederen, som sørger for at præsentere brugeren, svarende til testpersonen, for derefter at gennemgå interviewet ud fra de noter, der er taget, \parencite[ss. 106-107]{Book:CIInterpretationSession}. Er det mere end to dage siden interviewet fandt sted skal intervieweren sørger for at høre lydoptagelsen. Derudover deltager intervieweren i diskussionen, tilbyder indsigt, fortolkninger samt design idéer og sørger for pointer fanges og noteres, \parencite[s. 107]{Book:CIInterpretationSession}. Envidere er det intervieweren, som har det sidste ord i forhold til brugeren og brugeroplevelsen for på den måde at sikre at fortolkningen afspejler virkeligheden så godt som muligt, \parencite[s. 107]{Book:CIInterpretationSession}.

Den anden rolle handler om at tage noter, som udgør \textit{affinity notes}, dette dækker over at notere observationer, problemer, spørgsmål, huller, indsigter og design idéer, \parencite[s. 107]{Book:CIInterpretationSession}. Huller relaterer til at der mangler information, \parencite[s. 162]{Book:BuildingAnAffinity}. Derudover tilføjer personen, der tager noter, hvis der undervejs i fortolknings sessionen opstår ny demografisk viden til brugerprofilen, \parencite[s. 107]{Book:CIInterpretationSession}. For at påtage sig rollen kræver det, ifølge \textcite[s. 107]{Book:CIInterpretationSession}, at personen er i stand til at lytte, skrive og processere information samtidigt og personens evne til det vil styre hvor hurtig fortolkningssessionen foregår. Det er vigtigt at de andre deltagere ved hvem, der tager noter så de kan henvende sig til den person, hvis der opstår spørgsmål. I tillæg må personen, som tager noter, stille opklarende spørgsmål i forhold til hvordan en \textit{affinity note} bedst formuleres. I den forbindelse pointerer \textcite[s. 108]{Book:CIInterpretationSession}, at det er ineffektivt at diskutere hvor god en \textit{affinity note} er, hvorfor det er bedre at tilføje en ekstra \textit{affinity note}. 

De resterende deltagere vil påtage sig rollen som det generelle fortolknings team, hvis opgave består i at lytte til intervieweren, stille opklarende spørgsmål, tilbyder indsigt, fortolkninger og design idéer, \parencite[s. 108]{Book:CIInterpretationSession}. Derudover har de også et ansvar for at sikre, at vigtige pointer bliver noteret, gennemgå \textit{affinity notes} for at sikre at de er nøjagtige og samtidig sørger for at diskussionen holdes til emnet, \parencite[s. 108]{Book:CIInterpretationSession}. I tillæg skal de ydermere sørger for ikke at sammenligne dette interview med andre interviews, som endnu ikke er fortolket, \parencite[s. 108]{Book:CIInterpretationSession}. 

Èn af de resterende deltagere vil udover at have rollen som én af de generelle fortolknings deltagere også have rollen som moderator, hvis primære opgave er at holde diskussionen på sporet, \parencite[s. 108]{Book:CIInterpretationSession}. I tillæg er moderatorens opgave at sørger for at alle deltager i diskussionen, sikre at personen, som tager noter kan følge med og ikke kun noterer de inputs personen kan lide, samt sikre at intervieweren holder sig til noterne i kronologisk orden og ikke springer over noget for at besvare et spørgsmål, \parencite[ss. 108-109]{Book:CIInterpretationSession}. Med andre ord er det moderatorens opgave, at sørge for at fortolknings sessionen foregår efter reglerne. 

Fælles for alle deltagerer er, at de i fællesskab skal sørger for at samtalen holdes på sporet og derfor kun til hvad den specifikke testperson kommenterer. Oplever en af deltagerne at samtalen kører ud på et sidespor kan personen signalere til de andre, at nu er samtalen kørt af sporet. Dette betegnes som et \textit{rat hole} og kan eksempelvis være et flag eller anden illustration af et hul eller med teksten \textit{rat hole}, som smides på bordet, \parencite[s. 109]{Book:CIInterpretationSession}.\blankline
%   
Da brugeren, testpersonerne i denne sammenhæng, får oplyst at deres deltagelse behandles fortroligt er det favorabelt at anvende en kode og et nummer, som forbinder brugeren med data, \parencite[s. 111]{Book:CIInterpretationSession}. Koden angiver eksempelvis testperson, TP, hvor nummeret angiver hvilket nummer testpersonen er. Istedet for at testpersonen kaldes ved navn, vil den første testperson blive kaldt for TP01. Derudover bør hver \textit{affinity note} nummereres i forhold til den kronologiske rækkefølge den blev noteret i. Er der indsamlet demografisk data, såsom alder, køn, beskæftligelse, erfaring med robotter og lignede kan det være en idé at udarbejde en brugerprofil, som adskilles fra det resterende data, da det ikke anbefales at notere demografiske observationer på \textit{affinity notes}, \parencite[s. 109]{Book:CIInterpretationSession}. At udvikle brugerprofiler er det første step i fortolkningssessionen. 

Det næste step i sessionen er, at intervieweren præsenterer interviewet i kronologisk orden og uden at springe noget over eller sammenfatter noterne. Ifølge \textcite[s. 113]{Book:CIInterpretationSession} bør \textit{affinity notes} nedskrives i en teksteditor, som projiceres på en stor skærm eller væg så alle kan se hvad der noteres. Undervejs stiller de resterende deltagerer spørgsmål omkring hvad der foregår men aldrig omkring hvad der efterfølgende sker, \parencite[s. 114]{Book:CIInterpretationSession}.

Personen som står for at notere \textit{affinity notes} skal løbende sørger for, at de resterende deltagerer læser de noterede noter for at sikre at de er korrekte. Derudover er det vigtigt at hver \textit{affinity note} kun indeholer én tanke eller pointe samt en referencen til den specifikke testperson, \parencite[s. 115]{Book:CIInterpretationSession}. I tilfælde af at intervieweren ikke kan svare tilfredstillende på et spørgsmål er det personen, som står for at notere, der sørger for at notere det som en \textit{affinity note} med et \textit{Q}, så deltagerne ved at det er noget de skal være opmærksomme på senere, \parencite[s. 115]{Book:CIInterpretationSession}. Ønskes det at citerer testpersonerne gøres det ved at sætte \textit{" "} rundt om citatet. \textcite[s. 116]{Book:CIInterpretationSession} opstiller en tabel over \textit{Dos} og \textit{Don'ts} i forhold til at notere \textit{affinity notes}. Nogle af de ting, der, ifølge \textcite[s. 116]{Book:CIInterpretationSession}, bør opfyldes er:\blankline
%
\begin{itemize}
  \item Vær klar og tydelig omkring pronomener, det er ikke tilstrækkeligt at skrive \textit{han} eller \textit{hun}, da det efter et par dage vil være glemt hvem de var, derfor foreslåes det at notere noget, som identificere den person.
  \item Sørg for at sproget afspejler testpersonerne, fremfor det sprog der anvendes af projektgruppen.
  \item  Husk altid at testpersonerne er fortrolige, hvorfor deres navn ikke bør fremgå nogen steder. 
  \item Sørg for at der kun er én note på én \textit{affinity note}.
  \item Noterne skal være uafhængige af hinanden. 
  \item Sørg for at demografisk data nedskrives andetsteds og ikke på \textit{affinity notes}. 
  \item Hvis der er uenighed om et specifikt ord, så brug skråsteg og notér mulighederne.\blankline
\end{itemize}
%
Derudover kommenterer \textcite[s. 116]{Book:CIInterpretationSession},  at som tommelfingerregel bør der produceres 50 til 100 notater per to timer. For at gøre det nemmere for deltagerne at relatere sig til testpersonernes situation er det favorabelt, at de præsenteres for det fysiske miljø, eksempelvis med tegninger eller fotografier, \parencite[s. 119]{Book:CIInterpretationSession}. Da deltagerne i fortolkningssessionen i denne sammenhæng består af projektgruppen, som alle har været med ude i Aalborg Lufthavn, hvor feltundersøgelsen foretages, antages det at dette ikke er nødvendigt da de i forvejen er kendt med miljøet. 

Det sidste step i fortolkningssessionen handler om at indsamle indsigter. Ifølge \textcite[s. 119]{Book:CIInterpretationSession} skal indsigter repræsentere mønstre, situationer, behov og hverken løsninger eller design idéer. Indsigterne bygger på de deltagenes oplevelse af interviewet. \blankline
%
Det tilstræbes så vidt muligt at følge reglerne og processen for fortolkningssessionen, den vil dog tilpasses afhængigt af behov. 

\subsection{Udviklingen af \textit{affinity diagram}}
\label{ParametreUdviklingAfAffinity}
% 
Følgende bygger på hvordan de udarbejdede \textit{affinity notes} bruges til at udvikle et \textit{affinity diagram}, hvilket fremgår af \textcite[ss. 159-179]{Book:BuildingAnAffinity}. Istedet for at have foruddefineret kategorier bygger et \textit{affinity diagram} på en \textit{botttom-up} proces, hvor der tages udgangspunkt i data, som afspejler de problemer, behov og synspunkter som testpersonerne har, \parencite[s. 159]{Book:BuildingAnAffinity}. Kategorierne vil derefter få tildelt en label, der afspejler den måde hvorpå testpersonerne omtaler et bestemt problem, behov eller synspunkt på, \parencite[s. 159]{Book:BuildingAnAffinity}. Ud fra et \textit{affinity diagram} er det muligt at udlede hvilke krav potentielle brugere har til et produkt, hvor det i projektsammenhæng vil relatere sig til hvilke parametre danske rejsende tilskriver interaktionen med en social robot. 

For at følge den hierarkiske struktur, der opstår ved \textit{affinity diagram} vil \textit{affinity notes} blive nedskrevet på gule \textit{sticky notes}. Hver gruppering af \textit{affinity notes} vil derefter få tildelt en label, som nedskrives på blå \textit{sticky notes}. De blå labels nedskrives i førsteperson, som hvis det var brugeren selv, der fortalt hvad det specifikke problem, behov eller synspunkt vedrører, \parencite[s. 160]{Book:BuildingAnAffinity}. Derefter vil de blå labels blive kategoriseret efter sammenhæng og få tildelt en label, som nedskrives på pink \textit{sticky notes}, som igen skal afspejle brugerens måde at beskrive problemet, behovet eller synspunktet, \parencite[s. 160]{Book:BuildingAnAffinity}. Det sidste og højeste niveau i et \textit{affinity diagram} gengives med grønne labels, som bliver dannet ud fra en kategorisering af pink labels og som kan nedskrives ente i førsteperson, som afspejler brugeren eller med mere generelle termer, \parencite[s. 160]{Book:BuildingAnAffinity}. \blankline
%
Ifølge \textcite[s. 161]{Book:BuildingAnAffinity} er en tommelfinger regel, at der bør blive indsamlet kvalitativ data fra mellem otte og 10 testpersoner på to eller tre lokationer, som vil generere mellem 500 og 1000 \textit{affinity notes}. I projektsammenhæng vil feltundersøgelsen udelukkende blive afviklet ét sted; nemlig Aalborg Lufthavn. Det kan være en fordel at bygge sit \textit{affinity diagram} af to omgange, så efter halvdelen af interviewene er gennemført, bygges et \textit{affinity diagram}. Ved at bygge diagrammet af to omgange giver det, ifølge \textcite[s. 162]{Book:BuildingAnAffinity}, muligheden for at få overblik over om der mangler noget information ved specifikke områder og på den måde tilpasse de efterfølgende interviews. Dog forventes det at indsamle alt den nødvendige information på et enkelt besøg i Aalborg Lufthavn, hvorfor diagrammet bygges af én omgang. 

Sigtes der efter at bygge diagrammet på en enkelt dag er tommelfinger reglen, at for hver 50 til 80 \textit{affinity notes} skal der være en person, \parencite[s. 163]{Book:BuildingAnAffinity}. Derudover bør det ikke tage mere end to eller tre dage at bygge diagrammet, da teamet, som står for at bygge det vil miste motivationen, \parencite[s. 163]{Book:BuildingAnAffinity}.\blankline
%
Inden selve processen, udviklingen af \textit{affinity diagram}, starter er det vigtigt at have udført nogle opgaver inden: Alle \textit{affinity notes} skal printes eller nedskrives på \textit{sticky notes}, derefter skal de blandes så hver person i teamet har en blanding af noter fra forskellige testpersoner. Noterne vil blive udleveret i bunker af omkring 20 for at det ikke virker for overvældende. Sørg for at det listen med samtlige \textit{affinity notes} er let tilgængelig hvis der opstår forvirring, samt informationen om hvem testpersonerne er, \parencite[ss. 163-164]{Book:BuildingAnAffinity}. Derudover er det vigtigt at have fundet et rum, der er stort nok til at hænge alle \textit{affinity notes} op for at bygge diagrammet. 

Den første del af processen vedrører hvordan \textit{affinity notes} bliver kategoriseret på væggen. Baseret på fremgangsmåden beskrevet i \textcite[ss. 166-168]{Book:BuildingAnAffinity} skal følgende overholdes: \blankline
%
\begin{itemize}
  \item En gruppering af \textit{affinity notes} må ikke startes med en design idé \textit{DI} eller et spørgsmål \textit{Q}
  \item Når en gruppering er påbegyndt behandles \textit{DI} og \textit{Q} på samme vilkår som de øvrige \textit{affinity notes}
  \item Mangler der sammenhæng i en gruppe fjern de \textit{affinity notes}, som ikke passer ind og dan en ny gruppe
  \item Alle i teamet kan uden forklaring flytte en \textit{affinity note}
  \item Sigt efter at hver gruppe kun indeholder mellem tre og seks \textit{affinity notes}, indeholder gruppen mere end seks skal gruppen splittes op
  \item Hvis en \textit{affinity note} ikke giver mening gå tilbage til listen over \textit{affinity notes} til den specifikke testperson og læs noten før og efter. Giver det stadig ikke mening tal med personen som afviklede interviewet eller en person, som var med i fortolkningssessionen og hvis nødvendigt ret den pågældende \textit{affinity note}
  \item Indeholder én \textit{affinity note} to pointer skal den deles op, det gøres ved at skrive en ny \textit{affinity note} på en gul \textit{sticky note} og slet den overflødige information på den originale note
  \item Èn \textit{affinity note} kan kun indgå i én gruppering
  \item Kan en \textit{affinity note} ikke placeres i en gruppe lægges den til side og vænnes tilbage til senere, alternativ gives noten til en anden i teamet
  \item Indeholder en \textit{affinity note} demografisk data eller irrelevant information samles de i en kategori for sig selv, senere i processen er det muligt at vende tilbage til denne kategori i tilfælde af at en af noterne kan placeres i en af kategorierne \blankline
\end{itemize}
%
Ovenstående er retningslinjer for hvordan kategorierne dannes, hvis der opstår problemer med at forstå hvad en \textit{affinity note} betyder, hvordan dårlige \textit{affinity notes} sorteres fra, \parencite[ss. 166-168]{Book:BuildingAnAffinity}. Følgende retningslinjer relaterer sig til hvordan alle \textit{affinity notes} bliver kategoriseret på væggen og er beskrevet af \textcite[s. 168]{Book:BuildingAnAffinity}.\blankline
%
\begin{enumerate}
  \item En fra teamet læser sin første \textit{affinity note} op og placerer den på væggen. Denne note må hverken være \textit{DI} eller \textit{Q}
  \item De resterene teammedlemmer gennemgår deres bunke af \textit{affinity notes} for at finde én, der passe i denne kategori
  \item Findes der en \textit{affinity note}, som relaterer sig til den første der blev sat på væggen, læses denne note op og sættes op på væggen under den første \textit{affinity note}
  \item Findes der ikke en \textit{affinity note}, som relaterer sig til den første, læses en ny \textit{affinity note} op og placeres et andet sted på væggen for at danne en ny kategori
  \item Denne proces forstætter til at der er dannet omkring 10 kategorier med to til fire \textit{affinity notes} i hver
  \item Det er tilladt at en af teammedlemmerne flytter en \textit{affinity note}, hvis de vurderer at den ikke passer i den kategori, hvor den er placeret. Dette gøres uden diskussion eller argumentation\blankline 
\end{enumerate}
%
Ovenstående er retningslinjerne til at få processen, omkring at bygge sit \textit{affinity diagram}, påbegyndt og skal selvfølgelig ikke følges indtil samtlige \textit{affinity notes} er sat på væggen og kategoriseret. Denne del af processen er i højere grad med til at starte processen og sikre at alle i teamet forstår retningslinjerne for at bygge et \textit{affinity diagram}, \parencite[s. 168]{Book:BuildingAnAffinity}. Når denne del af processen er overstået, er det tid til at placere de resterende \textit{affinity notes} i kategorier eller danne helt nye kategorier, hvilket, ifølge \textcite[ss. 168-169]{Book:BuildingAnAffinity}, gøres efter følgende retningslinjer:\blankline
%
\begin{enumerate}
  \item Tilføj hele tiden nye \textit{affinity notes} uden at læse dem højt. Læs dem kun højt hvis de danner en ny kategori så alle bliver gjort opmærksomme på at en ny kategori er dannet
  \item Hold øje med kategorier, hvor der kun er én eller to \textit{affinity notes}. Sørg for at de eksisterende kategorier er fyldt før en ny startes
  \item Teammedlemmer kan spørger de andre fra teamet om de er stødt på et specifikt problem, behov eller synspunkt. Undgå dog at bruge nøglebegreber
  \item Når de tyder på at alle i teamet har forstået processen stop med at læse noterne højt
  \item Løber en fra teamet tør for \textit{affinity notes} giv personen nogle flere, eksempelvist fra et teammedlem som har en stor bunke tilbage. Er der mindre end 600 \textit{affinity notes} skal de alle sammen op på væggen inden de får blå labels
  \item Når næsten alle \textit{affinity notes} er sat op sæt en tidsfrist på mellem 10 og 15 minutter til at få de sidste op inden der holdes pause\blankline
\end{enumerate}
%
Ved at vente med at sætte de blå labels på indtil alle \textit{affinity notes} er sat på væggen vil det være nødvendigt for teamet at læse alle noterne i en bestemt kategori før de placerer en ny \textit{affinity note}, \parencite[s. 169]{Book:BuildingAnAffinity}. Ved at følge disse retningslinjer for processen vil alle \textit{affinity notes} blive kategoriseret og klar til at få tildelt en blå label, som er det næste step i at bygge sit \textit{affinity diagram}. \blankline
%
Når kategorierne skal tildeles deres blå labels 	splittes teamet op i par, hvor i projektsammenhæng vil der være et par med to og et par med tre. Ved at dele teamet op i par er det muligt at undgå blå labels som indeholder nøgleord fremfor testpersonernes egen oplevelse af et problem, behov eller synspunkt, \parencite[s. 170]{Book:BuildingAnAffinity}. Det er skal dog pointeres at de blå labels altid kan omformuleres, hvis der er behov for det. Hver gruppe får et område på væggen, hvor de skal tildele blå labels så der ikke opstår forvirring om hvilke kategorier hvem står for. Efter områderne er uddelt starter hver gruppe med den længste kolonne, den største kategori, hvor målet er, at hvis der er mere end 500 \textit{affinity notes} totalt, skal hver kategori indeholde mellem fire og fem \textit{affinity notes}, \parencite[s. 170]{Book:BuildingAnAffinity}. Antallet af \textit{affinity notes}, der må være i én kategori varierer dog i fremgangsmåden beskrevet af \textcite[ss. 159-179]{Book:BuildingAnAffinity}, hvor nogle gange blive der rådet til at der skal være mellem tre og seks \textit{affinity notes} i én kategori, \parencite[s. 167]{Book:BuildingAnAffinity}, hvor der andre gange bliver rådet til at der skal være mellem to og seks \textit{affinity notes} i én kategori. Derudover afhænger antallet af \textit{affinity notes}, der må være i én kategori af hvor mange \textit{affinity notes}, der totalt er. Det antages derfor at er der mere end 500 \textit{affinity notes} bør der minimum være to og maksimum seks \textit{affinity notes} i én kategori. 

Indeholder en kategori mere end seks \textit{affinity notes} skal den brydes op, hvilket gøres efter følgende retningslinjer fremsat af \textcite[s. 170]{Book:BuildingAnAffinity}:\blankline
%
\begin{enumerate}
  \item Gennemgå hver \textit{affinity note} i kategorien for finde de noter, der hænger sammen og dem som ikke gør og gruppere dem i mindre kategorier
  \item Diskuter indbyrdes mulige problemer eller idéer der kan opstå i forbindelse med nye kategorier
  \item Fjern de \textit{affinity notes}, der ikke passer ind og giv dem til de andre par, hvis de arbejder med det specifikke problem, behov eller synspunkt
  \item Start nye kategorier, hvis det er nødvendigt 
  \item Tildel de nye kategorier blå labels
  \item Hvis flere blå labels hænger sammen som en større enhed, tildel dem en pink label\blankline 
\end{enumerate}
%
En god blå label sørger for, at når det samlede \textit{affinity diagram} er bygget og brugerens historie læses fra toppen og ned, så er det ikke nødvendigt at læse hver enkelt \textit{affinity note}, \parencite[ss. 170-171]{Book:BuildingAnAffinity}. En blå label skal derfor afspejle det specifikke problem, behov eller synspunkt som den dækker over, for på den måde at gøre det muligt at inddrage det i designet. Ifølge \textcite[s. 171]{Book:BuildingAnAffinity} er de blå labels de vigtigste, fordi hvis de er for generelle eller dækker over for mange \textit{affinity notes}, vil brugerens problemer, behov eller synspunkter forsvinde i mængden af data og dermed er det ikke muligt at tage højde for dem i designet af det endelige produkt. 

Ifølge \textcite[s. 172]{Book:BuildingAnAffinity} skal følgende tilstræbes for at udarbejde gode labels: \blankline
%
\begin{itemize}
  \item Labels skal skrives som hvis brugeren fortalte det, undgå derfor at skrive labels i tredjeperson
  \item En god label fanger essens af kategorien, undgå derfor abstrakte labels
  \item Sørg for at være klar og tydelig så alle forstår hvad der står på labelen
  \item Sørg for at det er den indsamlede data, der fortæller historien om brugerens oplevelse, undgå derfor foruddefineret kategorier 
  \item Hvis en kategori er for usammenhængende til at en label kan formuleres så den afspejler hvad der faktisk foregår, skal kategorien splittes op istedet for at tvinge en label ned over det
  \item Sørg for at tydeliggøre vigtige design pointer \blankline  
\end{itemize}
%
Når alle de blå labels er sat på væggen foreslår \textcite[ss. 173-174]{Book:BuildingAnAffinity}, at danne midlertidige grønne labels. De grønne labels er som tidligere nævnt det højeste niveau i et \textit{affinity diagram}, \textcite[s. 160]{Book:BuildingAnAffinity}. At udarbejde midlertidige grønne labels hjælper teamet med at gruppere de blå labels i forhold til bestemte emner, der belyser brugerens problemer, behov eller synspunkter. Når de blå labels er grupperet under midlertidige grønne labels er det muligt at definere relevante pink labels og derefter de endelige grønne labels. Ligesom ved de blå labels vil teamet blive splittet i par, som arbejder på hver deres grønne midlertidige label. Når der arbejdes i det grønne område er opgaven som følger: \blankline
%
\begin{enumerate}
  \item Omstrukturere de blå labels for at fjerne det overflødige, sørg for at de blå kategorier har den rette længde, omskriv de blå labels for at få de endelige blå labels, fjern \textit{affinity notes} og kategorier, som ikke passer til det grønne område og giv dem til et par, som arbejder med det emne
  \item Gruppere de blå labels for at danne pink labels. For hver pink label skal der være mellem to og seks blå labels
  \item Gruppere de pink labels for at danne sammenhængende grønne labels og omskriv de grønne labels for at få de endelige grønne labels. For hver grøn label skal der være mellem fire og otte pink labels. For at opnå et godt \textit{affinity diagram} bør der være mellem fem og seks grønne labels\blankline
\end{enumerate}
%
Ligesom de blå labels skal kunne erstatte de \textit{affinity notes}, der er placeret under så det ikke er nødvendigt at læse alle \textit{affinity notes}, skal pink labels ligeledes kunne erstatte de blå labels, \parencite[s. 175]{Book:BuildingAnAffinity}. Når både de blå, pink og grønne labels er omskrevet til de endelige labels er diagrammet færdigt. Det færdige \textit{affinity diagram} vil have en hierarkisk struktur, hvor hvis det læses fra toppen og ned fortæller brugerens historie omkring hvilke problemer de oplever, \parencite[s. 25]{PDF:ConsolidationIdeationAffinity}



    


