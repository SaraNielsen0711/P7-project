\chapter{Databehandling}
\label{ParametreDatabehandling}
%
I dette kapitel er der fokus på, at behandle det indsamlede data fra feltundersøgelsen. Først beskrives hvilket ændringer, der er foretaget i forbindelse med testdesignet som resultat af testlokationen. Dernæst belyses det hvilke testpersoner, der har deltaget i undersøgelsen, hvor det efterfølgende vil blive beskrevet hvordan den indsamlede data behandles både teoretisk og i praksis. Efterfølgende vil kapitlet handle om at analysere det udarbejdede \textit{affinity diagram}, hvorfra det bør være muligt at udlede hvilke parametre danske rejsende tilskriver interaktionen med en social robot i en dansk lufthavn. 

\section{Ændringer af testdesign}
\label{ParametreTestdesign}
%
De ændringer der er foretaget i forbindelse med at udføre undersøgelsen i Aalborg Lufthavn vil være relateret til udstyr og fremgangsmåde. 

Grundet placeringen i lufthavnen, efter sikkerhedskontrollen, var det ikke muligt at opsætte bord og stole til observatører og robotstyre som de kunne sidde ved. Dog var der tre stole til rådighed, én til robotstyren, to til to af de tre observatører. Ellers var det muligt at sidde andre steder, hvor der stadig var udsyn til både testpersonen interaktion med robotten og til interviewet. Ved at have spredt observatørerne over et større areal har det været muligt at indsamle observationer fra flere vinkler og med forskellige formål. Derudover blev der ikke opsat et højbord til testleder og testperson, dette blev dog ikke betragtet som et problem, da testlederen alligevel ikke skulle tage noter undervejs i interviewet. Det var ydermere ikke muligt at foretage skærmdelingen mellem hvad der præsenteres på robotten og en anden computer end den hvorfra robotten styres, fordi det krævede et lukket netværk. Det blev derfor besluttet at minimum én af observatørerne skulle være placeret et sted, hvorfra det muligt at se hvad den rejsende trykkede på skærmen og signalere det til robotstyren, så robotten kunne styres derefter.

Forud for undersøgelsen var der ikke taget nogle beslutninger om hvordan det skulle håndteres hvis mere end én rejsende interagerede med robotten og efterfølgende blev interviewet. Det blev derfor besluttet i lufthavnen, at det ikke ville være et problem og såfremt der var mere end én der interagerede med robotten og efterfølgende fulgte robotten hen til interviewet blev de alle sammen inddraget i interviewet. Dog har det indflydelse på den indsamlede data, da flere testpersoner i ét interview nødvendigvis ikke alle svarer på det samme spørgsmål. Derudover tages der ikke højde for mulige påvirkning blandt testpersonerne, som deltog i det samme interview. Fordelen ved at lade mere end én testperson interagere med robotten og deltage i det efterfølgende interview er, at det vil afspejle potentielle naturlige situationer i en lufthavn, hvor de rejsende ikke nødvendigvis rejser alene.        

\section{Testpersoner}
\label{ParametreTestpersoner}
%

30 testpersoner, 16 kvinder og 14 mænd. Alders fra 8-62 år (gennemsnit: 37.9 år)
Mænd: 62 år, 14 år, Sidst i 30'erne, S2.1 (39 år), 8 år, 30 år, 10 år, 47 år, 47 år, 61 år, 49 år, 49 år, 40-45 år, S16 (43 år), 24 år, 9 år
Kvinder: Sidst i 30'erne, S2.2 (39 år), 23 år, Midt i 50'erne, S5 (55 år), 50 år, 59 år, 61 år, 51 år, 41 år, 40 år, 51 år, 46 år, 44 år, 14 år, 23 år, 37 år, 11 år


Måske skriv hvor mange der var "alene" og hvor mange der er i en gruppe. 

 Og skriv hvor lang tid det tog: omkring 5 minutter per person eller gruppe. Skriv at vi ikke har sagt at deres data er behandles fortroligt men at det gør vi.  

