\chapter{Delkonklusion}
\label{ParametreDelKonklusion}
%
For at besvare den første problemstilling: \textit{Ud fra hvilke parametre beskriver danske rejsende interaktion med en social robot i en dansk lufthavn?}, blev der udført en feltundersøgelse i Aalborg Lufthavn på danske rejsende. Her sås det at flere rejsende havde lyst til at interagere og få hjælp af robotten. Til feltundersøgelsen blev der dels opstillet nogle samtaleemner, hvorfra testpersonerne fik mulighed for at give deres spontane respons relateret til interaktionen med robotten, og dels formuleret nogle specifikke spørgsmål, som omhandlede specifikke parametre, der har betydningen for HRI. 
%
\begin{table}[H]
	\centering
	\begin{tabular}{ l|l }
		\centering
		Kategori & Parameter\\ \hline
		Skærmen virker ikke & Skærmens reaktion\\ \hline
		\multirow{2}{*}{Assisterer mennesker} & R kan hjælpe \\
		& Personlig hjælp \\ \hline
		\multirow{5}{*}{Væremåde} & Bevægelse \\
		& Hastighed \\
		& Irriterende \\
		& Menneskelig* \\
		& Anmassende** \\ \hline
		\multirow{4}{*}{Henvendelse} & Imødekommende \\
		& Afstand \\
		& I vejen \\
		& Overrasket\\ \hline
		\multirow{2}{*}{Udseende} & Højde \\
		& Elegant \\ \hline
		Interesse & Spændende\\ \hline
		\multirow{4}{*}{Postiv over for R} & Sød \\
		& Sjov \\
		& Sej \\
		& Betjent af R\\ \hline
		\multirow{2}{*}{Kendskab til teknologi} & At bruge R*** \\
		& Kendskab til teknologi/robotter \\ \hline
		\multirow{3}{*}{Tillid} & Forskrækket \\
		& Regnede med robotten \\
		& Tryg
	\end{tabular}
	\caption{Oversigt over de fundne og udvalgte parametre, samt hvilken kategori de stammer fra. Parameteren: \textit{Menneskelig*} stammer fra to kategorier: \textit{Væremåde} og \textit{Udseende}, \textit{Anmassende**} stammer fra tre kategorier: \textit{Væremåde}, \textit{Henvendelse} og \textit{Interagerer ikke med R}, hvor \textit{At bruge R***} stammer fra: \textit{Positiv over for R} og \textit{Kendskab til teknologi}. \textit{R} er en forkortelse for robot.}
	\label{tab:OversigtOverValgteParametre}
\end{table}
\noindent
%
For at udlede hvilke parametre danske rejsende tilskriver interaktionen med en social robot, blev testpersonernes respons kategoriseret og analyseret igennem et \textit{affinity diagram}, jævnfør \fullref{ParametreDatabehandlingAffinityDiagram}. Baseret på dette diagram blev der opstillet nogle potentielle skala spørgsmål, hvorfra udvælgelsen og udviklingen af skalaerne er baseret. Der er i alt udvalgt 24 parametre, som der er blevet dannet skalaer ud fra, jævnfør \autoref{tab:OversigtOverValgteParametre}. Sammenlignes de udvalgte parametre med de parametre, der er fundet til at have indflydelse på HRI i tidligere undersøgelser, fremgår det, at flere af parametrene går igen også for danske rejsende. Ydermere findes der parametre, som ikke er blevet nævnt i litteraturgennemgangen i \fullref{ParametreDatabehandlingAffinityDiagram}, men som medtages, når skalaerne dannes.\blankline
%
Til hver af de udvalgte parametre er der udviklet en skala hvorpå parameteren kan evalueres, jævnfør \fullref{ParametreDatabehandlingSkalaer}, hvilket er en del af at besvare den anden problemstilling: \textit{Hvordan kan de fundne parametre gengives i skalaer, som efterfølgende kan bruges til at evaluere HRI?}. Da disse skalaer, samt tilhørende skala spørgsmål, er udviklet specifikt til én parameter er der ikke taget højde for det samlede helhedsindtryk, hvorfor det vurderes at den anden problemstilling ikke er besvaret tilfredstillende. Det er derfor nødvendigt at danne et helhedsindtryk samt videreudvikle de 24 skalaer og tilpasse dem i forhold til følgende: Formuleringen af skala spørgsmålet, labels, skalatype, samt hvilken rækkefølge skalaerne skal præsenteres i, i den kommende undersøgelse.    





%EN ANDEN TYPE TABEL
%\begin{tabular}{ l|l|l|l|l }
%\hline
%\multicolumn{3}{ r }{Udvalgte parametre} \\
%\hline
%Kategori & Parameter & Venstre label & Midtpunkt & Højre label\\ \hline
%Skærmen virker ikke & Skærmens reaktion & Ekstremt dårligt & NL & Ekstremt godt \\ \hline
%\multirow{2}{*}{Assisterer mennesker} & Hjælp & Helt uenig & Neutral & helt enig \\
 %& Personlig hjælp & \makecell{Slet ikke\\ personlig} & - & Ekstremt personlig\\ \hline
%\multirow{5}{*}{Væremåde} & Bevægelse & Ekstremt rolige & NL & Ekstremt vilde \\
 %& Hastighed & Alt for langsom & Fin & Alt for hurtig \\
 %& Irriterende & Slet ikke irriterende & - & Ekstremt irriterende \\
 %& Menneskelig* & Slet ikke menneskelig & - & Ekstremt menneskelig \\
 %& Anmassende & Slet ikke anmassende & - & Ekstremt anmassende \\ \hline
%\multirow{4}{*}{Henvendelse} & Imødekommende & Ekstremt afvisende & NL & Ekstremt imødekommende \\
 %& Afstand & Alt for tæt på & NL & Alt for langt fra \\
 %& I vejen & Slet ikke i vejen & - & Ekstremt i vejen\\
 %& Overrasket & Slet ikke overrasket & - & Ekstremt overrasket\\ \hline
%\hline
%\end{tabular}