\section{Dannelse af skalaer}
\label{ParametreDatabehandlingSkalaer}
%
Ud fra de opstillede parametre, opstilles der skalaer, ved at gennemgå alle sætningerne markeret med S (for Scale). Kravet for opstilling af en skala er, at det er muligt at ændre på den specifikke parametre skalaen beskriver. Formålet er at opstille den optimale skala for hvert enkelt parameter, der er derfor ikke fastsat én specifik skala type på forhånd. 
%Det vurderes at en bipolær skala er mest beskrivende da den har den største spænd og denne vil derfor fortrækkes, i tilfælde hvor der findes både en bipolær og unipolær. 


Når parametrene er udledt fra hver af de grønne kategorier skal der udvælges de parametre, som vil blive anvendt i forbindelse med at udvikle skalaerne. 

Fælles for disse parametre er, at de skal opfylde følgende kriterier: \blankline
%
\begin{itemize}
  \item En designer skal kunne ændre på det
  \item 
\end{itemize}



NOTER FRA DANNELSEN AF SKALAERNE:
\textbf{Jeg føler mig tryg ved R}
DEnne er baseret på observationer og ikke direkte udtalelser fra testpersonerne. 
Ekstrem tryg/

\textbf{Jeg regner med at R følger mig hen til det sted jeg har valgt} 
helt enig/helt uenig med neutral i midten. 

Vilde bevægelser vælges som modpol til rolige bevægelser da det vurderes at være modsætningen, men også i høj grad fordi denne beskrivelse blev brugt af testpersonerne i undersøgelsen. 

Blev sorteret fra da den ikke er manipulerbar, da den kun henvender sig til hvad man foretrækker. 