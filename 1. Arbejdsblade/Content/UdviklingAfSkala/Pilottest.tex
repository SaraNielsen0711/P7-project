\section{Pilottest}
\label{ParametrePilottest}
%
Ved udviklingen af testdesignet er de forskellige delelementer i testen løbende blevet testet, primært på projektgruppens medlemmer. \blankline
%
\subsection*{Styring af robotten}
Robotstyreren skal øve sig i at styre robotten, derfor kørte vedkommende robotten rundt på Aalborg Universitet, Frederik Bajers Vej 7, mens robotstyreren sad i grupperummet med en computer forbundet til \textit{Double}. På den måde fik robotstyreren erfaring med at styre robotten ved brug af dens kamera. Robotten kørte rundt blandt studerende, så navigation blandt mennesker blev øvet. Det blev testet at dreje og køre i alle retninger og da gulvet er forskelligt rundt på Universitet, blev det også testet at køre robotten på forskellige underlag og over samlinger i gulvet. 
% 
Da det kan være svært at vurdere hvor tæt robotten kommer på personer omkring den, blev dens henvendelse testet og øvet ved at lade robotstyreren køre robotten hen til personer, for derefter at sammenligne afstanden mellem robotten og den person den henvender sig til, med hvad robotstyreren havde forventet at afstanden ville være. \blankline
%
\subsection*{Interfacet}
Interfacet er testet ved at lade et andet gruppemedlem end den der udviklede interfacet gennemgå alle mulige scenarier. Testningen af interfacet er fortaget i en iterativ proces, hvor tilføjelser og ændringer er foretaget med det samme, hvorefter interfacet er testet igen og dette gøres indtil projektgruppen vurderer at der ikke længere kan findes væsentlige fejl og mangler ved det. \blankline

Efter de forskellige delelementer af testen er på plads udføres den første sammenhængende pilottest. Her er det et gruppemedlem fra en anden gruppe end projektgruppen, der bliver rekrutteret til at deltage i testen. Efter udførelsen af denne test tilpasses testdesignet, og der udføres en pilottest mere, denne gang med gruppens vejleder, Dorte, som testdeltager. Formålet med at bruge gruppens vejleder som testperson i pilottesten er at det både giver mulighed for at teste designet og efterfølgende evaluere på testdesignet i samarbejde med vejlederen. Denne evalueringen resulterede i en større ændring ved testdesignet, hvor det blev besluttet at der ikke skal være interaktion mellem interviewer og testdeltageren før efter robotten har fulgt testpersonen hen til intervieweren. Fordelen ved at undgå kontakt mellem interviewer og testperson indtil selv interviewet er at testpersonens interaktion med robotten ikke bliver forstyrret og det bliver derfor en mere økologisk oplevelse for testpersonen. \blankline
%
Næste step i forhold til pilottestning burde være at udføre en pilottest i det miljø som selve testen skal udføres i. Da projektgruppen ikke har ubegrænset adgang til Aalborg Lufthavn, er det ikke en mulighed at udføre tests i lufthavnen over flere dage. Det vælges derfor at udføre selve testen i lufthaven, hvor der ved de første deltagerer i testen er fokus på om testdesignet fungerer som forventet. Det vælges at accepterer at der kan fortages tilpasninger i testdesignet undervejs i udførelsen at testen, som er nødvendige for at få det udbytte at testen som der ønskes. Da det er kvalitativt data der indsamles vurderes det, at det ikke har væsentlig betydning for resultaterne at testdesignet ikke er helt konsistent ved alle testene. Desuden vil der naturligt være variationer mellem alle de tests der udføres, da de udføres i et naturligt miljø, hvor det ikke er muligt, eller ønsket, at kontrollerer variabler der påvirker testen og testpersonernes interaktion med robotten. 

%Pilottest blev udført primært på gruppen selv i grupperummet og én enkelt anden testperson. Skriv om hvordan det gik. 

%Skriv om hvordan de første par test i lufthavnen ville være en pilottest hvis der opstod problemer, det er også her vi skriver om hvad vi har gjort for at tilpasse vores test design. 