\section{Pilottest}
\label{ParametrePilottest}
%
Ved udviklingen af testdesignet er de forskellige elementer i testen løbende blevet testet, primært på projektgruppens medlemmer. I følgende afsnit vil de forskellige elementer blive beskrevet.
%
\subsection{Styring af robotten}
\label{ParametrePilotStyringRobot}
%
Robotstyreren skal øve sig i at styre robotten, derfor kørte vedkommende robotten rundt på Aalborg Universitet, Frederik Bajers Vej 7, mens robotstyreren sad i grupperummet med en computer forbundet til \textit{Double}. På den måde fik robotstyreren erfaring med at styre robotten ved brug af dens kamera. Robotten kørte rundt blandt studerende, hvorfor navigation blandt mennesker blev øvet. Det blev testet at få robotten til at dreje og køre i alle retninger og da gulvet er forskelligt rundt på Universitet, blev det også testet at køre robotten på forskellige underlag og over samlinger i gulvet. 

Da det kan være svært at vurdere hvor tæt robotten kommer på personer omkring den, blev dens henvendelse testet og øvet ved at lade robotstyreren køre robotten hen til personer, for derefter at sammenligne afstanden mellem robotten og den person den henvender sig til, med hvad robotstyreren havde forventet at afstanden ville være.
%
\newpage
\subsection{Brugergrænseflade}
\label{ParametrePilotInterface}
%
Brugergrænsefladen, som er udviklet i \textit{Marvel}, blev testet ved at lade et gruppemedlem, som ikke havde været med til at designe brugergrænsefladen, gennemgå alle mulige scenarier. Testningen af brugergrænsefladen er foretaget som en iterativ proces, hvor tilføjelser og ændringer er foretaget med det samme, hvorefter brugergrænsefladen er testet igen. Dette gøres indtil projektgruppen vurderer, at der ikke længere kan findes væsentlige fejl og mangler ved det.
%
\subsection{Testdesignet}
\label{ParametrePilotTestdesign}
%
Efter de forskellige elementer af testen er på plads udføres den første sammenhængende pilottest. Her er det med gruppens vejleder, Dorte, som testdeltager. Formålet med at bruge gruppens vejleder som testperson i pilottesten er, at det både giver mulighed for at teste designet og efterfølgende evaluere på testdesignet i samarbejde med vejlederen. Denne evaluering resulterede i, at det blev fastlagt, at der ikke skal være nogen interaktion mellem interviewer og testperson før efter robotten har fulgt testpersonen hen til intervieweren. Fordelen ved at undgå kontakt mellem interviewer og testperson indtil selve interviewet er, at testpersonens interaktion med robotten derved ikke bliver forstyrret og det bliver derfor en mere økologisk oplevelse for testpersonen. \blankline
%
Efter den første sammenhængende pilottest udføres der i alt tre pilottests, hvor det er forskellige medlemmer fra projektgruppen der er testpersoner og interviewere. Disse pilottests har primært til formål, at give intervieweren erfaring med at udføre undersøgelsen og interviewet samt få tilpasset det således at undersøgelsen udføres som det er besluttet. Det er valgt at der ved udførelsen af undersøgelsen skal være to gruppemedlemmer, der skiftes til at være interviewer, derfor skal begge interviewere også prøve at have denne rolle i en pilottest. Udbyttet af disse pilottests er at gruppen får forventningsafstemt, hvordan udførelsen skal være for at det stemmer oversens med det der er besluttet.

Ved den sidste pilottest, der udføres, rekrutteres et gruppemedlem fra en anden gruppe end projektgruppen, til at deltage i testen. På den måde er testpersonen i denne pilottest ikke bekendt med formålet med testen og overvejelserne bag testdesignet, så forventningen er at feedbacken ikke vil være påvirket i samme grad som tilfældet er når det er projektgruppen eller vejlederen, der gennemgår testen. \blankline
%
Næste step i forhold til pilottestning burde være at udføre en pilottest i det miljø som undersøgelsen reelt skal foregå i. Da projektgruppen ikke har ubegrænset adgang til Aalborg Lufthavn er det ikke sikkert at der er mulighed for, at udføre flere undersøgelser i lufthavnen over flere dage. Det vælges derfor at udføre undersøgelsen i lufthaven, hvor der ved de første testpersoner er fokus på om testdesignet fungerer som forventet. Det vælges at accepterer, at der kan forekomme ændringer i testdesignet undervejs i interviewet, som er nødvendige for at få det optimale udbytte af undersøgelsen. Desuden vil der naturligt forekomme variationer fra testperson til testperson, da undersøgelsen udføres i et naturligt miljø, hvor det ikke er muligt, eller ønsket, at kontrollere variabler, der påvirker interviewet og testpersonernes interaktion med robotten. Det anslåes at interviewet varer omkring fem minutter, hvor interaktionen med robotten forventes at være kortere. 
