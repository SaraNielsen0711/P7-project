\section{Fremgangsmåde}
\label{ParametreFremgangsmaade}
%
Så snart en af de rejsende befinder sig i det aftalte område vil robotten henvende sig ved at køre hen til den rejsende og spørge om den kan hjælpe dem med at finde rundt i Aalborg Lufthavn. Svarer testpersonen \textit{Ja} vil der efterfølgende ganske kort stå, på skærmen, hvad formålet med testen er: At undersøge menneskers interaktion med robotter. Derefter har testpersonen mulighed for selv, at vælge et af de fire brugsscenarier beskrevet i \fullref{ParametreBrugsscenarier}. Uanset hvilket brugsscenarie, der bliver valgt vil det ende med at robotten opfordrer testpersonen til at følge efter. Istedet for at følge testpersonen det rigtige sted hen følger robotten testpersonen hen til testlederen, som tager over. Testlederen har til opgave at small-talke med testpersonen for at skabe en behagelig stemning og undervejs ledes small-talken over i samtaleemnerne, jævnfør \fullref{ParametreSamtaleemner}. Small-talken vedrører eksempelvis hvor testpersonen skal rejse hen, om de har rejst fra Aalborg Lufthavn før og lignende. I tillæg vil testpersonen introducere projektgruppen og at der sidder nogen og tager noter. 

Inden samtalen drejes ind på samtaleemnerne skal testlederen sørge for at give testpersonen tilstrækkelig information omkring testen til at testpersonen kan give et mundtligt samtykke, som tillader at der kan optages lyd. Som nævnt vil der undervejs blive taget noter både før testpersonen er i kontakt med testlederen og efter når de to konverserer. \blankline
%
Da det er robotten, som har stået for rekrutteringen og der ikke har været en menneske-menneske interaktion før robotten følger testpersonen hen til testlederen, vil det første samtaleemne vedrøre testpersonens førstehåndsindtryk dels af hvordan robotten henvendte sig til testpersonen og dels af hvordan det var at interagere med robotten. Da der anvendes elementer fra \textit{laddering} er det en fordel at testlederen gør det klart, at der vil blive stilt en del spørgsmål, som skal få testpersonen til at reflektere over sin oplevelse, så det ikke virker påtrængende at der bliver stilt alle de spørgsmål. Undervejs i samtalen vil robotstyren sørger for at robotten køre rundt på forskellige måder, varierer afstanden til testpersonen samt højden på robotten. Det forventes at det er med til at få testpersonerne til at kommentere på netop robottens bevægelse. 

Når alle samtaleemner er blevet inddraget og diskuteret er det testlederens opgave at afslutte undersøgelsen ved at debriefe testpersonen og ønske dem en god rejse. Under debriefingen vil testpersonen blive spurgt om alder og hvor ofte de rejser, hvor kønnet noteres af en af observatørerne. Der vil derudover bliver spurgt ind til nogle forskellige parametre, såsom robottens højde, hastighed, afstand, generelle bevægelse, udseende og indgangsvinkel, jævnfør \fullref{ParametreAfrundingDebriefing}.\blankline      
%
Trykker den rejsende derimod \textit{Nej} for ikke at deltage i undersøgelsen, ønsker robotten personen god rejse og forlader stedet og venter til at en ny rejsende befinder sig i området.\blankline
%
Efter hver testperson vil observatørerne konferere med hinanden og dele observationer eksempelvis i forhold til, hvad der kan være værd at fokusere på, men også i forhold til hvordan forskellige ting noteres, så der er en form for konsensus om noteringen. Ydermere vil der efter to eller tre testpersoner blive roteret roller blandt projektgruppen.  

