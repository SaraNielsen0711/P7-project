\section{Rollefordeling}
\label{ParametreRollefordeling}
%
For at udføre feltundersøgelsen er det nødvendigt at definere nogle roller, som relaterer sig til specifikke dele af undersøgelsen. Der vil i alt blive defineret fire roller, som vil rotere mellem projektgruppen.
%
\subsubsection*{Robot styre}
Da de rejsende skal interagere med en \textit{Double}-robot, som ikke kan forudprogrammeres er det nødvendigt, at have én til at styre robotten. For at de rejsende får et indtryk af at robotten er autonom og har en form for social intelligens er det favorabelt, at den rejsende ikke registrerer, at det er en person, som styrer robotten. For at efterkomme det vil personen, som styrer robotten være placeret så de rejsende ikke direkte kan se hvordan robotten styres, dog skal personen stadig have mulighed for at overvære og høre interaktionen mellem robot og den rejsende, så robotten kan styres derefter. Udover placeringen af personen, som styrer robotten, bliver de rejsende fortalt af testlederen, at personen vil tage supplerende noter. 

\subsubsection*{Testleder}
Testlederen har flere opgaver, først og fremmest er det testlederen de rejsende kommunikerer med, udover robotten, hvorfor det er vigtigt at testlederen i starten sørger for at den rejsende føler sig tryg, hvilket tilstræbes ved small-talk. Derefter gengiver testlederen formålet med undersøgelsen: At finde ud af hvordan danske rejsende beskriver interaktionen med en social robot i en lufthavn. I den forbindelse nævner testlederen at der vil blive optaget lyd og såfremt at testpersonerne giver mundtligt samtykke til det, starter testlederen lydoptageren

Testlederen har desuden til opgave at styre samtalen ind på de nævnte samtaleemner beskrevet i \fullref{ParametreSamtaleemner}. Afslutningsvist har testlederen til opgave, at debriefe den rejsende omkring undersøgelsen samt at afslutte forløbet. Da samtalen mellem testleder og testperson varierer udarbejdes der ikke specifikke instruktioner. Dog sørger testlederen for at introducere testpersonen til hvem projektgruppen er og at der vil blive taget noter undervejs. Fordi der anvendes elementer fra \textit{laddering} vil testlederen informere testpersonerne om at der vil blive spurgt en del ind til testpersonens udsagn.

Når robotten ikke interagerer med en testperson er det testlederens opgave, at holde øje med at robotten ikke kører ind i noget eller at andre rejsende ikke går ind i den.    

\subsubsection*{Observatør}
Der vil i alt være tre observatører til stede, hvor minimum én af dem vil observere interaktionen mellem testperson og robot før robotten leder testpersonen over til testlederen. Da det tilstræbes at interaktionen mellem testperson og robot er så naturlig som muligt, vil observatøren holde afstand til dem. De to andre observatører sidder med forskellig udsyn til både interaktionen mellem robot og testperson, og ved interviewet. Formålet med at have tre observatører er for at være sikker på, at der indsamles så meget data som muligt og fordi det forventes at observatørerne bemærker forskellige ting. Derudover vil der ikke optages video, hvorfor det er vigtigt at have gode notater. Ydermere vil det tilstræbes at én observatør fokuserer på testpersonernes mimik, én anden observatør fokuserer på kropsholdning og den sidste observatør fokuserer på decideret pointer koblet til en robot aktion. Er det muligt for observatørerne, at notere flere ting så gør de bare det. Observatørens noter vil blive nedskrevet på notespapir. 
