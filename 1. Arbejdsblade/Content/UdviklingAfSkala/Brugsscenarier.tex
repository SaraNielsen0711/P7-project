\section{Brugsscenarier}
\label{ParametreBrugsscenarier}
%
Da det tilstræbes at få et så naturligt scenarie, som muligt vil der ikke blive stilt en decideret opgave som testpersonerne skal løse men for at få de rejsende til, at sætte deres egne ord på oplevelsen af interaktionen med en social robot defineres der fire brugsscenarier, som anses for at forekomme naturligt i en lufthavn. De fire brugsscenarier henvender sig enten til at den rejsende skal finde gate information, toiletfaciliteter, indkøbsmuligheder eller forplejning. Det er derfor op til testpersonen selv at vælge, hvilket af de fire brugsscenarie, der gemmengåes. De fire brugsscenarier gengives i \textit{wireframe} designet i \textit{Marvel}, som præsenteres på \textit{Double}'s skærm. Det samlede \textit{wireframe} fremgår af \autoref{fig:SamledeWirefram}.
%
\begin{figure}[H]
\centering
\includegraphics[width = 0.6\textwidth]{Figure/SamledeWirefram} 
\caption{Oversigt over de forskellige skærmbilleder i det samlede \textit{wirefram}.}
\label{fig:SamledeWirefram}
\end{figure}
\noindent
%  

\subsubsection*{Gate information}
%
Vælger testpersonerne at finde information om deres gate vil følgende skærmbilleder blivet præsenteret på skærmen, jævnfør \autoref{fig:GateInformation}. 
%
\begin{figure}[H]
\centering
\includegraphics[width = 0.6\textwidth]{Figure/GateInformation} 
\caption{Oversigt over de forskellige skærmbilleder, der anvendes i brugsscenariet: Gate information.}
\label{fig:GateInformation}
\end{figure}
\noindent
%  

\subsubsection*{Toiletfaciliteter}
%
Vælger testpersonerne at finde information om toiletfaciliteter vil følgende skærmbilleder blivet præsenteret på skærmen, jævnfør \autoref{fig:Toiletfaciliteter}. 
%
\begin{figure}[H]
\centering
\includegraphics[width = 0.7\textwidth]{Figure/Toiletfaciliteter} 
\caption{Oversigt over de forskellige skærmbilleder, der er specifikke for brugsscenariet: Toiletfaciliteter.}
\label{fig:Toiletfaciliteter}
\end{figure}
\noindent
%  
\subsubsection*{Indkøbsmuligheder}
%
Vælger testpersonerne at finde information om indkøbsmuligheder vil følgende skærmbilleder blivet præsenteret på skærmen, jævnfør \autoref{fig:Indkoebsmuligheder}. 
%
\begin{figure}[H]
\centering
\includegraphics[width = 0.6\textwidth]{Figure/Indkoebsmuligheder} 
\caption{Oversigt over de forskellige skærmbilleder, der anvendes i brugsscenariet: Indkøbsmuligheder.}
\label{fig:Indkoebsmuligheder}
\end{figure}
\noindent
% 
\subsubsection*{Forplejning}
%
Vælger testpersonerne at finde information om forplejning vil følgende skærmbilleder blivet præsenteret på skærmen, jævnfør \autoref{fig:Forplejning}. 
%
\begin{figure}[H]
\centering
\includegraphics[width = 0.6\textwidth]{Figure/Indkoebsmuligheder} 
\caption{Oversigt over de forskellige skærmbilleder, der anvendes i brugsscenariet: Forplejning.}
\label{fig:Forplejning}
\end{figure}
\noindent
% 
