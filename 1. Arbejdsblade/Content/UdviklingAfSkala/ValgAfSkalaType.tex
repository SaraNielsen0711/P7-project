\section{Valg af skala type}
\label{ParametreSkalaType}
%
Det vurderes at en bipolær skala er mest beskrivende da den har den største spænd og denne vil derfor fortrækkes, i tilfælde hvor der findes både en bipolær og unipolær.\\
%
%OBS: SKRIV LIGE AT VI BRUGER VAS BI- OG UNIPOLÆR, HUSK AT SKRIV HVORDAN DE BYGGES OP og hvorfor vi har valgt at bruge VAS, hvorfor vi har nogle der er unipolære og hvorfor vi har nogle der er bipolære og hvorfor vi har midt punkter både med og uden labels - hvad er fordelene og ulemperne ved det og hvorfor vælger vi som vi gør. Argumenter for hvorfor vi enten bruge åbne eller lukkede endepunkter på VAS - kant effekter, man kan "bange" for at svare helt ude ved kanten, til gengæld kan det være med til at få spredt data på en bestemt måde. Hvorimod hvis skalaerne havde været åbne så giver det mere "tryghed" og en chance for at svare noget der er værre/bedre end det forgående selvom vi ikke beder dem om det - de ved det jo ikke at de ikke får en skala hvor de skal svare på det sammen. 
%Vi vil helst have dem som bipolære men hvis der ikke kan findes en naturlig og logisk modpol så vælges det at gøre det til en unipolær skala.
%
HER SKAL DER STÅ OM HVORFOR VI VÆLGER VAS - FORDELE OG ULEMPER\\
HVORFOR VI HAR NOGLE UNIPOLÆRE OG NOGLE BIPOLÆRE SKALAER\\
HVORFOR VI VÆLGER LUKKEDE ENDEPUNKTER FREMFOR ÅBNE ENDEPUNKTER - ARGUMENTER FOR VALG OG MULIGE ULEMPER OG FORDELE VED BEGGE.\blankline
%
I forhold til labels på skalaer så bruger vi f.eks. ekstremt og slet ikke, men det er jo ikke helt testpersonernes egne ord, men det gøres for at opnå et så stort spænd som muligt mellem de to punkter. Derudover hænger det også sammen med nogen af de retningslinjer der er for at opstille VAS - at endepunkterne skal være ekstremer. 

%OBS: nedenstående kan være en overgang til den næste sektion. 
Først vil hver af de 10 grønne kategorier blive præsenteret sammen med tilhørende potentielle skala spørgsmål, efterfulgt af tilhørende skala labels. For hvert potentielt skala spørgsmål vil det blive besluttet hvorvidt den specifikke parameter skal evalueres på en bipolær eller på en unipolær \textit{Visual Analoge Scale}, (VAS) med lukkede endepunkter. I tilfælde hvor parameteren evalueres på en bipolær skala vil der være markeret et midtpunkt, som enten kan være unavngivet eller navngivet. Derefter vil der fokuseres på udvælgelsen af de endelige skalaer.