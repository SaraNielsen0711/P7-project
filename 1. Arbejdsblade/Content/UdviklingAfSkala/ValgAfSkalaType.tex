\section{Valg af skala type}
\label{ParametreSkalaType}
%
Det vælges at anvende en \textit{Visual Analogue Scale} (VAS) til skalaspørgsmålene. Fordelen ved at bruge en VAS er, at det giver mulighed for mere nuancerede besvarelser. Testpersonerne har mulighed for at svare hvor som helst på skalaen, modsat eksempelvis en \textit{Likert}-skala, hvor svarmulighederne er afgrænsede til et bestemt antal bokse.
Dog ifølge \textcite[s. 73]{PDF:RatingScales}, er der ikke væsentlig forskel, når der er tale om UX research, på at bruge en 7-, 11-, eller 101-punkts skala, hvor 101-punkts skalaen er en VAS, der går fra 0-100.
%
\subsection{Endepunkter}
Ved VAS er der mulighed for at have lukkede eller åbne endepunkter. Fordelen ved at have åbne endepunkter er, at hvis samme testperson skal svare på det samme skalaspørgsmål gentagende gange, kan testpersonen få behov for at overgå tidligere afgivende besvarelser. Dette gør sig eksempelvis gældende, hvis testpersonen skal bedømme otte forskellige bolscher på de samme skalaer. Hvis de har bedømt et bolsche som værende stærkt, og der pludselig kommer et vanvittigt stækrt bolsche efterfølgende, så har de mulighed for at overgå deres første vurdering af, om bolschet er stærkt ved at sætte deres besvarelse udenfor endepunktet. Ved åbne endepunkter får testpersonen netop denne mulighed, da det er muligt at svare efter et endepunkt. Skalaerne i dette projekt skal testpersonerne kun svare på én gang for hver skala, hvilket gør at det ikke er så vigtigt at der er mulighed for at overgå tidligere besvarelser. \blankline
%
Det vælges at anvende skalaer med lukkede endepunkter da det vurderes, at der ikke er behov for at have åbne endepunkter. Desuden vurderes det, at det vil være nemmere for testpersonerne at bruge en skala med lukkede endepunkter, da der ikke skal tages stilling til, hvad der er efter endepunktet. Det kan ske, at der vil være ophobninger et stykke fra endepunkterne, da der muligvis vil være en tilbageholdenhed for at svare helt ude ved kanten. På den anden side er der også muligheden for at flere sætter ved endepunkter, fordi nogle måske kunne have ønsket at sætte længere ude på skalaen. 
%
%Hvorimod hvis skalaerne havde været åbne så giver det mere "tryghed" og en chance for at svare noget der er værre/bedre end det forgående selvom vi ikke beder dem om det - de ved det jo ikke at de ikke får en skala hvor de skal svare på det sammen. 
%
\subsection{Bipolær og unipolær}
Skalaerne kan opstilles som enten bipolær eller unipolær. Ved en unipolær skala vil det være samme ord for hele skalaen, hvor det er graden af dette ord, der bedømmes. Hvis skalaen er bipolær, vil det være to forskellige ord i hver ende af skalaen, hvor ordene er hinandens modpoler. \blankline
%
I forhold til dannelsen af skalaerne ønskes det for hvert skalaspørgsmål at tage stilling til, om der er mulighed for en bipolær skala, hvor de to ord er hinandens modpoler. Hvis det vurderes, at der ikke findes en naturlig og logisk modpol til ordet, vil der i stedet dannes en unipolær skala. \blankline
%
Årsagen til at det først undersøges om det er muligt at danne en bipolær skala fremfor en unipolær skala, er at der ved en bipolær skala dækkes over to udtryk, og det vil komme til udtryk hvor høj grad af modpolen, der er enighed med og ikke kun, at der ikke er enighed med den anden pol. Det er vigtigt, at der ved en bipolær skala er modpoler i hver ende, så der ikke opstår tilfælde, hvor det ikke er muligt at besvare skalaen, da personen måske er enig i begge poler og derfor ikke ved om de skal sætte besvarelsen i den ene eller i den anden ende. Derfor vil der kun blive dannet bipolære skalaer i tilfælde hvor der findes en naturlig og logisk modpol. 
%
\subsection{Labels}
%
De labels, der sættes på skalaerne dannes ud fra testpersonernes egne ord og formuleringer. 
Det ønskes dog at der i hver ende af skalaen skal være ekstremer, så der opnås så stort et spænd som muligt mellem de to punkter, så der ikke er tilfælde hvor skalaen ikke er dækkende i forhold til det testpersonen ønsker at svare. Derudover hænger det også sammen med, at der ved opstilling af VAS, er retningslinjer for at endepunkterne skal være ekstremer.\blankline
%
Det vælges at bruge testpersonernes ord, men hvor der ved den laveste grad af ordet sættes \textit{slet ikke} foran, og hvor det ved den højeste grad af ordet sættes \textit{ekstremt} foran. Det betyder at ved et eksempel, hvor ordet er \textit{glad}, så vil skalaens labels være \textit{Slet ikke glad} og \textit{Ekstremt glad}.


