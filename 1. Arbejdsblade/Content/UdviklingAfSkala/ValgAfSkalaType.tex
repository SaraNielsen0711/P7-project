\section{Valg af skala type}
\label{ParametreSkalaType}
%
Det vælges at anvende den samme skalatype til skalaspørgsmålene og denne type er en \textit{Visual Analoge Scale} (VAS). Fordelen ved at bruge en VAS er at det giver mulighed for mere nuancerede besvarelser, de testpersonerne har mulighed for at svare hvor som helst på skalaen, modsat eksempelvis en likert skala, hvor svarmulighederne er afgrænsede til et bestemt antal bokse. \blankline
%
Ifølge \textcite[s. 73]{PDF:RatingScales} er der, når konteksten er UX research, ikke væsentlig forskel på at bruge en 7-, 11- eller 101-punkts skala, hvor 101-punkts skalaen er en VAS der går fra 0 til 100.  
%
\subsection{Endepunkter}
Ved VAS er der mulighed for at have lukkede eller åbne endepunkter. Fordelen ved at have åbne endepunkter er, at hvis samme testperson skal svarer på den samme skalaspørgsmål gentagende gange, kan testpersonen få behov for at overgå tidligere afgivende besvarelser. Ved åbne endepunkter får testpersonen netop denne mulighed, da det er muligt at svarer efter et endepunkt. Skalaerne i dette projekt skal testpersonerne kun svare på én gang for hver skala, hvilket gør at det ikke er så vigtigt at der er mulighed for at overgå tidligere besvarelse. \blankline
%
Det vælges at anvende skalaer med lukkede endepunkter da det vurderes at der ikke er behov for at have åbne endepunkter. Desuden vurderes det, at det vil være nemmere for testpersonerne at bruge en skala med lukkede endepunkter, da der ikke skal tages stilling til hvad der er efter endepunktet. Det forventes ikke at der vil være ophobninger ved endepunkterne, da der muligvis vil være en tilbageholdenhed for at svare helt ude ved kanten. På den anden side er der også muligheden for at flere sætter ved endepunkter, fordi nogle måske kunne have ønsket at sætte længere ude. 
%
%Hvorimod hvis skalaerne havde været åbne så giver det mere "tryghed" og en chance for at svare noget der er værre/bedre end det forgående selvom vi ikke beder dem om det - de ved det jo ikke at de ikke får en skala hvor de skal svare på det sammen. 
%
\subsection{Bipolær og unipolær}
Skalaerne kan opstilles som enten bipolær eller unipolær. Ved en unipolær skala vil det være samme ord for hele skalaen, hvor det er graden af dette ord. Hvis skalaen er bipolær vil det være to forskellige ord i hver ende af skalaen, hvor ordene er hinandens modpoler. \blankline
%
I forhold til dannelsen af skalaerne ønskes det for hvert scale question at tage stilling til om der er mulighed for en bipolær skala, hvor de to ord er hinandens modpoler. Hvis det vurderes at det ikke findes en naturlig og logisk modpol til ordet vil der i stedet dannes en unipolær skala. \blankline
%
Årsagen til at det først undersøges om det er muligt at danne en bipolær skala fremfor en unipolær skala, er at der ved en bipolær skala dækkes over to udtryk og det vil komme til udtryk hvor høj grad af modpolen der er enighed med og ikke kun at der ikke er enighed med den anden pol. Det er vigtigt at der ved en bipolær skala er modpoler i hver ende, så der ikke opstår tilfælde hvor det ikke er muligt at besvarer skalaen, da personen måske er enig i begge poler og derfor ikke ved og de skal sætte besvarelsen i den ene eller i den anden ende. Derfor vil der kun blive dannet bipolære skalaer i tilfælde hvor der findes en naturlig og logisk modpol. 
%
\subsection{Labels}
%
De labels der sættes på skalaerne dannes ud fra testpersonernes egen ord og formuleringer. 
Det ønskes dog at der i hver ende af skalaen skal være ekstremer, så der opnås så stort et spænd som muligt mellem de to punkter, så der ikke er tilfælde hvor skalaen ikke er dækkende ift. det testpersonen ønsker at svarer. Derudover hænger det også sammen med at der ved opstilling af VAS, er retningslinjer for at endepunkterne skal være ekstremer.\blankline
%
Det vælges at bruge testpersonernes ord, men hvor der ved den laveste grad af ordet sættes ``slet ikke'' foran og hvor det ved den højeste grad af ordet sættes ``ekstremt'' foran. Det betyder at ved et eksempel hvor ordet er ``glad'', så vil skalaens labels være ``slet ikke glad'' og ``ekstremt glad''. \blankline

%Nedenstående kan være en overgang til den næste sektion. 
I det følgende afsnit vil hver af de 10 grønne kategorier først blive præsenteret sammen med tilhørende potentielle skala spørgsmål, efterfulgt af tilhørende skala labels. For hvert potentielt skala spørgsmål vil det blive besluttet hvorvidt den specifikke parameter skal evalueres på en bipolær eller på en unipolær \textit{Visual Analoge Scale}, (VAS) med lukkede endepunkter. I tilfælde hvor parameteren evalueres på en bipolær skala vil der være markeret et midtpunkt, som enten kan være unavngivet eller navngivet. Derefter vil der fokuseres på udvælgelsen af de endelige skalaer.