\section{Metodeovervejelse}
\label{ParametreMetodeovervejelser}
%
Formålet med undersøgelsen er at fastlægge hvilke parametre, der vurderes at have betydning for \textit{Human-Robot Interaction}(HRI) for danske rejsende ved interaktion med en social robot i lufthavnen. Det er en eksplorativ undersøgelse, hvor det ønskes, at undersøge om der er ny viden at finde og ikke nødvendigvis kun at bekræfte allerede fundne resultater. Den indsamlede data forventes derfor at være kvalitativ, da der primært vil blive undersøgt hvilke ord danske rejsende bruger, når de skal beskrive interaktionen med en social robot. 

Følgende afsnit beskriver, hvilket metodeovervejelser, der er gjort i forbindelse med udarbejdelsen af testdesignet til feltundersøgelsen. Der er undersøgt forskellige metoder, for at kunne fastlægge en passende tilgang og opstilling af undersøgelsen. De undersøgte metoder er beskrevet i det følgende. 

\subsection{Contextual Inquiry}
\label{ParametreContextualInquiry}
%
Den kvalitative metode \textit{Contextual Inquiry} anvendes til at indsamle data i den kontekst, hvor brugeren agerer. Brugeren og intervieweren, der indsamler data, undersøger sammen de problemer, der er eller opstår under interaktionen. Metoden er baseret på, at der ikke defineres en liste med konkrete spørgsmål, men i stedet opstilles nogle klare overvejelser eller problematikker, som belyses gennem undersøgelsen, \parencite[s. 1]{PDF:UsingCIToLearn}. Intervieweren vil både observere og tale med testpersonen, mens testpersonen anvender det produkt, der undersøges. 

Ved at bruge \textit{Contextual Inquiry} kan der indsamles detaljeret information om et produkts brugergruppe samt hvorfor de enten bruger eller ikke bruger et produkt eller specifikke funktioner ved produktet, \parencite[s. 2]{PDF:UsingCIToLearn}. \blankline
%
I undersøgelsen, hvor det ønskes at udlede hvilke parametre, der har indflydelse på HRI, er produktet, der interageres med robotten. Da robotten endnu ikke er implementeret i lufthavnen er det ikke muligt at udføre \textit{Contextual Inquiry}, efter de foreskrevne retningslinjer. Brugeren har ikke kendskab til robotten i den aktuelle kontekst før selve undersøgelsen. Selvom metoden som helhed ikke anvendes, er der elementer og tilgange fra den der bruges.

Testlederens rolle i undersøgelsen tilstræbes, at være som intervieweren er ved \textit{Contextual Inquiry}, dog med visse forbehold. Ved \textit{Contextual Inquiry} taler intervieweren som beskrevet med testpersonen, mens interaktionen foregår. Dette vil ikke nødvendigvis være tilfældet i feltundersøgelsen, da det ønskes at robotten selv rekruterer og interagerer med testpersonerne uden forstyrrelser fra en testleder. 

Efterfølgende ønskes det at foretage et interview med testpersonen, hvor interaktion med robotten stadig kan foregå, hvorfor nogle af elementerne fra \textit{Contextual Inquiry} medtages. Testlederen og testpersonen skal være på samme niveau, hvorfor testlederen ikke kommer til at fungere som en decideret leder, som titlen ellers indikerer. I stedet vil testlederen føre en samtale med testpersonen efter interaktionen og undervejs spørge ind til relevante interaktioner eller problemstillinger, samt uddybende kommenterer, hvis der er behov for det. Der opstilles ikke konkrette spørgsmål til selve interaktionen, men i stedet opstilles der samtaleemner, som testleder og testperson efterfølgende kan snakke om. Disse samtaleemner vil dække over overvejelser og problemstillinger i forhold til robotten, hvilket er inspireret af tilgangen der er ved \textit{Contextual Inquiry}.
%
\subsection{Laddering}
\label{ParametreLaddering}
%
Efter testpersonerne har interagerede med robotten, vil de blive bedt om at deltage i et interview. Under interviewet fokuseres der på at lade testpersonerne tale så meget som muligt, for at undgå at dreje samtalen væk fra de ting, som har optaget testpersonen. For at få uddybende information fra testpersonerne omkring de betydende parametre ved en social robot, ønskes det at stille spørgsmålene på samme måde som ved brug af \textit{UX Laddering}, der er en tilpasset \textit{Laddering} metode designet til undersøgelser af brugeroplevelse, \parencite[ss. 3-4]{PDF:LadderingTheUserExperience}. Metodens navn henviser til en stige eller trappe, der kan bruges til at bevæge sig opad og få en bedre forståelse af brugerens holdning til et produkt eller fænomen. Ved brug af \textit{UX Laddering} stilles testpersonen eksempelvis et spørgsmål om hvordan de oplevelede interaktionen med robotten. Når testpersonen svarer, bliver deres svar overvejet, hvorefter der stilles uddybende spørgsmål ind til det. Sådan fortsættes det op ad trappen, indtil eventuelle værdier omkring deres holdning til den sociale robot er blevet fastlagt. Når \textit{UX Laddering} skal bruges, er kontekst vigtig, \parencite[s. 3]{PDF:LadderingTheUserExperience}, hvorfor det er positivt, at testpersonerne er blevet præsenteret for robotten i en kontekst, som den reelt set kunne indgå i. 

Når en testperson starter med at besvare spørgsmål, er det sandsynligt at det foregår ved at opremse funktioner i stedet for specifikke egenskaber, \parencite[s. 3]{PDF:LadderingTheUserExperience}. I det henseende er det vigtigt at forsøge at få testpersonen til at træde et par skridt ned ad trappen igen, så spørgsmålene og svarene i denne undersøgelse starter ved robottens specifikke egenskaber. Det kan opnåes ved at stille spørgsmålet: \textit{Hvad skyldes det?}, frem for at spørge hvorfor testpersonen mener noget. Ydermere er det vigtigt, at interviewet ikke tager for lang tid, \parencite[s. 4]{PDF:LadderingTheUserExperience}. Testlederne skal naturligvis spørge videre ind, så testpersonen kommer til at tale om de bagvedliggende værdier, men kun hvis det virker naturligt i samtalen. Det er ikke alle oplevelser, der trigger virkelige værdier, hvorfor det kan være nødvendigt at stoppe med at stille spørgsmål, hvis det ikke længere virker naturligt. \blankline
%
\textit{UX Laddering} anvendes i feltundersøgelsen til at forme interviewet og forstå værdierne bag testpersonens udtalelser om robotten, hvorfra de vigtige parametre forventes at kunne udledes.

\subsection{Overvejelser til interviewer}
\label{ParametreOvervejelserTestleder}
%
Det kan være relevant at fortælle testpersonerne om den beskrevne \textit{Laddering} interview-metode, så de ikke bliver irriteret over de konstant uddybende og opklarende spørgsmål. Derudover kan det være relevant at fortælle testpersonerne, at det ikke er projektgruppen, der har designet \textit{Double}-robotten, hvorfor de frit kan fortælle hvad der falder dem ind, når de interagerer med robotten. Hvis testpersonerne begynder at snakke om generelle oplevelser eller hvordan andre vil forstå robotten, er det vigtigt at spørge dem ind til deres egne oplevelser frem for andres. \blankline
%
Ved afviklingen af interviewet er det vigtigt, at testlederen fremstår venlig og tilstedeværende, i stedet for at følge et manuskript. Derudover kan det være en god idé at "spille dum", for virkelig at få testpersonerne til at beskrive hvad de mener, frem for selv at tolke på det de siger. Det gør sig også gældende, selvom det de siger kan virke åbenlyst. Ydermere skal testlederen tænke over kropssprog og mimik, da det er nemt for testpersonen at fornemme, hvornår de har sagt noget, som testlederen havde håbet på og omvendt. Alle forudindtagelser og erfaringer fra tidligere tests skal derfor glemmes, så der på den måde kun fokuseres på testpersonens svar. 
