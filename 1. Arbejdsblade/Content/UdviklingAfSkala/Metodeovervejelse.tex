\section{Metodeovervejelse}
\label{ParametreMetodeovervejelser}
%OBS: Det er her vi (Emil?) skal skrive de overvejelser ind, der er gjort i forbindelse med metodevalg, altså at det ikke er én specifik metode vi anvender, men mere en kombination af flere metoder. Tag eventuelt noget af alt det teori der står om laddering i Interview. Det er også her der skal skrive om affinity diagrammer.\blankline
%
Følgende afsnit beskriver, hvilket metodeovervejelser der er gjort i forbindelse med udarbejdningen af testdesignet til feltundersøgelsen. Der er undersøgt forskellige metoder, for at kunne fastlægge en passende tilgang og opstilling af undersøgelsen. De undersøgte metoder er beskrevet i det følgende. \blankline
%
Formålet med undersøgelsen er at fastlægge hvilke parametre, der vurderes at have betydning for Human Robot Interaction (HRI) med en service robot i lufthavnen. Det er en eksplorativ undersøgelse, hvor det ønskes at undersøge om der er ny viden at finde og ikke nødvendigvis kun bekræfte allerede fundne resultater. 

Den indsamlede data forventes at være hovedsaligt kvalitativ, da der primært vil bliver undersøgt hvilke ord danske rejsende bruger, når de skal beskrive interaktionen med en social robot. 
%
%\subsection{Observations studier}
%\label{ParametreObservationsStudier}

%\subsection{Etnografiske studier}
%\label{ParametreEtnografiskeStudier}

\subsection{Contextual Inquiry}
\label{ParametreContextualInquiry}
%
Den kvalitative metode Contextual Inquiry anvendes til at indsamle data i den kontekst, hvor brugeren agerer. Brugeren og personen, der indsamler data, undersøger sammen de problemer, der er eller opstår under interaktionen. Metoden er baseret på, at der ikke defineres en liste med konkrete spørgsmål, men i stedet opstilles nogle klare overvejelser eller problematikker, som belyses gennem undersøgelsen \textcite[s. 1]{PDF:UsingCIToLearn}. Dataindsamleren, i dette tilfælde testlederen, vil både observere og tale med testpersonen, mens vedkommende anvender det produkt der undersøges. \blankline
%
Ved at bruge Contextual Inquiry kan der indsamles detaljeret information om et produkts brugergruppe samt hvorfor de enter bruger eller ikke bruger et produkt eller specifikke funktioner ved produktet \textcite[s. 2]{PDF:UsingCIToLearn}. \blankline
%
I undersøgelsen, hvor det ønskes at bestemme hvilke parametre, der har indflydelse på HRI, er produktet der interageres med robotten. Da robotten endnu ikke er implementeret i lufthavnen, der undersøges, er det ikke muligt at udføre Contextual Inquiry, som metoden foreskriver det. Brugeren ikke har kendskab til robotten i den givne kontekst før selve undersøgelsen. Selvom metoden som helhed ikke anvendes, er der elementer og tilgange fra den der bruges. \blankline
%
Testlederens rolle i undersøgelsen ønskes at være som dataindsamlerens rolle er ved Contextual Inquiry, dog med visse forbehold. Ved Contextual Inquiry snakker testlederen som beskrevet med testpersonen, mens interaktionen foregår. Dette vil ikke nødvendigvis være tilfældet her, da det ønskes at robotten selv rekrutere og interagere med testpersonerne uden forstyrrelser fra en forklarende testleder. Efterfølgende ønskes det at foretage et interview med testpersonen. Her kan interaktion med robotten stadig foregå, hvorfor nogle af elementerne fra Contextual Inquiry metoden medtages. Testlederen og testpersonen skal være på samme niveau, så testlederen kommer ikke til at fungere som en decideret leder, som titlen ellers indikerer. I stedet vil testlederen samtale med testpersonen efter interaktionen og undervejs spørge ind til relevante interaktioner eller problemstillinger, hvis der er behov for dette. Der opstilles ikke konkrette spørgsmål til selve interaktionen, men i stedet opstilles der samtaleemner, som testleder og testperson efterfølgende kan snakke om. Disse samtaleemner vil dække over overvejelser og problemstillinger i forhold til robotten, hvilket er inspireret af tilgangen der er ved Contextual Inquiry. \blankline
%
\subsection{Laddering}
\label{ParametreLaddering}
%
Når testpersonerne er blevet præsenteret for robotten udføres et interview. Under interviewet fokuseres der på at lade testpersonerne tale så meget som muligt, for at undgå at dreje samtalen væk fra de ting, som har optaget vedkommende. For at få uddybende information fra testpersonerne omkring de betydende parametre i en social robot, ønskes det at stille spørgsmålene på samme måde som ved brug af en UX laddering, der er en tilpasset laddering metode designet til research af brugeroplevelsen, \parencite[ss. 3-4]{PDF:LadderingTheUserExperience}. Ved brug af UX laddering metoden stilles testpersonen et spørgsmål om hvordan de eksempelvis oplevelede interaktionen med robotten. Når testpersonen svarer, bliver deres svar overvejet, hvorefter der stilles uddybende spørgsmål ind til dette. Sådan fortsættes det op ad stigen, indtil eventuelle værdier bag deres holdning til den sociale robot er blevet udvundet. Når UX laddering skal bruges, er kontekst vigtig, \parencite[s. 3]{PDF:LadderingTheUserExperience}, hvorfor det er positivt, at testpersonerne er blevet præsenteret for robotten i en kontekst den reelt set kunne indgå i. 

Når en testperson starter med at besvare spørgsmål, er det sandsynligt at vedkommende svarer ved at opstille funktionelle konsekvenser, \parencite[s. 3]{PDF:LadderingTheUserExperience}. Her er det vigtigt at forsøge at få testpersonen til at træde et par skridt ned ad trappen igen, så spørgsmålene og svarene starter ved konkrete egenskaber. Dette kan gøres ved at stille spørgsmålene "Hvad skyldes det?", frem for at spørge hvorfor testpersonen mener noget. Ydermere er det vigtigt, at interviewet ikke tager for lang tid, \parencite[s. 4]{PDF:LadderingTheUserExperience}. Testlederne skal naturligvis spørge videre ind, så de bagvedlæggende værdier udvindet, men kun hvis dette virker naturligt i samtalen. Det er ikke alle oplevelser, der trigger virkelige værdier, hvorfor det kan være nødvendigt at stoppe med at stille spørgsmål, hvis det ikke længere virker naturligt. \blankline
%
UX laddering bruges altså i denne test til at forme interviewet og forstå værdierne bag testpersonens udtalelser om robotten, hvorfra de vigtige parametre forventes at kunne udvindes.

\subsection{Overvejelser til interviewer}
\label{ParametreOvervejelserTestleder}
%
Det kan være relevant at fortælle testpersonerne om den beskrevne laddering interview-metode, så de ikke bliver irriteret over de konstant uddybende spørgsmål. Derudover kan det være relevant at fortælle testpersonerne, at det ikke er projektgruppen, der har designet \textit{Double}-robotten, hvorfor de frit kan fortælle hvad der falder dem ind, når de interagerer med robotten. Hvis testpersonerne begynder at snakke om generelle oplevelser eller hvordan andre vil forstå robotten, er det vigtigt at spørge dem ind til deres egne oplevelser frem for andres. \blankline
%
Ved udførelsen af interviewet er det vigtigt, at testlederen fremstår venlig og tilstedeværende, i stedet for at følge et manuskript. Derudover kan det være en god idé at "spille dum", for at få testpersonerne til virkelig at beskrive hvad de mener, frem for selv at tolke på det de siger. Dette gør sig også gældende, selvom det de siger kan virke åbenlyst. Ydermere skal testlederen tænke over kropssprog og mimik, da det er nemt for testpersonen at se, hvornår de har sagt noget, som testlederen havde håbet på. Alle forudindtagelser og erfaringer fra tidligere tests skal derfor glemmes, så der på den måde kun fokuseres på testpersonens svar. 
