\section{Testpersoner}
\label{ParametreTestpersoner}
%
Testpersonerne, de rejsende, bliver rekrutteret direkte i Aalborg Lufthavn i området efter sikkerhedskontrollen. Det er som udgangspunkt robotten, der står får rekrutteringen ved at henvende sig til de rejsende med spørgsmålet: \textit{Jeg kommer fra Aalborg Universitet. Må jeg hjælpe dig med at finde rundt i Aalborg Lufthavn?}, har den rejsende lyst til at deltage vil robotten føre dem igennem nogle foruddefineret brugsscenarier. Fordelen ved at få robotten til at rekruttere testpersoner er, at testpersonerne får et upåvirket førstehåndsindtryk af robotten, som det vil være tilfældet første gang de oplever robotten i lufthavnen. \blankline 
%
Det tilstræbes, at foretage underøgelsen på både kvinder og mænd, gerne med forskellige aldre og rejseformål; forretningsrejse eller ferierejse, hvis muligt vil det ydermere bestræbes at inddrage førstegangs rejsende. Derudover er det et krav, at den rejsende er dansktalende for at undgå, at vigtige pointer går tabt i oversættelsen, når data efterfølgende skal behandles. 

Antallet af testpersoner er ikke forudbestemt, da undersøgelsen i stedet afsluttes, når der opnåes mætning, hvorved der ikke indsamles ny viden. Dette vurderes af de tilstede værende gruppemedlemmer. Det forventes dog, at udføre undersøgelsen på minimum fem testpersoner.     