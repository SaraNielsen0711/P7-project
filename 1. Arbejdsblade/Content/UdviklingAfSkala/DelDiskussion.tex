\chapter{Diskussion}
\label{ParametreDiskussion}
%
I følgende afsnit vil feltundersøgelsen i AAL diskuteres. Da de forskellige valg løbende er blevet diskuteret, vil følgende indeholde diskussion af observationer gjort under feltundersøgelsen, resultater fra affinity, hvordan udvælgelsen af skalaer har foregået og hvad der kan gøres bedre til næste test.\blankline
%
\subsection{Observationer og resultater fra feltundersøgelsen}
\label{DiskussionerObservationer}
%
Feltundersøgelsen i AAL var succesfuld, da flere testpersoner havde lyst til at interagere med robotten og efterfølgende havde tid til at snakke om den med testlederen. Som forventet var der dog også mange rejsende, der ikke havde lyst til at interagere med robotten. Nogle rejsende så slet ikke robotten, hvilket kunne skyldes at den både var lav og ikke altid i bevægelse, men også at den stod med ryggen eller siden til og derfor ikke blev set som noget der kunne interageres med. Mange rejsende både så, snakkede om og grinede af robotten, men havde enten ikke lyst til at interagere med robotten eller trykkede nej til at få hjælp med at finde rundt i AAL. Dette kan både skyldes at de bare ikke have lyst til at interagere med robotten, men også at de ikke vidste at robotten henvendte sig til dem og at de havde muligheden for at interagere med den. Det var svært for robotstyreren altid at komme tæt på de rejsende, fordi udsynet til robotten ikke altid var lige godt og det ikke var ønsket at køre ind i nogen. Det kan have haft en betydning for om de rejsende følte den henvendte sig til dem eller ej. Ved udvikling af social intelligens er det derfor nødvendigt at brugerteste på hvordan en robot skal henvende sig til mennesker, for at de forstår denne henvendelse.

Robotten kørte rundt i lufthavnen med en skærm hvorpå spørgsmålet "Kan jeg hjælpe dig med at finde rundt i Aalborg Lufthavn?" blev præsenteret. Nogle af de rejsende trykker "Nej" på skærmen, hvilket kan skyldes at de ikke havde brug for hjælp til at finde rundt i lufthavnen, men betyder ikke nødvendigvis at de ikke i en anden situation ville have lyst til at interagere med og få hjælp af robotten. Spørgsmålet kan derfor være lidt misvisende og afskære nogle af de rejsende, som ellers ville have relevante og vigtige pointer i forhold til en social robot. Det blev diskuteret, hvorvidt det ville være en god idé at stoppe de rejsende, som trykker nej, til næste test, for at få deres begrundelse til dette. Dog blev det efterfølgende besluttet ikke at forsøge at indhente og interviewe de mennesker, der trykker nej, da det er svært at spørge ind til uden at lyde bebrejdende. Ydermere kan de rejsende have trykket nej, fordi de ikke har tid til at interagere, hvorfor det vil være ubelejligt hvis projektgruppen stopper dem for at stille spørgsmål.\blankline
%
På trods af at der er flere, der ikke vil intereagere med robotten, er der stadig rejsende, som gerne vil. Projektgruppen blander sig ikke i hvem der bliver rekrutteret som testpersoner på andre måder end ved at køre en robot hen til dem, hvilket forsøges at gøre på alle, der ikke stormer af sted. Feltundersøgelsen er på den måde meget økologisk, da det kun er de rejsende, som vil interagere med en social robot i en lufthavn, der deltager i undersøgelsen. Dog er projektgruppen ikke helt usynlig i lufthavnen, da observatørene sidder rundt omkring med en blok papir og flere af de rejsende smiler hen til testleder, robotstyrer eller observatører efter at have set robotten. Ydermere fortæller robotten, at den kommer fra Aalborg Universitet og at der er en undersøgelse i gang.\blankline

Selvom feltundersøgelsen er meget økologisk kan projektgruppen kun påvirke de rejsende med de ting, som robotten kan. Det kan være svært at påpege vigtige parametre, hvis helheden fungerer godt, hvorfor der kan være parametre, som ikke kommer frem, fordi robotten opfører sig som forventet.

Et parameter der bliver nævnt flere gange er dog, at robottens hastighed har betydning. Der er flere af testpersonerne, der synes robotten går alt for langtsomt. Nogen gange så langsomt, at det er besværligt at gå bag ved den eller følge dens tempo, når testpersonerne går ved siden af den. Denne hastighed kan ikke ændres på \textit{Double}-robotten, men ellers ville det være relevant at øge makshastigheden en anelse, så fremtidige testpersoner ikke er ved at falde over robotten.\blankline
%
Et andet parameter der bliver nævnt flere gang er, hvor godt touchskærmen på robotten virker. Det ses både under observationerne og når affinity diagrammet bygges. Under interaktionen med robotten var det tydeligt, at flere rejsende blev irriteret enten over at skærmen ikke reagerede på deres berøring eller at robotten ikke ville acceptere et nej, når de prøvede at trykke på "Nej". Dette havde betydning for at flere af de rejsende stoppede interaktionen og gik videre, selvom de gerne ville have hjælp af robotten. Grunden til at touchskærmen ikke altid reagerede på berøring er grundet \textit{wireframet} designet i \textit{Marvel}. For det første kræver dette setup en blid berøring af skærmen, hvilket ikke er det første de rejsende tænker, når teknologien ikke virker. For det andet vises \textit{wireframet} gennem \textit{Double}-appen, hvorfor komminukationen godt kan være en anelse langsommere end mennesket nu til dags er vant til. Til den kommende test kan det derfor være en fordel at overveje, hvordan de rejsende vil få nemmere ved at interagere med touchskærmen. Dette kan eventuelt gøres ved at implementere et stykke tekst, der fortæller at der skal trykkes blidt på skærmen.\blankline
%
Generelt viste feltundersøgelsen, at en social robot på denne måde kan være en god idé, både i lufthavne såvel som andre steder. I lufthavne er robotten god som vejviser, specielt til mennesker, der ikke forstår det talte sprog eller børn, der skal følges til den rigtige gate. Specielt børn er meget begejstret overfor robotten, hvorfor det kan være logisk at have en social robot, der kan hjælpe børn. Udover lufthavnssituationen blev der eksempelvis foreslået at robotten kan bruges på hospitaler, storcentre eller på museer som vejviser, underholdning eller guide. Ud fra det opbyggede affinity diagram ses det, at kendskab til teknologi kan spille en rolle i hvor glad man er for at henvende sig eller interagere med en robot som \textit{Double}. Flere testpersoner kunne se robotten give mening i forhold til deres profession, nogle var interesseret i den fordi det var en kørende iPad og nogle viste interesse, fordi de selv arbejdede med teknologi eller robotter til dagligt. Generelt tyder resultaterne på, at mennesker med et godt forhold til teknologi har nemmere ved at acceptere en social robot på denne måde og mere lyst til at interagere med den. 

\subsection{Skalaer}
\label{DiskussionSkalaer}
Da skala spørgsmålene i første omgang er dannet, er dette gjort på baggrund af hvad testpersonerne selv har sagt og udtrykt som værende vigtige parametre. Efterfølgende er nogle af skala spørgsmålene blevet slået sammen, fordi disse dækkede over det samme. Når de tilhørende skalaer er blevet designet, er der taget udgangspunkt i det enkelte skala spørgsmål frem for at kigge på helheden. Dog er der til alle spørgsmålet lavet en VAS, hvor det kun er ankerpunkter og labels der varierer. Dette er gjort, for at udvikle de bedste skalaer til de enkelte parametre, så det bedst mulige svar på spørgsmålet kan fås. Modsat vil fordelen ved at gøre skalaerne mere ens være, at de er mere konsistente, når testpersonerne skal se og svare på dem. Som skalaerne er bygget op får de spørgsmålene stillet på forskellige måder med varierende endepunkter, hvilken kan skabe forvirre, misforståelser og i værste fald påvirke testpersonernes markering på skalaerne. 

Skalaerne er blevet sat op med lukkede endepunkter. Dette er gjort fordi testpersonerne ikke bliver præsenteret for robotten på forskellige måder eller skalaerne flere gang. Det er derfor lidt usandsynligt at testpersonerne vil have brug for at svare mere ekstremt end de gjorde før, netop fordi spørgsmålene kun vil blive besvaret én gang. Derudover vil lukkede endepunkter skabe nogle klare rammer for hvor man kan svare, og det forventes derfor at data bliver mere normalfordelt. Ulempen ved at bruge lukkede endepunkter er, at testpersonerne kan blive kantforskrækket og svare længere væk fra endepunktet end de egentlig mener. 

For at mindste misforståelser af skalaerne vælges det at præsentere de skalaer der minder mest om hinanden sammen i den efterfølgende test.