\chapter{Diskussion}
\label{ParametreDiskussion}
%
I følgende afsnit vil feltundersøgelsen i AAL diskuteres. Da de forskellige valg løbende er blevet diskuteret, vil følgende indeholde diskussion af observationer gjort under feltundersøgelsen, resultater fra affinity, hvordan udvælgelsen af skalaer har foregået og hvad der kan gøres bedre til næste test.
%
\subsection{Observationer og resultater fra feltundersøgelsen}
\label{DiskussionerObservationer}
%
Feltundersøgelsen i AAL var succesfuld, da flere danske rejsende havde lyst til at interagere med robotten og efterfølgende havde tid til at snakke om den med testlederen. Som forventet var der dog også mange rejsende, der ikke havde lyst til at interagere med robotten. Nogle rejsende så slet ikke robotten, hvilket kunne skyldes at den både var lav og ikke altid i bevægelse, men også at den stod med ryggen eller siden til og derfor ikke blev set som noget der kunne interageres med. Mange rejsende både så, snakkede om og grinede af robotten, men havde enten ikke lyst til at interagere med robotten eller trykkede nej til at få hjælp med at finde rundt i AAL. Dette kan både skyldes, at de bare ikke havde lyst til at interagere med robotten, men også at de ikke vidste at robotten henvendte sig til dem og at de havde mulighed for at interagere med den. Det var svært for robotstyreren altid at komme tæt på de rejsende, fordi udsynet til robotten ikke altid var lige godt og det ikke var ønsket at køre ind i nogen, hvilket kan have haft en betydning for om de rejsende følte den henvendte sig til dem eller ej. Ved udvikling af social intelligens er det derfor nødvendigt at brugerteste på hvordan en robot skal henvende sig til mennesker, for at de forstår denne henvendelse og reagerer på den.

Robotten kørte rundt i lufthavnen med en skærm hvorpå spørgsmålet: \textit{Kan jeg hjælpe dig med at finde rundt i Aalborg Lufthavn?}, blev præsenteret. Nogle af de rejsende trykkede \textit{Nej} på skærmen, hvilket kan skyldes at de ikke havde brug for hjælp til at finde rundt i lufthavnen, men betyder ikke nødvendigvis at de ikke i en anden situation ville have lyst til at interagere med og få hjælp af robotten. Spørgsmålet kan derfor være lidt misvisende og afskære nogle af de rejsende, som ellers ville have relevante og vigtige pointer i forhold til en social robot. Det blev diskuteret, hvorvidt det ville være en god idé at stoppe de rejsende, som trykker \textit{Nej}, til næste test, for at få deres begrundelse for hvorfor de trykkede, som de gjorde. Dog blev det efterfølgende besluttet ikke at forsøge at indhente og interviewe de rejsende, der trykkede \textit{Nej}, da det er svært at spørge ind til uden at lyde bebrejdende. Ydermere kan de rejsende have trykket \textit{Nej}, fordi de ikke har tid til at interagere, hvorfor det vil være ubelejligt hvis projektgruppen stopper dem for at stille spørgsmål.\blankline
%
På trods af at der er flere, der ikke vil interagere med robotten, er der stadig rejsende, som gerne vil. Projektgruppen havde ikke indflydelse på hvem, der blev rekrutteret udover robotstyren, som kører robotten hen til de rejsende, hvilket forsøges at gøre på alle, der ikke stormer af sted eller virker optaget. Feltundersøgelsen er på den måde økologisk, da det kun er de rejsende, som vil interagere med en social robot i en lufthavn, der deltager i undersøgelsen. Dog er projektgruppen ikke helt usynlig i lufthavnen, da observatørerne sidder rundt omkring med en blok papir og flere af de rejsende smiler hen til testleder, robotstyrer eller observatører efter at have set robotten. Ydermere fortæller robotten, at den kommer fra Aalborg Universitet og at der er en undersøgelse i gang.\blankline
%
Selvom feltundersøgelsen er økologisk kan projektgruppen kun påvirke de rejsende med de ting, som robotten kan. Det kan være svært for testpersoner at påpege vigtige parametre, hvis helheden fungerer godt, hvorfor der kan være parametre, som ikke kommer frem, fordi robotten opfører sig som forventet.

Et parameter, der bliver nævnt flere gange er dog, at robottens hastighed har betydning. Der er flere af testpersonerne, der synes robotten bevæger sig alt for langsomt. Nogen gange så langsomt, at det er besværligt at gå bag ved den eller følge dens tempo, når testpersonerne går ved siden af den. Denne hastighed kan ikke ændres på \textit{Double}-robotten, men ellers ville det være relevant at øge makshastigheden, så fremtidige brugere ikke er ved at falde over robotten eller bliver frustreret over at robotten er for langsom.\blankline
%
Et andet parameter, der bliver nævnt flere gang er, hvor godt touchskærmen på robotten virker. Det fremgår både gennem observationerne og når affinity diagrammet udvikles. Under interaktionen med robotten var det tydeligt, at flere rejsende blev irriteret enten over at skærmen ikke reagerede på deres berøring eller at robotten ikke ville acceptere et nej, når de prøvede at trykke på \textit{Nej}. Dette havde betydning for, at flere af de rejsende stoppede interaktionen og gik videre, selvom de formentlig gerne ville have hjælp af robotten. Grunden til at touchskærmen ikke altid reagerede på berøring skyldes \textit{wireframet} designet i \textit{Marvel}. For det første kræver den opsætning en blid berøring på skærmen, hvilket ikke var det første de rejsende gjorde, da teknologien ikke virkede. For det andet eksekveres \textit{wireframet} gennem \textit{Double}s applikation, hvorfor kommunikationen kan være langsommere end hvad de rejsende er vant til. Til den kommende test kan det derfor være en fordel at overveje, hvordan de rejsende vil få nemmere ved at interagere med touchskærmen. Dette kan eventuelt gøres ved at implementere et stykke tekst, der fortæller at der skal trykkes blidt på skærmen.\blankline
%
Baseret på feltundersøgelsen, tyder det kraftigt på, at en social robot kan være en god idé, både i lufthavne såvel som på andre offentlige lokationer. I lufthavne er robotten god som vejviser, specielt til rejsende, der ikke forstår det talte sprog eller til børn, der rejser alene og skal følges til den rigtige gate. Specielt børn er meget begejstret for robotten, hvorfor det kan være eftertragtet at have en social robot, der kan hjælpe børn. Udover lufthavnssituationen blev der eksempelvis foreslået, at robotten kan bruges på hospitaler, i storcentre eller på museer som vejviser, underholdning eller guide. Ud fra det opbyggede affinity diagram fremgår det, at kendskab til teknologi kan spille en rolle i hvorvidt den rejsende har lyst til at henvende sig eller interagere med en robot som \textit{Double}. Flere testpersoner gav udtryk for at robotten gav mening i forhold til deres profession, nogle var interesseret i robotten fordi det var en kørende iPad og nogle viste interesse, fordi de selv arbejdede med teknologi eller robotter til dagligt. Generelt tyder resultaterne på, at brugere med et positivt forhold til teknologi har nemmere ved at acceptere en social robot og mere lyst til at interagere med den. 

\subsection{Skalaer}
\label{DiskussionSkalaer}
%
Da skala spørgsmålene i første omgang blev dannet, er det hovedsageligt gjort på baggrund af hvad testpersonerne selv har sagt og udtrykt som værende vigtige parametre. Efterfølgende er nogle af skala spørgsmålene blevet slået sammen, fordi de dækkede over det samme. Da de tilhørende skalaer blev designet, blev der taget udgangspunkt i det enkelte skala spørgsmål frem for at betragte helheden. Dog er der til alle spørgsmål designet en VAS, hvor det kun er ankerpunkter og labels, der varierer. Dette er gjort, for at udvikle den bedste skala til hver af de enkelte parametre. Modsat vil fordelen ved at gøre skalaerne mere konsistente være, at når testpersonerne præsenteres for dem og skal angive deres respons på dem, så skal de ikke forholde sig til flere forskellige typer af skalaer. Som skalaerne er bygget op stilles skala spørgsmålene på forskellige måder med varierende endepunkter, hvilket kan skabe forvirring, misforståelser og i værste fald påvirke testpersonernes markering på skalaerne. 

Skalaerne er blevet sat op med lukkede endepunkter. Dette er gjort fordi testpersonerne hverken bliver præsenteret for robotten på forskellige måder under interaktionen eller bliver præsenteret for den samme skala flere gang. Det vurderes derfor at være usandsynligt, at testpersonerne vil få brug for at kunne svare mere ekstremt end den tidligere besvarelse, netop fordi skala spørgsmålene kun vil blive præsenteret én gang. Derudover vil lukkede endepunkter skabe nogle klare rammer for, hvor testpersonerne kan svare, og det forventes derfor at data bliver mere normalfordelt end hvis skalaerne havde åbne endepunkter. Ulempen ved at bruge lukkede endepunkter er, at testpersonerne kan blive kantforskrækket og svare længere væk fra endepunktet end de egentlig mener. 

For at mindske misforståelser af skalaerne vælges det at præsentere de skalaer, der minder mest om hinanden sammen i den efterfølgende test.