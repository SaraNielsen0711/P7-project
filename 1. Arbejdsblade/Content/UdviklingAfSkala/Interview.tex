\section{Samtaleemner}
\label{ParametreSamtaleemner}
%
Mens undersøgelsen står på er det vigtigt, at holde samtalen i gang for at indsamle data, der gengiver testpersonernes oplevelse af interaktionen med robotten. For at sikre at testlederen har relevante emner at spørge ind til er det valgt, at opstille samtaleemner for undersøgelsen. Samtaleemnerne er inddelt i tre grupper: Small-talk, løbende og afsluttende samtaleemner. Alle samtaleemner er vejledende, hvorfor de ikke nødvendigvis bliver formuleret ordret af testlederen.

\subsection{Small-talk emner} 
\label{ParametreISmallTalk}
%
Small-talk emnerne har til formål at få samtalen i gang mellem testperson og testleder. Samtalen fungerer primært som en ice-breaker, hvor de to parter har mulighed for at small-talke og hvor det er muligt, at få testpersonen til at involvere sig og stille spørgsmål til den interaktion, der lige er foregået. Derudover er formålet med small-talk at skabe en god stemning før den reelle samtale, samtalen om robotten, går i gang. Stall-talking behøver ikke støtte sig op af specifikke spørgsmål, men kan handle om testpersonens rejse, erfaring med at rejse eller lignende. Small-talk emnerne vil ikke være ens for de forskellige testpersoner, men vil afhænge af hvad der falder naturligt i testsituationen.  
%
\subsection{Samtaleemner} 
\label{ParametreSamtaleemner}
%
Samtaleemner henvender sig primært til emner, der forventes at være relevante for testpersonernes oplevelse af interaktionen med robotten. Disse samtaleemner udgør interviewet af testpersonen, som skal afsløre hvilke parametre, der er vigtige for testpersonen i en social robot. Samtaleemnerne er som følger:\blankline
%
\begin{itemize}
\item Førstehåndsindtryk af robotten - fra rekrutteringen
\item Måden hvorpå robotten henvender sig
\item Hvad testpersonen synes om robotten
\item Hvad testpersonerne tror andre rejsende tænker om interaktionen 
\item Robottens relevans
\item Robottens pålidelighed
\item Normal oplevelse i en lufthavn uden hjælp fra en robot 
\end{itemize}
%
Da der gøres brug af elementer fra laddering, jævnfør \fullref{ParametreMetodeovervejelser} vil de listede samtaleemner udgøre grundlag for samtalen, men ikke nødvendigvis være det eneste der bliver snakket om. 
%

\subsection{Afrunding og debriefing} 
\label{ParametreAfrundingDebriefing}
%
De afsluttende samtaleemner vil blive anvendt som debriefing af testpersonen og afrundning på undersøgelsen. Ydermere vil opklarende spørgsmål omkring parametre, som testpersonen ikke selv har nævnt blive stillet, hvis testlederen føler behovet opstår.  Samtaleemnerne er som følger: \blankline
%
\begin{itemize}
\item Hvad synes du om..
	\begin{itemize}
		\item Robottens hastighed?
		\item Robottens højde?
		\item Robottens afstand til dig?
		\item Robottens generelle bevægelse?
		\item Robottens udseende?
		\item Den retning robotten henvendte sig til dig fra?
	\end{itemize}
\item Hvor gammel er du?
\item Hvor ofte flyver du?
\item Spørgsmål og/eller kommentarer til undersøgelsen 
\item Spørgsmål og/eller kommentarer til vores projekt
\item Afrunding og ønsk testpersonerne god rejse
\end{itemize}

