\section{Problemformulering}
\label{Problemformulering}
%
Ulempen ved de førnævnte undersøgelser er dels, at HRI foregår over en kort periode, hvorfor interaktionen med robotten er kortvarig og dels, at undersøgelserne som udgangspunkt foretages i laboratorier. I ingen af tilfældene er det muligt for testpersonerne at danne et decideret tilhørsforhold til robotten, da forholdet stort set kun bygger på førstehåndsindtrykket. Derudover præsenteres robotterne i specifikke kontekster, som ikke nødvendigvis er den kontekst robotten er designet til at indgå i. Som tidligere diskuteret er der både store kulturelle-, alders- og kønsforskelle, hvorfor de parametre testpersonerne i de førnævnte undersøgelser vurderer robotten ud fra, ikke nødvendigvis er de parametre, der har betydning for danske brugere. På baggrund af dette kan følgende problemstillinger opstilles:\blankline
%
\begin{quotation}
	\noindent
	\textit{Ud fra hvilke parametre beskriver danske rejsende interaktion med en social robot i en dansk lufthavn?\blankline
		%
		Hvordan kan de fundne parametre gengives i skalaer, som efterfølgende kan bruges til at evaluere HRI?}\blankline
\end{quotation}
%
For at besvare den første problemstilling er det nødvendigt, at foretage en form for interview med danske rejsende i en dansk lufthavn, hvor de skal gengive deres subjektive oplevelse af HRI. Baseret på de rejsendes respons skal der foretages en kvalitativ analyse for at kategorisere den givne respons og dermed udlede, hvilke parametre, der har indflydelse på de danske rejsendes interaktion med en social robot. Disse parametre kan derefter sammenholdes med tidligere foretaget undersøgelser, for at afgøre om interaktionen mellem danske rejsende og en social robot er påvirket af samme parametre, som gennemgået i \fullref{InteraktionSocialeRobotter}. 

For at besvare den anden problemstilling vil der blive udarbejdet skalaer, hvorpå de fundne parametre kan evalueres. Efterfølgende forsøges det at anvende de udviklede skalaer til at evaluere interaktionen med robotten. Dette gøres ved at definere nogle brugsscenarier samt bevægelsesmønstre for robotten, som de rejsende præsenteres for. Baseret på de rejsendes evalueringer bør det være muligt, at undersøge hvilke parametre korrelerer med hinanden. 


