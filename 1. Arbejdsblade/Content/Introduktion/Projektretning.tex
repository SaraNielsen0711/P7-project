\chapter{Projektretning}
\label{ProjektRetning}
%
Følgende kapitel er en redegørelse over projektets retning. Indholdet er blevet sendt til vejleder Dort Hammershøi, den 5. oktober 2017, samt Karl Damkjær Hansen, som står bag projektforslaget, den 10. oktober 2017. Da dette er første udkast omkring projekt retning, skal der tages forbehold for ændringer.
%
\section{Formål}
\label{ProjektRetningFormaal}
%
Projektets formål er at undersøge den subjektive oplevelse af en interaktion mellem mennesker og sociale robotter i forskellige kontekster, med henblik på implementering af primitiv social intelligens, som skal gøre robotten i stand til at interagere med mennesker på en social acceptabel måde.\blankline
%
Projektforslaget startede med et besøg i Karl Damkjær Hansens robotlaboratorium. Karl udvikler sociale robotter som kan køre hen til mennesker og indgå i forskellige interaktioner med dem. Karls primære arbejdsområde er det tekniske aspekt i at få robotterne til at køre som de skal. Han har indtil videre arbejdet med nogle forskellige modeller, blandt andet en robotstøvsuger og en segway lignende platform. For at give robotterne en mere menneskelig adfærd, er der et ønske om at gøre dem mere dynamiske i deres bevægelser og ydermere gøre dem i stand til at bevæge sig i alle retninger. Begge dele gør dem i stand til at bevæge sig tilsvarende den måde mennesker underbevidst bevæger sig på, når de eksempelvis skifter vægten fra det ene ben til det andet eller gør en samtalekreds større, for at gøre plads til flere mennesker. 

Derfor arbejder Karl nu på at udvikle en robot som balancerer på en kugle og bruger inerti til at kunne bevæge sig i alle retninger. Kuglens store diameter gør, at robotten kan køre ubesværet og naturligt på de fleste overflader og da den hele tiden står og holder balancen på kuglen, føles den også lidt mere levende at interagere med.
%
\section{Case: Lufthavn}
\label{CaseLufthavn}
%
Karl har indtil videre arbejdet med en case, hvor robotterne skulle indgå i et genoptræningsprogram og hjælpe folk til at lave øvelserne rigtigt. En anden case han har arbejdet med er Hotel Scandic i Aalborg, som ønskede at have en robot som kan byde gæster velkommen og checke dem ind, men også tage imod bestillinger fra hotelbarens gæster. På nuværende tidspunkt handler projektet for Karl om at få robotterne til at bevæge sig på en hensigtsmæssig måde, så de kan indgå i en social interaktion. Fra februar starter han et nyt overbyggende projekt i samarbejde med Combine og Københavns lufthavn. Formålet med dette projekt er at få udviklet robotterne, så det er nemt for udviklere eller designere at programmere robotterne til den specifikke kontekst den skal indgå i. Idéen er efterfølgende, at digitale bureauer, eksempelvis Combine, kan tilbyde deres kunder en robot, på lige fod med at de sælger applikationer og hjemmesider. Som en case for dette projekt, bruger de Københavns lufthavn, som allerede nu er meget interesserede i at få en robot ud og servicere deres kunder.\blankline
%
I lufthavne kan der hurtigt opstå en kaotisk stemning, når folk løber rundt på gangene for at nå deres fly. Et af de største problemer er ofte, at folk ikke er klar over hvor god tid de egentlig har og hvor langt de skal gå, og derfor skynder sig ned mod gaten. Når de når derned, er de langt væk fra salgsområderne, som typisk ligger centralt i lufthavne, hvilket resulterer i at de enten ikke køber noget, eller at de går frem og tilbage endnu en gang. Det påvirker blandt andet folks oplevelse af lufthavnen og rejsen, men også butikkernes omsætning. Københavns lufthavn vil gerne forsøge at holde folk i de centrale dele af lufthavnen længst muligt og dermed minimere trafikken på gangarealerne ud mod de forskellige gates. Det er her robotterne kommer ind i billedet, da disse kan fungere som serviceassistenter. Det forestilles, at robotterne kan køre rundt og fortælle folk, hvor længe der er til deres gate lukker, hvor langt der er derhen samt hvor de kan finde eller aflevere bagage. Ydermere forestilles det, at robotterne kan vise vej til gaten og eventuelt skabe mersalg i de centrale områder.
%
\section{Vores rolle}
\label{VoresRolle}
%
Ved møde med Karl blev det gjort klart, at der stadig er nogle udfordringer i forbindelse med den subjektive oplevelse af interaktionen med sociale robotter, før robotterne er klar til at komme ud og interagere med rigtige brugere. 

Det overordnede formål er at undersøge, hvordan robotterne skal indgå i en social kontekst og interagere med mennesker, hvortil følgende konkrete eksempler på problemer kan opstilles:\blankline
\begin{itemize}
  \item Hvordan skal robotten bevæge sig, sådan at dem vi vil interagere med føler, at den henvender sig til dem, uden at dem vi ikke vil interagere med føler at den også henvender sig til dem?
  \item Hvordan skal robotten forholde sig under en interaktion? Eksempelvis vippe, flytte sig fra side til side, dreje sig mod den der snakker, være dynamisk, stå stille osv.
  \item Hvordan skal robotten forholde sig, når nogen prøver at interagere med den fysisk? Eksempelvis når man skal trykke på touchskærmen.
  \item Hvordan skal robotten ændre adfærd, når der kommer nye mennesker ind i billedet?
  \item Hvordan kan folk ændre eksempelvis afstanden til robotten, hvis de føler at den er for nærgående eller for langt væk?
  \item Hvordan kan udseendet og interfacet på skærmen påvirke folks oplevelse?\blankline
\end{itemize}
\noindent
%
Ud fra ovenstående spørgsmål er det tydeligt, at der fra Karls side er meget fokus på at besvare spørgsmål i forbindelse med robottens overordnede sociale adfærd, som ikke er rettet mod en specifik case. Der ønskes at lave en generel løsning, som gør robotten i stand til at gebærde sig i mange forskellige situationer. Denne løsning omtales som robottens sociale intelligens.\blankline 
%
Der blev overvejet hvilke aspekter, der er vigtige i forbindelse med sociale robotters adfærd i de tre kontekster, som Karl arbejder eller har arbejdet med, Københavns lufthavn, Hotel Scandic Aalborg og et genoptræningscenter. Her blev der hurtigt enighed om, at robotten sandsynligvis ikke skal interagere med mennesker på samme måde i alle kontekster. Man kunne eksempelvis forestille sig, at folk i genoptræning, som inviterer robotten ind i deres hjem, vil have en anden oplevelse af robotten end folk i en lufthavn. Robotten bør i stedet tilpasses afhængigt af hvem den skal interagere med, hvad dens formål med interaktionen er og i hvilken kontekst interaktionen foregår.

For at gøre dette, kan der med fordel udvikles en metode eller en skala, som designere og udviklere i fremtiden vil kunne bruge til at evaluere den subjektive oplevelse i den specifikke kontekst, hvor robotten skal indgå. Ved evaluering af oplevelsen kan optimale bevægelsesmønstre findes, så de kan programmeres ind i robottens adfærd. Dette stemmer godt overens med Karls kommende projekt, som handler om at gøre robotter mere tilgængelige for almene virksomheder, ved at gøre det muligt for dem selv at programmere og tilpasse robotterne.

De udviklere og designere, som Karl har været i kontakt med, giver dog udtryk for at de ikke er interesserede i at involvere sig i, hvordan robotten bevæger sig. De vil foretrække at robotten har nogle foruddefinerede indstillinger for den grundlæggende måde robotten bevæger sig på, der gør at de kun skal tage stilling til hvad robotten skal gøre og ikke hvordan den skal gøre det. På den måde er det nemt og hurtigt for dem at programmere robotten til specifikke formål.

For at robotterne kan nå et niveau, hvor virksomhederne ønsker at de skal interagere med deres kunder, er det dog vigtigt at brugeroplevelsen optimeres til den givne kontekst, selvom det tager ekstra tid. Man kan derfor ikke altid bare bruge en generel model, som designeren eller udvikleren ellers ønsker. Samtidig er det dog værd at overveje, at hvis det er for omstændigt og dyrt at implementere robotterne, vil de digitale bureauer eller deres kunder sandsynligvis slet ikke købe dem. Det er derfor nødvendigt at finde et kompromis, som sikrer både en nem implementering og en skræddersyet brugeroplevelse, for at tilfredsstille alle interessenterne. 

\section{Potentielt løsningsforslag}
\label{PotentieltLoesningsforslag}
%
For at kunne udvikle denne primitive sociale intelligens i robotten, er det nødvendigt at forstå, hvilke parametre brugeren mener er vigtige i en social robots adfærd. Ved at kigge på kontekster som eksempelvis lufthavne eller supermarkeder, hvor der er en meget bred målgruppe, kan det være muligt at udpege hvilke faktorer, der generelt er vigtige for robottens adfærd i disse situationer. På den måde behøves der ikke nødvendigvis laves dybdegående brugerundersøgelser og modificeringer hver gang robotten skal bruges i en ny kontekst, men der kan derimod generaliseres fra andre kontekster med samme karakteristika.

Når robotten skal bruges i kontekster med en mere snæver målgruppe og med et mere specialiseret formål, kan det derimod være nødvendigt at lave brugerundersøgelser for den enkelte situation og modificere robottens adfærd herefter. 

I dette projekt tages der udgangspunkt i casen om Københavns lufthavn, hvor det undersøges, hvordan folk oplever interaktionen med robotten. Potentielle brugere vil blive interviewet, med henblik på at finde generelle tendenser i deres subjektive oplevelser med den sociale robot. Disse tendenser bruges til at udvikle en eller flere skalaer til evaluering af brugeroplevelsen. Skalaerne kan herefter verificeres med andre potentielle brugere. Hvis de vigtige parametre kan generaliseres til også at gælde andre lignende situationer, som eksempelvis et shoppingcenter, sygehus eller supermarked, vil de udviklede skalaer sandsynligvis kunne bruges til at teste robottens adfærd i nævnte, lignende situationer.

Der startes med et lille studie, der forsøger at afdække hvordan folk snakker om oplevelsen med robotten og især hvilke ord folk bruger til at beskrive deres oplevelse. Det gøres for at være sikker på, at skalaerne udvikles ud fra de reelle brugeres egne forståelser. Derefter vil et forsøgsdesign blive opstillet som tester præcist de parametre, som vi har fundet ud af er vigtige for robottens sociale adfærd.
