\section{Ændringer af testdesign}
\label{TestAfSkalaerAendringerTD}
%
Efter at have kørt den første test i lufthavnen den 01/12-2017 var der indsamlet data fra 22 testpersoner. Det blev derfor besluttet at tage ud i lufthavnen og indsamle mere data den 05/12-2017. Information om afgangene blev derfor opdateret, så de passede til den tilsvarende dag testen blev kørt. 

Ydermere blev det under testen observeret, at det var svært at styre hvor tæt på og fra hvilken vinkel robotten henvendte sig. Selvom det blev prøvet at følge skemaet, var der mange rejsende, som undveg robotten, når den kom for tæt på eller generelt ikke virkede til at have lyst til at interagere med den. Det blev derfor besluttet at prøve at variere de forskellige typer henvendelser, men uden at følge et bestemt skema. Da de rejsende også generelt virkede til at være mindre interesserede i robotten disse to testdage sammenlignet med de rejsende i feltundersøgelsen, blev det besluttet at det var vigtigere at få nogle testpersoner frem for at følge et bestemt skema. Højden på robotten blev i starten varieret efter skemaet, men som der blev mangel på testpersoner blev det besluttet at variere højden oftere, så alle højder blev testet.

I lufthavnen var der en testperson, som gerne ville vide hvor toilettet var, men ikke ville følges derhen. Vedkommende virkede lidt forvirret da robotten efterfølgende spurgte hvad den så kunne hjælpe med, hvorefter vedkommende gik. Det valgtes derfor, at hvis der trykkes nej til gældende spørgsmål, så skifter skærmbilledet til: "Hav en god rejse" og robotten spørger derfor ikke hvad den skal hjælpe med endnu en gang.