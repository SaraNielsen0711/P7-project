\section{Andet besøg i Lufthavnen}
\label{TestAfSkalaLufthavnsBesog}
%
Vi var i lufthavenen fredag d. 01-12-2017 fra 07:00-15:15. Det virkede meget sværere at få folk til at interagere med robotten denne gang. Vi havde et mål om at få kørt ca. 40 igennem, men det lykkedes kun at få 22. I løbet af dagen varierede vi højden på robotten, samt indgangsvinklen således, at vi kunne få testet skalaerne så bredt som muligt.\\
Vi havde ikke løst problemet med, at skærmen på Double reagerede skidt. Derudover var der også få problemer med det skalaprogram, som vi havde lavet. Ofte reagerede det ikke på første klik og der skulle dobbeltklikkes. Få gange skulle programmet genstartes på første side, hvilket var en skam, da deres oprindelige første indskydelse blev forandret. De svarede anderledes efter en genstart på trods af, at det kun tog 5 sekunder.



\subsubsection{Observationer}
Det virkede som om, at Robotten ikke var spændende nok til at interagere med, når den kørte ret frem og tilbage. Med andre ord virkede det bedre, at styre robotten, så den kom skråt fra siden.
Selvom mange virkede til at have travlt, så eksisterede der stadig en generel interesse for robotten. Mange smilte, pegede og virkede overraskede selvom de ikke interagerede med robotten. Der var også mange, der virkede interesserede, men efter at have læst på skærmen blot gik videre eller trykkede nej. Årsagen til dette kan bunde ud i indholdet på teksten. Skærmen spørger om den må hjælpe med at finde rundt i lufthavnen. Aalborg lufthavn er ret lille sammenlignet med mange andre lufthavne og det er derfor ikke svært at finde rundt. Der var en person, der forsøgte at anvende gestikker til at stoppe robotten. Han viftede hånden foran skærmen på den.\\
Der var en del, der stoppede og tog billeder eller video med deres telefoner. Flere lavede gesturer som om robotten var en hund, og klappede sig på lårene og piftede for at få den til at følge efter. Vi mistede mange deltagende igen som en konsekvens af, at Double skærmen reagerede dårligt. Hvis ikke skærmen reagerer ved første tryk, trykker folk hårdere og hurtigere. Ofte tager de fat med fingrene bagom skærmen og trykker med tommelfingeren, muligvis for at kunne trykke hårdt samtidigt med at stabilisere robottens svaj.

\subsubsection{Testpersoner}
\begin{itemize}
\item \textbf{Testperson 1} blev forstyrret under hans besvarelse af spørgeskemaet, da det reagerede meget dårligt. Efter tre besvarelser, blev programmet genstartet. Det blev observeret, at han nu svarede markant anderledes end sin oprindelige indskydelse på de første tre. 
\item \textbf{Testperson 2}  misforstod skalaens midtpunkt og antog, at det var en slider. 
\item \textbf{Testperson 3} nævnte at skærmen på robotten reagerede bedre end skalaprogrammet.
\item \textbf{Testperson 4} udfylder skemaet sammen med sin kone. De konfererer omkring svarene, selvom det kun var manden, der trykkede på skærmen.
\item 	\textbf{Testperson 10} er et ældre ægtepar, der trykkede for hårdt på skærmen, og blev afbrudt af en bservatør, der guidede dem igennem deres besvarelse. Efterfølgende nævnte de også, at de ville have opgivet hurtigt, hvis ikke de havde fået hjælp. Hun nægtede at menneskeliggøre robotten ved at kalde den sød, som hun forbandt med en menneskelig egenskab.
\item \textbf{Testperson 11} interagerede først med robotten, men gik. Han kom dog tilbage 5 minutter efter derfor havde han ikke lige interageret med robotten, da der blev besvaret på skalaerne. Halvvejs inde i besvarelsen opdager han, at den ikke kører selv, hvilket måske kan have påvirket hans besvarelse.
\item \textbf{Testperson 12} var muligvis ansat i lufthavnen, da hun gik i civil sammen med personale.
\item 	\textbf{Testperson 13} er en far og søn, der begge interagerer med robotten. Det er kun faderen der svarer på skala.
\item \textbf{Testperson 14} fik hjælp til at trykke på skærmen på Double, da hun trykkede for hårdt og holdt fingeren inde på skærmen. Hun blev ikke fulgt afsted af robotten, da forsøgslederen stod lige ved siden af. Hun spørger ind til, hvad der menes med hastighed.
\item \textbf{Testperson 15} regnede med at kunne bruge midterpunkt som slider. Dobbeltklikkede på en skala under midterpunktet, da den ikke reagerede, flyttede musen over midten og satte sin markør der, før han gik videre.
\item \textbf{Testperson 16} forstod ikke skalaen, der handler om hvor langt fra/tæt på robotten stoppede. Udfylder skalaerne med sin kammerat ind til side 4 af 7. Derefter udfyldes det af en enkelt person.
\item \textbf{Testperson 17} havde problemer med skalaprogrammet - det reagerer ikke godt. Spørger ind til midterpunkt-label "fin" og om det også var fin alle andre steder. Forsøgslederen forklarer, at det bygger på folks udtalelser, de steder det er angivet.
\item \textbf{Testperson 18} havde ikke lige interageret, da personen skulle på toilet først.
\item 	\textbf{Testperson 19} ventede et par minutter på at besvare skalaerne pga. på Testperson 18 stadig var igang. Kommenterede, at robotten var meget hyggelig; lidt som en hund.
\item 	\textbf{Testperson 20} interagerede med robotten på dens højeste niveau.
\item \textbf{Testperson 21} går først forbi robotten men vender sig om til sin mand og de interagerer sammen samt udfylder skalaerne sammen. De misforstår skalaspørgsmålet, der spørger ind til hvor langt fra/tæt på den stoppede. De troede det betød om den stoppede for langt fra/for tæt på ende destinationen.
\item \textbf{Testperson 22} et par, der udfydlte skalaerne sammen. De grinte rigtigt meget, da de fulgte efter robotten.
\end{itemize}

Der var efterhånden langt imellem afgangene, så vi blev nødt til at komme igen for at få flere testpersoner.

\subsection{3. Besøg i Lufthavnen}
Vi var i lufthavnen d. 05-12-2017 fra 08:00-15:30. Ligesom sidst virkede folk til at have travlt, og der gik lidt langt tid inden vi fik en eneste deltager. 

\subsubsection{Observationer}
Der var få ikke-dansktalende rejsende, der interagerede med robotten. En enkelt britte, der kommenterede, at det var ret imponerende og at han elskede det. En pige på 12 år virkede oprigtigt interesseret i robotten og ville interagere. Hendes mor sagde, at hun ikke kunne lide sådan noget og trykkede mange gange og ivrigt på nej.
Igen var skærmen ofte et problem på både Double og computeren med skala program.

\subsubsection{Testpersoner}
\begin{itemize}
\item \textbf{Testperson 23} var en far og en mindre dreng. Faderen forsøgte først at trykke på skærmen, men trykkede ofte for hårdt. Han løftede drengen op (robot var 151 cm) og den virkede. Kommenterede at skalaprogrammet heller ikke virkede så godt. Kommenterer, at han ikke lagde mærke til den, før den pludselig var der og siger den ikke skal være irriterende som en gadesælger.
\item \textbf{Testperson 24} var en familie på 4. Det var en pige, der trykkede på Double og udfyldte skalaerne med hjælp fra sin familie.
\item	\textbf{Testperson 25}. Både Double og skalaprogram reagerede rigtigt dårligt.
\item \textbf{Testperson 26}. "Hvis [Double] var pyntet op til jul, så var den ekstremt imødekommende".
\item \textbf{Testperson 27} blev ikke fulgt afsted. Han kom til at minimere Double appen, så iPad'ens baggrund viste sig.
\item \textbf{Testperson 28} var sammen med en kammerat. Han tastede ikke selv på Double. Han skifter sin placering på skalaen, når den ikke reagerer første gang.
\item \textbf{Testperson 29} studerer innovationsdesign på AAU.
\item	\textbf{Testperson 31} var 4 personer i alt. Hun brugte meget sine rejsemakkere til at udfylde skalaerne. Hun fik forklaret hvad anmassende betød af sin rejsemakker. Hun talte flydende dansk, men det var ikke hendes modersmål
\item \textbf{Testperson 33} mener robotten er for langsom.
\item \textbf{Testperson 38}. Det hele gik uden problemer. Double og skalaprogram reagerede hver gang, der blev trykket.
\item \textbf{Testperson 39}. Hun henvendte sig selv til robotten. Hun reflekterede over, hvad der blev ment med hvor mange gange man flyver: er det pr. tur eller fysisk fly? Hun endte med at tælle alle fly med hun satte sig ind i i løbet af et år.
\item \textbf{Tesperson 40} reflekterede over ordet sjov. De synes den var sjov, men den fortalte ingen vittigheder, så den var ikke \textit{sjov}-sjov.
\item \textbf{Testperson 41} mente det ville være nemmere at kunne tale til den. Brugte "Find whiskey" som et eksempel.
\item \textbf{Testperson 42} blev forvirret, da gateinformation om København ikke var blevet opdateret. Han forsøgte at swipe, da han så tiden ikke passede og derefter trykke på Aalborg Lufthavn logoet i bunden (måske for at komme tilbage eller opdatere siden.)


\end{itemize}