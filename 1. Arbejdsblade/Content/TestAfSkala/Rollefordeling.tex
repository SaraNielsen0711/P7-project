\section{Rollefordeling}
\label{TestAfSkalaRollefordeling}
%
For at strukturer testen opstilles der specifikke roller hvortil specifikke opgaver tildeles hver rolle. Der opstilles i alt tre forskellige roller. 
%
\subsubsection*{Robotstyre}
Som ved feltundersøgelsen der er udført tidligere vælges det at fjernstyrer en \textit{Double}-robot og det er derfor nødvendigt at have en person til at styre robotten, som vil blive betegnet Robotstyrer. \blankline
%
Robotstyren har til opgave at køre robotten rundt i lufthavnen og få den til at henvende sig til personer i området. Ved denne test er valgt at robotstyren skal variere på følgende parametre\fxnote{Disse parametre er foreløbige, og tilpasses/ændres sandsynligvis} ved robotten: 
\begin{itemize}
	\item Højden på robotten. 
	\item Afstanden fra robotten til personen den henvender sig til. 
	\item Om det er robotten eller testpersonen der henvender sig. 
\end{itemize}
Formålet med at robotstyreren skal ændre på disse fysiske parametre er at det ønskes at undersøge om andre parametre også ændrer sig når disse fysiske parametre gør. 

\subsubsection*{Testleder}
Personen der står for at fange testpersonen og bede dem besvarer skalaerne betegnes testlederen. Denne rolle har til opgave at introducere testpersonerne for testen og give dem tabletten til besvarelse af skalaerne. \blankline
%
Testlederen har desuden til opgave at observere og svare på spørgsmål under besvarelsen, hvis testpersonen måtte have nogle. Hvis der er spørgsmål eller relevante observation noter testlederen disse ned på en notesblok. Det er vigtigt at der i tilfælde hvor testpersonen har spørgsmål til Scale questions, ikke forklares projektgruppens forståelse af spørgsmålet, men i stedet lader det være op til testpersonen, ved eksempelvis at sige: “det er ud fra din forståelse af det” eller “Hvad synes du det betyder?” \blankline
% 
Testleder skal efter besvarelsen af skalaerne stille nogle enkelte afsluttende spørgsmål, for derefter at takke testpersonen for deltagelse og sige farvel.  

\subsubsection*{Nej-sigere} \fxnote{find ny titel denne rolle}
I tilfælde hvor personer vælger at trykke ``Nej'' på skærmen i forhold til om robotten kan hjælpe dem, skal personen med rollen som Nej-sigere fange personerne og spørge indtil hvorfor de valgt at sige nej. Det er vigtigt at tonen og henvendelsen er ekstra venlig, så det undgås at personen der har sagt nej føler at de bliver bebrejdet for at have gjort det. Formuleringen af spørgsmålet som Nej-sigere kan stille er følgende: 
\begin{itemize}
	\item Må jeg hurtigt hører hvad der fik dig til at trykke ``nej'' på robotten? 
\end{itemize}