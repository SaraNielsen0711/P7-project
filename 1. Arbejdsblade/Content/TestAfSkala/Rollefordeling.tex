\section{Robottens bevægelse}
\label{RobottensBevaegelse}
%
Det vælges, at robotstyreren skal variere på robottens højde, afstand til testpersonen samt indgangsvinkelen hvorved robotten henvender sig til en testperson. At netop de tre parametre er valgt skyldes, at det er muligt at ændre på dem ved brug af \textit{Double}. Formålet med at ændre på de tre fysiske parametre er for, at kunne undersøge om de har indflydelse på nogen af de andre parametre.\blankline
%
Den laveste højde robotten kan have er 118 cm, målt fra gulvet til det øverste punkt på rammen hvori iPad'en sidder, hvor robotten maksimalt kan indstilles til en højde på 151 cm, målt fra gulv til top. De to højde vil derfor agerer som de to ekstremer: \textit{Alt for lav} og \textit{Alt for høj}. På den bipolære skala, hvor robottens højde evalueres, er midtpunktet navngivet med: \textit{Fin}, som afspejler en højde svarende til omkring albuehøjde, jævnfør \fullref{ParametreDatabehandlingSkalaer}. Denne label er med til at kalibrer skalaen, da der ikke findes en naturlig og logisk højde mellem de to ekstremer, hvorfor det vælges at én af højderne så vidt muligt skal afspejle, hvad testpersonerne forbinder med \textit{Fin}. For at have et kvalificeret estimat af albuehøjden tages der udgangspunkt i den gennemsnitligehøjde for danske mænd, som er 181.4 cm, og for danske kvinder, som er 167.2 cm, \parencite{WEB:DanskersHoejde}. Omregnes de to højder til en gennemsnitshøjde for danskere er højden 174.3 cm. Albuehøjden blev derfor målt ved, at en fra projektgruppen, som er ca. 174.3 cm høj, bukkede albueleddet i en vinkel på ca. 90$^\circ$ og interagerede med robotten i en højde, der føltes behageligt. Robottens højde blev målt til 129 cm når skærmen befandt sig i albuehøjde, hvilket svarer til midtpunktet på skalaen angivet med \textit{Fin}. Denne højde blev bekræftet af flere fra studiet, som blev spurgt om hvad du synes om robottens højde, hvortil de svarede: \textit{Fin}. 

For at få yderligere variation i robottens højde og for at præge testpersonerne til at bruge mere af skalaen vælges det er inkludere to ekstra højder. Den ene højde er midtimellem minimum højden og albuehøjden, hvilket giver en højde på 123.5 cm målt fra gulv til top. Den anden højde er midtimellem albuehøjden og maks højden, hvilket giver en højde på 140 cm. Det tilstræbes, at indstille robotten i disse højder, dog kan der forekomme en lille variation i forhold til den eksakte højde. At der kan forekomme en lille variation i højden vurderes, at være acceptabelt da formålet med at ændre højden på robotten er, at undersøge hvilken indflydelse det kan have på andre parametre.\blankline           
%
I forhold til robottens afstand til testpersonerne vil tre afstande være gældende: Tæt på, tilpas og langt fra. Afstanden angives ikke i centimeter, da det ikke er muligt, at sikre en specifik afstand når robotten kører rundt i lufthavnen, hvertfald ikke uden at forstyrre interaktionen mellem testperson og robot. Dog gør følgende retningslinjer sig gældende: Tæt på afspejler en afstand hvor robotten kommer så tæt på testpersonen, at testpersonen vil træde et skridt tilbage for at kunne interagere ordentlig på skærmen. En tilpas afstand afspejler en afstand, hvor testpersonen har mulighed for, at nå skærmen uden nødvendigvis at strække armen helt ud og uden at skulle træde et skridt tilbage. Afstanden langt fra afspejler at testpersonen med strakt arm ikke kan nå skærmen og derfor er nødsaget til at træde et skridt nærmere. Det forventes dog, at det ude i lufthavnen kan blive svært at overholde disse retningslinjer fuldt ud, da der vil forekomme situationer hvor de rejsende selv henvender sig til robotten og på den måde selv bestemmer en afstand. 

Den sidste af de tre parametre, der ændres på er: Robottens indgangsvinkel når den henvender sig til en potentiel testperson. Det vil blive forsøgt, at få robotten til at henvende sig forfra, fra den rejsendes højre side og fra den rejsendes venstre side. Der sættes ikke nogle krav op til at indgangsvinklen skal være præcis en samme hver gang den kommer fra enten højre eller venstre. Ligesom med afstanden forventes det dog, at det kan blive svært fuldstændigt at kunne kontrollere indgangsvinklen, da det dels afhænger af hvor mange rejsende der er i test området på det pågældende tidspunkt, hvor den rejsende befinder sig i forhold til robotten og dels afhænger af om den rejsende selv henvender sig til robotten, da det i så fald vil være den rejsende, der dikterer indgangsvinklen.     

I \autoref{tab:Raekkefoelge} fremgår en plan for hvilken afstand robotten skal holde til testpersonen samt hvilken indgangsvinkel den skal henvende sig med når robotten er på sit laveste 118 cm. For at de fem højder alle præsenteres med de tre forskellige afstande og indgangsvinkler, resulterer det i 45 kombinationer. For de fire resterende højder vil rækkefølgen for afstand og indgangsvinkel være magen til den, der fremgår af \autoref{tab:Raekkefoelge}.    
%
\begin{table}[H]
	\centering 
	\begin{tabular}{c|c|c|c}
		Testperson  & Højde & Afstand & Indgangsvinkel \\\hline
		1   & 118 cm & Tæt på & Forfra  \\\hline
		2   & 118 cm & Tilpas & Forfra \\ \hline
		3   & 118 cm & Langt fra  & Forfra \\ \hline
		4   & 118 cm & Tæt på & Højre side \\ \hline
		5   & 118 cm & Tilpas & Højre side \\ \hline
		6   & 118 cm & Langt fra & Højre side \\ \hline
		7   & 118 cm & Tæt på & Venstre side \\ \hline
		8   & 118 cm & Tilpas & Venstre side \\ \hline
		9   & 118 cm & Langt fra  & Venstre side 
	\end{tabular} 
	\caption{Plan for hvilken afstand robotten skal have til testpersonen og hvilken indgangsvinkel den skal have ved henvendelse, når robotten er 118 cm høj. Tilsvarende er gældende for de fire resterende højder: 123.5 cm, 129 cm, 140 cm og 151 cm.}
	\label{tab:Raekkefoelge}       
\end{table}
\noindent
%
For at opnå de 45 kombinationer af højde, afstand og indgangsvinkel kræver det deltagelse fra minimum 45 testpersoner. Det tilstræbes, at have alle 45 testpersoner på én dag, men da der kun blev udført 18 interviews i feltundersøgelsen, jævnfør \fullref{ParametreFaktiskeTestpersoner}, er det ikke sikkert at det er muligt. Er det ikke muligt, at indsamle data fra minimum 45 testpersoner på én dag vil testen foretages over flere dage til det ønskede antal testpersoner er opnået.   

\section{Rollefordeling}
\label{TestAfSkalaRollefordeling}
%
For at strukturer testen opstilles der specifikke roller hvortil specifikke opgaver tildeles hver rolle. Der opstilles i alt tre forskellige roller. 
%
\subsubsection*{Robotstyrer}
Som ved tidligere udført feltundersøgelse vælges det at fjernstyre en \textit{Double}-robot. Det er derfor nødvendigt at have en person til at styre robotten. \blankline
%
Robotstyren har til opgave at køre robotten rundt i lufthavnen og få den til at henvende sig til personer i området. Som beskrevet i \fullref{RobottensOpfoersel} skal robotstyreren følge \autoref{tab:Raekkefoelge}, når vedkommende styrer robotten.

\subsection{Observatør af interaktion på skærmen}
Denne observatør skal holde øje med hvad de rejsende trykker på skærmen. Hvis der trykkes “nej” skal observatøren indikere til robotstyreren at robotten skal køre væk og henvende sig til andre personer. Dette indikeres ved at vende tommelfingeren nedad. 
Hvis personen, der interagerer med robotten, trykker "ja", skal observatøren holde øje med hvornår der står “følg venligst efter mig” på skærmen. Når der står dette på skærmen indikerer observatøren til robotstyreren, ved at vende tommelfingeren opad og robotstyreren lader robotten køre mod shopping området. 

\subsection{Observatør af robotten}
Personen med denne rolle har til formål at observere, hvordan robotten bevæger sig, og notere dette. De parametre, der er vigtige at notere, er følgende: 
%
\begin{itemize}
	\item Robottens højde
	\item Hvorfra robotten henvender sig
	\item Hvordan robotten bevæger sig 
	\item Formålet for denne observatør er at indsamle data der er relevant for hvordan robotten agerede ved den enkelte person. 
\end{itemize}
%
\subsubsection*{Testleder}
Personen der står for at henvende sig til testpersonen, når robotten følger dem mod shoppingområdet og bede dem besvarer skalaerne betegnes testlederen. Denne rolle har til opgave at introducere testpersonerne for testen og give dem tabletten til besvarelse af skalaerne. \blankline
%
Testlederen har desuden til opgave at observere og svare på spørgsmål under besvarelsen, hvis testpersonen måtte have nogle. Hvis der er spørgsmål eller relevante observation noter testlederen disse ned på en notesblok. Det er vigtigt at der i tilfælde hvor testpersonen har spørgsmål til Scale questions, ikke forklares projektgruppens forståelse af spørgsmålet, men i stedet lader det være op til testpersonen, ved eksempelvis at sige: “Det er ud fra din forståelse af det” eller “Hvad synes du det betyder?” \blankline
% 
Testleder skal efter besvarelsen af skalaerne stille nogle enkelte afsluttende spørgsmål, for derefter at takke testpersonen for deltagelse og sige farvel.  



