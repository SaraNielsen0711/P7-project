\section{Fremgangsmåde}
\label{TestAfSkalaFremgangsmaade}
%
Så snart en af de rejsende befinder sig i det aftalte område vil robotten henvende sig ved at køre hen til den rejsende og spørge om den kan hjælpe dem med at finde rundt i Aalborg Lufthavn. Svarer testpersonen \textit{Ja} vil der efterfølgende ganske kort stå, på skærmen, hvad formålet med testen er: At undersøge menneskers interaktion med robotter.
Derefter har testpersonen mulighed for selv at vælge et at de fire brugsscenarier. Uanset hvilket brugsscenarie, der bliver valgt vil det ende med at robotten opfordre testpersonen til at følge efter. Kort efter testpersonen er begyndt at følge efter robotten vil testleder stoppe dem, og komme med en kort introduktion der indeholder følgende: 
%
\begin{quotation}
\noindent
\textit{Hej, vi kommer fra Aalborg universitet. Vi er i gang med at undersøge menneskers interaktion med robotter. Har du to minutter* til at svare på nogle spørgsmål i forhold til robotten? Det foregår på denne tablet. Hvis du har nogle spørgsmål undervejs, så bare stil dem til mig. *tiden skal testes inden og rettes til.}
\end{quotation}
%
Efter introduktion giver testlederen en tablet til testpersonen hvorpå de skal vurdere robotten på de udviklede skalaer. Hvis testpersonen stiller spørgsmål til besvarelsen af skalaerne undervejs, vil testlederen noter spørgsmålet, samt hvilken skala det er stilt til. \blankline
%
Besvarelsen på tabletten indeholder både de 24 udviklede skalaer samt afsluttende spørgsmål og demografi. \blankline
%
Efter endt besvarelse vil testleder stille følgende spørgsmål: 
\begin{itemize}
	\item Var der noget du synes der manglede ved vurderingen af robotten? 
	\item Hvordan var det at besvare de opstillede skalaer? 
\end{itemize} \blankline
%
Efterfølgende afsluttes testen ved at testlederen siger tak for deres deltagelse og ønske testpersonen en god rejse.