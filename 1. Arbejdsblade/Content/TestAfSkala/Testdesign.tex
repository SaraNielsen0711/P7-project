\chapter{Testdesign}
\label{TestAfSkalaTestdesign}
%
For dels at kunne besvare den anden problemstilling: \textit{Hvordan kan de fundne parametre gengives i skalaer, som efterfølgende kan bruges til at evaluere HRI?}, og dels for at få erfaring med samt for at evaluere de udviklede skalaer opstilles der en test, hvori skalaerne anvendes til evaluering af robotten samt HRI. 

Testdesignet, der er gældende for denne test baseres på testdesignet for den allerede udførte feltundersøgelse, som er præsenter i \fullref{ParametreTestdesign}. Der vil i dette kapitel kun blive præsenteret de elementer af testen, der er forskellig fra feltundersøgelsen. De elementer, der som udgangspunkt ikke ændres fra feltundersøgelsen til denne test er: Type af testpersoner samt rekrutteringen af dem, brugsscenarierne samt testlokation, som er beskrevet i henholdsvis: \autoref{ParametreTestpersoner}, \autoref{ParametreBrugsscenarier} og \autoref{ParametreTestlokationOgUdstyr} 
%
\section{Testens omfang}
\label{TestAfSkalaTestensOmfang}
%
Med udgangspunkt i og for at besvare problemstillingen: \textit{Hvordan kan de fundne parametre gengives i skalaer, som efterfølgende kan bruges til at evaluere HRI?}, vil omfanget af denne test primært indebære at lade danske rejsende i Aalborg Lufthavn interagere med den social robot, for efterfølgende at evaluere robotten samt interaktionen med den. Testpersonernes evaluering vil foregå på de nu tilpassede 24 skalaer, der er udviklet på baggrund af den tidligere feltundersøgelse. De 24 skalaer, samt rækkefølgen hvorved de præsenteres fremgår af \fullref{TilpasningSkalaer}. Modsat feltundersøgelsen, hvor der blev indsamlet kvalitativ data, vil der i denne test primært kun blive indsamlet kvantitativ data i form af responsen på skalaerne. Baseret på den indsamlede data vil det blive undersøgt, om der forekommer korrelation mellem parametrene. Fokus for denne test er derfor ikke at undersøge, hvordan og om testpersonerne er i stand til at anvende skalaerne eller om de forstår de angivne labels.
\newpage