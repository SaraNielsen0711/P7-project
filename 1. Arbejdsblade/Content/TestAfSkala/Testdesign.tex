\chapter{Testdesign}
\label{TestAfSkalaTestdesign}
%
For at få erfaring med og evaluere de udviklede skalaer opstilles der en test, hvori skalaerne anvendes til evaluering af robotten og HRI. \blankline
%
Det ønskes at undersøge om testpersonerne har problemer med brugen af skalaerne og forståelsen af de valgte labels. Det ønskes desuden at undersøge om der kan være korrelation eller sammenhæng mellem nogle af parametrene. \blankline

Testdesignet for denne test baseres på testdesignet for den allerede udførte feltundersøgelse, som er præsenter i \fullref{ParametreTestdesign}. Der vil i dette kapitel kun blive præsenteret de elementer af testen der er forskellig fra feltundersøgelsen. De elementer der ikke ændres fra feltundersøgelsen til denne test er; Testpersonerne, brugsscenarierne samt testlokation og udstyr, som er beskrevet i henholdsvis \autoref{ParametreTestpersoner}, \autoref{ParametreBrugsscenarier} og \autoref{ParametreTestlokationOgUdstyr} 
%
\section{Testens omfang}
\label{TestAfSkalaTestensOmfang}
%
Formålet med denne test er at lade rejsende i Aalborg lufthavn interagere med en social robot, for efterfølgende at bede dem evaluere robotten og interaktionen med den på 24 forskellige skalaer, der er udviklet på baggrund af en feltundersøgelse i lufthavnen. Der ønskes at indsamle kvalitativ data i form af besvarelser på samt kvalitativ data i form af udtalelser til skalaerne og observation af anvendelsen af skalaerne. 