\chapter{Databehandling}
\label{TestAfSkalaDatabehandling}
%
I følgende kapitel vil den indsamlede data blive behandlet og analyseret med forskellige metoder samt efter forskellige emner. Det samlede data forefindes i \fullref{ElektroniskBilagExcel}. Det skal dog pointeres, at fordi denne undersøgelse er eksplorativ vil databehandlingen ligeledes være det. Ligesom ved den foregående felt undersøgelse, vil ændringer af testdesign blive beskrevet først. Dernæst hvilke testpersoner, der har deltaget i undersøgelsen, som er efterfulgt af de observationer, der er foretaget i AAL. Inden der foretages en decideret behandling og analyse af testpersonernes besvarelser, vil der frasorteret manglende datapunkter, som afspejler manglende besvarelser. Derefter vil der blive fokuseret på fordelingen af besvarelserne. For at undersøge hvordan de varierede fysiske parametre har indflydelse på testpersonernes respons og for at undersøge hvorvidt der forekommer korrelation mellem to eller flere parametre bliver der udført en \textit{Principal Component Analysis} (PCA), først generelt for hele datasættet og derefter i forhold til de tre fysiske parametre: Højde, afstand og indgangsvinkel. Når PCA udføres kan det være en fordel, at have det tre dimensionelle \textit{Bi}-plot ved siden af, der forefindes 3D \textit{Bi}-plots for det samlede datasæt, højde og indgangsvinkel i \fullref{ElektroniskBilag3D}. Der er ikke et 3D \textit{Bi}-plot for afstand, da der kun er to \textit{Principal Components} (PCs). Derudover kan det være en god idé, at have oversigten over skala spørgsmålene og skalaerne ved siden af, da disse vil blive noteret med SQ efterfulgt af nummer, oversigten forefindes i \fullref{ElektroniskBilagSkalaOversigt}. Derefter fokuseres der på, at sammenligne de korrelerede parametre, hvorefter der fokuseres på hvilken indflydelse testpersonernes demografi har.        
