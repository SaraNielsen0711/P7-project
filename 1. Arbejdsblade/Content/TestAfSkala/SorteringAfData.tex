\section{Sortering af data}
\label{TestAfSkalaSorteringAfData}
%
%OBS: Jeg ved ikke om det skal være en sektion eller en undersektion endnu. 
%
Testpersonernes respons på skalaerne omregnes til procent, hvorfor det venstre endepunkt er 0 \% og det højre endepunkt er 100 \%. Ud fra datasættet fremgår det, at nogle skalaer har én eller flere besvarelser på 0 \%, problemet med det er, at det ikke vides med sikkerhed om testpersonen har svaret 0 eller om testpersonen slet ikke har angivet en respons på den pågældende skala. At det ikke er muligt at adskille de to tilfælde fra hinanden skyldes, at programmet automatisk gemmer værdien 0, hvis en skala ikke er besvaret. Det har derfor været nødvendigt at gennemgå datasættet og vurdere hver enkelte tilfælde i forhold til om det er en reel besvarelse eller om det er en manglende besvarelse. Dette afgøres på baggrund af to kriterier: 1) Hvis alle skalaerne, der er præsenteret på én af de syv sider, ikke er besvaret vil de alle være angivet med 0 i datasætte, hvorfor det kan fastslåes at det ikke er en reel besvarelse men derimod en manglende besvarelse. Årsagen til at der forekommer tilfælde, hvor en testperson ikke har besvaret en eneste skala på en af de præsenterede sider skyldes formentlig, at der har været problemer med programmet. Disse problemer relaterer sig hovedsageligt til \textit{Næste}-knappen i programmet, som reagerede dårligt og i nogle tilfælde kun ved dobbeltklik, så hvis en testperson klikkede flere gange på knappen er det muligt, at de sprang over en side og slet ikke blev præsenteret for de pågældende skalaer. 2) Hvis der ved en skala generelt er høje procentsatser og en besvarelse på 0 \% virker usandsynlig vil det betragtes som en manglende besvarelse. Er det tilfældet vil 0'et blive behandlet som en outlier, der er tre standard afvigelser fra det resterende data.

I henhold til datasættet (LAV HENVISNING TIL DATASÆTTET) vil situationer, der opfylder kriterie 1) vedrørende manglende besvarelser, være markeret med en rød celle. Summeret forekommer det 16 gange, hvor største delen af fejlene er opstået på side syv, hvor hverken TP8, TP10, TP18, TP29, TP34 og TP36 har afgivet en besvarelse på hver af de fire præsenterede skalaer. 

Der er 28 tilfælde hvor det er uvist hvorhvidt der enten er afgivet en besvarelse på 0 \% eller om 0 \% afspejler en manglende besvarelse. DET KIGGER VI PÅ NU... 

 

  
