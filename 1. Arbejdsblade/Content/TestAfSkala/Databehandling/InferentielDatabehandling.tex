\section{Inferentiel databehandling}
\label{sec:inferentiel}
%
Først laves en \textit{Principal Component Analysis} (PCA) analyse på det samlede datasæt, for at undersøge om der er korrelation mellem nogle af de 23 skala spørgsmål. For at undersøge hvor meget af variansen, der kan forklares ud fra hver \textit{Principal Components} (PC) udarbejdes der et \textit{Scree}-plot, jævnfør \autoref{fig:Scree}. 
%
\begin{figure}[H]
\centering
\includegraphics[width=\textwidth]{Figure/DatabehandlingSkalaer/PCAfigures/Scree.png}
\caption{\textit{Scree}-plot, hvorpå sammenhængen mellem antallet af \textit{Principal Components} og \textit{Variance Explained [\%]} fremgår.}
\label{fig:Scree}
\end{figure}
\noindent
%
Ud fra \textit{Scree}-plottet på \autoref{fig:Scree} fremgår det, at der ikke er en tydelig sammenhæng i variansen fra datasættet, da kun 29.9 \% af variansen kan forklares af PC1 og 13.4 \% af PC2. Selv hvis de tre første PCs medtages, er det kun 53.1 \% af variansen, som forklares af modellen. Dette skyldes formentligt, at der er for mange variable, som bliver varieret på én gang og at det ikke har været muligt at kontrollere alle variable lige godt, og at der tilmed forekommer forskellige grupperinger i forhold til de variable, der tilnærmelsesvist var mulige at kontrollere. På baggrund af det er det svært at vurdere hvilke parametre, der korrelerer.\blankline 
%
Ud fra \textit{Score}-plottet på \autoref{fig:Score} fremgår det også at PC1 kun forklarer en lille del af variansen, da der er meget spredning i datapunkterne og spredningen ikke er systematisk. Hvis PC1 havde forklaret en større del af variansen, ville data være mere centreret omkring den horisontale akse.
%
\begin{figure}[H]
\centering
\includegraphics[width=\textwidth]{Figure/DatabehandlingSkalaer/PCAfigures/Scores}
\caption{\textit{Score}-plot for PC1 og PC2.}
\label{fig:Score}
\end{figure}
\noindent
%
For at undersøge hvilke parametre, der bidrager til hver PC opstilles et \textit{Bi}-plot, hvori både \textit{scores} og \textit{loadings} fremgår. \textit{Bi}-plottet på \autoref{fig:Biplot} giver det bedste overblik, da det er nemmest at fortolke i to dimensioner, dog er det tredimensionelle plot vedlagt i \fullref{ElektroniskBilag3D}. 
%
\begin{figure}[H]
\centering
\includegraphics[width=\textwidth]{Figure/DatabehandlingSkalaer/PCAfigures/Biplot}
\caption{\textit{Bi}-plot med både \textit{loadings} (angivet med blå) og \textit{scores} (angivet med rød) fremgår i forhold til robottens højde.}
\label{fig:Biplot}
\end{figure}
\noindent
%
Baseret på \autoref{fig:Biplot} tyder det på, at der forekommer en positiv korrelation mellem SQ14, vedrørende hvor personlig robottens hjælp opleves, og SQ15, vedrørende hvor overrasket testpersonerne blev over robottens henvendelse. Lignende korrelation forefindes mellem SQ17, vedrørende hvor elegant robotten opleves, og SQ22, vedrørende hvor sjov robotten opleves. Derudover tyder det på, at der forekommer en positiv korrelation mellem SQ8, vedrørende hvorvidt testpersonerne føler, at robotten kan hjælpe dem, SQ19, vedrørende hvor sød robotten opleves, og SQ20, vedrørende hvor sej robotten opleves. De tre parametre tyder på, at være negativ korrelerede med SQ16, vedrørende hvor irriterende robotten opleves. En negativ korrelation forefindes også mellem SQ4, vedrørende hvordan robottens bevægelser opleves, og SQ13, vedrørende hvorvidt testpersonerne regnede med at robotten fulgte dem hen til det valgte sted. 

Baseret på \autoref{fig:Biplot} fremgår det, at SQ7, vedrørende robottens høj, SQ5, vedrørende robottens afstand, og SQ3, vedrørende hvordan det var at bruge robotten, kun forklare en meget lille del af variationen. 



Ud fra disse to plots kan det ses at der er korrelation mellem SQ14 og SQ15, SQ17 og SQ22 samt SQ9 og SQ21.





Ud fra 3D plottet kan man se at SQ7 ikke bidrager meget til nogle af de tre PCs. Som nævnt i \fullref{TestAfSkalaVarians} på \autoref{fig:Varians} har SQ7 en meget lille varians, og det er derfor ikke overraskende at den ikke bidrager mere til de tre principal components end den gør resultat er derfor ikke overraskende. Det betyder at spørgsmål 7 har fået nogenlunde samme besvarelser på tværs af grupperne og at den derfor ikke er særlig vigtig for den samlede oplevelse. Det kan dog både være fordi at den ikke er et vigtigt parameter, men det kan også være, at der bare ikke er nogle af testpersonerne, som blev udsat for en oplevelse, der påvirkede dette parameter, men at det måske er vigtigt i andre sammenhænge. 

Resultaterne fra PCA'en fra det samlede datasæt giver ikke nogle håndgribelige konklusioner. For at undersøge om der er nogle tendenser i forhold til de objektive parametre som blev justeret undervejs i testen, højde, retning og afstand, undersøges hvordan hver af disse parametre påvirker en PCA-analyse. Dette gøres ved at kigge på et parameter ad gangen og dele det op i de grupper som allerede findes.