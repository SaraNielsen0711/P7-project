\section{Varians}
\label{TestAfSkalaVarians}
%
Skriv noget om variansen i data samt præsenter grafer med oversigt over variansen for hver SQ. 
%
\subsection{Standardisering af besvarelser}
Variansen er ikke ens for data til de forskellige SQ. Hvis variansen skulle udlignes vil der skulle laves en standardisering af data. Formålet med at gøre dette er at data med lav varians i rå data får lige betydning i PCA som rå data med stor varians. 

Vi vælger ikke at standardisere vores data, da det kan give skævvridninger som ikke er ønsket. Testpersonerne har reelt haft muligheden for at besvarer på hele skalaen, men har valgt at give besvarelser i nærheden af hinanden, hvilket tolkes som at de har været mere enige i dette tilfælde. 

Vores data er målt med samme ``måleenhed'', da det er målt på samme skala. Det er 0-100 det hele. 

Hvis der er større variation ved nogle skalabesvarelser, vægter de mere i PCA. Det mener vi er fint. 
