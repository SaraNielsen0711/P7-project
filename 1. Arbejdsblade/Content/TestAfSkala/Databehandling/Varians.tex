\section{Fordeling af besvarelserne}
\label{TestAfSkalaFordeling}
%
%
I følgende afsnit vil fordelingen af besvarelserne blive præsenteret. Dette dækker over den gennemsnitlige besvarelse til hver af de 23 skalaer samt eksempler på forskellige fordelinger, der er med til at vise, at der eksisterer stor varians på tværs af besvarelserne. Til sidst vil data blive præsenteret i et boksplot og diskuteret.
%
\begin{figure}[H]
\centering
\includegraphics[width = \textwidth]{Figure/DatabehandlingSkalaer/DataPresentation/MeanBarplot} 
 \caption{Søjlediagram over den gennemsnitlige besvarelse (\%) til hvert skala spørgsmål (SQ) samt barer, der viser standardafvigelsen for hver SQ.}
\label{fig:BarPlotGennemsnit}
\end{figure}
\noindent
%
På \autoref{fig:BarPlotGennemsnit} fremgår den gennemsnitlige besvarelse for hver af de 23 skalaer, angivet med \textit{SQ} efterfulgt af nummer. På grafen er standardafvigelsen for vurderingerne til hvert skala spørgsmål præsenteret, for at give indblik i spredningen af data for hver skala. \blankline
%
Der er lavet et kombineret histogram og normalfordelingsplot for hver skala, for at få en mere nøjagtig forståelse for, hvordan besvarelserne er fordelt. Enkelte af dem vil blive beskrevet i dette afsnit, mens resten findes samlet i \fullref{ElektroniskBilagHistNormal}.

\begin{figure}[H]
\centering
\includegraphics[width = 0.7\textwidth]{Figure/DatabehandlingSkalaer/HistogramNormalFordeling/SQ17} 
\caption{Histogrammet viser hyppigheden af besvarelser (y-aksen) målt som relativ frekvens (Antal/SampleSize), inden for de fastsatte intervaller (x-aksen). Den sorte kurve viser den underliggende normalfordelingskurve, som er baseret på middelværdien og standardafvigelsen}
\label{fig:histogram17}
\end{figure}
\noindent
%
\autoref{fig:histogram17} viser histogrammet for SQ17 med tilhørende normalfordelingskurve og er et eksempel på hvordan resten af hisogrammerne er visualiseret. Da en stor del af data befinder sig under den sorte kurve, tyder det på, at besvarelserne for SQ17 er tilnærmelsesvist normalfordelte. Dette undersøges dog ikke med en signifikanstest, \textit{Shapiro-Wilk Normality Test}, da formålet med gennemgangen blot er at få et overblik over fordelingerne.\blankline
%
Kurverne i flere af plottene ser meget flade ud, hvilket skyldes, at akserne holdes konstante på tværs af plottene, for bedre at kunne sammenligne dem indbyrdes. Dette er dog med undtagelse af histogrammer og normalfordelinger for SQ5, SQ7 og SQ11, da data er meget centreret i midten og til venstre på disse histogrammer. Det kan derfor være svært at se fordelingerne på de resterende histogrammer, hvis akserne defineres ud fra SQ5, SQ7 og SQ11.\blankline
%
I mange af plottene er der en ret stor varians, hvilket viser at en stor del af skalaen er blevet anvendt. Her skal det tilføjes at besvarelserne er givet ud fra forskellige højder, indfaldsvinkler og afstande og at variansen derfor ikke nødvendigvis er et udtryk for at folk er uenige, men snarere at de har vurderet forskellige stimuli. Et eksempel er givet på \autoref{fig:histogram14}.\blankline
%
\begin{figure}[H]
\centering
\includegraphics[width = 0.7\textwidth]{Figure/DatabehandlingSkalaer/HistogramNormalFordeling/SQ14} 
\caption{Fordelingerne af besvarelserne på skalaspørgsmål 14}
\label{fig:histogram14}
\end{figure}
\noindent
%
I disse historgram- og normalfordelingsplot blev der kigget på, hvordan besvarelserne er fordelt for hvert enkelt spørgsmål, men for at få et samlet overblik, kigges på \autoref{fig:boksplots}, hvor der er lavet boksplots for alle skalaspørgsmålene. 
%
\begin{figure}[H]
\centering
\includegraphics[width = \textwidth]{Figure/DatabehandlingSkalaer/BoksplotUden0er} 
\caption{Boksplot over fordelinger af besvarelserne for hvert skalaspørgsmål.}
\label{fig:boksplots}
\end{figure}
\noindent
%

Boksplottet deler alle besvarelserne for det pågældende skalaspørgsmål op i fire kvartiler. Medianen adskiller de to midterste kvartiler, og kanterne på kasserne adskiller disse fra de to yderste kvartiler. Outliers er ikke medregnet i kvartilerne for boksplottet, men er stadig plottet for at give en idé om hvordan de fordeler sig i forhold til resten af besvarelserne. \blankline
%
Ud fra boksplottet på \autoref{fig:boksplots} kan det ses, at fordelingerne adskiller sig meget fra hinanden. Nogle af dem udnytter en stor del af skalaen (SQ1, SQ4, SQ8, SQ12, SQ14, SQ19, SQ22, SQ23), hvorimod andre fordeler sig over mindre områder af skalaen (SQ2, SQ3, SQ5, SQ7, SQ9, SQ11, SQ13, SQ16). Nogle boksplot ser tilnærmelsesvis normalfordelte ud (SQ1, SQ2, SQ6, SQ17, SQ20, SQ21), mens andre er skævevredet den ene eller anden vej (SQ8, SQ9, SQ10, SQ11, SQ13, SQ18). Der ses også enkelte boksplots, hvor datapunkterne ophober sig meget omkring endepunkterne eller midtpunktet (SQ5, SQ7, SQ9, SQ10, SQ11, SQ13). \blankline
%
Den lave varians, som ses i nogle af boksplottene, kan skyldes forskellige ting. Det kan enten skyldes, at variablen ikke er vigtig for oplevelsen af interaktion og at folk derfor har svaret neutralt. Det kan også skyldes, at det ikke er en variabel som er blevet påvirket af de specifikke stimuli testpersonerne blev udsat for, men at den kan være vigtig i andre situationer med robotten. Effekten forstærkes især ved SQ5 og SQ7, som er bipolare skalaer. Midtpunkterne på disse skalaer kan fungere som et anker, hvor testpersonerne ofte vil centrere deres vurderingerne omkring. Labellet ”Fin”, som anvendes i SQ7 blev valgt på baggrund af testpersonernes udtalelser i feltundersøgelsen. Det kan dog være, at ordet ”Fin” dækker over for bredt et spænd af skalaen, hvilket kan medføre at folk har sat en markering i midten, selvom robotten har været en lille smule for høj eller lav. Dette kan også være med til at centrere datapunkterne omkring midtpunktet og dermed mindske variansen i vurderingerne. \blankline
%
De lukkede endepunkter, som blev anvendt på skalaerne, kan ophobe data meget omkring disse. Ophobningen af data omkring endepunkterne betyder også, at fordelingen bliver skævvredet, da den ofte stopper brat i den ende med endepunktet, men har en lang hale i den anden ende.  \blankline
%
Nogle steder bruges kun øverste halvdel, hvor der andre steder kun bruges nederste halvdel. Det hænger sammen med formuleringerne på endepunkterne, da den positive label sommetider vil være ved 100 (Eks.: Ekstremt sød) og i andre tilfælde ved 0 (Eks.: Slet ikke anmassende).


%%\subsection{Varians}
%%
%%For at få et overblik over hvor meget besvarelserne variere og hvor stor forskel, der er mellem variationen ved de forskellige SQ, beregnes variansen for hver SQ beregnes med formlen: \blankline
%%
%%\begin{equation}
%%	Var = \frac{\sum_{i=1}^{n}(x_i-\overline{x})^{2}}{(n-1)}
%%\end{equation}
%%\noindent
%%
%%\fxnote{Indsæt kilde (Field bog)}
%%Hvor $Var$ er varians, $x_i$ er målingen for nummer $i$, $\overline{x}$ er middelværdien og $n$ er antal målinger. 
%%En oversigt over variansen for hver SQ kan ses på \autoref{fig:Varians}. 
%%
%%\begin{figure}[H]
%%\centering
%%\includegraphics[width = \textwidth]{Figure/DatabehandlingSkalaer/Varians} 
%%\caption{Søjlediagram over variansen for besvarelserne til hvert Scale Question.}
%%\label{fig:Varians}
%%\end{figure}
%%\noindent
%%
%%Det tydeligt at se ud fra \autoref{fig:Varians} at der er stor forskel mellem variansen ved de forskellige SQ. For eksempel er variansen for SQ5 og SQ7 meget lav i forhold til SQ1 og SQ8. 
%%
\subsection{Standardisering af besvarelser}
Når variansen ikke er ens for ens data kan det i nogle tilfælde vælges at udligne variansen. Hvis variansen skulle udlignes vil der skulle laves en standardisering af data. Formålet med at gøre dette er at data med lav varians i rå data får lige betydning i PCA som rå data med stor varians. \blankline
%
Det vælges ikke at standardisere data, da det kan give skævvridninger som ikke er ønsket. Testpersonerne har reelt haft muligheden for at svare på hele skalaen, men har valgt at give besvarelser i nærheden af hinanden, hvilket tolkes som at de har været mere enige i dette tilfælde. \blankline

%Al data er målt med samme måleenhed, da det er målt på samme skala, der går fra 0 til 100 ved aflæsningen. Dette er en del af begrundelsen for ikke at standardisere, da det ofte gøre i tilfælde hvor målingerne ikke er målt med samme måleenhed. \blankline
%
%Hvis der er større variation ved nogle skalabesvarelser, vil de vægte mere i en PCA, men det vurderes at være acceptabelt og ønsket. Det er dog vigtigt at være opmærksom på dette, når resultaterne af PCA analyseres. 
