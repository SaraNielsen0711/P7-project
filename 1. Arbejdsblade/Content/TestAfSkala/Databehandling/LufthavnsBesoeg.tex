\section{Observationer i AAL}
\label{TestAfSkalaLufthavnsBesog}
%
Projektgruppen var i lufthavnen fredag d. 01-12-2017 fra 07:00-15:15 og tirsdag d. 05-12-2017 fra 08:00-15:30. Det virkede meget sværere at få de rejsende til at interagere med robotten, da de virkede til at have travlt. I løbet af de to dage blev højden på robotten, afstanden til de rejsende samt indgangsvinklen varieret for at få testpersonerne til at bruge mere af skalaerne. I de følgende afsnit vil observationer fra de to test dage blive beskrevet.\blankline
%
Problemet med at skærmen på \textit{Double} reagerede dårligt blev ikke fuldstændig løst af at tilføje teksten: \textit{Tryk blidt på mig} i wireframet. Der var stadig problemer med at skærmen ikke reagerede på testpersonernes berøring. Derudover var der også problemer med det udviklede skalaprogram. Ofte reagerede det ikke på første klik, hvorfor der skulle dobbeltklikkes. I nogle tilfælde skulle programmet genstartes, hvilket resulterede i at testpersonens besvarelser blev overskrevet, så de skulle angive en ny respons. Det blev observeret, at den nye respons i nogle tilfælde var meget anderledes end den oprindelige indskydelse.

\subsection{Observationer fra d. 01-12-2017}
Det virkede ikke som om, at robotten var spændende nok til at interagere med, når den kørte ret frem og tilbage. Med andre ord virkede det bedre, at styre robotten, så den kom skråt fra siden. Selvom mange rejsende virkede til at have travlt, så eksisterede der stadig en generel interesse for robotten. Mange smilte, pegede og virkede overraskede selvom de ikke interagerede med robotten. Der var også mange, der virkede interesserede, men efter at have læst på skærmen blot gik videre eller trykkede nej til at robotten måtte hjælpe dem med at finde rundt. Årsagen til det kan være formuleringen af teksten. Skærmen står der om robotten må hjælpe med at finde rundt i lufthavnen. Aalborg Lufthavn er ret lille sammenlignet med mange andre lufthavne og det er derfor ikke svært at finde rundt, hvorfor det kan være årsagen til at de rejsende ikke har lyst til at deltage. Der var en dansk rejsende, der forsøgte at anvende gestikker til at stoppe robotten, ved at vifte hånden foran skærmen.

Der var en del, der stoppede og tog billeder eller optog video med deres telefoner. Flere lavede gestikker som om robotten var en hund, og klappede sig på lårene og piftede for at få den til at følge efter. Vi mistede mange testpersoner igen som en konsekvens af, at \textit{Double} skærmen reagerede dårligt. Hvis ikke skærmen reagerer ved første tryk, trykkes der hårdere og hurtigere, hvilket skærmen slet ikke reagerer på. Ofte tager de rejsende fat med fingrene bagom skærmen og trykker med tommelfingeren, muligvis for at kunne trykke hårdt og præcist samtidigt med at stabilisere robottens bevægelser.

\subsubsection{Testpersoner (TP)}
\begin{itemize}
\item \textbf{TP01} blev forstyrret under sin besvarelse af skalaerne, da programmet reagerede meget dårligt. Efter tre skalabesvarelser, blev programmet genstartet. Programmet reagerede efterfølgende hurtigere, men det blev observeret, at testpersonen svarede markant anderledes end sin oprindelige indskydelse på de første tre skalaer. 
\item \textbf{TP02} misforstod skalaens midtpunkt og antog, at det var en slider. 
\item \textbf{TP03} nævnte at skærmen på robotten reagerede bedre end skalaprogrammet.
\item \textbf{TP04} besvare skalaerne sammen med sin kone. De konfererer omkring svarene, selvom det kun var manden, der trykkede på skærmen.
\item \textbf{TP10} er et ældre ægtepar, der trykkede for hårdt på skærmen, og blev afbrudt af en observatør, der assisterede dem igennem deres interaktion med robotten. Efterfølgende nævnte de også, at de ville have opgivet hurtigt, hvis ikke de havde fået hjælp. Kvinden nægtede at menneskeliggøre robotten ved at kalde den sød, som hun forbandt med en menneskelig egenskab.
\item \textbf{TP11} interagerede først med robotten, men gik. Testpersonen kom dog tilbage omkring 5 minutter efter, hvorfor testpersonen ikke lige havde interageret med robotten, da skalaerne blev besvaret. Halvvejs inde i besvarelsen opdagede testpersonen, at robotten ikke kører af sig selv, hvilket måske kan have påvirket besvarelserne.
\item \textbf{TP12} var muligvis ansat i lufthavnen, da testpersonen gik i civil sammen med personale.
\item \textbf{TP13} er en far og søn, der begge interagerer med robotten. Det er kun faderen der svarer på skalaerne.
\item \textbf{TP14} fik hjælp til at trykke på robottens skærm, da testpersonen trykkede for hårdt og holdte fingeren inde på skærmen. Testpersonen blev ikke fulgt afsted af robotten, da testlederen stod lige ved siden af. Testpersonen spørger ind til, hvad der menes med hastighed.
\item \textbf{TP15} regnede med at kunne bruge midterpunktet som slider. Dobbeltklikkede på en skala under midterpunktet, da den ikke reagerede, flyttede musen over midten og satte sin markør der, før testpersonen gik videre.
\item \textbf{TP16} forstod ikke skalaen, der handler om hvor langt fra/tæt på robotten stoppede. Udfylder skalaerne med sin kammerat ind til side 4 af 7. Derefter besvares skalaerne af en enkelt testperson.
\item \textbf{TP17} havde problemer med skalaprogrammet fordi det reagerede dårligt. Spørger ind til midtpunktets label: \textit{Fin} og om det også var fin alle andre steder. Testlederen forklarer, at det bygger på andre danske rejsendes udtalelser, de steder det er angivet.
\item \textbf{TP18} havde ikke lige interageret med robotten da skalaerne blev besvaret, da testpersonen skulle på toilet først.
\item \textbf{TP19} ventede et par minutter på at besvare skalaerne fordi TP18 stadig var igang. Kommenterede, at robotten var meget hyggelig; lidt som en hund.
\item \textbf{TP20} interagerede med robotten, som havde en højde på 151 cm.
\item \textbf{TP21} går først forbi robotten men vender sig om til sin mand og de interagerer sammen samt besvarer skalaerne sammen. De misforstår skalaspørgsmålet, der spørger ind til hvor langt fra/tæt på robotten stoppede. De troede det betød om robotten stoppede for langt fra/for tæt det sted de havde valgt den skulle følge dem hen til.
\item \textbf{TP22} et par, der besvarer skalaerne sammen. De grinte rigtigt meget, da de fulgte efter robotten.\blankline
\end{itemize}
\noindent
%
Der er ikke foretaget nogle særlige observationer for TP05 til TP09, hvorfor de ikke fremgår af ovenstående. Efter TP22 var der lang tid mellem diverse afgange, hvorfor det blev valgt at stoppe testen og tage ud i lufthavnen en anden dag og få flere testpersoner. 

\subsection{Observationer fra d. 05-12-2017}
Der var få ikke-dansktalende rejsende, der interagerede med robotten. En enkelt britte kommenterede, at robotten var ret imponerende og at han elskede den. En pige på 12 år virkede oprigtigt interesseret i robotten og ville interagere. Hendes mor sagde, at hun ikke kunne lide sådan noget og trykkede mange gange og ivrigt på nej. Igen var skærmen ofte et problem på både \textit{Double} og computeren hvorpå skalaerne præsenteres.

\subsubsection{Testpersoner (TP)}
\begin{itemize}
\item \textbf{TP23} var en far og en mindre dreng. Faderen forsøgte først at trykke på skærmen, men trykkede ofte for hårdt. Han løftede drengen op (robotten var 151 cm) så drengen kunne trykke, hvorefter skærmen reagerede. Kommenterede at skalaprogrammet heller ikke virkede så godt. Kommenterer, at han ikke lagde mærke til robotten, før den pludselig var der og påpeger at den ikke skal være irriterende som en gadesælger.
\item \textbf{TP24} var en familie på fire. Det var en pige, der trykkede på skærmen og udfyldte skalaerne med hjælp fra sin familie.
\item \textbf{TP25} Både skærmen på robotten og skalaprogrammet reagerede rigtigt dårligt.
\item \textbf{TP26} kommenterer: Hvis [Double] var pyntet op til jul, så var den ekstremt imødekommende.
\item \textbf{TP27} blev ikke fulgt afsted. Testpersonen kom til at minimere \textit{Double} applikationen, så iPad'ens baggrund viste sig.
\item \textbf{TP28} var sammen med en kammerat. Testpersonen tastede ikke selv på \textit{Double}. Testpersonen ændrer sin placering på skalaen, når den ikke reagerer første gang.
\item \textbf{TP29} studerer Innovationsdesign på AAU.
\item \textbf{TP31} var fire personer i alt. Testpersonen brugte meget sine rejsemakkere til at besvare skalaerne. Testpersonen fik forklaret hvad anmassende betød af sin rejsemakker. Testpersonen talte flydende dansk, men det var ikke hendes modersmål, hvorfor dette accepteres.
\item \textbf{TP33} mener at robotten er for langsom.
\item \textbf{TP34} var nordmand, men forstod udmærket dansk, hvorfor dette accepteres.
\item \textbf{TP38} det hele gik uden problemer. Robottens skærm og skalaprogrammet reagerede hver gang, der blev trykket.
\item \textbf{TP39} henvendte sig selv til robotten. Testpersonen reflekterede over, hvad der menes med hvor mange gange testpersonen flyver; er det per tur eller fysisk fly? Ender med med at tælle alle fly med som testpersonen satte sig ind i i løbet af et år.
\item \textbf{TP40} reflekterede over ordet \textit{sjov}. De synes robotten var sjov, men den fortalte ingen vittigheder, så den var ikke \textit{sjov}-sjov.
\item \textbf{TP41} mente det ville være nemmere at kunne tale til robotten. Brugte \textit{Find whiskey} som et eksempel.
\item \textbf{TP42} blev forvirret, da gateinformation om København ikke var blevet opdateret. Testpersonen forsøgte at swipe, da testpersonen bemærkede at tiden ikke passede og derefter trykkede på Aalborg Lufthavn logoet i bunden (måske for at komme tilbage eller opdatere siden.)
\item \textbf{TP43} interagerede med robotten et andet sted end alle andre testpersoner, ved et pakkebord under informationsskærmen
\end{itemize}
%
Igen er der testperson, hvor der ikke er foretaget nogle særlige observationer hvorfor de ikke fremgår af ovenstående. Selvom det blev tilstræbet at følge skemaet over de planlagte højde, afstande og indgangsvinkler var det ikke muligt, at overholde det, hvorfor det ligeledes accepteres at der ikke er 45 testpersoner.  
\newpage
