\chapter{Diskussion}
\label{TestAfSkalaDiskussion}
%
Denne diskussion vil tage udgangspunkt i den foregående databehandling og vedrører forskellige emner, som ikke allerede er diskuteret.  

\section{Skalaerne}
\label{DiskussionSkala}
%
Som det er beskrevet \fullref{TestAfSkalaProgramSkala} opstod der flere problemer med programmet hvori skalaerne blev præsenteret. For testpersonerne betød det at programmet ikke reagerede når de prøvede at angive deres respons, hvorfor en testleder i flere tilfælde måtte træde til og genstarte programmet. Dette forårsagede desværre, at det i flere tilfælde blev observeret, at testpersonerne ændrede deres respons anden gang de svarede på en skala. Dette er noteret for TP1 og TP28. Derudover var der flere testpersoner, som gav verbalt udtryk for at programmet reagerede for langsomt, men også i deres kropssprog blev det observeret, at de var irriteret over, at programmet ikke reagerede og de som følge deraf måtte bruge længere tid på at besvare skalaerne. 

Derudover opstod der situationer hvor flere testpersoner fejlagtigt troede at midtpunktet var en slider, som de kunne bruge til at angive deres respons. Det kan muligvis skyldes den dårlige reaktion, så når programmet ikke reagerede første gang ved et musseklik på skalaen, så prøvede testpersonerne at tage fandt i midtpunktet for at rykke det. Dette blev noteret for TP2 og TP15.\blankline 
%
TP17 er den eneste noteret testperson, som har kommenteret på det label, der er på to af de bipolære skalaer: \textit{Fin}, hvor testpersonen spurgte ind til om det var meningen, at det skulle stå alle steder, hvorfor testlederen måtte forklare, hvorfor det er valgt at anvende netop det label. Dog tyder det på, at både midtpunkt og labels har haft en indflydelse på besvarelserne, særligt ved SQ5, vedrørende robottens afstand, og SQ7, vedrørende robottens højde, hvor variationen er meget lille, hvilket gengiver at testpersonerne har svaret tæt på eller direkte på midtpunktet, jævnfør \autoref{fig:boksplots}. Selvom SQ5 ikke har haft labelen \textit{Fin}, som de to andre skalaer præsenteret på side to, så kan det være at det har haft en indflydelse i form af, at testpersonerne har troet, at det skulle stå på den skala. Derudover så har testpersonerne selv kontrolleret afstanden til robotten, hvorfor det ligeledes er forståeligt, at deres respons samler sig omkring midtpunktet. At den lille variation forekommer til SQ7 kan formentlig skyldes, at robottens højde aldrig når en højde, som reelt ændre testpersonernes opleves, men det kan også være at testpersonerne bare ikke har en holdning til det, hvorfor de angiver en respons omkring midtpunktet \textit{Fin}. Dette forudsætter, at der naturligt vil forekomme en del flere outliers, fordi der ikke skal særlig meget til for at afvige fra den resterende respons centreret omkring midtpunktet. Når variationen for de to skala spørgsmål er så lille som den er på \autoref{fig:boksplots} og at ingen af dem forklarer en særligt stor del af variationen i de foretaget PCA, indikerer det at de formentlig ikke har særlig stor betydning for danske rejsende, hvert fald ikke som et selvstændigt skala spørgsmål. Det skyldes formentlig, at de to parametre har indflydelse på andre parametre eller indirekte bliver målt ved andre parametre. 

Foruden de to skala spørgsmål forekommer der generelt en stor spredning på skalaerne, hvilket skyldes at testpersonerne har haft forskellige oplevelser af interaktionen med robotten. Det betyder også at det er lykkedes at påvirke testpersonernes oplevelse ud fra de fysiske parametre, der blev ændret undervejs. Dog medfører den store varians, at det kan være svært, at fastslå noget konkret ud fra de fundne korrelationer. Det er også derfor, at der skal tages forbehold for tendenslinjerne, som særligt påvirkes af variationen men også af den måde data sorteres på for at lave graferne, der blandt andet præsenteres i \fullref{DatabehandlingSammenligningKorrelerede}. 
%



\section{Testpersoner}
\label{DiskussionTestpersoner}
%
Havde det været muligt, at kontrollere robottens afstand til testpersonerne samt indgangsvinklen ville der blive sigtet efter at få 45 testpersoner. Da det ikke var en mulighed og det tog væsentlig længere tid overhovedet, at få nogle testpersoner, hvorfor testen blev afviklet over to dage, resulterede det i, at der kun blev indsamlet data fra 43 testpersoner, hvilket er relativt få taget i betragtning af hvor mange timer der blev afsat til testen i lufthavnen, og hvor mange rejsende, der bare gik forbi robotten.

Selvom 43 testpersoner i nogle henseender kan være tilstrækkeligt vurderes det, at det ikke har været tilfældet i denne test. Det skyldes hovedsagligt, at der bliver testet 23 forskellige skalaer, ændret på tre fysiske parametre; højde, afstand og indgangsvinkel, men også at den måde hvorpå testpersonerne er inddelt i grupperne. Hvor grupperne referer til antallet af testpersoner repræsenteret ved de fem højder, tre afstande og fire indgangsvinkler. Med så mange forskellige scenarier burde der have været langt flere testpersoner for, at kunne fastslå noget endegyldigt.\blankline
%
På gruppen blev det diskuteret hvorvidt rejsende, som ikke interagerede med robotten men tydeligt registrerede den var der og rejsende, som starter en interaktion med robotten men som enten svarer nej til om den må hjælpe dem med at finde rundt i AAL eller som af en eller anden grund stopper interaktionen undervejs, skulle stoppes og spørges om hvorfor de ikke havde lyst til at interagere med robotten. Årsagen til at dette ikke blev gjort er, at det for de rejsende kan virke meget bebrejdende, at blive spurgt ind til hvorfor de ikke har interageret med robotten, og fordi det ikke ønskes, at forstyrre dem, hvis de nu har travlt. Derudover skal det også respekteres, hvis de rejsende bare ikke har lyst til at deltage. 

Fordelen ved at have stoppet dem for at spørge ind til hvorfor de valgte ikke at interagere med robotten eller valgte aktivt at trykke nej til hjælpen eller stoppede interaktionen undervejs, havde været at andre synspunkter kunne være kommet til udtryk. Det kan jo være, at årsagen til at nogle af de rejsende, som ikke interagerede med robotten, er, at de ikke føler sig trygge ved robotten, eller ikke føler at den kan hjælpe dem. Dette kommer ikke nødvendigvis til udtryk ved de indsamlede besvarelser, da det formentlig godt kan antages, at de rejsende, som har interageret med robotten og besvaret skalaerne, i højere grad har følt sig trygge ved robotten sammenlignet med nogen af dem, der ikke har. Ved at have stoppet de rejsende kan det have været med til, at give et mere nuanceret billede af hvordan HRI opleves, da nogle af de mere negative parametre, såsom at robotten er anmassende og irriterende, formentlig ville være mere tydelige. \blankline
%
Da testpersonerne besvarede skalaerne var, der som regel en testleder i nærheden for at sikre, dels at programmet fungerede og dels for at besvare eventuelle spørgsmål. Dette kan potentielt have påvirket nogle af testpersonernes besvarelser, da de måske har været bevidste om at testlederen kunne se besvarelserne, hvorfor de måske enten har vurderet nogle parametre lavere eller højere, end hvad de reelt mener bare for at tilfredsstille projektgruppen. Derudover var der også nogle få testpersoner, eksempelvis TP10, som modtog hjælp i forhold til at trykke på robottens skærm under interaktionen, da de formentlig ellers havde opgivet og gået videre. Dette kan ligeledes have påvirket deres besvarelser, uden at det vides hvordan. 

Andre ting, der kan have påvirket testpersonernes opleves relaterer sig blandt andet til, at alle testpersoner ikke blev fulgt lige langt, enten fordi de vidste hvor de rent faktisk skulle hen og robotten fulgte dem den forkerte vej, eller fordi testpersonerne startede interaktionen i nærheden af Duty Free, hvor robotten som regel stoppede. Dette gør sig blandt andet gældende for TP14 og TP27, som ikke blev fulgt afsted af robotten, da testlederen stod lige ved siden af dem. I relation til det, fremgår det fra observationerne, jævnfør \fullref{TestAfSkalaLufthavnsBesog}, at TP21 misforstod SQ5: \textit{Jeg synes, at robotten stoppede}, hvor de to rejsende tror det handler om, hvorvidt robotten stoppede tæt på det sted, som de havde valgt at den skulle følge dem hen til. Det virker også til at TP16 ikke forstod SQ5. 

Andre testpersoner har haft en samtale med testlederen samtidig med, at de evaluerede interaktionen med robotten på de præsenterede skalaer, dette, dog uvist, kan have haft en indflydelse på deres besvarelser, både fordi samtalen kan have hjulpet dem til at forstå skala spørgsmålene, men også fordi det måske, har påvirket dem til at svare mere positivt end hvad de reelt mener. Derudover har der også været situationer, hvor en familie enten skiftes til at trykke på robottens skærm, eller hvor det kun er én, der trykker på skærmen, men hvor det er et andet familiemedlem, som svarer på skalaerne. Derfor er der situationer hvor en testperson har svaret på skalaerne, men ikke interagerede med robotten. Det er uvist hvilken indflydelse det har haft og om det overhovedet har haft en indflydelse.  


\section{Fysiske parametre}
\label{DiskussionFysiskeParametre}
%
I \fullref{RobottensBevaegelse} er der beskrevet og argumenteret for valget af de højder, som robotten havde under testen. Dog kan det være, at det har været unødvendigt at anvende alle fem højder, særligt 123.5 cm, som kun er 5.5 cm fra den laveste højde (118 cm) og midtpunktet (129 cm). Ved at have undladt denne højde, kunne antallet af testpersoner præsenteret for de fire andre højder være steget. Derudover forefindes der ikke en markant tendens til at testpersonerne, som interagerede med robotten på 123.5 cm svarer anderledes end de testpersoner, som interagerede med robotten på 129 cm. Dog med forbehold for at der generelt forekommer stor variation i besvarelserne. 

I forhold til robottens højde, så blev det observeret, at det havde en indflydelse på hvor hurtigt robotten kørte. Når robotten var på sit laveste (118 cm) kørte den væsenligt hurtigere end når den var på sit højeste (151 cm). Problemet med at hastigheden robotten kører med afhænger af dens højde er, at det ikke vides hvordan de hver især påvirker testpersonernes opleves, og at der ikke er nogen kontrol over robottens hastighed. Det betyder også at testpersonerne, der har interagerede med robotten når den var 118 cm, har interagerede med en hurtigere robot end dem der interagerede med robotten når den var 151 cm, hvor hastigheden var langsommere. Dette kommer blandt andet til udtryk på \autoref{fig:TendensHeightSQ6}, hvor det netop fremgår, at når robotten er 118 cm bliver den vurderet hurtigere end når den er 151 cm. \blankline
%
I forhold til robottens afstand var det ikke muligt, at foruddefinere bestemte afstande, hvilket formentlig heller ikke havde været hensigtmæssigt, da testpersonerne så ikke selv kunne have indflydelse på afstanden fordi robotten konstant vil tilpasse sin afstand afhængigt af testpersonens placering. Dog havde fordelen med foruddefineret afstanden været, at det på samme måde som ved højde, havde været muligt at inddele testpersonerne i mere specifikke grupper end hvad der har været tilfældet, hvor de er inddelt i langt fra, tilpas og tæt på. Den manglende kontrol over afstanden kan være årsagen til, at SQ5 forklarer så lidt af variationen i de forskellige PCA'er.

Fordi det har været så svært, at kontrollere afstanden til testpersonerne og notere hvorvidt afstanden var langt fra, tilpas eller tæt på, og fordi testpersonerne undervejs i interaktionen har mulighed for at ændre afstanden, er det ikke sikkert at en PCA er det bedste valg. Det samme gør sig gældende for indgangsvinkel, som ligeledes har været svært at kontrollere, specielt at få robotten til at henvende sig fra venstre i AAL grundet placeringen. Derudover, har det ikke været muligt at kontrollere hvilken indgangsvinkel testperson, som selv henvendte sig til robotten havde. 


\section{Redundans}
\label{DiskussionRedundansSkalaer}
%
Som beskrevet i \fullref{DatabehandlingSammenligningKorrelerede} korrelerer nogle af parametrene. Det er derfor være relevant at undersøge, hvorvidt nogle af de korrelerede parametre dækker over det samme og derfor kan slås sammen under ét dækkende parameter. Det vælges kun at diskutere de parametre, hvor det giver mening at slå dem sammen. 

Jævnfør \autoref{fig:SammenligningSQ12SQ18}, hvor der forefindes en positiv korrelation mellem at kunne lide at blive betjent af robotten og hvor spændende den er, samt \autoref{fig:SammenligningSQ12SQ21}, hvor der forefindes en negativ korrelation mellem at kunne lide at blive betjent af robotten og hvor anmassende den er, kan det overvejes at erstatte skala spørgsmålet vedrørende at blive betjent af robotten med parametrene spændende og anmassende. Det kan være en tilfældighed, at der forefindes en korrelation, men hvis de rejsende ikke synes, at robotten er spændende kan de formentlig heller ikke lide at blive betjent af den. Det samme gør sig gældende med anmassende. Hvis de rejsende kan lide at blive betjent af robotten, så synes de højst sandsynligt heller ikke at den er anmassende. På den måde kan udledes hvorvidt testpersonerne kan lide at blive betjent af robotten, ved at spørge dem om de synes, at robotten er anmassende og spændende, uden at skulle spørge direkte om deres oplevelse af betjeningen.

Fokuseres der på \autoref{fig:SammenligningSQ20SQ22}, for sej og sjov, \autoref{fig:SammenligningSQ19SQ20}, for sød og sej samt \autoref{fig:SammenligningSQ18SQ20}, for spændende og sej, forefindes der positiv korrelation for de sammenlignede parametre. Under feltundersøgelsen kom det til udtryk, at sjov er et parameter, der dækker over mange forskellige ting, blandt andet at noget kan være humoristisk sjovt, men også underligt sjovt. Dette parameter er derfor svært at måle på, hvorfor det kan give mening at sammensætte det med sej, da disse korrelerer. Ydermere kan sej sammensættes med enten sød eller spændende. Det virkede til, at børn oftest bedømte robotten som værende sej, hvorimod ældre rejsende ikke nødvendigvis kunne relatere til at robotten var sej. Da det bedømmes, at en robot ikke kan være sej uden at være spændende og da parametrene korrelerer positivt, kan det vælges at medtage parametrene sød og spændende dækkende over både sød, spændende, sej og sjov.  

Undersøges hvor tryg testpersonerne var ved robotten op imod om de regnede med at den fulgte dem det rigtige sted hen og hvorvidt robotten kunne hjælpe dem, jævnfør \autoref{fig:SammenligningSQ10SQ12} og \autoref{fig:SammenligningSQ8SQ10}, forefindes der en positiv korrelation herimellem. Det kan overvejes hvorvidt tryghedsparameteret og at de rejsende regnede med robotten fulgte dem det rigtige sted hen kan slås sammen til et parameter omhandlende tillid til robotten. På den måde kan de tre parametre slås sammen til to; tillid til robotten og hvorvidt robotten kan hjælpe den rejsende. Det kan dog være problematisk at slå to skala spørgsmål sammen til et nyt, for at danne en ny skala, da forståelsen af tillid til robotten ikke er testet på samme måde som de to andre er, og fordi tillid og tryghed ikke nødvendigvis er det samme. 

