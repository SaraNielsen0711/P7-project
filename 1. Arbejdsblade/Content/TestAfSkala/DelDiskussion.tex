\chapter{Diskussion}
\label{TestAfSkalaDiskussion}
% Nedenstående er tanker gjort under lufthavnsbesøgene (1/12 og 5/12). Det kan bruges som inspiration til diskussion, men er ikke punkter der skal diskuteres. 
Folk forsøger at ramme et endepunkt - det er den effekt endepunkter har.\\
Folk var ikke lige så glade i starten af december i lufthavnen, derfor har vi ikke haft lige så mange testpersoner igennem som vi gerne ville.\\
Det fungerede ikke så godt at forudbestemme R, for hvad nu hvis man kun vil interagere, når den kommer fra en bestemt vinkel - men så ikke gør det, fordi den henvender sig forkert. Det er bedre at lade der være naturlig variation, og så ændre højden mellem testpersonerne.\\
At robotten rekrutterer gør at rekrutteringen går meget langsommere, men igen er det meget økologisk.\\
Der er virkelig mange testpersoner der siger nej\\
Vi mangler næsten et "den henvendte sig for meget" parameter, så vi kan se om den er anmasende, fordi den presser på og henvender sig, når man ikke ligner nogen der har brug for hjælp.\blankline
%
Skriv om at vi kan undvære SQ5 og SQ7, både fordi de ikke rigtig siger noget, eller varierer - det skyldes formentlig labels på midtpunktet. 

Vi kan se at der er svaret lige omkring midten for spørgsmål 5 og 7 på tværs af alle de demografiske faktorer. 5 og 7 er derfor ikke særligt beskrivende for den oplevelse folk har haft, da alle har svaret cirka det samme både på tværs af demografi og på tværs af de stimuli de har været udsat for (forskellige højder, afstande og vinkler). 5=robotten er for tæt på/langt væk, 7=robotten er for høj/lav.
