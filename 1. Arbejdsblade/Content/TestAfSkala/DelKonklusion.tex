\chapter{Delkonklusion}
\label{TestAfSkalaDelKonklusion}
%
Ud fra foregående afsnit, \autoref{TestAfSkalaDatabehandling} og \autoref{TestAfSkalaDiskussion}, er det muligt at konkludere på denne del af projektet, der testede de opstillede skalaer.\blankline
%
Først og fremmest kan det konkluderes, at det tog betydeligt længere tid end forventet at teste de udviklede skalaer i naturlige omgivelser, hvorfor der er nøjedes med 43 testpersoner. Ydermere har det været svært at kontrollere fysiske parametre som afstand til testpersonerne og indgangsvinkel, da robotten selv skulle henvende sig til de rejsende i lufthavnen, og disse ofte gik udenom eller selv kom hen, hvis de ville interagere med robotten. Skal fysiske parametre som afstand og indgangsvinkel testes mere kontrolleret kan det derfor være fordelagtigt at opstille en test i et laboratorium.

Robottens interface og skærm reagerede også i anden test dårligt, og det virkede ikke til at hjælpe at fortælle de rejsende, at de skulle trykke blidere på skærmen. Dette har medvirket til at flere potentielle testpersoner har opgivet interaktionen og gået væk igen.\blankline
%
Det ses fra resultaterne og oberservationerne, at der er forskel på folks forståelse af skalaerne. Når der er et midterpunkt på skalaen forstår nogle testpersoner denne som en slider og når der er et label på midterpunktet kan et label som "fin" være for bredt og medvirke til at alle testpersonerne placerer deres svar heromkring. Ydermere ses det fra resultaterne, at testpersonernes svar tilnærmelsesvist har fordelt sig mere normalt ved de bipolære skalaer (1-7) end de unipolære skalaer (8-23). \textbf{Okay ,jeg prøver bare, jeg ved ikke om man kan sige det der. Her skal stå noget smart om labels og kanter.}\blankline
%
Som beskrevet i \fullref{sec:inferentiel} er det ikke fordeltagtigt at lave \textit{Principal Component Analysis} på hele datasættet, hvorfor data grupperes efter henholdsvis robottens højde, afstand den stoppede fra testpersonerne og den indgangsvinkel interaktionen blev initieret. Fra PCA ses det hvilke parametre der er positivt og negativt korreleret, hvilket er vist i \autoref{tab:CorrelationsFromPCA}. Ydermere er de parametre, hvor der er fundet en korrelation imellem, blevet sammenlignet, hvilket kan ses i \autoref{tab:CorrelationsFromGraphs}. Disse sammenligninger er blevet lavet på alt data, hvorfor der ikke er blevet taget højde for grupperinger.\blankline
%
\begin{table}[H]
	\centering
	\begin{tabular}{ c|c|c }
		\centering
		PCA & Positive korrelationer & Negative korrelationer \\ \hline
		\multirow{5}{*}{Højde} & SQ08  + SQ17 & SQ02  + SQ09 \\
		& SQ10 + SQ13 & SQ04 + SQ12 \\
		& SQ12 + SQ18 & SQ12 + SQ21 \\
		& SQ14 + SQ15 & SQ16 + SQ19 \\
		& SQ20 + SQ22 & SQ18 + SQ21\\ \hline
		\multirow{6}{*}{Afstand} & SQ01 + SQ12 & SQ02 + SQ09 \\
		& SQ07 + SQ17 & SQ05 + SQ21 \\
		& SQ08 + SQ21 & SQ10 + SQ13 \\
		& SQ10 + SQ22 & SQ13 + SQ22 \\
		&  & SQ14 + SQ16 \\	
		&  & SQ19 + SQ20 \\ \hline	
		\multirow{5}{*}{Indgangsvinkel} 
		& SQ05 + SQ07 & SQ01 + SQ12 \\
		& SQ08 + SQ10 & SQ06 + SQ23 \\
		& SQ09 + SQ14 & SQ09 + SQ10 \\
		& SQ18 + SQ20 & SQ10 + SQ14 \\
		&  & SQ13 + SQ21
		
	\end{tabular}        
\caption{Korrelationer fra PCA}
\label{tab:CorrelationsFromPCA}
\end{table}
\noindent
%
\begin{table}[H]
	\centering
	\begin{tabular}{ c|c }
		\centering
		Positive korrelationer & Negative korrelationer \\ \hline
		SQ04 + SQ09 & SQ02 + SQ09 \\ 
		SQ08 + SQ10 & SQ09 + SQ10 \\ 
		SQ08 + SQ17 & SQ12 + SQ21 \\ 
		SQ10 + SQ13 & SQ13 + SQ21 \\ 
		SQ12 + SQ18 & SQ18 + SQ21	\\	
		SQ18 + SQ20 & 							\\
		SQ20 + SQ22 & 
	\end{tabular}        
\caption{Korrelationer fra sammenligningsgrafer}
\label{tab:CorrelationsFromGraphs} 
\end{table}
\noindent
%
Udover de beskrevne korrelationer viser resultaterne også forskellige tendenser, når information fra demografien sammenlignes med ratings på skalaerne. Her ses det, at når robotten bliver højere, så opleves bevægelserne roligere og hastigheden langsommere. Når robotten er på sit laveste stoler testpersonerne mere på den samt at den opleves mere elegant og sjov. Desto lavere robotten er, desto sødere opleves den ydermere. Kigges der på afstand ses det, at når robotten kommer tæt på bliver testpersonerne mere overrasket end ellers. 

Jo ældre testpersonerne er og jo større højdeforskel på testperson og robot der er, jo mere nærmer bedømmelsen af hastighededen sig et midterpunkt på skalaen, hvilket må svare til en tilpas hastighed. Derudover ses det, at jo ældre testpersonerne er, jo mindre sjov og spændende bliver den bedømt. Kigges der videre på højdeforskel ses der en tendens til, at desto større højdeforskel, desto vildere opleves robottens bevægelser, men desto sødere er robotten også.

Tjekkes der for kønsforskelle i skalabesvarelserne ses det, at SQ21, der vedrører hvor anmasende robotten er, er det eneste spørgsmål med så stor forskel på besvarelser, at konfidensintervallet ikke overlapper. Der kan derfor forekomme en kønsforskel ved dette parameter. Selvom konfidensintervallerne overlapper en smule, når der kigges på SQ4, omhandlende robottens bevægelser, og SQ11, omhandlende hvor forskrækket robotten gjorde en, ses det at der også her kan være en tilnærmelsesvis kønsforskel.