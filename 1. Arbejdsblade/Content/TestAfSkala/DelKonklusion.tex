\chapter{Delkonklusion}
\label{TestAfSkalaDelKonklusion}
%
Ud fra foregående afsnit, \autoref{TestAfSkalaDatabehandling} og \autoref{TestAfSkalaDiskussion}, er det muligt at konkludere på denne del af projektet og samtidig besvare den anden problemstilling: \textit{Hvordan kan de fundne parametre anvendes til at evaluere HRI?}. De fundne parametre blev i \fullref{ParametreDatabehandlingSkalaer} gengivet i skalaer, som efterfølgende blev tilpasset for at optimere formuleringerne og for at tilstræbe et helhedsindtryk, hvilket blev gjort i \fullref{TestAfSkalaTilpasningAfSkalaer}, hvorefter rækkefølgen for hvordan de præsenteres ligeledes blev besluttet. For at evaluere HRI blev de valgte skalaer anvendt af danske rejsende i AAL efter de havde interageret med robotten.\blankline 
%
Først og fremmest kan det konkluderes, at det tog betydeligt længere tid end forventet at afvikle testen, hvorfor der kun var 43 testpersoner. Ydermere har det været svært at kontrollere fysiske parametre såsom afstand og indgangsvinkel, hvor de rejsende ofte gik udenom eller selv henvendte sig, hvis de ville interagere med robotten. Skal fysiske parametre som afstand og indgangsvinkel testes mere kontrolleret kan det derfor være fordelagtigt at opstille en test i et laboratorium eller tilføje en afstands måler til robotten, så det efterfølgende er muligt at vide hvilken afstand robotten havde.

Ligesom ved foregående feltundersøgelse reagerede robottens skærm ofte dårligt, og det virkede ikke til at hjælpeteksten: \textit{Tryk blidt på mig}, havde den ønskede effekt. Dette har medvirket til at flere potentielle testpersoner opgav interaktionen og gik deres vej.\blankline
%
Af \fullref{TestAfSkalaLufthavnsBesog} og \fullref{DiskussionSkala} fremgår det, at testpersonerne havde en forskellig forståelse af skalaerne. Nogle testpersoner troede at midtpunktet på de bipolære skalaer var en slider, som de kunne bruge til at angive deres respons. \blankline 
%
Som beskrevet i \fullref{DatabehandlingPCA} var det ikke fordelagtigt at udføre \textit{Principal Component Analysis} på hele datasættet, hvorfor data grupperes efter henholdsvis robottens højde, afstand og indgangsvinkel. Fra PCA fremgår det hvilke parametre, der er positivt og negativt korreleret, hvilket er listet i \autoref{tab:CorrelationsFromPCA}. 
%
\begin{table}[H]
	\centering
	\begin{tabular}{ c|c|c }
		\centering
		PCA & Positive korrelationer & Negative korrelationer \\ \hline
		\multirow{5}{*}{Højde} & SQ08  + SQ17 & SQ02  + SQ09 \\
		& SQ10 + SQ13 & SQ04 + SQ12 \\
		& SQ12 + SQ18 & SQ12 + SQ21 \\
		& SQ14 + SQ15 & SQ16 + SQ19 \\
		& SQ20 + SQ22 & SQ18 + SQ21\\ \hline
		\multirow{6}{*}{Afstand} & SQ01 + SQ12 & SQ02 + SQ09 \\
		& SQ07 + SQ17 & SQ05 + SQ21 \\
		& SQ08 + SQ21 & SQ10 + SQ13 \\
		& SQ10 + SQ22 & SQ13 + SQ22 \\
		&  & SQ14 + SQ16 \\	
		&  & SQ19 + SQ20 \\ \hline	
		\multirow{5}{*}{Indgangsvinkel} 
		& SQ05 + SQ07 & SQ01 + SQ12 \\
		& SQ08 + SQ10 & SQ06 + SQ23 \\
		& SQ09 + SQ14 & SQ09 + SQ10 \\
		& SQ18 + SQ20 & SQ10 + SQ14 \\
		&  & SQ13 + SQ21
	\end{tabular}        
\caption{Korrelationer fra PCA, \textit{SQ} angiver skala spørgsmål.}
\label{tab:CorrelationsFromPCA}
\end{table}
\noindent
%
Ydermere er de parametre, hvor der er fundet en korrelation, blevet sammenlignet, hvilket fremgår af \autoref{tab:CorrelationsFromGraphs}. Disse sammenligninger er foretaget på alt data, hvorfor der ikke er blevet taget højde for grupperinger.
%
\begin{table}[H]
	\centering
	\begin{tabular}{ c|c }
		\centering
		Positive korrelationer & Negative korrelationer \\ \hline
		SQ04 + SQ09 & SQ02 + SQ09 \\ 
		SQ08 + SQ10 & SQ09 + SQ10 \\ 
		SQ08 + SQ17 & SQ12 + SQ21 \\ 
		SQ10 + SQ13 & SQ13 + SQ21 \\ 
		SQ12 + SQ18 & SQ18 + SQ21	\\	
		SQ18 + SQ20 & 							\\
		SQ20 + SQ22 & 
	\end{tabular}        
\caption{Korrelationer fra sammenligningsgrafer \textit{SQ} angiver skala spørgsmål.}
\label{tab:CorrelationsFromGraphs} 
\end{table}
\noindent
%
En gennemgående tendens for de tre PCAer for henholdvist højde, afstand og indgangsvinkel er at både SQ5, vedrørende afstand, og SQ7, vedrørende højde, kun forklarer en meget lille del af variationen. Sammenholdes det med \autoref{fig:Boksplots}, hvorpå det fremgår at spredningen i testpersonernes besvarelser til de to skala spørgsmål er meget lille og centreret omkring midtpunktet. Dette kan betyde flere ting; 1) at de ikke er nogle parametre som har specielt stor betydning for danske rejsende, 2) at begge parametre indirekte måles ved et eller flere andre parametre, 3) at skalaen har påvirket testpersonerne til at afvige en respons omkring midtpunktet eller 4) at det simpelthen ikke har været muligt, at variere nok på afstanden og højden til at testpersonerne får en anderledes oplevelse. Det er dog stadig uvist, hvilken årsag det rent faktisk er.\blankline  
%
Udover de beskrevne korrelationer forekommer der også forskellige tendenser, når information fra demografien sammenlignes med besvarelserne på skalaerne. Det blev fundet, at når robotten bliver højere opleves bevægelserne roligere og hastigheden langsommere, hvilket formentlig hænger sammen med når \textit{Double} er på sit laveste (118 cm) så kører den rent faktisk hurtigere end når den er på sit højeste (151 cm). 

Derudover er det fundet, at når robotten er på sit laveste (118 cm), så stoler testpersonerne mere på, at den følger dem hen til det sted de har valgt, end når robotten er på sit højeste (151 cm), jævnfør \autoref{fig:TendensHeightSQ13}. Lignende er tilfældet for hvor elegant og sød robotten opleves, hvor den opleves som værende mere sød og elegant når den er lav end når den er høj, jævnfør \autoref{fig:TendensHeightSQ17} og \autoref{fig:TendensHeightSQ19}. I forhold til robottens afstand oplever testpersonerne robotten som værende mere overraskende, når den kommer tæt på end når den er langt fra. Sammenholdes testpersonernes alder og højdeforskel med deres vurdering af robottens hastighed fremgår det, at desto større højdeforskellen er og desto yngre testpersonerne er, desto langsommere oplever de robottens hastighed. Dog fremgår det at 75 \% af testpersonerne besvarelser befinder sig mellem \textit{Alt for langsom} og \textit{Fin}, jævnfør \autoref{fig:Boksplots}, hvorfor det formentlig kan konkluderes, at robottens hastighed godt kan øges. I tillæg fremgår det, at desto ældre testpersonerne er desto mindre spændende og sjov opleves robotten, jævnfør \autoref{fig:AgeSQ18} og \autoref{fig:AgeSQ22}. Ydermere fremgår det af \autoref{fig:HeightRatioSQ19}, at desto større højdeforskellen er desto sødere opleves robotten. Af \autoref{fig:Gender} tyder det på, at den eneste kønsforskel der er fundet, hvor CI ikke overlapper, er i forhold til hvor anmassende robotten opleves, hvor mændende oplever robotten som værende mere anmassende end kvinderne.
