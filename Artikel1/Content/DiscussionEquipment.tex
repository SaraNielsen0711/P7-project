\subsection{Equipment}
%Her skrives noget om udstyr fx at skærmen ikke virkede ordentligt
Due to the Marvel wireframe running inside the double application, the touch screen was less responsive than expected. This resulted in some travellers not wanting further interaction with the robot. It also affected the experience for those who chose to interact with it anyway.

Controlling the robot in a busy area of the airport proved to be more difficult than expected. It was hard to control it via the camera, which meant that the controller had to do it by looking directly at the robot. It was controlled from a distance of approximately 10 meters, and sometimes travellers walked in front of the robot and obstructed the controller's visibility. Also, the controller had to be very cautious about the robot not running into anyone. This resulted in a somewhat hesitating behaviour of the robot. For example, subject 03 in the first stated that the robot appeared hesitating and that it should move smoothly and adapt its movement to the people surrounding it.\\
The scaling software written for the second test had some response issues as well which ultimately led to incomplete datasets because the program often failed to notice mouse-clicks. This is not ideal for a scale which relies on an immediate impulse response without much thought. This have almost definitely shifted some subjects' responses in either direction of the scale due to them clicking in another location on the scale than initially chosen.