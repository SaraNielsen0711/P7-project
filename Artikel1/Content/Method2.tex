\section{Method - Scale Testing}
\label{Method2}
The test was conducted over the course of two days and in the same way as the first test described in \autoref{MethodElicitation}. However, in this test instead of the robot leading them directly to an interviewer, it led them in the direction of their chosen location. After a short distance, an experimenter politely stopped them and informed them about the ongoing study. This was done to give the subjects a more natural experience, where the robot appeared to lead them in the right direction. After they accepted to participate, they were led to a PC nearby to rate their experience on the developed scales. The experimenter was ready to take notes while they rated. The order in which the scales were presented is described in \autoref{ResultsElicitation}. The physical variables of the robot were varied over the course of the test. This included the angle of approach, the robot's height (and speed), and the distance to the subjects. As a consequence of the study being ecologically performed in an actual airport it was not possible to precisely control the distance to the subjects or the angle of approach. These had to rely on subjective judgement from the researchers. Height was measured using a measuring tape. The robot height was set to either of the following five heights: 118 cm, 123.5 cm, 129 cm, 140 cm, or 151 cm. The first and last mentioned heights are dictated by the robot's minimum and maximum height, whereas 129 cm was chosen according to what the subjects in previous study mentioned as fine, which roughly corresponds to human elbow height according to the mean height of the Danish population. 

\subsection{Materials}
The same materials were used as those used in the first test described in \autoref{MethodElicitation} along with the same \textit{Double} and tilted head mount. The same wireframe was used except with a few adjustments. Additionally, software was developed in Processing 3.3.6 (\url{www.processing.org}) to be able to collect data from the scales. The program was presented on a \textit{Microsoft Surface Pro (5)} with a wireless mouse. The 24th scale was presented on a sheet of paper along with a few other questions to collect demographic information. 

\subsection{Scale Program}
The scale program consisted of 23 scales distributed on seven pages. The number of scales presented on a single page were maximum four and minimum two. The subjects were instructed to set a marker representing their response on the scales using the provided mouse. The scales were organised to be as consistent on each page as possible e.g. the same type of scales with a midpoint were presented on one page and internally the scales were organised such that two similar attributes did not appear right after each other.

\subsection{Subject Recruitment}
Over the course of two days 43 subjects participated. They were aged from 10 to 72 (M=40.1, SD=13.4) and distributed among 16 females and 27 males. Subjects reported that they travel between 1 and 100 times per year (M=15.3, SD=18.1). The subjects were again recruited by the robot with the same wireframe as in the previous test.

\subsection{Roles}
Four researchers were present during the test at AAL. One controlled the \textit{Double} robot, one instructed the subjects to answer the scales, one observed when the robot should start leading the subjects towards the shopping area and one noted the physical variables of the robot such as height, direction of approach, and distance to the subjects. As with the previous test the robot controller had no way to know what happened on the screen, one of the observers signaled to the robot controller when to start leading the subjects to towards the shopping area. 

\subsection{Data Processing}
Data was gathered in Processing 3.3.6 and for each subject data were saved in a .csv format and analysed using MATLAB, Excel, and RStudio. Data were analysed using boxplots and Principal Component Analysis (PCA), which were further analysed in order to compare correlated attributes. The PCAs were conducted on robot's height, distance, and direction separately in order to assess how the subjects responded to the physical changes of the robot.
