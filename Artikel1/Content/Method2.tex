\section{Method}
\label{Method2}
The test was conducted over the course of two days and otherwise as described in \autoref{MethodElicitation}. The location stayed the same along with the overall procedure. This time, in stead of being led to an interview by the robot, the subjects were led a small distance in the direction of most of the available options in the wireframe (Food, Shopping, Gates). Here an experimenter politely stopped them and informed them about the ongoing study. After they accepted to participate, they were led to a PC nearby to rate their experience on the developed scales . An observer was ready to take notes while they rated. The physical parameters of the robot were varied over the course of the test. This included: the angle of approach, the robot height (and speed), and the distance to the participant. As a consequence of the study being ecologically performed in an actual airport it was not possible to precisely control the distance to the participant and the angle of approach. These had to rely on subjective judgement on the researchers part. Height was measured using a measuring tape.

\subsection{Materials}
The same materials were used as described in \autoref{MethodElicitation} along with the same \textit{Double} and tilted headmount.  Additionally a program \fxnote{Software i stedet?} was developed in Processing (\url{www.processing.org}) to be able to collect data from the scales. The program was presented on a \textit{Microsoft Surface Pro (5)} with a simple wireless mouse. To collect demographic information a sheet of paper with a few questions was used.

\subsection{Scale Program}
The scale program consisted of 23 scales distributed on 7 pages. The maximum amount of scales presented on a single page was four and minimum two. The subjects were instructed to set a fitting marker on the scales using the provided mouse. The scales were ordered to be as consistent on each page as possible e.g. the same type of scale with a mid-point was presented on one page.

\subsection{Subject Recruitment}
Over the course of two days, 43 subjects participated aged from 10 to 72 (M=40.1, SD=13.4) distributed among 16 females and 27 males. The subjects were again recruited by the robot with the same wireframe as in the previous test.

\subsection{Roles}
Four researchers were present during the test at AAL. One controlled the \textit{Double} robot, one instructed the subjects to answer on the scales, one observed when the robot should start leading the subjects towards duty free and one noted the physical parameters of the robot such as height, direction of approach and distance to the subjects.

\subsection{{\color{red}Data Processing}}
Data was gathered via the scale program for each participant in a .csv format and analysed using both Matlab and R. It was analysed using multivariate Principal Component Analysis (PCA) and boxplots on the overall data. in order to assess how the participants responded to the physical changes of the robot they were generally categorised in three groups (three groups within age, height ratio, distance, direction etc.)