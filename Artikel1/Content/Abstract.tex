\label{Abstract}
% As a general rule, do not put math, special symbols or citations
% in the abstract
Social robots are expected to play a much bigger role in the near future. This calls for research in determining how these social robots are supposed to behave. This paper presents an ecological field study and investigates the subjective experience of social robots in Aalborg Airport (AAL). Travellers were recruited by a remote controlled robot from Double Robotics, Inc., which had a tablet with an interface asking if it may help them with wayfinding in AAL. When the participants had chosen their location they were asked to follow the robot and led to a semi-structured interview about their first impressions. In total the study includes 30 participants from 8 to 62 years (M=37.9, SD=17.1). The observations and the participants' statements were coded using an Affinity Diagram. 567 affinity notes were sorted and ended up with 10 categories of which the main categories revolved around appearance, behaviour, approach and trust. 


%method
%result
%conclusion
