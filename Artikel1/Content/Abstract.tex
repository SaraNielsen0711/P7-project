\label{Abstract}
% As a general rule, do not put math, special symbols or citations
% in the abstract
%Social robots are expected to play a much bigger role in the near future. This calls for research in determining how these social robots are supposed to behave. 
This paper investigates the subjective experience of a social robot at Aalborg Airport (AAL) by conducting an ecological field study. Travellers were recruited by a remote controlled robot from Double Robotics, Inc., which had an iPad with an interface asking if it may help the travellers with wayfinding at AAL. When the subjects had chosen the desired location they were asked to follow the robot, which led them to a semi-structured interview about their experience with the robot. The behaviour of the subjects were observed throughout the interaction with the robot and the interview. 

The observations and the subjects' statements was interpreted and coded using an affinity diagram. 567 affinity notes were sorted by a bottom-up procedure into ten categories of which four categories revolved around appearance, trust, behaviour,  and approach. For each of the four categories variables were formulated as scale questions which will be used in an upcoming study. 

%method
%result
%conclusion
