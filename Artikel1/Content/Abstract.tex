\label{Abstract}
% As a general rule, do not put math, special symbols or citations
% in the abstract
%Social robots are expected to play a much bigger role in the near future. This calls for research in determining how these social robots are supposed to behave. 
This paper investigates the subjective experience of social robots in Aalborg Airport (AAL) by conducting an ecological field study at the airport. Travelers were recruited by a remote controlled robot from Double Robotics, Inc., which had a tablet with an interface asking if it might help the travelers with wayfinding at AAL. When the participants had chosen the desired location they were asked to follow the robot and led to a semi-structured interview about their experience with the robot. The behavior of the participants were observed throughout the interaction with the robot and the interview. 

The observations and the participants' statements were interpreted and coded using an affinity diagram. 567 affinity notes were sorted by bottom-up procedure into 10 categories of which the main categories revolved around appearance, behavior, approach and trust. For each of the four categories variables was formulated in to scale question witch will be used in a upcoming study. 

%method
%result
%conclusion
