\label{Abstract}
% As a general rule, do not put math, special symbols or citations
% in the abstract
%Social robots are expected to play a much bigger role in the near future. This calls for research in determining how these social robots are supposed to behave. 
This paper investigates the subjective experience of interacting with a social robot at Aalborg Airport (AAL) by first conducting an ecological field study where 24 important variables in Human Robot Interaction (HRI) are elicitated. Afterwards scales are developed from these variables and tested in AAL. 

Travellers were in both tests recruited by a remote controlled Double robot, which had an iPad with an interface asking if it may help the travellers with wayfinding at AAL. When the subjects had chosen the desired location they were asked to follow the robot, which in the first test led them to a semi-structured interview about their experience and in the second test to an experimenter who asked them to rate their experience on the developed scales. 


In the first test the observations and the subjects' statements were interpreted and coded using an affinity diagram. 567 affinity notes were sorted by a bottom-up procedure into ten categories from which the 24 scale questions and scales were developed. 

The scales were used in a test at AAL, where 43 subjects rated the robot and the interaction on the 24 scales. The ratings were analysed with Principal Component Analysis (PCA), which showed both positive and negative correlation within the groups \textit{robot's height}, \textit{distance}, and \textit{direction}. 


%This paper investigates the subjective experience of a social robot at Aalborg Airport (AAL) by conducting an ecological field study. 
%
%{\color{red} Her tilføjes en kort beskrivelse af at der er udført to test og hvorfor. }
%
%Travellers were recruited by a remote controlled robot from Double Robotics, Inc., which had an iPad with an interface asking if it may help the travellers with wayfinding at AAL. When the subjects had chosen the desired location they were asked to follow the robot, which led them to a semi-structured interview about their experience with the robot. The behaviour of the subjects were observed throughout the interaction with the robot and the interview. 

%The observations and the subjects' statements was interpreted and coded using an affinity diagram. 567 affinity notes were sorted by a bottom-up procedure into ten categories of which four categories revolved around appearance, trust, behaviour,  and approach. For each of the four categories variables were formulated as scale questions which will be used in an upcoming study. 

%{\color{red} Her tilføjes beskrivelse af overgangen fra resultaterne fra test 1 til opstilling af test 2. Derefter beskrives test 2 og resultaterne her fra opsumeres.}

%method
%result
%conclusion
