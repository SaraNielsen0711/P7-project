\section{Results}
\label{Results}
%
Of the 10 green categories we chose to exemplify four categories which are related to the robot’s appearance (udseende), trust (tillid), behaviour (væremåde), and approach (henvendelse). Based on these categories a variable can be elicitated according to the criterion of a) being an adjustable variable and b) the possibility of formulating the variables as a scale question. Because the field study is conducted with Danish speaking test subjects the variables are listed in both English and Danish. For each of the four main categories the following variables can be elicitated and formulated as a scale question:\\
%
\begin{itemize}
\item Appearance
\begin{enumerate}
  \item I think R's height is (Jeg synes at R's højde er)
  \item I think R is elegant (Jeg synes R er elegant)
  \item I think R looks human (Jeg synes R ser menneskelig ud)
  \item I like R's appearance (Jeg kan godt lide R's udseende)\\
\end{enumerate}
\item Trust 
\begin{enumerate}
  \item I feel safe around R (Jeg føler mig tryg ved R)
  \item I count on R to follow me to the right place (Jeg regner med at R følger mig hen til det sted jeg har valgt)
  \item R scared me (R gjorde mig forstrækket)\\
\end{enumerate}
\item Behaviour
\begin{enumerate}
  \item I think R's movements are (Jeg synes R's bevægelser er)
  \item I think R's speed is (Jeg synes at R's hastighed er)
  \item I think R is annoying (Jeg synes R er irriterende)
  \item I think R is alive (Jeg synes R er levende)
  \item I think R is intrusive (Jeg synes R er anmassende)\\
\end{enumerate}
\item Approach 
\begin{enumerate}
  \item I think R is accommodating (Jeg synes R er imødekommende)
  \item I thought that R came too close to me (Jeg synes R kom for tæt på)
  \item I thought that R was obstructive (Jeg synes R stod i vejen)
  \item I was surprised by R's approach (Jeg blev overrasket over R's henvendelse)
  \item I thought that R was intimidating (Jeg synes R er intimiderende)\\
\end{enumerate}
\end{itemize}
%
In the above scale questions \textit{R} is short for robot. When comparing the variables for HRI found in this study with variables for HRI from previous conducted studies on social robots \cite{PDF:ExploringInfluencingVariable}, \cite{PDF:SharingALifeHarvey}, \cite{PDF:InTheCompanyofRobots}, \cite{PDF:CloseButNotStuck}, \cite{PDF:TheImpactOfTraveler}, \cite{PDF:HumanRobotEmodiedInteraction}, \cite{PDF:RecommendationEffects}, variables like distance, anthropomorphism, height, speed, movement, trust, and usefulness recur. New variables found related to appearance, in this study compared with previous mentioned studies, is elegance. According to behavior the new variables found is how annoying and intrusive the robot is, and how calm or wild the movements of the robot are. Focusing on approach the new variables are how accommodating, obstructive, surprising, and intimidating the robot is perceived. There is only found one new variable which influence trust that is if the robot scared you. In the following tabels each individual scale question, noted with \textit{SQ}, is listed alongside the labels on the specific scale. If the scale does not contain a mid point it will be noted with \textit{-}, if it contains an unlabeled mid point it will be noted with \textit{No label}, whereas if the mid point is label the specific label is noted. 
%
\begin{table}[H]
	\centering
\caption{Specific scale labels for each scale question concerning the robots appearance.}
	\label{tab:AppearanceScale} 
	\begin{tabular}{l|c|c|c}
		SQ     & Left label & Mid point & Right label \\\hline
		1   & \makecell{Too low \\(for lav)}  & \makecell{Appropriate \\(fin)} & \makecell{Too high \\(for høj)}        \\\hline
		2   & \makecell{Not at all elegant \\(slet ikke elegant)} & - & \makecell{Extremely elegant \\(ekstremt elegant)}         \\\hline
		3   & \makecell{Not at all human \\(slet ikke menneskelig)} & - & \makecell{Extremely human \\(ekstremt menneskelig)}         \\\hline
	 	4   & \makecell{Completely disagree \\(helt uenig)} & - & \makecell{Completely agree \\(helt enig)}         \\\hline
		5   & \makecell{Not at all scared \\(slet ikke forskrækket)} & -  & \makecell{Extremely scared \\(ekstremt forskrækket)}           
	\end{tabular}        
\end{table}
\noindent
%
%
\begin{table}[H]
	\centering
\caption{Specific scale labels for each scale question concerning the users trust in regard of the robot.}
	\label{tab:TrustScale} 
	\begin{tabular}{l|c|c|c}
		SQ  & Left label & Mid point & Right label \\\hline
		1   & \makecell{Extremely unsafe\\ (ekstremt utryg)} & No label & \makecell{Extremely safe \\(ekstremt tryg)}          \\\hline
		2   & \makecell{Completely disagree \\(helt uenig)} & Neutral &\makecell{Completely agree \\(helt enig)} 
	\end{tabular}        
\end{table}
\noindent
%
%
\begin{table}[H]
	\centering
\caption{Specific scale labels for each scale question concerning the robots behavior.}
	\label{tab:BehaviorScale} 
	\begin{tabular}{l|c|c|c}
		SQ     & Left label & Mid point & Right label \\\hline
		1   & \makecell{Too calm \\(for rolige)} & No label & \makecell{Too wild \\(for vilde)}           \\\hline
		2   & \makecell{Too slow \\(for langsom)} & \makecell{Appropriate \\(fin)} & \makecell{Too fast \\(for hurtig)}   \\\hline
		3   & \makecell{Not at all annoying \\(slet ikke irriterende)} & - & \makecell{Extremely annoying \\(ekstremt irriterende)}          \\\hline
	 	4   & \makecell{Not at all alive \\(slet ikke levende)} & - & \makecell{Extremely alive \\(ekstremt levende)}         \\\hline
		5   & \makecell{Not at all intrusive \\(slet ikke anmassende)} & - & \makecell{Extremely intrusive \\(ekstremt anmassende)}             
	\end{tabular}
\end{table}
\noindent
%
%
\begin{table}[H]
	\centering
\caption{Specific scale labels for each scale question concerning the robots approach.}
	\label{tab:ApproachScale} 
	\begin{tabular}{l|c|c|c}
		SQ     & Left label & Mid point & Right label \\\hline
		1   & \makecell{Very rejective \\(meget afvisende)} & No label & \makecell{Very accommodating \\(meget imødekommende)}          \\\hline
		2   & \makecell{Too far \\(for langt væk)} & \makecell{Appropriate \\(tilpas)} & \makecell{Too close \\(for tæt på)}          \\\hline
		3   & \makecell{Not at all obstructive \\(slet ikke i vejen)}& -  & \makecell{Extremely obstructive \\(ekstremt i vejen)}  \\\hline
	 	4   & \makecell{Not at all surprised \\(slet ikke overrasket)} &  -  & \makecell{Extremely surprised \\(ekstremt overrasket)}       \\\hline
		5   & \makecell{Not at all intimidating \\(slet ikke intimiderende)} & - & \makecell{Extremely intimidating \\(ekstremt intimiderende)}           
	\end{tabular}
\end{table}
\noindent
%
The scale questions can either be presented on a bi- or unipolar \textit{Visual Analoge Scale} (VAS) with open anchor points with or without mid point markers. If the scale is bipolar a midt point will be marked either with or without a labeled. Each scale consists of its own specific labels which are listed in \autoref{tab:AppearanceScale} for robot appearance, \autoref{tab:BehaviorScale} for robot behavior, \autoref{tab:ApproachScale} for robot approach, and \autoref{tab:TrustScale} for robot trust. In writing, the scales have not been properly developed but are expected to appear as shown on \autoref{fig:Height} for a bipolar scale with labeled mid point. 
%
\begin{figure}[H]
\centering
\includegraphics[width = 0.49\textwidth]{Figure/HeightHoejde} 
\caption{Example of a bipolar scale with a labeled mid point relevant for the scale question: \textit{I think R's height is}.}
\label{fig:Height}
\end{figure}
\noindent
% 
A bipolar scale without a label is expected to appear as shown on \autoref{fig:Calm}.  
%
\begin{figure}[H]
\centering
\includegraphics[width = 0.49\textwidth]{Figure/CalmWild} 
\caption{Example of a bipolar scale without a labeled mid point relevant for the scale question: \textit{I think R's movements are}.}
\label{fig:Calm}
\end{figure}
\noindent
%
An example of an unipolar scale is illustrated on \autoref{fig:HumanMenneskelig}.
%
\begin{figure}[H]
\centering
\includegraphics[width = 0.49\textwidth]{Figure/HumanMenneskelig} 
\caption{Example of an unipolar scale relevant for the scale question: \textit{I think R looks human}.}
\label{fig:HumanMenneskelig}
\end{figure}
\noindent
%
