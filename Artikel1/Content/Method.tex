\section{Method}
\label{Method}
%
In order to understand people's experiences and the words they use to describe the interaction with a social robot, travelers interacting with a robot in Aalborg Airport was investigated. In total 30 travelers, including 16 women and 14 men, was interview during 18 interviews. 11 interview were done on a single persons, where the remaining seven was done on groups of two or more. The participants' ages range from 8 to 62 years (M=37.9, SD=17.1) and have all been traveling more than once.

During the tests a double robot was used. Double is basically a segway merged with an iPad and in this study a new headmount was used, so the iPad was tilted a little upwards towards the participants, see figure (indsæt billede af double og referencen hertil). On the iPad travelers was shown a wireframe, intended to help them find a location in the airport of their choosing. In order to create a natural experience, the robot was used to recruit participants by asking them if it could help them find their was around Aalborg Airport. When travellers were willing to participate the robot led them to the interviewer instead of the chosen gate or restaurant. By doing this, the interviewer could start the interview by asking participants how their first meeting with the robot was without having to first set the scene for the participants. 

By doing a contextual field study, including interviews and observations, the experience is captured. Afterwards the interviews are transcribed and affinity notes are made for building an affinity diagram. By building an affinity diagram these notes can be sorted into a hierarchy of different categories and subcategories, which will tell the users perspective of the interaction. The affinity diagram represent some of the main areas that are important for travelers when interacting with the robot. These areas are used to create scales, which then are tested with new users in the airport. After gathering data on the chosen parameters, it can then be evaluated in which degree the different attributes contribute to the overall experience of the interaction, and how they relate to each other.
