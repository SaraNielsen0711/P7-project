\section{Method}
\label{Method}
%
The purpose of this study is to elicitate the variables that have an affect on the interaction between Danish travellers and a social robot in a natural setting. 


The underlying purpose of the study is to observe\fxnote{formålet er vel ikke at observere men mere at høst parametre? Vi bruger observation som er værktøj til at vurdere hvilke parapetre?-JN} potential users interacting with a dummy robot in a setting and environment which resemble the use situation. 



In this case the setting is in Aalborg Airport (AAL) and the potential users are the travellers. The travellers who interact with the robot are asked to participate in a semi-structured interview about their first impressions while being observed. 

\subsection{Materials}
For the study a Double 1 robot from Double Robotics Inc. with an iPad Air is used. The iPad has a wireframe interface from marvel (\url{www.marvelapp.com}) made to look like a wayfinding tool at the airport. The Double 1 was connected to a PC via WiFi connection and its own software. The Double 1 software allows a link to be displayed containing a web-page with the wireframe on the iPad while the observer still is able to control the Double 1 from the PC. It is not possible to control the Double 1 when an application other than the Double 1 software is running.

\subsection{Subject Recruitment}
In total the study includes 30 participants from 8 to 62 years (M=37.9, SD=17.1). The subjects are recruited by the robot itself which is remote controlled by an observer present. The observer who controls the robot let he robot approaches the potential subjects. The wireframe on the iPad asks the subjects if it may help them find their way around AAL and presents a "Yes" or "No" option. If "Yes" is pressed, the subjects are presented with four wayfinding options: Food, Shopping, Toilets, or Gate. The subjects are then asked to kindly follow the robot after they have chosen their preferred option. The robot then leads the subjects to an observer who shortly informs them of the study and receives verbal consent to record audio during a semi-structured interview.
 
 \subsection{Semi-structured Interview}
 The interview was semi-structured in two parts. The first part consisted of probing the subjects for their first impression and experience of interacting with the robot. The second part consisted of asking specific questions relating to physical parameters like height, approach, and speed. The two parts was conducted in continuation of each other and the questions in the second part were only asked if the subjects had not previously answered them by themselves.
 
\subsection{Observation}
Five researchers were present during the study at AAL. One controlled the Double 1 robot, one conducted the interview and the rest observed the travelers as they interacted or walked passed the robot. 
%One observer also acted as the interviewer when the Double 1 successfully guided subjects to the allotted interview area after the security check. 

\subsection{Data Processing}
The interviews were transcribed and coded along with observations into affinity notes. The purpose is to create an affinity diagram \cite{Wendell2005} which brings insights and issues into a wall-sized hierarchical diagram based on subjects' statements and behavior. In the end this Affinity Diagram will be pivotal in creating the scales to be used for further work.




%Real travellers were observed interacting with a robot in AAL and interviewed in order to get a sense of peoples' experience and the words they use to describe interacting with a robot in an airport. In total 30 travellers, including 16 women and 14 men, was interviewed during 18 interviews. 11 interviews were performed including a single traveller, where the remaining seven was done on groups of two or more. The participants' ages range from 8 to 62 years (M=37.9, SD=17.1) and have all been travelling more than once.
%
%During the tests a Double robot was used from Double Robotics Inc. Double is basically a segway merged with an iPad and in this study a new head mount was used, so that the iPad was tilted upwards towards the participants, see figure (indsæt billede af double og referencen hertil). Travellers were shown a wireframe on the iPad, intended to help them find a location in the airport of their choosing. In order to create a natural experience, the robot was used to recruit participants by asking them if it could help them find their way around AAL. When travellers were willing to participate the robot led them to the interviewer instead of the chosen gate or restaurant. By doing this, the interviewer could start the interview by asking participants how their first meeting with the robot was without having to first set the scene for the participants and interfering with the natural first impression.
%
%The user experience is then documented like a contextual field study with observations and a semi-structured interview. The interviews are then transcribed and affinity notes are made for building an affinity diagram. By building an affinity diagram these notes can be sorted into a hierarchy of different categories and subcategories, which will tell the user's perspective of the interaction. The affinity diagram represent some of the main areas that are important for travellers when interacting with the robot. These areas are used to create scales, which then are tested with new users in the airport. After gathering data on the chosen parameters, it can then be evaluated in which degree the different attributes contribute to the overall experience of the interaction and how they relate to each other.
