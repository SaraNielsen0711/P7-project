\section{Method}
\label{Method}
%
In order to understand people's experiences and the words they use to describe the interaction with a social robot, travelers interacting with the double robot in Aalborg Airport was investigated. In total 30 travelers, including 16 women and 14 men, was interview during 18 interviews. 11 interview were done on a single persons, where the remaining seven was done on groups of two or more. The participants' ages range from 8 to 62 years (M=37.9, SD=17.1) and have all been traveling more than once.


Investigating people's experiences and the words they use to describe the interaction with a social robot, to create user relatable words/scales for future evaluation of the interaction with social robots. Evaluating the scales with new users.
During the tests, a double robot is used, which is basically a segway merged with an iPad. On the iPad, the travellers will see a wireframe, intended to help them find a location in the airport of their choosing.
In order to create a natural experience, the robot is tested in its intended context; the airport. By doing a contextual field study, including interviews and observations, the experience is captured. The interviews is transcribed and affinity notes are made. Using affinity diagrams, these notes can be sorted into a hierarchy of different categories and subcategories, which tells the user’s story.
The output of the affinity diagram represent some of the main areas that people talked about after interacting with the robot. These are used to create scales, which are then tested with new users in the airport. After gathering data on the chosen parameters, it can then be evaluated in which degree the different attributes contribute to the overall experience of the interaction, and how they relate to each other.