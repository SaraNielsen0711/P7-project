\section{Part II - Scale Testing}
\label{MethodScaleTesting}
%{\color{red} Metoden for anden test beskrives i dette afsnit. Hvis der er gentagelser ift første test vil der blot laves en henvisning til det forrige metode afsnit}
This section describes the second part of the study where the scales were tested at AAL. The purpose was to get the travellers to rate their experience with the interaction on the developed scales. Afterwards the results might show if any scales are redundant or needs to be reformulated.
%

\section{Method}
\label{Method2}
The test was conducted over the course of two days and otherwise as described in \autoref{MethodElicitation}. The location stayed the same along with the overall procedure. This time, instead of being led to an interview by the robot, the subjects were led a small distance in the direction of Duty Free where most of the available options in the wireframe were (Food, Shopping, Gates). Here an experimenter politely stopped them and informed them about the ongoing study. After they accepted to participate, they were led to a PC nearby to rate their experience on the developed scales. The experimenter was ready to take notes while they rated. The order in which the scales were presented is described in \autoref{ResultsElicitation}. The physical parameters of the robot were varied over the course of the test. This included: the angle of approach, the robot height (and speed), and the distance to the participant. As a consequence of the study being ecologically performed in an actual airport it was not possible to precisely control the distance to the participant or the angle of approach. These had to rely on subjective judgement on the researchers part. Height was measured using a measuring tape.

\subsection{Materials}
The same materials were used as described in \autoref{MethodElicitation} along with the same \textit{Double} and tilted headmount. Additionally, software was developed in Processing (\url{www.processing.org}) to be able to collect data from the scales. The program was presented on a \textit{Microsoft Surface Pro (5)} with a simple wireless mouse. The 24th scale was presented on a sheet of paper along with a few questions to collect demographic information.

\subsection{Scale Program}
The scale program consisted of 23 scales distributed on 7 pages. The number of scales presented on a single page were maximum four and minimum two. The subjects were instructed to set a fitting marker on the scales using the provided mouse. The scales were organised to be as consistent on each page as possible e.g. the same type of scales with a mid-point were presented on one page.

\subsection{Subject Recruitment}
Over the course of two days 43 subjects participated. They were aged from 10 to 72 (M=40.1, SD=13.4) and distributed among 16 females and 27 males. Subjects reported that they travel between 1 and 100 times per year (M=15.3, SD=18.1). The subjects were again recruited by the robot with the same wireframe as in the previous test.

\subsection{Roles}
Four researchers were present during the test at AAL. One controlled the \textit{Double} robot, one instructed the subjects to answer the scales, one observed when the robot should start leading the subjects towards Duty Free and one noted the physical parameters of the robot such as height, direction of approach and distance to the subjects.

\subsection{Data Processing}
Data was gathered via the scale program for each participant in a .csv format and analysed using both Matlab and R. It was analysed using multivariate Principal Component Analysis (PCA) and boxplots on the overall data. The subjects were generally categorised in three groups (three groups within age, height ratio, distance, direction etc.) in order to assess how the participants responded to the physical changes of the robot \fxnote{rettes til}.

%\subsection{Materials}
%\label{MaterialsScaleTesting}
%%
%The same \textit{Double} robot with the same headmount and wireframe was used in this test. Besides that a \textit{Microsoft Surface Pro (5)} computer was used to show and collect answers from the 23 scales. To collect demographic information a sheet of paper with a few questions was used.
%\subsection{Subject Recruitment}
%43 subjects from the age of 10 to 72 years (M=40.1, SD=13.4) distributed among 16 females and 27 males participated. The subjects were recruited with the same procedure and with the same questions in this test as in the previous. 
%\subsection{Roles}
%Four researchers were present during the test at AAL. One controlled the \textit{Double} robot, one got the subjects to answer on the scales, one kept an eye on when the robot should start leading the subjects towards duty free and one noted the physical parameters of the robot.
%\subsection{{\color{red}Data Processing}}

