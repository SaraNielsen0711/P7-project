\subsection{Labels and scale questions}
%
It appears that there was some confusion regarding some of the scales e.g. where SQ05 was misinterpreted. The question was meant as to how close the robot stopped in relation to the subject but some subjects thought it was meant in relation to their chosen location which they obviously could not answer. A few also commented on the attributes funny and cute. Some subjects expressed that the robot was funny, but not humorous funny and therefore had trouble rating the scale question because of this duality. Another subject commented that she refused to rate the robot as cute, because she regarded it as a human quality and that she refused to humanise the robot.

For the scale rating to SQ02, none of the subjects rated under 50 \%. The half of the scale regarding unwelcoming might therefore be redundant. Instead an unipolar scale regarding welcoming could be developed and this would give a wider range to answer on the scale to the attribute welcoming.

The low variance and centering around the midpoint on some of the scales might be that the subjects did not find the scale questions relevant or did not understand them and therefore chose a neutral rating. It might also be that the selected stimuli was not varied enough. This could be reinforced by the midpoints showing the subjects were the middle is. If the ratings are constantly located at the same point, they do not contain much information about how different variables and parameters affect the experience of interacting with the robot.