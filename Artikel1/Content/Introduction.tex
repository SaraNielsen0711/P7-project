\section{Introduction}
\label{Introduction}
%
This study originated from a social robot research project at Aalborg University with the aim of developing and implementing robots in a variety of contexts. This raises technological, normative, and empirical questions as to how these robots are to behave and in which settings they might be useful. It is necessary to assess the user experience in these specific situations and design for the people interacting with the robot in order to make the robots interact appropriately in the given context. Several studies have already shown that both utilitarian and hedonic variables influence the way the users interact with social robots \cite{PDF:ExploringInfluencingVariable}, \cite{PDF:SharingALifeHarvey}, and few have already conducted research on social robots in an airport setting \cite{PDF:InvestigatingPositioning}. Utilitarian variables involves the utility, practicality, and usability of the robot \cite{PDF:ExploringInfluencingVariable}, whereas hedonic variables reflect the user experience, enjoyment, and acceptance while interacting with the robot. Furthermore some of the cultural differences that are found to affect the interaction with a social robot are the willingness to accept robots \cite{PDF:InTheCompanyofRobots}, and the intimate distance which is smaller for southern Europeans and Japanese compared to Americans and northern Europeans \cite{PDF:HowMayIServeYou}. Gender also affect how one experience the human-robot interaction (HRI), with females being less willing to accept and interact with robots than males, who also perceive the robots as more human-like compared to females \cite{PDF:ExploringInfluencingVariable}.

It is unknown whether or not these results are applicable to Danish travellers because the influencing variables vary, depending on cultural and contextual differences. Since the users have different experiences and might articulate them differently depending on both the context and their culture, it is advantageous to use their own words to elicitate which variables are important.

This study aims to develop relevant scales based on the elicitated variables which can be used by designers when developing the robots to a specific context and afterwards when evaluating the user experience in that context. These scales should be based on the users statements, which is obtained in the desired context.



%To increase acceptance of the robots, it is necessary to investigate the user experience concerned with the interaction with these robots.
%[\textbf{Exploring influencing variables.., 2013}],  [\textbf{Sharing a life with Harvey, 2015}] 

%develop usable scales based on observations and the participants' own statements about their first impressions.

%This is something that is hard to do in a controlled laboratory setting and is why this paper presents an ecological ethnographic study conducted with a remote controlled robot from Double Robotics, Inc. in Aalborg Airport (AAL). 

%The field of technology and robotics are moving faster for every day. So is the prospect of having to engage with social robots in every day life. This raises technological, normative and empirical questions as to how these robots are to behave and in which settings they might be useful. These questions will likely have to be answered with interdisciplinary methods by joining cognitive psychology, phenomenology, and anthropology with social robotics and engineering. Social robotics needs Humanities in order to gain insights in human interaction and behaviour and as a means to guide the technological design of the robots themselves. In writing, a social robot for an Airport setting is being developed at Aalborg University and there seems to be missing a framework as to how such a robot should behave in a crowded area or simply how to approach travellers. Some research have tried to make a protocol for the acceptance of social robots by defining several important variables for the acceptance[\textbf{Exploring influencing variables.., 2013}], [\textbf{Sharing a life with Harvey, 2015}]. Both studies investigated the acceptance of social robots in the privacy of their own home. It is hard to know how big an influence culture or the setting has on these resulting variables though previous studies show that it can play a big role [\textbf{Mangler kilde her}]. Few have already conducted research on social robots in an airport setting [\textbf{PhD. Thesis, Michiel Joosse, 2017]} but it is not known whether or not these results are applicable to Danish travellers. It is important to define and document which parameters are important for Human Robot Interaction (HRI) in the Airport. This is something that is hard to do in a controlled laboratory setting and is why this paper presents an ecological ethnographic study conducted with a remote controlled robot from Double Robotics, Inc. in Aalborg Airport (AAL). The robot appeared to help with wayfinding in AAL and the purpose was to develop usable scales based on observations and the participants' own statements about their first impressions.




% Idéer fra Lucca
%Kort introduktion af kugle-robotten og at det er den vi vil undersøge parametre omkring
% Der er brug for ny teknologi til at gøre lufthavnsopelvelsen bedre.
%Karakteristika ve den social robot
%Sociale robotter kan hjælpe med selskab og stress i dagligdagen, så måske de også kan hjælpe i lufthavnen.
%Vi er interesseret i hvilke parametre, der er relevante ved en social robot, for derefter at udvikle og teste skalaer bygget på disse parametre.
%Kulturelle forskelle (Vi kigger på danskere, blandt andet fordi vi i den vestlige verden er mere bange for robotter end i asien. Der kan derfor være andre parametre der spiller ind, end dem vi finder i literaturen.)

%Social robots in airports, helping people, wayfinding, measuring subjective experience and stuff like that

%\subsection{Motivation}
%- Why robots?
%Karl’s robot
%Exploring disruptive technology
%Robots solves real business problems
%Robots is the future
%
%- Why airports?
%Busy and often overcrowded area. Overwhelming amount of information
%
%- Airport business goals
%Enhancing traveler experience
%Keeping people in sales area and away from gates and hallways to increase sales and decrease chaos.
%Being innovative and using disruptive technology to appear more modern and futuristic.
%
%\subsection{Problem framing}
%- What do we want to do and how do we want to do it?
%Cultural difference between how people experience and express themselves about robots, therefore we start out by finding out how danes talk about robots and which parametres are important in social robots. Afterwards scales will be designet, on which test subjects will be able to rate their experience with the robot. 