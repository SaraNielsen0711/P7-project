\section{Introduction}
\label{Introduction}
% Idéer fra Lucca
%Kort introduktion af kugle-robotten og at det er den vi vil undersøge parametre omkring
% Der er brug for ny teknologi til at gøre lufthavnsopelvelsen bedre.
%Karakteristika ve den social robot
%Sociale robotter kan hjælpe med selskab og stress i dagligdagen, så måske de også kan hjælpe i lufthavnen.
%Vi er interesseret i hvilke parametre, der er relevante ved en social robot, for derefter at udvikle og teste skalaer bygget på disse parametre.
%Kulturelle forskelle (Vi kigger på danskere, blandt andet fordi vi i den vestlige verden er mere bange for robotter end i asien. Der kan derfor være andre parametre der spiller ind, end dem vi finder i literaturen.)
Social robots in airports, helping people, wayfinding, measuring subjective experience and stuff like that

\subsection{Motivation}
- Why robots?
Karl’s robot
Exploring disruptive technology
Robots solves real business problems
Robots is the future

- Why airports?
Busy and often overcrowded area. Overwhelming amount of information

- Airport business goals
Enhancing traveler experience
Keeping people in sales area and away from gates and hallways to increase sales and decrease chaos.
Being innovative and using disruptive technology to appear more modern and futuristic.

\subsection{Problem framing}
- What do we want to do and how do we want to do it?
Cultural difference between how people experience and express themselves about robots, therefore we start out by finding out how danes talk about robots
