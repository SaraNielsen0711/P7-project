\section{Conclusion}
\label{Conclusion}
%
The research conducted in this study reveals ten main categories relating to HRI. Variables were found for each of these ten categories which describe the user experience of interacting with a social robot at Aalborg Airport (AAL). The observations and the 30 subjects' statements were interpreted and coded using an affinity diagram. 567 affinity notes were sorted by a bottom-up procedure into ten categories which revolved around appearance, trust, behaviour, approach, problems with touch screen, avoidance of interaction, personal interest, positivity towards the robot, usefulness, and tech-experience.

New variables within these categories were discovered compared to previous conducted studies: Within Appearance, the variable elegant; within Trust, the variable startled; within Behaviour, how annoying and how intrusive the robot seemed along with how calm the movements of the robot was; within Approach, how welcoming, intrusive, and surprising the approach was and to what extent the robot was obstructing. Besides the newly found variables some variables were found to match previously found variables. These variables regards distance, anthropomorphism, height, speed, movement, trust, and usefulness.

24 variables are found and scales are developed from the found variables. Doing a PCA with different groupings on the scale ratings from the second test shows 10 instances of correlation relating to the height grouping, 10 instances of correlation relating to the distance grouping, and 9 instances of correlation relating to the direction grouping. 12 instances of correlation is found when comparing variables that correlates when doing PCA. From these correlations a few of the variables should be reevaluated due to redundancy.

Furthermore the label fine, the use of midpoints and scale questions as "I think the robot stopped..." should be reconsidered to improve the users understanding of the scale and minimize misunderstandings. 


%\subsection{Future work}
%{\color{red} Dette afsnittet slettes helt, da det nu er udført}
%\sout{During the next month, the evaluation of the scales will take place at AAL. This involves creating the scales from the results of the affinity diagram, bringing the robot to the airport, letting travellers interact with it, measuring their experience on the developed scales, and analysing the results.}