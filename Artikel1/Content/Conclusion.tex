\section{Conclusion}
\label{Conclusion}
%
The research conducted in this study revealed 10 main categories relating to HRI. Attributes were found for each of these 10 categories which describe the user experience of interacting with a social robot at Aalborg Airport. The observations and the 30 subjects' statements were interpreted and coded using an affinity diagram. 567 affinity notes were sorted by a bottom-up procedure into 10 categories which revolved around appearance, trust, behaviour, approach, problems with the touch screen, avoidance of interaction, personal interest, positivity towards the robot, usefulness, and tech-experience.

New attributes within the 10 categories were discovered compared to previous mentioned studies: Elegance, cuteness, coolness, startling, excitement, welcoming, obstructive, whether the robot help is seen as personal, and whether the encounter with the robot was surprising. Besides the newly found attributes some attributes were found to match previously found attributes. These attributes regards distance, anthropomorphism, height, speed, movement, trust, and usefulness.

23 attributes were elicited and scales were developed from these. After the second test, 10 instances of correlations were found for both height and distance and 9 for direction based on PCA with different groupings. 12 instances of correlation were found when comparing attributes that correlates when doing PCA. From these correlations a few of the attributes should be reevaluated due to redundancy.

Furthermore the label fine, the use of midpoints and scale questions as "I think the robot stopped..." should be reconsidered to improve the users understanding of the scale and minimize misunderstandings. 


%\subsection{Future work}
%{\color{red} Dette afsnittet slettes helt, da det nu er udført}
%\sout{During the next month, the evaluation of the scales will take place at AAL. This involves creating the scales from the results of the affinity diagram, bringing the robot to the airport, letting travellers interact with it, measuring their experience on the developed scales, and analysing the results.}