\section{Conclusion}
\label{Conclusion}
%
The research conducted in this study reveals 10 categories of which four categories are exemplified. For each of these four categories there are found variables which describe the user experience of a social robot in Aalborg Airport (AAL): Appearance, Trust, Behaviour, and Approach.

New variables within the categories were discovered compared to previous conducted studies: Within appearance, the variable \textit{elegant (elegant)}; within behaviour, how \textit{annoying (irriterende)} and how \textit{intrusive (anmassende)} the robot seemed along with how \textit{calm (rolige)} the movements of the robot was; within approach, how \textit{welcoming (imødekommende)}, \textit{intimidating (intimiderende)}, and \textit{surprising (overraskende)} the approach was. Also to what extent the robot obstructed one's way \textit{obstructive (i vejen)}. Within trust only one new variable was elicited: \textit{scared (forskrækket)}. Besides the newly found variables some variables were found to match previously found variables. These variables regards distance, anthropomorphism, height, speed, movement, trust, and usefulness.

Each of the variables within the categories are formulated as scale questions which will be sorted and used in an upcoming study. The results of this study are context dependent and based on the specific location and test design. It is not expected to transfer the results to other contexts without further testing. 

The study is usable in development of a robot for AAL, where the variables found in this study might be used to evaluate the user experience of the robot. The variables and scales found in this study could also be tested to determine the robot design for optimal user experience.

\subsection{Future work}
During the next month, the evaluation of the scales will take place at AAL. This involves creating the scales from the results of the affinity diagram, bringing the robot to the airport, letting travellers interact with it, and measuring their experience on the developed scales and analyzing the results.

