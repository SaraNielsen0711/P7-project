\section{Conclusion}
\label{Conclusion}
%Possible implications and the impact
%https://explorable.com/writing-a-conclusion}
%New variables found under appearance, in this study compared with previous mentioned studies, is elegance. Under behavior the new variables found is how annoying and intrusive the robot is, and how calm and pleasant the movements of the robot are. Looking at approach the new variables include how inviting, intimidating and surprising the approach is, if it is either the person or the robot who approaches, and if the robot was blocking the way. Looking at trust only one new variable was discovered; if the robot scared you.


The research conducted in this study reveals four categories of variables found to describe the user experience of a social robot in Aalborg Airport (AAL): Appearance, Behaviour, Approach and Trust.

%udkast 1
New variables within the categories were discovered compared to previous conducted studies: Within Appearance, the variable \textit{elegant (elegant)}; within Behaviour, how \textit{annoying (irriterende)} and how \textit{intrusive (anmassende)} the robot seemed along with how \textit{calm (rolig)} and how \textit{pleasing (behagelig)} the movements of the robot was; Within Approach, how \textit{welcoming (imødekommende)}, \textit{intimidating (intimiderende)}, and \textit{surprising (overraskende)} the approach was and further whether it was preferred that the robot or oneself approached. Also to what degree the robot blocked one's way \textit{obstructive (i vejen)}. Within Trust only one new variable was revealed: \textit{scared (forskrækket)}

%udkast 2
%The variables \textit{elegant (elegant)}, \textit{annoying (irriterende)}, \textit{intrusive (anmassende)}, \textit{calm (rolig)}, \textit{ pleasant (behagelig)}, \textit{welcoming (imødekommende)}, \textit{intimidating (intimiderende)}, \textit{surprising (overraskende)}, \textit{obstructive (i vejen)}, \textit{scared (forskrækket)} were new compared to previous conducted studies. 
Each of the variables within the categories are formulated as scale questions which will be sorted and used in an upcoming study. The results of the study are context dependent and based on the specific location and test design. It is not expected to transfer the results to other contexts without further testing. 

The study is usable in development of a robot for AAL, where the variables found in this study might be used to evaluate the user experience of the robot. The variables and scales found in this study could also be tested to determine the robot design for optimal user experience .

%The conclusion aims to answer the following questions: \\
%What Has Your Research Shown?\\
%How Has It Added to What is Known About the Subject?\\
%What Were the Shortcomings?\\
%Has Your Research Left Some Unanswered Questions?\\
%Are My Results of Any Use in the Real World?

\subsection{Future work}
During the next month, the evaluation of the scales in the airport will take place. This involves creating the scales from the results of the affinity diagram, bringing the robot to the airport, letting people interact with it, and measuring their experience on the developed scales and analyzing the results

