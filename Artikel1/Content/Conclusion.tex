\section{Conclusion}
\label{Conclusion}
%
The research conducted in this study reveals ten categories of which four categories are exemplified. Variables were found for each of these four categories which describe the user experience of interacting with a social robot at Aalborg Airport (AAL). The categories were Appearance, Trust, Behaviour, and Approach. {\color{red} Rettes til med de efterfølgende ti kategorier og måske correlation}

New variables within these categories were discovered compared to previous conducted studies: Within Appearance, the variable \textit{elegant (elegant)}; within Trust, the variable \textit{startle (forskrækket)}; within Behaviour, how \textit{annoying (irriterende)} and how \textit{intrusive (anmassende)} the robot seemed along with how \textit{calm (rolige)} the movements of the robot was; within Approach, how \textit{welcoming (imødekommende)}, \textit{intimidating (intimiderende)}, and \textit{surprising (overraskende)} the approach was. Also to what extent the robot obstructed one's way \textit{obstructive (i vejen)}. Besides the newly found variables some variables were found to match previously found variables. These variables regards distance, anthropomorphism, height, speed, movement, trust, and usefulness.

{\color{red} Her fra og ned slettes og der skrives om resultaterne fra anden test i stedet. For efterfølgende at konkludere mere bredt/samlet på hele studiet (begge tests)}

\sout{Each of the variables within these categories are formulated as scale questions which will be sorted and used in an upcoming study. The results of this study are context dependent and based on the specific location and test design. It is not expected to transfer the results to other contexts without further testing. }

\sout{The study is usable in development of a robot for AAL, where the variables found might be used to evaluate the user experience. The variables and scales found in this study could also be used to determine a robot design for optimal user experience.}

%\subsection{Future work}
%{\color{red} Dette afsnittet slettes helt, da det nu er udført}
%\sout{During the next month, the evaluation of the scales will take place at AAL. This involves creating the scales from the results of the affinity diagram, bringing the robot to the airport, letting travellers interact with it, measuring their experience on the developed scales, and analysing the results.}