\section{Conclusion}
\label{Conclusion}
%
The research conducted in this study reveals ten main categories relating to HRI. Variables were found for each of these ten categories which describe the user experience of interacting with a social robot at Aalborg Airport (AAL). The observations and the 30 subjects' statements were interpreted and coded using an affinity diagram. 567 affinity notes were sorted by a bottom-up procedure into ten categories which revolved around appearance, trust, behaviour, approach, problems with touch screen, avoidance of interaction, personal interest, positivity towards the robot, usefulness, and tech-experience {\color{red} Virker det underligt, at det ikke er nævnt i artiklen før nu? og bør det nævnes i results?}

New variables within these categories were discovered compared to previous conducted studies: Within Appearance, the variable elegant (elegant); within Trust, the variable startle (forskrækket); within Behaviour, how annoying (irriterende) and how intrusive (anmassende)  the robot seemed along with how calm (rolige) the movements of the robot was; within Approach, how welcoming (imødekommende), intrusive (anmassende), and surprising (overraskende) the approach was. Also to what extent the robot obstructed one's way obstructive (i vejen). Besides the newly found variables some variables were found to match previously found variables. These variables regards distance, anthropomorphism, height, speed, movement, trust, and usefulness.

{\color{red} Her fra og ned slettes og der skrives om resultaterne fra anden test i stedet. For efterfølgende at konkludere mere bredt/samlet på hele studiet (begge tests)}


%\subsection{Future work}
%{\color{red} Dette afsnittet slettes helt, da det nu er udført}
%\sout{During the next month, the evaluation of the scales will take place at AAL. This involves creating the scales from the results of the affinity diagram, bringing the robot to the airport, letting travellers interact with it, measuring their experience on the developed scales, and analysing the results.}