\section{Method - Elicitation of attributes}
\label{MethodElicitation}
%
The first test was conducted on Danish travellers who interacted with a social robot in a natural setting. The test was conducted at Aalborg Airport (AAL) after the security check and before they reached the shopping and dining area at the airport. The travellers who interacted with the robot were asked to participate in a semi-structured interview about their first impressions while being observed during both the interaction and the interview. 

\subsection{Materials}
For the test a \textit{Double} robot from Double Robotics Inc, \cite{WEB:Double}, with an iPad Air 2, \cite{WEB:iPadAir2}, was used. Based on an small pretest it was decided to change the head mount so that the iPad is angled upwards, to make the robot appear more welcoming. The modified \textit{Double} robot is shown on \autoref{fig:ModificeretDoubleFront} and \autoref{fig:ModificeretDoubleSideClose}. The \textit{Double} robot was connected to a laptop via Wi-Fi connection and its own software. The \textit{Double} application allows a web page to be superimposed on the iPad screen while the controller is still able to control the robot from the laptop. The web page presented on the iPad was a wireframe developed in Marvel, (\url{www.marvelapp.com}), aiming to depict the potential usage of the robot as a wayfinding tool at AAL. The entire wireframe was formulated in Danish.

\subsection{Subject Recruitment}
30 subjects from the age of 8 to 62 years (M=37.9, SD=17.1) distributed among 16 females and 14 males participated. The subjects were recruited by the robot itself, which was remote controlled by a present controller, to provide a more ecological and undisturbed interaction between robot and subject. The robot approached potential subjects and the wireframe on the iPad asked them if it might help them find their way around AAL and presented a "Yes/No" option. If "No" was pressed, the robot wished the traveller a pleasent journey and left. If "Yes" was pressed, the subjects were presented with four wayfinding options: Food, Shopping, Toilets, or Gate information, as shown on \autoref{fig:brug}. The subjects were then kindly asked to follow the robot after they had chosen their preferred option. The robot then led the subjects to an interviewer who shortly informed them of the study and received verbal consent to record audio during the semi-structured interview. In total 18 interviews were conducted of which 11 were on single travellers and seven were on groups of travellers.
%
\begin{figure}[H]
\centering
\begin{minipage}{.25\textwidth}
  \centering
  \includegraphics[width=\linewidth, angle =-90]{Figure/ModificeretDoubleFront}
  \caption{\textit{Double}'s front.}
  \label{fig:ModificeretDoubleFront}
\end{minipage}%
\begin{minipage}{.25\textwidth}
  \centering
  \includegraphics[width=\linewidth, angle =-90]{Figure/ModificeretDoubleSideClose}
  \caption{\textit{Double}'s profile.}
  \label{fig:ModificeretDoubleSideClose}
\end{minipage}
\end{figure}
\noindent
%
%
\begin{figure}[H]
\centering
\includegraphics[width = 0.25\textwidth]{Figure/brug}
\caption{Options presented for the subjects in the interface.}
\label{fig:brug}
\end{figure}
\noindent
% 
\subsection{Semi-structured Interview}
The interview was a two part semi-structured interview. The first part consisted of probing the subjects for their first impression and experience of interacting with the robot in regard to their thoughts about the robot itself and what they think other travellers might think about the interaction. In addition to these conversation topics the subjects were asked about their opinion regarding the robots approach, usefulness, and reliability. Afterwards, the subjects were asked about previous experiences and problems at airports, in order to gather potential use cases where the robot might be helpful. 

%The last topic of conversation in the first part of the interview regards the subjects' previous experiences at an airport, to gather potential use cases for the robot.

%The following are only guidance to the conversation topics and not specific questions:\\ 
%
%\begin{itemize}
%\item First impression of the robot. SKREVET IND
%\item How the robot approached the subject. SKREVET IND
%\item What the subject think about the robot. SKREVET IND
%\item What the subject think other travellers think of their interaction with the robot. SKREVET IN
%\item Robot relevance. SKREVET IND
%\item Robot reliability. SKREVET IND
%\item Experiences at an airport where robot help might have been useful.\\
%\end{itemize}
%\noindent
%
The second part consisted of asking specific questions relating to the robots physical characteristics such as speed, height, distance, movements, appearance, and approach. These questions were asked because these attributes have been found to affect the experience of HRI \cite{PDF:HowMayIServeYou}.

The two parts were conducted in continuation of each other and the questions in the second part were only asked if the subjects had not previously answered them spontaneously.
 
\subsection{Roles}
Five researchers were present during the test at AAL. One controlled the \textit{Double} robot, one conducted the interview, and the remaining three observed the travellers as they interacted with or walked past the robot. 

\subsection{Data Processing}
The interviews were transcribed and coded along with observations into affinity notes. The purpose was to create an affinity diagram, which brings insights and issues into a hierarchical diagram based on subjects' statements and behaviour \cite{Wendell2005}. This affinity diagram is pivotal in eliciting the attributes that affect HRI, and thereafter in creating the scales to be used for further work. To include more of the gathered insigths, it was decided to use both the spontaneous answers from the conversation topics and the answers from the specific questions in the affinity diagram. 




%Real travellers were observed interacting with a robot in AAL and interviewed in order to get a sense of peoples' experience and the words they use to describe interacting with a robot in an airport. In total 30 travellers, including 16 women and 14 men, was interviewed during 18 interviews. 11 interviews were performed including a single traveller, where the remaining seven was done on groups of two or more. The participants' ages range from 8 to 62 years (M=37.9, SD=17.1) and have all been travelling more than once.
%
%During the tests a Double robot was used from Double Robotics Inc. Double is basically a segway merged with an iPad and in this study a new head mount was used, so that the iPad was tilted upwards towards the participants, see figure (indsæt billede af double og referencen hertil). Travellers were shown a wireframe on the iPad, intended to help them find a location in the airport of their choosing. In order to create a natural experience, the robot was used to recruit participants by asking them if it could help them find their way around AAL. When travellers were willing to participate the robot led them to the interviewer instead of the chosen gate or restaurant. By doing this, the interviewer could start the interview by asking participants how their first meeting with the robot was without having to first set the scene for the participants and interfering with the natural first impression.
%
%The user experience is then documented like a contextual field study with observations and a semi-structured interview. The interviews are then transcribed and affinity notes are made for building an affinity diagram. By building an affinity diagram these notes can be sorted into a hierarchy of different categories and subcategories, which will tell the user's perspective of the interaction. The affinity diagram represent some of the main areas that are important for travellers when interacting with the robot. These areas are used to create scales, which then are tested with new users in the airport. After gathering data on the chosen parameters, it can then be evaluated in which degree the different attributes contribute to the overall experience of the interaction and how they relate to each other.
