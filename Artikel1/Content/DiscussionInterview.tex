\subsection{Interview}
%Her disukterer vi interview metoderne der er anvendt 
When planning the interview, the idea was to let the subjects do most of the talking, in order to assess what was important to them and what was not. {\color{red} Sætningen ``not knowing'' skal skrives om, så det kommer til udtryk at det er fordi samtale er forskellig fra person til person. Nogle snakker meget andre næsten ikke - Måske inddrag erfaringer fra de udførte pilottests}Not knowing how this would work, it was decided to have some predetermined questions as a backup, in case some of the topics that seemed important in the literature remained unaddressed by the subjects. In reality, however, it became more like a semi-structured interview with the subjects waiting for the next question. This introduced some interviewer bias, since the predetermined questions dragged the conversation in a certain direction. That meant subjects talked a lot about a specific aspect of the interaction, because they were asked directly about it.