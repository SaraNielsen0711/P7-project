\makepagestyle{AAU}
\makepsmarks{AAU}{%
	\createmark{chapter}{left}{shownumber}{}{. \ }
	\createmark{section}{right}{shownumber}{}{. \ }
	\createplainmark{toc}{both}{\contentsname}
	\createplainmark{lof}{both}{\listfigurename}
	\createplainmark{lot}{both}{\listtablename}
	\createplainmark{bib}{both}{\bibname}
	\createplainmark{index}{both}{\indexname}
	\createplainmark{glossary}{both}{\glossaryname}
}
\nouppercaseheads
% Ingen Caps ønskes

\makeevenhead{AAU}{\groupname}{}{\leftmark}
% Definerer lige siders sidehoved {Navn}{Venstre}{Center}{Højre}

\makeoddhead{AAU}{\rightmark}{}{\universityname}
% Definerer ulige siders sidehoved {Navn}{Venstre}{Center}{Højre}

\makeevenfoot{AAU}{\thepage}{}{}
% Definerer lige siders sidefod {Navn}{Venstre}{Center}{Højre}

\makeoddfoot{AAU}{}{}{\thepage}
% Definerer ulige siders sidefod {Navn}{Venstre}{Center}{Højre}

\makeheadrule{AAU}{\textwidth}{0.5pt}
% Tilføjer en streg under sidehovedets indhold

\makefootrule{AAU}{\textwidth}{0.5pt}{1mm}
% Tilføjer en streg under sidefodens indhold

\copypagestyle{AAUchap}{AAU}
% Sidehoved for kapitelsider defineres som standardsider, men med blank sidehoved

\makeoddhead{AAUchap}{}{}{}
\makeevenhead{AAUchap}{}{}{}
\makeheadrule{AAUchap}{\textwidth}{0pt}

\aliaspagestyle{chapter}{AAUchap}
% Den ny stil vælges til at gælde for kapitler
															
\pagestyle{AAU}
% Valg af sidehoved og sidefod
