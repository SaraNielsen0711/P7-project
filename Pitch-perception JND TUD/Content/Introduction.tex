\chapter*{Pitch perception JND}
This paper aims to do a comparison of two stimuli presentation methods by comparing the measured difference threshold of pitch perception, known as the Just Noticeable Difference (JND). The two methods used are the Method of Constant Stimuli and the Transformed Up/Down Method. Webers law states that the smallest detectable change in perception, $\Delta$\textit{I}, is equal to a constant, \textit{k}, times the physical stimulus intensity, \textit{I}.
%
\begin{equation}
\Delta I= k \cdot I
\end{equation}
%
The constant, \textit{k}, is known as the Weber Fraction for a particular sensory dimension.



%The purpose of this study is to evaluate the difference threshold, known as Just Noticeable Difference (JND), of pitch perception using \textit{Transformed Up/Down Method}. Webers law states that the smallest detectable change in perception, $\Delta$\textit{I}, is equal to a constant, \textit{k}, times the physical stimulus intensity, \textit{I}.
%%
%\begin{equation}
%\Delta I= k \cdot I
%\end{equation}
%%
%The constant, \textit{k}, is known as the Weber Fraction for a particular sensory dimension.