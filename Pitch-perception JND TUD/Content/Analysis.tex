\section*{Analysis}
A 2-AFC method is used to collect data which leaves a 50 \% chance of guessing the right answer of the two possibilities. The point for JND is read at 70.7 \% because of the transformed method 1-up/2-down where you have to guess right two times in a row for $\Delta$f to decrease.
\%

From the fitted psychometric function for each subject the JND is read at 70.7 \% correct answered. In \autoref{tab:JND} the individual test subjects JND are presented along with a calculated mean JND which is xx Hz. 
%
\begin{table}[H]
	\centering
	\begin{tabular}{l|c}
		Subject     & JND ($\Delta$f at 70.7\%) \\\hline
		Subject 1   & 3.3  Korn            \\\hline
		Subject 2   & 3.3 Emil            \\\hline
		Subject 3   & 9.0 Lucca             \\\hline
		Subject 4   & 9.6 Juliane           \\\hline
		Subject 5   & 3.6 Sara              \\\hline
		mean JND &        
	\end{tabular}
	\caption{JND for all test subjects and mean JND.}
	\label{tab:JND}         
\end{table}
\noindent
%
From Webers law the Weber fraction, \textit{k}, can be calculated. According to this sample, the smallest detectable change in pitch perception, \textit{$\Delta$I}, is the mean JND at xx, when compared to the physical stimulus intensity, \textit{I}, of 800 Hz.
% 
\begin{equation}
xx = k \cdot 800 Hz \Rightarrow k = \frac{xx}{800 Hz} = xx
\end{equation}
%
This Weber fraction allows for calculating differential threshold at other intensities, \textit{I}. 
%
