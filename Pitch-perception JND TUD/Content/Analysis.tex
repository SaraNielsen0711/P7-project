\section*{Analysis}
\label{analysis}
%
A 2-AFC method is used to collect data which leaves a 50 \% chance of guessing the right answer of the two possibilities. The point for JND is read at 70.7 \% because of the \textit{Transformed 1-Up/2-Down Method}, which prescribe that the test subject have to have two consecutive correct answer before a decrease in $\Delta$f is obtained.\blankline
%
JND, or in this case the differential threshold, is calculated by taking the average of the last four reversals. In \autoref{tab:JND} the individual test subjects JND are presented along with a calculated mean JND which is 5.76 Hz. 
%
\begin{table}[H]
	\centering
	\begin{tabular}{l|c}
		Subject     & JND ($\Delta$f at 70.7\%) \\\hline
		Subject 1   & 9.00 Hz          \\\hline
		Subject 2   & 3.30 Hz            \\\hline
		Subject 3   & 9.60 Hz              \\\hline
		Subject 4   & 3.60 Hz            \\\hline
		Subject 5   & 3.30 Hz               \\\hline
		mean JND &    5.76 Hz   
	\end{tabular}
	\caption{JND for all test subjects and mean JND.}
	\label{tab:JND}         
\end{table}
\noindent
%
From Webers law the Weber fraction, \textit{k}, can be calculated. According to this sample, the smallest detectable change in pitch perception, \textit{$\Delta$I}, is the mean JND at 5.76 Hz, when compared to the physical stimulus intensity, \textit{I}, of 800 Hz.
% 
\begin{equation}
5.76 Hz = k \cdot 800 Hz \Rightarrow k = \frac{5.76 Hz}{800 Hz} = 0.0072
\end{equation}
%
This Weber fraction allows for calculating differential threshold at other intensities, \textit{I}.\blankline 
%
When turning to the results from the previous assignment listed in \autoref{tab:JND_constant} where the \textit{Method of Constant Stimuli} was used, it is quite clear that the individual JND's differ between the two methods. The lowest JND measured using the \textit{Method of Constant Stimuli} is 1.28 whereas the lowest JND measured with the \textit{Transformed 1-Up/2-Down Method} is 3.30 Hz. The largest individual difference between the two methods is found with test subject 4, who had an estimated JND of 9.42 Hz with \textit{Method of Constant Stimuli} and an estimated 3.60 Hz JND with \textit{Transformed 1-Up/2-Down Method}. The difference between the two estimates is 5.82 Hz.   
%
\begin{table}[H]
\centering
\begin{tabular}{l|c}
Subject     & JND ($\Delta$f at 75\%) \\\hline
Subject 1   & 4.40 Hz                 \\\hline
Subject 2   & 1.28 Hz                 \\\hline
Subject 3   & 7.56 Hz                 \\\hline
Subject 4   & 9.42 Hz                 \\\hline
Subject 5   & 3.88 Hz                 \\\hline
mean JND & 5.31 Hz       
\end{tabular}
\caption{JND for all test subjects and mean JND from the previous results using the \textit{Method of Constant Stimuli}.}
\label{tab:JND_constant}         
\end{table}
\noindent