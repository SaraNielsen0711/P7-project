\section*{Discussion and conclusion}
When comparing the results from this test with the results from the test with constant stimuli there is some changes to be aware of. The test setup is differnt, and that is what we would like to discuss, . Due to technical issues the computer was changed between the two tests. The effect of this could be, that the volume of the tone changed, even though it was set at same level. 

When comparing the two test and the two mean JND it's shown that these don't differentiate much. The mean JND when using the same test subjects is 5.31 Hz for method of constant stimuli and 5.76 Hz for transformed 1-Up/2-Down method. It's positive that the two means are similar, because that means that transformed 1-Up/2-Down method can be used instead of method of constant stimuli. Benefits of using transformed 1-Up/2-Down method is that this method takes much shorter time and is better at 'zooming' in at the subject's differential JND. When using method of constant stimuli you have to define all the presented stimuli before hand

%Så behøver vi ikke bruge så lang tid, fordele ved at vi zoomer ind på tærsklen, lidt større usikkerheder ved method of constant stimuli.

%Comparing Transformed method 1-Up/2-Down with method of constant stimuli

%Different test setup on screen can result in different results

%Samme testpersoner - kan man vænne sig til lydene?

%Forskellig interface man kigger på, forskellig måde at vælge sit svar. 

%Tæt på samme JND (og samme k-værdi)

%Forskellige metoder, fordele og ulemper? Hvis man bliver distraheret i transformed method up/down og svarer forkert to gange i træk har man ikke så god mulighed for at komme ned igen (og da man bruger de sidste 4 vendepunkter ryger ens gennemsnit meget hurtigere op)

