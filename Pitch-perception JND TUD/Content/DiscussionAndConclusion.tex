\section*{Discussion and conclusion}
%
When comparing the results from this test with the results obtained from the test with \textit{Method of Constant Stimuli} there are some changes to be aware of e.g. the experimental setup is different. Due to technical issues the computer was changed between the two setups. This could potentially affect the volume that the tones are presented with, even though it was fixed at the same level of 10 \%. \blankline
%
Compared to the mean JND, 5.31 Hz, found with \textit{Method of Constant Stimuli}, the mean JND, 5.76 Hz, found with a \textit{Transformed 1-Up/2-Down Method} does not yield a big difference. The difference is 0.45 Hz which is practically inaudible. The individual differential threshold found with each method does only yield a small variation in stimulus intensity, $\Delta$f. This can be due to confounding variables such as ambient noise, fatigue, learning improvements and motivation, just to name a few.  

It is favorable to use the method which have the lowest load on the test subjects because the two methods produce somewhat similar results. In this case \textit{Transformed 1-Up/2-Down Method} would be the favorable choice. Compared to \textit{Method of Constant Stimuli}, where we presented 100 trials to each test subject, the average number of trials needed with a \textit{Transformed 1-Up/2-Down Method} were 28.8 trials. This is calculated by averaging the number of trials needed for each of the five participants in the current study. The reduction in the number of trials needed to estimate the differential threshold leads to a reduction in time in which the test subjects must focus on their task, namely perceiving which of the two tones had the highest pitch. Therefore \textit{Transformed 1-Up/2-Down Method}  is more efficient compared to \textit{Method of Constant Stimuli}.

Furthermore \textit{Transformed 1-Up/2-Down Method} accounts for criterion problems, which \textit{Method of Constant Stimuli} does not. If the test subject, for some reason, changes his or hers response during the test, the \textit{Transformed 1-Up/2-Down Method} adjust the level of stimulus intensity accordingly. Other advantages by using \textit{Transformed 1-Up/2-Down Method} instead of \textit{Method of Constant Stimuli} are that the former mentioned method does not need a qualified guess on where the threshold might be or pre-defined stimuli intensities because they are adjusted throughout the test according to both the up/down rules and the step size. For these reasons \textit{Transformed 1-Up/2-Down Method} is more accurate compared to \textit{Method of Constant Stimuli}.\blankline
%
One advantage from \textit{Method of Constant Stimuli} which is not applicable with a \textit{Transformed 1-Up/2-Down Method} is the order in which the stimuli are presented. In \textit{Method of Constant Stimuli} the stimuli intensities are randomized which is not the case with a \textit{Transformed 1-Up/2-Down Method}. Another advantage when using \textit{Method of Constant Stimuli} is that each test subject is presented with exactly the same stimuli intensities, which is not the case with a \textit{Transformed 1-Up/2-Down Method}. 

Because the threshold can be extracted directly from raw data there is no need for fitting a psychometric function. If, however, a psychometric function is fitted there would be detailed information around the threshold and barely non information above and below threshold. This lack of information further from threshold is not an issue with \textit{Method of Constant Stimuli} because stimuli intensities are presented well above and below threshold and therefore gives more information further from threshold compared to \textit{Transformed 1-Up/2-Down Method}.      

