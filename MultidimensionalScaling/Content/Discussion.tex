\section*{Diskussion}
\label{Diskussion}
%
For at gøre det lettere at fortolke plottet er det valgt at der kun skal være to dimensioner i stedet for tre dimensioner, som ellers ville passe bedre, da den har en lavere stress værdi. Det vil dog være en del svære at fortolke et plot med tre dimensioner, så derfor vil det altid være et kompromis mellem disse, når antallet af dimensioner vælges. 

De to labels der er sat på dimensionerne, negativitet og fraværende, er ikke blevet tjekket om hvor godt de passer. En måde hvorpå det kunne undersøges, er ved at få forsøgspersoner til at vurdere disse udtryk i forhold til hvert enkelt ansigtsudtryk. Data for vurderingerne kan derefter analyseres og plottes som vektorer i MDS-plottet. Retningen og længden af vektorerne kan aflæses, og derved give en indikation for hvor godt de passer. Der kunne med fordel måles andre attributter, som muligvis kunne vise sig at være en bedre label til dimensionerne. I analysen er de indsatte vektorer indsat på baggrund af egen fortolkning. Det vil derfor være en fordel om at undersøge om denne tolkning er korrekt. 
\\\\ 
Andre mulige attributter der kan undersøges kan være fysiske parametre, som eksempelvis hvor åbne øjnene er eller vinklen på hovedpositionen. Disse parametre passer ikke på nogle af dimensionerne for det todimensionale plot, men kan være relevante at undersøge yderligere. 
\\\\
Det er tilladt at roterer plottet, for derved at finde passende beskrivelser for dimensionerne. Dette er ikke gjort i analysen, da det ikke blev fundet nødvendigt. Men hvis det ønskes at undersøge flere (eller andre) mulige beskrivelser for dimensionerne, ville det være en oplagt mulighed at undersøge. I denne sammenhæng vil det være relevant at se på de indsatte vektorer, der næsten alle fire går i diagonale retninger, og derfor muligvis kunne bruges som labels på dimensionerne, hvis plottet blev drejet. 