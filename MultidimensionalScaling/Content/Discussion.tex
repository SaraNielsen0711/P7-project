\section*{Diskussion}
\label{Diskussion}
%
For at gøre det lettere at fortolke data, illustreret på \autoref{fig:MDS}, er det valgt at anvende to dimensioner fremfor tre dimensioner, til trods for at tre dimensioner passer bedre og har en lavere \textit{stress}-værdi, jævnfør \autoref{fig:ScreePlot}. Ulempen ved at vælge tre dimensioner er, at det bliver væsentligt svære at fortolke data, hvorfor det er nødvendigt at finde et kompromis mellem, hvor meget mere information der opnåes ved at vælge et højere antal dimensioner i forhold til, hvor svært det bliver at fortolke resultaterne. \blankline
%
De to labels, der gengiver dimensionerne er \textit{Negativity} og \textit{Absent-minded}, jævnfør \autoref{fig:MDSA}, er ikke blevet undersøgt i forhold til hvor godt de rent faktisk passer. En måde hvorpå det kunne undersøges er ved, at få testpersonerne til at vurdere de 13 ansigtsudtryk med udgangspunkt i de to labels. Derudover er de fire attributter, gengivet med retningsvektorer på \autoref{fig:MDSA}, heller ikke testet men valgt ud fra egen fortolkning. De fire attributter \textit{Enjoyment}, \textit{Relaxed}, \textit{Unpleasant} og \textit{Fear} kan undersøges på samme måde som dimensionerne; ved at få testpersonerne til at vurdere ansigtsudtrykkende i forhold til hver attribut, eksempelvis på en skala. Baseret på det data vil det være muligt, at afgøre den korrekte længde og retning på vektorerne. I den sammenhæng kunne det ydermere være en fordel, at måle andre attributter både til dimensionerne men også til retningsvektorerne. 

Eksempelvis kunne det være interessant at undersøge fysiske attributter, såsom hvor åbne øjnene er eller vinklen på hovedet.\blankline     
%
Det er tilladt at roterer plottet og retningsvektorerne, så det i højere grad stemmer overens med dimensionerne. Dette er dog ikke gjort i analysen, da det ikke blev fundet nødvendigt. I tilfælde af at der undersøges én eller flere attributter, vil det være en oplagt mulighed.