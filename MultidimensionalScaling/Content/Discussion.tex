\section*{Diskussion}
\label{Diskussion}
%
For at gøre det lettere at fortolke plottet er det valgt at der kun skal være to dimensioner i stedet for tre, som ville passe bedre, da den har en lavere stress værdi. Det vil dog være en del svære at fortolke et plot med tre dimensioner. 

De to labels der er sat på dimensionerne, negativitet og fraværende, er ikke blevet tjekket om hvor godt de passer. En måde hvorpå det kunne undersøges, er ved at få forsøgspersoner til at rate disse udtryk i forhold til hvert enkelt ansigtsudtryk. Data for ratingerne kan derefter analyseres og plottes som vektorer i MDS-plottet. Retningen og længden af vektorerne kan aflæses, og derved give en indikation for hvor godt de passer. Der kunne med fordel måles andre attributter, som muligvis kunne vise sig at være en bedre label til dimensionerne. 
Andre mulige attributter kan være fysiske parametre, som eksempelvis hvor åbne øjnene er eller vinklen på hovedpositionen. Disse parametre passer ikke på nogle af dimensionerne for det todimensionale plot, men kan relevante at undersøge. 
\\\\
Det er tilladt at roterer plottet, for derved at finde passende beskrivelser for dimensionerne. Dette er ikke gjort i analysen, det da ikke blev fundet nødvendigt. Men hvis det ønskes at undersøge flere (eller andre) mulige beskrivelser for dimensionerne, ville det være en oplagt mulighed at undersøge. 