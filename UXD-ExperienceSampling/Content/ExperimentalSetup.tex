\chapter*{Undersøgelses setup}
Dette afsnit indeholder beskrivelse og begrundelse for valg taget i forhold til setupet for undersøgelsen.  
\section*{Parametre}
Protocolen vælges til at være interval-contingent, hvor det er bestemte tidspunkter på dagen at deltagerne skal beskrive deres studieaktivitet. Begrundelse for valget er, at det nemt at opsætte og forventes at være en fordel for deltagerne at de ved hvornår det forventes at de svarer. For at tilpasse besvarelsen til deltagerne, og derved øge sandsynligheden for fuld besvarelse fra deltagerne, er det valgt at spørge hver deltager ved undersøgelsens start, hvornår det passer dem bedst at dokumentere. \\
Det er på forhånd valgt et interval hvori besvarelserne skal angives, for at gøre det nemmere at sammenligne data efterfølgende. Tidspunkter, med intervaller, er som følger: 
\begin{itemize}
	\item Morgen (kl. 8.00 til 9.00): Besvarelse angående forventningerne til studieaktiviteten for dagen. 
	\item Eftermiddag (kl. 16.00 til 17.00): Besvarelse angående studieaktiviteten for første del af dagen. 
	\item Aften (kl. 21.00 til 23.00): Besvarelse angående studieaktiviteten for sidste del af dagen.
\end{itemize}

\section*{Testpersoner}
Det ønskes at undersøge studerenes studieaktiviteter og deres forventninger til deres studieaktiviteter samt undersøge om dette er forskelligt mellem udvalgte institutter på Aalborg Universitet. Det vælges at sammenligne tre institutter og der skal derfor bruges tre grupper af testpersoner. Målet er at have fem personer i hver gruppe, da dette er realistisk i forhold til undersøgelsens omfang og samtidig nok til at bygge antagelser på. \\
Da sandsynligheden for frafald vurderes til at være høj, på grund af undersøgelsens længde, vælges det at der rekruteres én ekstra deltager i hver gruppe. Deltagerne vil derfor bestå af:
\begin{itemize} 
	\item 18 deltager i alt. 
	\item Delt ind i tre grupper, baseret på institut af deltagerens studie. 
\end{itemize}
\section*{Introduktion}
Undersøgelsen starter med et indledende spørgeskema, hvori data om deltagerne indsamles. Spørgeskemaet er udarbejdet i Survey Monkey og indeholder følgende spørgsmål, hvor der i den efterfølgende parentes står besvarelsesmulighederne: 
\begin{enumerate}
	\item Angiv dit Subject-ID (01-18)
	\item Køn (Kvinde/Mand)
	\item Alder (18-38 med et-års interval)
	\item Angiv studieretning (tekst)
	\item Angiv semester (1.-10. semester)\\
	\textit{I løbet af undersøgelsen skal du besvare nogle spørgsmål tre gange om dagen; morgen, eftermiddag og aften. I den forbindelse vil vi gerne tilpasse tidspunkterne efter dine præferencer.} 
	\item Hvilket tidspunkt passer dig bedst om morgnen? (8 til 9, med 15 minutters interval)
	\item Hvilket tidspunkt passer dig bedst om eftermiddagen? (16 til 17, med 15 minutters interval)
	\item Hvilket tidspunkt passer dig bedst om aftenen? (21 til 23, med 30 minutters interval)
\end{enumerate}
\section*{Software}
Det er valgt at bruge programmet PIEL Survey. 
\section*{Spørgsmål og skala}
Morgen: 
\begin{itemize}
	\item Hvor mange timer forventer du at være på Universitetet i dag? 
	\item Hvor mange timer, hvor du ikke er på Universitet, forventer du at bruge på dit studie i dag? 
\end{itemize}

\noindent Eftermiddag: 
\begin{itemize}
	\item Hvor mange timer har du være på Universitetet i dag? 
	\item Hvor mange timer, hvor du ikke var på Universitet, har du brugt på dit studie i dag? 
\end{itemize}

\noindent Aften: 
\begin{itemize} 
	\item Hvor mange timer har du brugt på dit studie i dag siden sidste besvarelse? 
\end{itemize}

\noindent Morgen og eftermiddags spørgsmålene svares på følgende intervalskala: 
\begin{itemize}
	\item Ingen
	\item Mindre end 2
	\item 2 - 4
	\item Mere end 4 - 6 
	\item Mere end 6 - 8
	\item Mere end 8 
\end{itemize}

\noindent Aften spørgsmålet svares på følgende intervalskala: 
\begin{itemize}
	\item Ingen
	\item Mindre end 2
	\item 2 - 4
	\item Mere end 4 - 6 
	\item Mere end 6
\end{itemize}