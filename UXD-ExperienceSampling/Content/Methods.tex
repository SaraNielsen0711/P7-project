\section*{Methods}
\label{Methods}
%
Det ønskes at dokumentere studerenes studieaktiviteter og deres forventninger til deres studieaktiviteter, for derved at sammenligne den episodiske og semantiske repræsentation af samme aktivitet. Til at dokumentere den episodiske repræsentation af aktiviteten er det valgt at anvende metoden \textit{Experience Sampling}. Det er en metode der anvendes til at undersøge fænomener, når de opstår i individers hverdag \cite[p. 53]{PracticalGuide}. Metoden består af en undersøgelse, hvor deltagerne selv rapporter deres tanker, følelser og/eller aktiviteter i forhold til det fænomen der undersøges. Rapporteringen kan ske på forskellige måder, som kaldes protocol. Typen af rapportering og derved protocol kan enten være bestemte tidspunkter på dagen (Interval-contingent), ved signal forskellige tidspunkter på dagen (Signal-contingent) eller lige efter en aktivitet (Event-contingent). 
Valget af type protocol bør tages på baggrund af hvor ofte fænomenet forekommer, hvor stor indflydelse retrospektiv hukommelse vurderes at have og hvad der kan forventes af deltagerne i undersøgelsen. Disse faktorer er også afgørende ved bestemmelse af hvor lang tid undersøgelsen skal strække sig over (Sampling periode). 

%Describe the target group - students from at least three different study areas.
%Plan the study and recruit at least 5 students in each group
%Choose approach and tools
%Prepare info materials and run the study for one or two weeks per students
