\section*{Discussion}
\label{Discussion}
%
The number of SST was relatively high, namely five out of 20 possible violations, which corresponds to 25 \% of the possible violations. There does not exist a definite rule which describes the acceptable amount of SST violations before the BTL-model does not fit, but since the Chi-square gives a very small p-value the BTL model can't be used on this dataset.  

The number of MST and WST was low, one and zero respectively. This contribute to the choice of using a preference tree on the data. This kind of model is a bit harder to interpret, but gives a good representation on how the drugs can be compared. It can be difficult to make a preference tree which fits with the data, especially if the analyst don't have the same mental idea of the health risk of drugs as the participants. A lot of different preference trees including braches as legal and illegal, found in nature and chemically manufactured was suggested, but non of theese fit. \blankline 
%
The two preference trees found only has p-values on 0.12 and 0.14 respectively. Theese aren't very high p-values and a preference tree fitting even better might be found if all possible relations was tested. \blankline 
%
As shown in \fullref{Analyse} heroine is considered to have a much bigger health risk than any other drug in this study. Cocain and ecstasy is also rated to have a high health risk compared to alcohol, tobacco and cannabis, but not as high as heroine.

As seen on both preference trees alcohol has a branch for itself, which indicates that this can't be compared with any of the other drugs. The preference trees also shows that both tobacco and cannabis, even though grouped with the harder drugs, are evaluated as less of a health risk than alcohol.