\section*{Discussion}
\label{Discussion}
%
The Preference model is used to analyse the data. This kind of model is a bit harder to interpret, but gives a good representation on how the drugs can be compared. It can be difficult to make a Preference Tree which fits the data, especially if the analyst does not have the same mental idea of the health risk of drugs as the participants. A lot of different Preference Trees including branches as legal and illegal, organic and chemically manufactured was suggested, but non of these was a found to be a fit. \blankline 
%
The two found Preference Trees only has p-values on 0.12 and 0.14 respectively. These are not particular high p-values and thus it is possible that another Preference Tree might fit better if all possible relations were tested. \blankline 
%
As shown in \fullref{Analyse} heroin is considered to be of a higher health risk than any other drug in this study. Cocaine and ecstasy is also rated to have a high health risk compared to alcohol, tobacco and cannabis, but not as high as heroin.

As seen on both Preference Trees alcohol has a branch for itself, which indicates that this drug can not be compared with any of the other drugs. The Preference Trees also shows that both tobacco and cannabis, even though grouped with harder drugs, are evaluated to be of lower health risk than alcohol.