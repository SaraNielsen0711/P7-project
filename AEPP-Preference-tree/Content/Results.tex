\section*{Results}
\label{Results}
%
The probability of choosing one stimuli over another is calculated for each comparison as explained in \autoref{eq:probability}. The results for the probabilities are shown in \autoref{tab:prob}. 
%
\begin{table}[H]
\centering
\begin{tabular}{@{}llllllll@{}}
\toprule
Drugs      & No. & 1    & 2    & 3    & 4    & 5    & 6    \\ \midrule
Alcohol    & 1   & 0    & 0.58 & 0.73 & 0.21 & 0.08 & 0.15 \\
Tobacco    & 2   & 0.42 & 0    & 0.38 & 0.04 & 0    & 0.06 \\
Cannabis   & 3   & 0.27 & 0.63 & 0    & 0.06 & 0.02 & 0    \\
Ecstasy    & 4   & 0.79 & 0.96 & 0.94 & 0    & 0.02 & 0.35 \\
Heroin     & 5   & 0.91 & 1    & 0.98 & 0.98 & 0    & 0.92 \\
Cocaine    & 6   & 0.85 & 0.94 & 1    & 0.65 & 0.83 & 0    \\ \bottomrule
\end{tabular}
\caption{The probability that one stimulus is chosen over another stimulus.}
\label{tab:prob}
\end{table} 
\noindent 
%
To find out how many times the stochastic transitivities is violated, each of the three types is investigated and the number of times there is a violation is counted. The results are shown in \autoref{tab:Stocha}. 
%
\begin{table}[H]
\centering
\begin{tabular}{@{}ll@{}}
\toprule
Stochastic transitivity     & Violations \\ \midrule
WST      & 0   \\
MST      & 1   \\
SST      & 5   \\ \bottomrule
\end{tabular}
\caption{Results for the number of violations of the three stochastic transitivities out of 20 possible.}
\label{tab:Stocha}
\end{table}

\subsection*{BTL-model}
To check how good a fit the BTL-model is, a Chi-square test is conducted and the results are shown in \autoref{tab:Chi}. 
%
\begin{table}[H]
\centering
\begin{tabular}{@{}lll@{}}
\toprule
$\chi^{2}$   & Df  & p-value \\ \midrule
24.9423      & 10  & 0.0055* \\ \bottomrule
\end{tabular}
\caption{Results from the Chi-square test. The p-value of 0.0055 suggests that the BTL model does not fit}
\label{tab:Chi}
\end{table} 
\noindent 
The results from the Chi-square test shows a p-value at 0.0055 and therefore the conclusion is that this model is significant worse at predicting the data than the statistic model. The model is not accepted and instead the Preference Tree-model will be used. 

\subsection*{Preference Tree-model}


To check how good a fit the Preference Tree-model is, a Chi-square test is conducted and the results are shown in \autoref{tab:Chi}. 
%
\begin{table}[H]
\centering
\begin{tabular}{@{}lll@{}}
\toprule
$\chi^{2}$   & Df  & p-value \\ \midrule
12.7157      & 8   & 0.1220  \\ \bottomrule
\end{tabular}
\caption{Results from the Chi-square test. The p-value of 0.1220 suggests that the Preference model fits.}
\label{tab:Chi}
\end{table} 
\vfill
