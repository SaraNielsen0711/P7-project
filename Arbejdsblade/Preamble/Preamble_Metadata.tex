% Kommandoerne kan benyttes overalt i rapporten, og synkroniseres således overalt hver gang de opdateres her.

% OBS: Kommandokald som efterfølges af et mellemrum, skal afsluttes med "\" ala; "\groupname\ er en gruppe fra \studyname".

\newcommand{\groupname}{Group 782}
\newcommand{\institutionname}{Electronic Systems}
\newcommand{\adress}{Fredrik Bajers vej 7}
\newcommand{\city}{9220 Aalborg}
\newcommand{\universityname}{Aalborg University}
\newcommand{\studyname}{Produkt- og Designpsykologi}
\newcommand{\groupemail}{17gr782@es.aau.dk}
\newcommand{\semestername}{Produkt- og Designpsykologi}
\newcommand{\semester}{seventh}
\newcommand{\projectname}{Arbejdsblade}
\newcommand{\projectnameextension}{- NY: TITEL PÅ PROJEKT-} % Eventually leave empty
\newcommand{\projectnameextended}{\projectname \projectnameextension} %This one is defined by the others.

\newcommand{\supervisor}{Dorte Hammershøj}
\newcommand{\groupmemberI}{Andreas Kornmaaler Hansen}
\newcommand{\groupmemberII}{Lucca Julie Nellemann}
\newcommand{\groupmemberIII}{Emil Bonnerup}
\newcommand{\groupmemberIIII}{Juliane Nilsson}
\newcommand{\groupmemberIIIII}{Sara Nielsen}


%\newcommand{\groupmembers}{\groupmemberI og \groupmemberII}

\newcommand{\finishdate}{Den 18. December}
\newcommand{\beginyear}{} %Leave empty if same as \endyear
\newcommand{\finishyear}{2017} 
\newcommand{\projectperiod}{\begindate\beginyear\ til \finishdate\ \finishyear} %This one is defined by the others.


\newcommand{\appendixnamecustom}{Bilag}
\newcommand{\chapternamecustom}{Kapitel}
\newcommand{\sectionnamecustom}{Afsnit}
\newcommand{\subsectionnamecustom}{Underafsnit}
\newcommand{\figurenamecustom}{Figur}
\newcommand{\tablenamecustom}{Tabel}
\newcommand{\equationnamecustom}{Ligning}
\newcommand{\tocnamecustom}{Indholdsfortegnelse}