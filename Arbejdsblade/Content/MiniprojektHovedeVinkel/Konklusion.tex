\section*{Konklusion}
\label{Konklusion}
%
Position 2 (65$^{\circ}$) og position 3 (35$^{\circ}$) blev begge vurderet til at være mere indbydende end position 1 og position 4, som generelt blev vurderet på skalaens nedrehalvdel. Der blev udført en \textit{One-way Repeated Measures ANOVA}, som indikerede at der forekommer en signifikant forskel mellem hovedpositionerne i forhold til hvor indbydende robotten perciperes. Efterfølgende blev der foretaget en \textit{Pairwise Comparison}, hvorfra det blev fundet, at der er signifikant forskel mellem position 1 og position 2, mellem position 1 og position 3, samt en forskel mellem position 3 og position 4. Baseret på resultaterne er det muligt at afvise nul hypotesen, $H_0$, jævnfør \fullref{Hypotese}. Der er dog ikke endegyldigt belæg for at acceptere den alternative hypotese, $H_a$, da der ikke er fundet en signifikant forskel mellem position 2 og position 4. 

Dog kan det konkluderes, at for at robotten perciperes mest indbydende, i forhold til de fire valgte hovedpositioner, skal den være indstillet enten i position 2 eller i position 3. Dette gengiver at robottens hovede er vinklet skråt opad, hvilket formentlig giver testpersonerne i følelse af, at de har øjenkontakt med den.       
