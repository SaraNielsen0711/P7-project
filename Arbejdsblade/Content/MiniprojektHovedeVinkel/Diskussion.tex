\section{Diskussion}
\label{SkaleringseksperimentDiskussion}
%
Foruden information omkring køn, alder og studieretning kunne det have været en fordel at måle testpersonernes højde, da den ene af testpersonerne, baseret på testledernes vurdering, var betydeligt højere end de resterende, samt en testperson, som var betydeligt lavere end de restende testpersoner. Disse højdeforskellen kan potentielt have en indflydelse på hvor indbydende testpersonerne perciperede robotten afhængigt af dens fire hovedpositioner. 

Under en samtale antages det, at samtalepartnerne generelt vil søge efter at opnå øjenkontakt og opretholde en form for øjenhøjde. Er dette tilfældet kan det ligeledes antages at robottens hovedposition positionelt har indflydelse på, hvorvidt testpersonerne betragter robotten som en form for samtalepartner og ydermere hvilke interaktionsmuligheder robotten tillader. Det vil derfor være interessant at undersøge sammenhængen mellem brugerens højde og hvor indbydende robotten perciperes afhængigt af hovedposition.\blankline  
%
Testpersonerne angav på en \textit{Visual Analogue Scale} (VAS), hvor indbydende de perciperede robotten afhængigt af dens hovedposition. Skalaen var designet med åbne endepunkter samt to ankerpunkter angivet med \textit{Slet ikke} og \textit{Ekstremt}. Som tidligere nævnt kommenterede testperson 1 i pilottesten at ankerpunktet \textit{Ekstremt} ikke var det mest passende ord at anvende som label, da det for testpersonen forbindes med at være attraktiv, hvilket testpersonen gav udtryk for robotter ikke er. Selvom det blev vurderet ikke at ændre på skalaen, kunne det have været en fordel at udføre gentagende pilottests for at undersøge om andre har en lignende holdning og på baggrund af det ændre labelen.    

Ydermere kan ordet \textit{ekstremt} også betragtes som værende et negativt ladet ord i den forstand, at det oftest benyttes til at beskrive kraftige vejrforhold, naturkatastofer eller politiske holdninger langt fra, hvad der antages for normalt, \parencite{WEB:Oxford}. Det kunne derfor have været en fordel at undersøge om der kunne findes en erstatning for \textit{Ekstremt} på den anvendte skala. Dog er en af fordelene ved at vælge et ord som \textit{Ekstremt}, at der formentlig ikke vil forekomme stimuli, der er endnu mere ekstreme end før. Taget i betragtning af at det andet ankerpunkt er \textit{Slet ikke}, hvor der det er svært at finde et alternativ som er mindre end slet ikke, så er det oplagt at vælge et ord til det andet ankerpunkt som dækker det samme. Med de to valgte ankerpunkter har det været muligt at dække hele skalaen, hvor hvis der var brugt \textit{Slet ikke} og \textit{Meget}, så ville der potentielt være stimuli som er mere end meget, hvorfor der formentlig vil forekomme samlinger omkring endepunktet, hvilket ikke har været tilfældet med de to valgte ankerpunkter.\blankline
%
Testpersonerne blev bedt om at vurdere hvor indbydende de perciperede robotten til hver af de fire hovedpositioner, hvor der antages en fælles forståelse for ordet \textit{indbydende}. Det blev observeret af testlederne, at et par af testpersonerne spurgte ind til hvad der mentes med indbydende, hvor testlederne var nødsaget til at uddybe forklaringen. En måde hvorpå der kunne skabes en form for fællesforståelse kunne have været, at præcisere yderligere hvor robotten er tiltænkt samt hvad dens formål er. Det kunne have indebåret en opgave, som testpersonerne skulle løse ved hjælp af robottens ansigt, som i det tilfælde skulle erstattes med en touchskærm. Ved at anvende en touchskærm kunne det potentielt have medvirket dels til en fælles forståelse for ordet \textit{indbydende} men også en forståelse for hvordan interaktionen med robotten foregår. \blankline
%
Testen blev kun udført af otte testpersoner, hvilket har indflydelse på præcisionen af de statiske analyser. Dette afspejles blandt andet af den store variation i respons, særligt ved position 2 og position 4, jævnfør \autoref{fig:boksplot}. Havde flere testpersoner deltaget kunne det have medført en mere pålidelig sammenhæng mellem hvor indbydende robotten perciperes afhængigt af dens hovedposition. 



