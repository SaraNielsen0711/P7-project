\section*{Pilottest}
\label{Pilottest}
%
I forbindelse med at udvikle skalaer er det vigtigt, at sikre sig at skalaerne fungerer og at testpersonerne er i stand til anvende dem. Formålet med pilottesten er derfor ikke at undersøge hvor indbydende robotten perciperes, men derimod at undersøge om testpersonerne er i stand til at anvende den designede VAS og dertil om de forstår de angivne ankerpunkter. \blankline
%
Foruden instruktionerne, der fremgår i \fullref{Instruktioner}, bliver testpersonerne i pilottesten ydermere spurgt om:\blankline   
%
\begin{itemize}
	\item Kan du forstå hvordan du skal bruge skalaen?
	\item Hvordan synes du det var at bruge skalaen?
	\item Havde du nogen problemer med at forstå skalaen?
	\item Kunne du forstå de to labels på skalaen?
	\item Kunne du forstå spørgsmålet?
	\item Har du andre kommentarer?
\end{itemize} 
%
Pilottesten blev afviklet med to testpersoner; to mandlige studerende på Aalborg Universitet. Den første pilottest blev afviklet i kantinen på Frederik Bajers vej 7, hvorefter de restende tests blev afviklet i forhallen på Frederik Bajers vej 7H. 
%
\subsection*{Databehandling af pilottest}
%
Baseret på testpersonernes respons var det klart hvordan skalaen skulle anvendes og der opstod hverken problemer med at bruge eller forstå skalaen, når testpersonerne skulle angive hvor indbydende robotten var. I tillæg gav testpersonerne udtryk for at spørgsmålet: \textit{Hvor indbydende synes du robotten er?} var letforståeligt. I henhold til spørgsmålet omkring hvorvidt testpersonerne kunne forstå de angivne labels, gav testpersonerne udtryk for at det ikke var et problem at forstå dem. 

Dog gav testperson 1 udtrykt for nok aldrig at ville bruge udtrykket \textit{ekstremt indbydende} om en robot, da udtrykket forbindes med at være attraktiv, hvilket testpersonen ikke synes en robot er. På trods af denne kommentar vælges det at bibeholde \textit{Ekstremt}, som det højre ankerpunkt.

I forhold til skalatype, gav testperson 2 udtryk for bedre at kunne lide en kontinuerlig skala, hvilket VAS er, fremfor en skala, der er punktopdelt. Hvilket kommer til udtryk ved følgende kommentar:
% 
\begin{quotation}
  \textit{[...] synes den her skala er bedre end sån en 1-10 skala.}, testperson 2.
\end{quotation}



 