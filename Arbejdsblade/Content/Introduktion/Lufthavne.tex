\section{Lufthavne}
\label{Lufthavne}
%
I følge \textcite[s. 203]{PDF:FutureAirportTerminals} tyder det på, at lufthavne konstant skal fornye sig og adoptere ny teknologi for at forbedre rejse oplevelse, ved at gøre det hurtigere, mindre stressfuldt, mere sikkert og effektivt at rejse. Det fremgår af \textcite[s. 203]{PDF:FutureAirportTerminals}, at rejsende ofte oplever flyrejser som værende tidskrævende, ukomfortable, stressende og dyre. Ifølge \textcite[s. 351]{PDF:TheImpactOfTraveler} har amerikanske rejsende haft et dårligt forhold til lufthavnsindustrien, da den ikke lever op til deres forventninger. Derudover oplever de rejsende, at de ansatte er inkompetente, der er for lange ventetider, dårlige navigationssystemer, manglende information samt ubelejlige flyoversigter. 

Dette lægger ekstra pres på lufthavnene, da antallet af rejsende vil stige i fremtiden, \textcite[s. 203]{PDF:FutureAirportTerminals}. En af de nuværende teknologiskeløsninger luftenhaven gør brug af er applikation til smartphones, så de rejsende konstant kan modtage information omkring deres rejse, eller finde information omkring diverse butikker, \parencite[s. 203]{PDF:FutureAirportTerminals}. En af fordelene ved at benytte lufthavnsapplikationer er at omkring 76 \% af de rejsende er i besiddelse af en smartphone, \parencite[s. 203]{PDF:FutureAirportTerminals}. En anden måde at forbedre rejseoplevelsen er ved at gøre endnu mere brug af selvbetjening, så alt der vedrører check-in udelukkende består af selvbetjening, \parencite[s. 205]{PDF:FutureAirportTerminals}. Ifølge \textcite[s. 351]{PDF:TheImpactOfTraveler}, forventes det at inden for de næste par år vil alt der vedrører check-in af passager og bagage foretages udelukkende ved brug af selvbetjening.

Ved at investere i nye innovative teknologier, der hjælper med check-in, bagageaflevering samt lokalisering af gaten, vil det, ifølge \textcite[s. 352]{PDF:TheImpactOfTraveler}, forbedre rejseoplevelsen. Det kan derfor være en fordel, at investere i sociale robotter, som er i stand til at interagere med de rejsende og hjælpe dem med lige netop deres behov, hvad end det vedrører boarding, gate information eller information omkring indkøbsmuligheder.\blankline   
%
I en undersøgelse foretaget af \textcite{PDF:CustomerPerceptionOfService} fokuseres der på den rejsendes perciperede oplevelse af service kvaliteten i Chiles største lufthavn; Santiago de Chile's Airport. Baseret på resultaterne konkluderer \textcite[s. 213]{PDF:CustomerPerceptionOfService}, at den vigtigste parametre er \textit{Image perception of airport}, hvilket dækker over hvor innovativ lufthavnsterminalerne er, sikkerhed, passager opmærksomhed, terminal vedligeholdelse samt handicap faciliteter. Derudover finder \textcite[s. 213]{PDF:CustomerPerceptionOfService} ydermere at information- og kommunikationsløsninger har en indflydelse på den perciperede service kvalitet. Denne parameter dækker over fly- og lufthavnsinformation via skærme eller skiltning. I tillæg forventer \textcite[s. 210]{PDF:CustomerPerceptionOfService}, at de rejsende værtsætter integration af nye teknologier i lufthavne. 

Hvis det antages at disse resultater er generaliserbare samt forventingen om at rejsende værtsætter integration af nye teknologier er korrekt, så de ligeledes gør sig gældende for Københavns Lufthavn tyder det på, at det bør være muligt at introducere den sociale robot i Københavns Lufthavn. Integrationen af den sociale robot bør derfor øge den perciperede service kvalitet, da det er ny og innovativ teknologi, som er i stand til at levere informationer og kommunikere med de rejsende.  \blankline
%





% DISKUTER design forslag til hvordan disse teknologier kan udvikles
Det her kan bruges til at argumentere for hvilke parametre der er vigtige i forbindelse med flyrejser: .. Because uncertainty is an important element of traveler frustration (Grupe and Nitschke, 2013), airport technologies should be designed to reduce anxiety and increase travelers' confidence... individuals with technology anxiety are less satisfied with self-service technologies, the perceived quality of self-service technologies generally leads to increased satisfaction and intention to use these technologies... \textcite[s. 352]{PDF:TheImpactOfTraveler}

Handler om tillid: Despite arguments that SSTs reduce relational benefits, they tend to increase customers' confidence benefits, which represent one dimension of customer relationship benefits in the framework established by (Gwinner et al., 1998). Gwinner et al. (1998) defined confidence benefits as the state of reduced perceptions of anxiety and risk, and enhanced faith in the service provider. In other words, confidence benefits represent “a sense of knowing what to expect” and “that what goes wrong will be taken care of” (Gremler and Gwinner, 2015). Moreover, confidence benefits have proven to be very important to customers, since they shape their feelings of comfort, security, and certainty of expectations regarding the service outcome. Such findings are also congruent with Berry's (1995) conclusion that a positive relationship with the service provider is the consequence of risk reduction. Berry (1995) showed that the concept of confidence benefits is closely related to the term “trust.” According to Moorman et al. (1993, p. 82), trust is “a willingness to rely on an exchange partner in whom one has confidence.” A trustworthy service provider is thus one who instills confidence, and displays strong reliability and integrity (Morgan and Hunt, 1994). The state of confidence/trust in a service provider is a critical factor when there is corresponding competition in the market (Gwinner et al., 1998). \textcite[s. 353]{PDF:TheImpactOfTraveler}