\section{Lufthavne}
\label{Lufthavne}
%
Skriv lidt om hvad de har af teknologier, hvordan forskellige nationaliteter vurderer oplevelsen, hvordan kan det gøres bedre mm. Måske skal det rykkes op til projekt idé


Københavns lufthavn case
I lufthavne kan der hurtigt opstå en kaotisk stemning, når folk løber rundt på gangene for at nå deres fly. Et af de største problemer er ofte, at folk ikke er klar over hvor god tid de egentlig har og hvor langt de skal gå, og derfor skynder sig ned mod gaten. Når de når derned, er de langt væk fra salgsområderne, som typisk ligger centralt i lufthavne, hvilket resulterer i at de enten ikke køber noget, eller at de går frem og tilbage endnu en gang. Det påvirker blandt andet folks oplevelse af lufthavnen og rejsen, men også butikkernes omsætning. Københavns lufthavn vil gerne forsøge at holde folk i de centrale dele af lufthavnen længst muligt og dermed minimere trafikken på gangarealerne ud mod de forskellige gates. Det er her robotterne kommer ind i billedet, da disse kan fungere som serviceassistenter. Det forestilles, at robotterne kan køre rundt og fortælle folk, hvor længe der er til deres gate lukker, hvor langt der er derhen samt hvor de kan finde eller aflevere bagage. Ydermere forestilles det, at robotterne kan vise vej til gaten og eventuelt skabe mersalg i de centrale områder.