\section{Lufthavne}
\label{Lufthavne}
%
I følge \textcite[s. 203]{PDF:FutureAirportTerminals} tyder det på, at lufthavne konstant skal fornye sig og adoptere ny teknologi for at forbedre rejse oplevelse, ved at gøre det hurtigere, mindre stressfuldt, mere sikkert og effektivt at rejse. Det fremgår af \textcite[s. 203]{PDF:FutureAirportTerminals}, at rejsende ofte oplever flyrejser som værende tidskrævende, ukomfortable, stressende og dyre. Dette lægger ekstra pres på lufthavnene, da antallet af rejsende vil stige i fremtiden, \textcite[s. 203]{PDF:FutureAirportTerminals}. En af de nuværende teknologiskeløsninger luftenhaven gør brug af er applikation til smartphones, så de rejsende konstant kan modtage information omkring deres rejse, eller finde information omkring diverse butikker, \parencite[s. 203]{PDF:FutureAirportTerminals}. En af fordelene ved at benytte lufthavnsapplikationer er at omkring 76 \% af de rejsende er i besiddelse af en smartphone, \parencite[s. 203]{PDF:FutureAirportTerminals}. En anden måde at forbedre rejseoplevelsen er ved at gøre endnu mere brug af selvbetjening, så alt der vedrører check-in udelukkende består af selvbetjening, \parencite[s. 205]{PDF:FutureAirportTerminals}. 

En måde hvorpå selvbetjening både i afgangshallen og i transitten kan forbedres er ved at indfører sociale robotter, som er i stand til at interagere med de rejsende og hjælpe dem med lige netop deres behov, hvad end det vedrører boarding, gate information eller information omkring indkøbsmuligheder.\blankline  
%
I en undersøgelse foretaget af \textcite{PDF:CustomerPerceptionOfService} fokuseres der på den rejsendes perciperede oplevelse af service kvaliteten i Chiles største lufthavn; Santiago de Chile's Airport. Baseret på resultaterne konkluderer \textcite[s. 213]{PDF:CustomerPerceptionOfService}, at den vigtigste parametre er \textit{Image perception of airport}, hvilket dækker over hvor innovativ lufthavnsterminalerne er, sikkerhed, passager opmærksomhed, terminal vedligeholdelse samt handicap faciliteter. Derudover finder \textcite[s. 213]{PDF:CustomerPerceptionOfService} ydermere at information- og kommunikationsløsninger har en indflydelse på den perciperede service kvalitet. Denne parameter dækker over fly- og lufthavnsinformation via skærme eller skiltning. I tillæg forventer \textcite[s. 210]{PDF:CustomerPerceptionOfService}, at de rejsende værtsætter integration af nye teknologier i lufthavne. 

Hvis det antages at disse resultater er generaliserbare samt forventingen om at rejsende værtsætter integration af nye teknologier er korrekt, så de ligeledes gør sig gældende for Københavns Lufthavn tyder det på, at det bør være muligt at introducere den sociale robot i Københavns Lufthavn. Integrationen af den sociale robot bør derfor øge den perciperede service kvalitet, da det er ny og innovativ teknologi, som er i stand til at levere informationer og kommunikere med de rejsende.  




























Københavns lufthavn case
I lufthavne kan der hurtigt opstå en kaotisk stemning, når folk løber rundt på gangene for at nå deres fly. Et af de største problemer er ofte, at folk ikke er klar over hvor god tid de egentlig har og hvor langt de skal gå, og derfor skynder sig ned mod gaten. Når de når derned, er de langt væk fra salgsområderne, som typisk ligger centralt i lufthavne, hvilket resulterer i at de enten ikke køber noget, eller at de går frem og tilbage endnu en gang. Det påvirker blandt andet folks oplevelse af lufthavnen og rejsen, men også butikkernes omsætning. Københavns lufthavn vil gerne forsøge at holde folk i de centrale dele af lufthavnen længst muligt og dermed minimere trafikken på gangarealerne ud mod de forskellige gates. Det er her robotterne kommer ind i billedet, da disse kan fungere som serviceassistenter. Det forestilles, at robotterne kan køre rundt og fortælle folk, hvor længe der er til deres gate lukker, hvor langt der er derhen samt hvor de kan finde eller aflevere bagage. Ydermere forestilles det, at robotterne kan vise vej til gaten og eventuelt skabe mersalg i de centrale områder.