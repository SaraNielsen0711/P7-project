\section{Teknologier i lufthavne}
\label{Lufthavne}
%
Ifølge \textcite[s. 203]{PDF:FutureAirportTerminals} tyder det på, at lufthavne konstant skal fornye sig og adoptere ny teknologi for at forbedre rejseoplevelsen. Dette kan ske ved at gøre det hurtigere, mindre stressfuldt, mere sikkert og effektivt at rejse. Det fremgår af \textcite[s. 203]{PDF:FutureAirportTerminals}, at rejsende ofte oplever flyrejser som værende tidskrævende, ukomfortable, stressende og dyre. Ifølge \textcite[s. 351]{PDF:TheImpactOfTraveler} har amerikanske rejsende haft et dårligt forhold til lufthavnsindustrien, da den ikke lever op til deres forventninger. Derudover oplever de rejsende, at de ansatte er inkompetente, at der er for lange ventetider, dårlige navigationssystemer, manglende information samt ubelejlige flyoversigter. 

Dette lægger ekstra pres på lufthavnene, da antallet af rejsende vil stige i fremtiden, \textcite[s. 203]{PDF:FutureAirportTerminals}. En af de nuværende teknologiske løsninger luftenhaven gør brug af er en applikation til smartphones, så de rejsende konstant kan modtage information omkring deres rejse, eller finde information omkring diverse butikker, \parencite[s. 203]{PDF:FutureAirportTerminals}. En af fordelene ved at benytte lufthavnsapplikationer er, at omkring 76 \% af de rejsende er i besiddelse af en smartphone, \parencite[s. 203]{PDF:FutureAirportTerminals}. En anden måde at forbedre rejseoplevelsen er ved at gøre endnu mere brug af selvbetjening, så alt der vedrører check-in udelukkende består af selvbetjening, \parencite[s. 205]{PDF:FutureAirportTerminals}. Ifølge \textcite[s. 351]{PDF:TheImpactOfTraveler} forventes det, at inden for de næste par år vil alt, der vedrører check-in af passager og bagage foretages udelukkende ved brug af selvbetjening.

Ved at investere i nye innovative teknologier, der hjælper med check-in, bagageaflevering samt lokalisering af gaten, vil det, ifølge \textcite[s. 352]{PDF:TheImpactOfTraveler}, forbedre rejseoplevelsen. Det kan derfor være en fordel, at investere i sociale robotter, som er i stand til at interagere med de rejsende og hjælpe dem med lige netop deres behov, hvad end det vedrører boarding, gate information eller information omkring indkøbsmuligheder.\blankline   
%
I en undersøgelse foretaget af \textcite{PDF:CustomerPerceptionOfService} fokuseres der på den rejsendes perciperede oplevelse af service kvaliteten i Chiles største lufthavn; Santiago de Chile's Airport. Baseret på resultaterne konkluderer \textcite[s. 213]{PDF:CustomerPerceptionOfService}, at den vigtigste parameter er \textit{Image perception of airport}, hvilket dækker over hvor innovativ lufthavnsterminalerne er, sikkerhed, passager opmærksomhed, terminal vedligeholdelse samt handicap faciliteter. Derudover finder \textcite[s. 213]{PDF:CustomerPerceptionOfService} ydermere at information- og kommunikationsløsninger har en indflydelse på den perciperede service kvalitet. Denne parameter dækker over fly- og lufthavnsinformation via skærme eller skiltning. I tillæg forventer \textcite[s. 210]{PDF:CustomerPerceptionOfService}, at de rejsende værtsætter integration af nye teknologier i lufthavne. 

Hvis det antages at disse resultater er generaliserbare samt forventingen om at rejsende værtsætter integration af nye teknologier er korrekt, så de ligeledes gør sig gældende for Københavns Lufthavn tyder det på, at det bør være muligt at introducere den sociale robot i Københavns Lufthavn. Integrationen af den sociale robot bør derfor øge den perciperede service kvalitet, da det er ny og innovativ teknologi, som er i stand til at levere informationer og kommunikere med de rejsende.  \blankline
%
Når nye teknologier, som skal indgå i en lufthavn, skal designes er der nogle parametre, der bør tages højde for. Ifølge \textcite[s. 352]{PDF:TheImpactOfTraveler} skyldes en stor del af de rejsendes frustration usikkerhed, hvorfor nye teknologier bør designes så de reducerer angst og øger den rejsendes selvtillid. Den rejsendes selvtillid har stor indflydelse på følelser såsom sikkerhed, forventninger og hvor komfortabel den rejsende føler sig ved at interagere med en service teknologi, \parencite[s. 353]{PDF:TheImpactOfTraveler}. Sådan en teknologi skal ifølge \textcite[s. 353]{PDF:TheImpactOfTraveler}, udstråle integritet og pålidelighed, for at opnå den rejsendes tillid, som er med til at øge selvtilliden.

Da formålet med dette projekt er at undersøge, hvordan en social robot kan integreres i en lufthavn er det relevant at undersøge interaktionen med denne type robot, særligt i forhold til om de førnævnte parametre kan inkluderes. \blankline
%
SKRIV OM LGS ROBOT I LUFTHAVNE OG ANDRE KILDER OM ROBOTTER I LUFTHAVNE OG RET TIL EFTERFØLGENDE\blankline
%

