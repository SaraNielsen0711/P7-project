\section{Udfordringer ved sociale robotter}
\label{UdfordringerSocialeRobotter}
%
Kom ind på at det er relativt nyt at designe sociale robotter, det er ikke noget der er allemands ege, folk (specielt i europa, husk kilde) er stadig lidt bange for tanken om robotter, folk skal stole på teknologien og vænne sig til den. Udseendet kom ind på uncanny valley. Kom ind på de begrænsede test muligheder, de fleste undersøgelser er relativt korte (undtagen Harvey) så man får ikke den del med når brugeren har vænnet sig til hvordan det er at interagere med robotten.  \blankline
%
Måske inddrag at det er svært at teste folks subjektive oplevelse af hvordan det er at interagere med en robot, så vi kan lede det over i hvorfor vi ikke bare tager en af de metoder som allerede er brugt, her kan det være en idé at inkludere kulturelle forskelle, det er ikke sikkert at danskere forbinder "comfort" med deres subjektive oplevelse af robotten, det kan være de bruger andre ord til at beskrive det (det er ikke mening at det skal være et decideret metode afsnit men nærmere en måde at koble alt det her teoretiske viden til hvorfor vi gør som vi har tænkt os, altså inddrag semester emnet).