\section{Udfordringer ved sociale robotter}
\label{UdfordringerSocialeRobotter}
%
Sociale robotter er en relativ ny teknologi, eksempelvis blev den første version af AIBO lanceret i 1999, \textcite{WEB:AIBO}, og i 2012 angav 87 \% af borgerne i Europa, at de aldrig har været i kontakt med en robot, jævnfør \fullref{InteraktionSocialeRobotterGenerelleTendenser}. Selvom teknologien har udviklet sig en del igennem de sidste fem år, antages det, at procentdelen af europæiske borgere, der ikke har været i kontakt med en robot stadig er høj. En af årsagerne til at sociale robotter ikke har floreret på det europæiske marked i lige så høj grad som det eksempelvis har på det asiatiske, skyldes formentlig den europæiske frygt for robotterne, \parencite[s. 28]{PDF:InTheCompanyofRobots}. I tillæg er der en frygt for, at robotter vil overtage ens job, \parencite[s. 42]{PDF:PerceptionAcceptance}. Det tyder på, at frygten blandt europærer enten skyldes uvidenhed omkring, hvilke muligheder der er med sociale robotter eller at de har et forældet syn på robotter. En måde at reducere frygten og fornye synet på sociale robotter, kan være ved at inddrage potentielle brugere i designfasen. Ved at involvere potentielle brugere i designfasen, kan de få indflydelse på hvordan robotten skal designes for at de stoler på den og ikke frygter den.      

I forhold til frygten for at sociale robotter tager ens arbejde, bør det ikke være et problem i denne sammenhæng, hvor den sociale robot implementeres i en lufthavn og som primært varetager opgaver, som normalt ikke udføres af mennesker. Ifølge \textcite[s. 352]{PDF:TheImpactOfTraveler} vil de rejsende opleve, at i de fleste lufthavne vil der ikke være en menneske-menneske interaktion før de boarder flyet, da de nødvendige opgave håndteres ved hjælp af en form for teknologi. \blankline
%
Baseret på de undersøgelser, der er inddraget i det forrige kapitel er der kun én, som strækker sig over mere end en dag; \textcite[s. 3]{PDF:SharingALifeHarvey}, som strækker sig over tre perioder på hver 10 dage. Nogen undersøgelser, såsom \textcite[s. 273]{PDF:PerceptionAcceptance} og \textcite[s. 23]{PDF:CloseButNotStuck}, præsenterer testpersonerne for billeder, hvorefter de skal vurdere robotterne ud fra nogle foruddefineret parametre. \textcite[s. 62]{PDF:PerceptionAcceptance} præsenterede deres testpersoner for et videoklip, som enten skulle få testpersonerne til at percipere robotten som værende forudsigelig eller uforudsigelig, når de efterfølgende skulle interagere med robotten. 

Hvor i undersøgelser foretaget af \textcite[s. 1480]{PDF:ExploringInfluencingVariable}, \textcite[ss. 190-191]{PDF:PsychologicalEffects}, \textcite[s. 173]{PDF:HowMayIServeYou} og \textcite[ss. 786-787]{PDF:HowSocialDistanceShapesHRI} præsenteres testpersonerne for en social robot. I undersøgelsen foretaget af \textcite[s. 1480]{PDF:ExploringInfluencingVariable} skal testpersonerne deltage i en samtale med robotten, som styrer samtalen. \textcite[ss. 190-191]{PDF:PsychologicalEffects} undersøger hvordan testpersonernes komfortabilitet påvirkes af hvordan robotten nærmer sig testpersonen samt robottens hastighed og afstand. Lignende undersøges af \textcite[s. 173]{PDF:HowMayIServeYou}, hvor testpersonerne skal interagere med en social robot. Hvor testpersonerne i sidstnævnte enten skal samarbejde eller konkurrerer med en robot, som enten agerer \textit{supervisor} eller \textit{subordinate}.\blankline
%
Med udgangspunkt i de foregående afsnit omkring teknologier i lufthavne, jævnfør \fullref{Lufthavne}, interaktion med sociale robotter, jævnfør \fullref{InteraktionSocialeRobotter}, samt dette afsnit vedrørende udfordringerne ved sociale robotter, er det muligt at udarbejde en problemformulering. 
  
