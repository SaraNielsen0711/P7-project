\section{Udfordringer ved sociale robotter}
\label{UdfordringerSocialeRobotter}
%
Sociale robotter er en relativ ny teknologi, eksempelvis blev den første version af AIBO lanceret i 1999, \textcite{WEB:AIBO}, og i 2012 angav 87 \% af borgerne i Europa, at de aldrig har været i kontakt med en robot, jævnfør \fullref{InteraktionSocialeRobotterGenerelleTendenser}. Selvom teknologien har udviklet sig en del igennem de sidste fem år, antages det, at procentdelen af europæiske borgere, der ikke har været i kontakt med en robot stadig er høj. En af årsagerne til at sociale robotter ikke har floreret på det europæiske marked i lige så høj grad som det eksempelvis har på det asiatiske, skyldes formentlig den europæiske frygt for robotterne, \parencite[s. 28]{PDF:InTheCompanyofRobots}. I tillæg er der en frygt for, at robotter vil overtage ens job, \parencite[s. 42]{PDF:PerceptionAcceptance}. Det tyder på, at frygten blandt europærer enten skyldes uvidenhed omkring, hvilke muligheder der er med sociale robotter eller at de har et forældet syn på robotter. En måde at reducere frygten og fornye synet på sociale robotter, kan være ved at inddrage potentielle brugere i designfasen. Ved at involvere potentielle brugere i designfasen, kan de få indflydelse på hvordan robotten skal designes for at de stoler på den og ikke frygter den.      

I forhold til frygten for at sociale robotter tager ens arbejde, bør det ikke være et problem i denne sammenhæng, hvor den sociale robot implementeres i en lufthavn og som primært varetager opgaver, som normalt ikke udføres af mennesker. Ifølge \textcite[s. 352]{PDF:TheImpactOfTraveler} vil de rejsende opleve, at i de fleste lufthavne vil der ikke være en menneske-menneske interaktion før de boarder flyet, da de nødvendige opgave håndteres ved hjælp af en form for teknologi. \blankline
%




 Kom ind på de begrænsede test muligheder, de fleste undersøgelser er relativt korte (undtagen Harvey) så man får ikke den del med når brugeren har vænnet sig til hvordan det er at interagere med robotten.  \blankline
%
Måske inddrag at det er svært at teste folks subjektive oplevelse af hvordan det er at interagere med en robot, så vi kan lede det over i hvorfor vi ikke bare tager en af de metoder som allerede er brugt, her kan det være en idé at inkludere kulturelle forskelle, det er ikke sikkert at danskere forbinder "comfort" med deres subjektive oplevelse af robotten, det kan være de bruger andre ord til at beskrive det (det er ikke mening at det skal være et decideret metode afsnit men nærmere en måde at koble alt det her teoretiske viden til hvorfor vi gør som vi har tænkt os, altså inddrag semester emnet).\blankline
%

