\section{Interaktion med sociale robotter}
\label{InteraktionSocialeRobotter}
%
Inden der dykkes ned i hvilke parametre, der har indflydelse på hvordan interaktionen med sociale robotter perciperes og accepteres, vil nogle mere generelle tendenser blive diskuteret. Det dækker blandt andet over køns-, alders- samt kulturelleforskelle i henhold til synet på sociale robotter. Dernæst gives der nogle eksempler på, hvordan relaterede studier har målt brugerens perception og accept af sociale robotter.
%
\subsection{Generelle tendenser}
\label{InteraktionSocialeRobotterGenerelleTendenser}
%
I den vestlige verden er vi i langt mindre grad villige til at acceptere sociale robotter end hvad der eksempelvis er tilfældet i Japan, \parencite[s. 28]{PDF:InTheCompanyofRobots}. Det skyldes, ifølge \textcite[s. 28]{PDF:InTheCompanyofRobots}, at der er en indgroede frygt for maskiner og følelsen af manglende kontrol, hvilket ikke er tilfældet i den Japanske kultur, som åndeliggøre robotter. At mennesker i den vestlige verden frygter robotter kan, blandt andet, retfærdiggøres med at i 2012 angav 87 \% af borgerne i Europa, at de aldrig har været i kontakt med en robot, hverken i hjemmet eller på ens arbejdsplads, \parencite[s. 40]{PDF:PerceptionAcceptance}. I en undersøgelse, beskrevet af \textcite[s. 41]{PDF:PerceptionAcceptance}, fremgår det at synet på robotter ikke har ændret sig de sidste 35 år. Når robotterne, ved brug af tegninger, visualiseres så minder de i høj grad om robottypen: \textit{Non assistive robots}, illustreret på \autoref{fig:CategorizationOfRobots}. Endvidere tyder det på, at Europærer frygter, at de vil miste deres job til robotten og at robotten er til for at erstatte mennesket, \parencite[s. 22]{PDF:RobotShiftFromIPtoSR}.   

Derudover er der en tendens til, at mænd i højere grad perciperer robotten som menneskeagtig sammenlignet med kvinder, som i langt højere grad perciperer robotten som en maskine, \parencite[s. 28]{PDF:InTheCompanyofRobots}. Ifølge \textcite[s. 1479]{PDF:ExploringInfluencingVariable}, perciperer mænd robotter som værende mere brugbare, de har større intention om at bruge dem og de er mere villige til at acceptere robotter end kvinder er.  

Ydermere argumenterer \textcite[s. 2]{PDF:SharingALifeHarvey} for, at den ældre population i højere grad accepterer sociale robotter, sammenlignet med den yngre population. Det antages, at en af årsagerne til det formentlig skyldes, at der anvendes sociale robotter i ældreplejen eksempelvis til mindske følelsen af ensomhed, \fullref{EksisterendeSocialeRobotter}. I mere praktiske situation, som eksempelvis rengøring, tøjvask og lignende, foretrækker ældre robotter fremfor mennesker, hvorimod hvis opgaverne er omsorgsrelateret foretrækkes mennesker, \parencite[s. 22]{PDF:RobotShiftFromIPtoSR}.

\subsection{Parametre, der har indflydelse på accept og interaktion med sociale robotter}
\label{InteraktionSocialeRobotterParametre}
%
\textcite[s. 1477]{PDF:SharingALifeHarvey} opstiller tre forskellige problemstillinger, der bør overvejes når brugerens accept af en social robot undersøges. I de følgende afsnit undersøges hvilke parametre, der har indflydelse på brugerens accept i forhold til de tre problemstillinger:\blankline 
%
\begin{quotation}
\textit{
  \begin{enumerate}
  \item the likely positive or negative consequences of the behavior
  \item the approval or disapproval of the behavior by respected individuals or groups, and
  \item the factors that may facilitate or impede performance of the behavior.
\end{enumerate}}
\textcite[s. 1477]{PDF:SharingALifeHarvey}\blankline
\end{quotation}
\noindent
%
Overvejelserne afspejler henholdvist brugerens evaluering af robotten, hvilke sociale normative overbevisninger, der tilskrives robotten ved brug samt hvilke kontekstuelle faktorer, der spiller ind ved brug, \parencite[s. 1477]{PDF:SharingALifeHarvey}. Baseret på \textcite[ss. 1477-1478]{PDF:SharingALifeHarvey} tyder det på, at der er to overordnede kategorier af parametre, som har indflydelse på den første problemstilling: Utilitaristiske og hedoniske parametre. Førstnævnte dækker over det praktiske og anvendelige aspekt ved at interagere med en social robot, hvor sidstnævnte relateres til brugerens oplevelse af at anvende den sociale robot, \parencite[s. 1476]{PDF:SharingALifeHarvey}.

I følgende to afsnit vil de utilitaristiske og hedoniske parametre undersøges nærmere, hvorefter de sociale normative overbevisninger og efterfølgende hvilke parametre, der har indflydelse på præstationsevnen, undersøges.
%   

\subsubsection*{Utilitaristiske parametre}
\label{InteraktionSocialeRobotterParametreUtilitarian}
%
Som nævnt dækker utilitaristiske parametre over det praktiske og anvendelige aspekt ved at interagere med en social robot. Derudover har disse parametre ligeledes indflydelse på den specifikke adfærd. I følgende afsnit belyses hvilke parametre, der har indflydelse på det praktiske såvel som det anvendelige aspekt. Der tages primært udgangspunkt i \textcite[s. 1477]{PDF:SharingALifeHarvey}.
%
\begin{figure}[H]
\centering
\includegraphics[width = 0.75\textwidth]{Figure/UtilitarianParameters} 
\caption{Sammenhængen mellem de seks forskellige parametre, der har indflydelse på det praktiske og anvendelige aspekt. Pilene indikerer hvilken retning indflydelsen er mellem to parametre.}
\label{fig:UtilitarianParameters}
\end{figure}
\noindent 
%
Baseret på \textcite[s. 1477]{PDF:SharingALifeHarvey} defineres \textit{usefulness} som værende brugerens overbevisning om, at robotten vil forbedre de daglige aktiviteter. Ifølge \textcite[s. 11]{PDF:SharingALifeHarvey} tyder det på, at \textit{usefulness} er en vigtig parametre, der kan bidrage til langtidssigtet forhold mellem bruger og robot. \textit{Ease of use} defineres som værende brugerens overbevisning om, at det nemt at anvende robotten. Derudover argumenterer \textcite[s. 1477]{PDF:SharingALifeHarvey} for, at i situationer hvor robotten skal indgå i en social interaktion med mennesker, er det nødvendigt at robotten gengiver menneskeagtige træk, for at brugeren føler sig komfortabel nok til at indgå i interaktionen. Robottens evne til at tilpasse sig den sociale kontekst afhængigt af brugeres behov, defineres som \textit{perceived adaptability}. Ifølge \textcite[s. 1477]{PDF:SharingALifeHarvey}, har \textit{perceived adaptability} indflydelse på \textit{perceived usefulness}, \textit{use attitude}, \textit{use intentions}, som er repræsenteret på \autoref{fig:UtilitarianParameters}. Derudover har \textit{perceived adaptability} indflydelse på \textit{enjoyment}. Ydermere tyder det på at \textit{intelligence}, har en effekt på hvor realistiske robotten perciperes, \parencite[s. 1477]{PDF:ExploringInfluencingVariable}.   
%
\subsubsection*{Hedoniske parametre}
\label{InteraktionSocialeRobotterParametreHedonic}
%
Som nævnt dækker hedoniske parametre over brugerens oplevelse af at anvende den sociale robot. Derudover har disse parametre ligeledes indflydelse på den specifikke adfærd. I følgende afsnit belyses hvilke parametre, der har indflydelse på det dette. Der tages primært udgangspunkt i \textcite[ss. 1477-1478]{PDF:SharingALifeHarvey}. På \autoref{fig:HedonicParameters} illustreres forskellige parametre, samt deres indbyrdes forhold.
%
\begin{figure}[H]
\centering
\includegraphics[width = 0.75\textwidth]{Figure/HedonicParameters} 
\caption{Sammenhængen mellem de otte forskellige parametre, der har indflydelse på brugerens oplevelse. Pilene indikerer hvilken retning indflydelsen er mellem to parametre.}
\label{fig:HedonicParameters}
\end{figure}
\noindent 
%
Baseret på \textcite[s. 1477]{PDF:ExploringInfluencingVariable} har både \textit{enjoyment}, der defineres som værende følelsen af fornøjelse eller glæde forbundet med brug, og \textit{attractiveness}, der defineres som værende den positive evaluering af robottens udseende, på flere af de parametre gengivet på \autoref{fig:UtilitarianParameters}. \textit{Enjoyment} har indflydelse på \textit{ease of use}, \textit{use attitude} samt \textit{use intentions}. Derimod har \textit{attractiveness} indflydelse på \textit{usefulness} og \textit{ease of use}. \blankline 
%
Antropomorfisering defineres som værende evnen til at tilskrive naturfænomener, guder, overnaturlige væsner og dyr menneskelige egenskaber, såsom følelser og motiver, \parencite{WEB:DefAntropomorisering}. Dog defineres antropomorfisering i HRI sammenhæng, som værende evnen til at tildele og beskrive objekter med menneskelige egenskaber, for at rationalisere objektets adfærd, \parencite[s. 1478]{PDF:ExploringInfluencingVariable}. Ifølge \textcite[s. 1478]{PDF:ExploringInfluencingVariable} har anthropomorfisme indflydelse på \textit{usefulness}, \textit{use attitude} samt \textit{use intention}, som er repræsenteret på \autoref{fig:UtilitarianParameters}. Derudover har \textit{anthropomorphism} ligeledes indflydelse på \textit{attitude toward robots}, \textit{social influence} samt \textit{companionship}. Ifølge \textcite[s. 19]{PDF:CloseButNotStuck} resulterer antropomorfisering i, at mennesket betragter en social robot som en social enhed, hvorfor robotten behandles som et menneske.

Der er forskellige årsager til at mennesker antropomorfiserer sociale robotter. Antropomorfisering kan forekomme i situationer hvor mennesket oplever en manglende kontrol og usikkerhed, hvor mennesket i højere grad har en tendens til at antropomorfisere sociale robotter, for at kunne forstå, kontrollere samt forudse robottens adfærd, \parencite[s. 1478]{PDF:ExploringInfluencingVariable}. Dette høre under \textit{effectance motivation}, der defineres som værende ønsket om effektivt at kunne interagere med ens omgivelser, \parencite[s. 62]{PDF:EffectsOfAnticipatedHRI}.

Derudover har mennesker med en teknologisk baggrund en tendens til at tildele robotten sin egen personlighed, hvilket ikke er tilfældet med mennesker uden teknologisk baggrund, \parencite[s. 19]{PDF:CloseButNotStuck}. I tillæg argumenterer \textcite[s. 2]{PDF:SharingALifeHarvey} for, at desto mere antropomorfisering brugeren oplever, desto bedre er evalueringen af robotten, desto mere fornøjet er de med interaktionen og desto større er chancen for at de oplever robotten som en ledsager.

Ifølge \textcite[s. 61]{PDF:EffectsOfAnticipatedHRI} har ensomme mennesker en stærk tendens til at antropomorfisere kæledyr og teknologiske objekter, såsom robotter. Det skyldes, at mennesket har et behov for både tilknytning og et tilhørsforhold. Dette hører under \textit{sociality motivation}, der defineres som værende ønsket og behovet for at skabe en social relation til andre, \parencite[s. 61]{PDF:EffectsOfAnticipatedHRI}. I det henseende vil en robot med et mere livagtigt udtryk perciperes som værende en venlig ledsager, \parencite[s. 1478]{PDF:ExploringInfluencingVariable}. Dette gengives som \textit{companionship}, der defineres som brugerens perciperede mulighed for at bygge et forhold til robotten. \textit{Companionship} er illustreret på \autoref{fig:HedonicParameters} og udover at have indflydelse på \textit{user experience}, har \textit{companionship} indflydelse på den vedvarende interaktion med robotten. \blankline
%
Baseret på \textcite[s. 1478]{PDF:ExploringInfluencingVariable} tyder det på, at \textit{realism} kan forbedre HRI og har dermed indflydelse på \textit{user experience}, \autoref{fig:HedonicParameters}. En robots \textit{realism} afspejler i hvilken grad brugeren tro på, at robotten reagerer og opfører sig realistisk. Desto mere realistisk robotten perciperes, desto mere intelligent perciperes den, \parencite[s. 1478]{PDF:ExploringInfluencingVariable}. 

\textit{Sociability} defineres som værende brugerens overbevisning om hvorvidt robotten besidder de sociale, emotionelle samt kognitive færdigheder, der er nødvendige for en succesfuld tilvænning af robotten, \parencite[s. 1478]{PDF:ExploringInfluencingVariable}. Denne parameter har, ifølge \textcite[s. 1478]{PDF:ExploringInfluencingVariable}, ligeledes indflydelse på \textit{use attitude} og \textit{usefulness}.\blankline
%
I henhold til teknologier i lufthavne argumenterer \textcite[s. 352]{PDF:TheImpactOfTraveler} for, hvorfor der bør tages højde for \textit{hedonic} parametre. En stor del af den teknologi, som kan findes i lufthavne bygger primært på \textit{utilitarian} parametre, hvilket formentlig har været årsagen til en dårlig brugeroplevelse. For at forbedre dårlige oplevelser, blandt andet ved lange ventetider ved sikkerhedskontrollen, har lufthavne forsøgt at foretage nogle tiltag, der skal gøre oplevelsen bedre, hvilket eksempelvis kommer til udtryk ved et øget brug af smartphone applikationer. Derudover understreger \textcite[s. 352]{PDF:TheImpactOfTraveler}, at det er nødvendigt at skabe fornøjelige lufthavns oplevelser. Fornøjelighed gengives på \autoref{fig:HedonicParameters}, som \textit{enjoyment}.
%

\subsubsection*{Sociale normer}
\label{InteraktionSocialeRobotterParametreSocialeNormer}
% 
I henhold til problemstilling 2: \textit{the approval or disapproval of the behavior by respected individuals or groups}, fremsat af \textcite[s. 1477]{PDF:SharingALifeHarvey}, vil følgende afsnit undersøge hvilke parametre, der har indflydelse på det.\blankline
%
De sociale normer dækker over de synspunkter mennesket har i relation til deres egen adfærd og dækker ydermere de synspunkter og regler, der er gældende for en gruppe for hvorvidt en bestemt adfærd er passende eller upassende, \parencite[s. 1478]{PDF:ExploringInfluencingVariable}. Ifølge \textcite[s. 1478]{PDF:ExploringInfluencingVariable} dækker de sociale normer over \textit{social influence} og \textit{image}. \textit{Social influence} er karakteriseret ved brugerens perception af hvad andre tænker omkring brugen af robotten. Denne parameter har indflydelse på \textit{usefulness}, \textit{ease of use}, \textit{use attitude}, \textit{use intention} samt \textit{actual use}, \parencite[s. 1478]{PDF:ExploringInfluencingVariable}, jævnfør \autoref{fig:UtilitarianParameters}. Det er særligt mennesker i ens tætte omgangskreds; familie, partner og venner, hvis mening har indflydelse på hvorvidt en bestemt teknologi vil blive brugt, i dette tilfælde en social robot, \parencite[s. 1478]{PDF:ExploringInfluencingVariable}. 

\textit{Image} karakteriseres ved brugerens overbevisning om at interaktionen med robotten kan lede til større anerkendelse og social status blandt ens omgangskreds, \parencite[s. 1478]{PDF:ExploringInfluencingVariable}. Derudover har \textit{image} indflydelse på \textit{perceived usefulness}, illustreret på \autoref{fig:UtilitarianParameters}.\blankline
%
I relation til sociale normer og tilnærmelsesvis \textit{social influence} undersøger \textcite{PDF:HowSocialDistanceShapesHRI}, hvilken indflydelse \textit{social distance} har på HRI. \textit{Social distance} referer til hvilken grad mennesker perciperer manglende intimitet grundet forskellige egenskaber såsom etnicitet, race, religion, beskæftligelse, \parencite[s. 784]{PDF:HowSocialDistanceShapesHRI}. \textit{Social distance} dækker over to forskellige parametre; \textit{social structural distance} og \textit{physical distance}.

\textit{Social structural distance} referer til mennesket perception af og adfærd rettet mod andre, som er afhænger af hvordan mennesket kategoriserer andre, særligt i forhold til deres kompetance og varme, \parencite[s. 784]{PDF:HowSocialDistanceShapesHRI}. Derudover kategoriseres mennesker afhængigt af deres indbyrdes forhold, er der forskel i status, foregår interaktionen som samarbejde eller som en konkurrence. \textit{Social structural distance} kan inddeles i yderligere to kategorier; \textit{power distance} og \textit{task distance}. Førstnævnte har stor indflydelse på den individuelle's adfærd og perception afhængigt af forholdet til interaktionspartneren, som afhænger af personligheder, roller og deres individuelle sociale klasse, \parencite[s. 784]{PDF:HowSocialDistanceShapesHRI}. \textit{Task distance} referer til at mennesker afhænger af hinanden for at opnå deres mål, det kan enten komme positivt til udtryk når mennesker samarbejder for opnår deres fælles mål, eller negativt når mennesker konkurrerer mod hinanden og når de forhindre hinanden i at opnå deres individuelle mål, \parencite[s. 784]{PDF:HowSocialDistanceShapesHRI}. 

\textit{Physical distance} kan også beskrives som \textit{proxemic distance} og referer til den fysiske adskillelse mellem mennesker, \parencite[s. 784]{PDF:HowSocialDistanceShapesHRI}. Der opstår både bedre kommunikation mellem mennesker og den individuelle vil være i stand til at løse egne opgave bedre, \parencite[s. 785]{PDF:HowSocialDistanceShapesHRI}.\blankline
%
\textcite[s. 794]{PDF:HowSocialDistanceShapesHRI} undersøger forholdet mellem \textit{power distance} og \textit{proxemic distance} i forhold til HRI. \textit{Power distance} udtrykkes ved at robotten enten agerer som \textit{supervisor} eller som \textit{subordinate}, hvor \textit{proxemic distance} udtrykkes ved at afstanden mellem robot og menneske er lille eller stor. Baseret på disse resultater tyder det på, at \textit{user experience} blev bedre når interaktionen foregik med en \textit{supervisor} robot tæt på og når interaktionen foregik med en \textit{subordinate} robot langt væk, \parencite[s. 785]{PDF:HowSocialDistanceShapesHRI}. Dog finder \textcite[s. 785]{PDF:HowSocialDistanceShapesHRI}, at præstationen i opgaven blev forværret desto tætter robotten var på testpersonen, uafhængigt af \textit{power distance}. 

Derudover finder \textcite[s. 785]{PDF:HowSocialDistanceShapesHRI}, at \textit{user experience} blev forbedret når robotten konkurrerede med testpersonen tæt på og når robotten samarbejde med testpersonen langt væk, hvilket var imod forventningen, \parencite[s. 785]{PDF:HowSocialDistanceShapesHRI}.   
%
\subsubsection*{Indflydelse på præstationsevnen}
\label{InteraktionSocialeRobotterParametrePraestation}
%
I henhold til problemstilling 3: \textit{the factors that may facilitate or impede performance of the behavior}, fremsat af \textcite[s. 1477]{PDF:SharingALifeHarvey}, vil følgende afsnit undersøge hvilke parametre, der har indflydelse på præstationsevnen.\blankline
%
\textit{Control beliefs} referer til brugerens overbevisning om hvilke ressourcer, muligheder og forhindringer, der er tilstede eller fraværende og som vil have en positiv eller negativ indflydelse på præstationsevnen, \parencite[s. 1478]{PDF:ExploringInfluencingVariable}. \textit{Control beliefs} kan inddeles i tre kategorier, som alle har indflydelse på \textit{user acceptance}: \textit{Perceived behavioral control}, \textit{anxiety} og \textit{experience}, begge rettet mod robotter, \parencite[s. 1478]{PDF:ExploringInfluencingVariable}. 

\textit{Perceived behavioral control} defineres som værende brugerens perciperede oplevelse af hvor let eller svært det var at interagere med robotten, \parencite[s. 1478]{PDF:ExploringInfluencingVariable}. \textit{Perceived behavioral control} har indflydelse på \textit{perceived ease of use}, \textit{use intentions} samt \textit{actual use}. \textit{Anxiety} rette mod robotter defineres som værende ængstelige eller andre negative emotioner forbundet med HRI og som forværres afhængigt at tidligere oplevelser, \parencite[s. 1478]{PDF:ExploringInfluencingVariable}. \textit{Anxiety} har en negativ effekt på \textit{perceived ease of use}, men \textit{anxiety} kan reduceres med \textit{enjoyment}, \parencite[s. 1478]{PDF:ExploringInfluencingVariable}. 

\textit{Experience} med robotter defineres som den opnåede oplevelse med robotter både direkte, i form af HRI, og indirekte via et medie, såsom nyhedsartikler og science-fiction film, \parencite[s. 1479]{PDF:ExploringInfluencingVariable}. Som nævnt i \fullref{InteraktionSocialeRobotterGenerelleTendenser}, har 87 \% af borgerne i Europa aldrig indgået i en interaktion med en robot, hvorfor det antages at den direkte oplevelse med robotter er begrænset, hvert fald i Europæiske lande.

Tidligere erfaring med robotter har, ifølge \textcite[s. 1479]{PDF:ExploringInfluencingVariable}, en positiv effekt på: \textit{Usefulness}, \textit{ease of use}, \textit{attitude towards robots}, \textit{use intention} samt \textit{actual use}. I henhold til tidligere erfaringer med robotter, undersøger \textcite{PDF:CloseButNotStuck} hvorvidt \textit{online community} og erfaring med \textit{avatars}, har en indflydelse på evnen til at antropomorfisere og acceptere robotter. \textit{Community} defineres som forholdet mellem individet og den sociale struktur de tilhører, \parencite[ss. 20-21]{PDF:CloseButNotStuck}. Det essentielle ved \textit{community} dækker over support, \textit{sociability}, information, dannelse af social identitet og tilhørsforhold, \parencite[s. 21]{PDF:CloseButNotStuck}. Det eneste der, adskiller \textit{community} fra et \textit{online community} er, at sidstnævnte foregår på virtuelt. \textcite[s. 25]{PDF:CloseButNotStuck} fandt de testpersoner, som enten var en del af \textit{online community} eller mere involveret med \textit{avatars}, i højere grad var i stand til at genkende menneskelige træk ved robotten, sammenlignet med testpersoner uden disse erfaringer. Ydermere fandt \textcite[s. 26]{PDF:CloseButNotStuck} at testpersoner, som enten var en del af \textit{online community} eller mere involveret med \textit{avatars}, begge var mere villige til at acceptere robotter, som en del af deres sociale og fysiske miljø.  
%

\subsection{Bevægelsesmønstre og udseende}
\label{InteraktionSocialeRobotterParametreBevaegelsesmoenstre}
%
Som nævnt i \fullref{InteraktionSocialeRobotterGenerelleTendenser} forekommer der store kulturelle forskelle i forhold til hvordan sociale robotter perciperes. Ikke nok med det, forekommer der ligeledes store kulturelle forskelle i hvordan robotter bør bevæge sig, blandt andet i forhold til hvor tæt de må komme på. Først vil afstanden mellem robot og menneske blive diskuteret, og derefter hvilken hastighed robotten bør bevæge sig med. 

\subsubsection*{Afstand mellem robot og menneske}
\label{InteraktionSocialeRobotterParametreBevaegelsesmoenstreAfstand}
%
Ifølge \textcite[s. 178]{PDF:HowMayIServeYou} er den intime grænse for hvor tæt et andet menneske, tætte venner og familie, må komme på en mellem 20 cm og 30 cm for syd europærer og japanere. Hvorimod denne grænse for amerikanere og nord europæer er 46 cm til 122 cm. Ydermere har mennesker, der er opvokset i et landdistrikt ligeledes behov for en øget intim grænse, modsat mennesker, der er opvokset i bymiljøer, \textcite[s. 178]{PDF:HowMayIServeYou}. Endvidere tyder det på, at kvinder normalvis er tættere på hinanden og vendt mere direkte ansigt-til-ansigt, sammenlignet med mænd, \parencite[s. 178]{PDF:HowMayIServeYou}. Det må derfor antages at kvinder, har en mindre intim grænse end mænd.

I undersøgelsen foretaget af \textcite{PDF:HowMayIServeYou}, undersøger de blandt andet hvor tæt en robot må komme før den overskrider den intime grænse, afhængigt af indgangsvinkel: Frontal, højre eller venstre. Robotten holdte en afstand svarende til 50 cm $\pm$ 10 cm. I den første undersøgelse foretaget af \textcite[s. 174]{PDF:HowMayIServeYou}, angav 76 \% testpersonerne at afstanden mellem dem og robotten var \textit{about right}, 19 \% angav at afstanden var for stor og 5 \% angav at robotten kom for tæt på. I den anden undersøgelse foretaget af \textcite[s. 175]{PDF:HowMayIServeYou}, angav 53 \% af testpersonerne, at når robotten nærmede sig frontalt, at den kom for tæt på, 27 \% angav at afstanden var \textit{about right} og 20 \% angav af afstanden var for stor. Da robotten nærmede sig fra venstre, angav 80 \% af testpersonerne at det var \textit{about right} og 20 \% angav at afstanden var for stor. Da robotten nærmede sig fra højre, angav 60 \% af testpersonerne at afstanden var \textit{about right} og 40 \% angav at afstanden var for stor. I alle tilfælde foregår interaktionen mellem robotten og et siddende menneske.

I henhold til hvilken retning robotten skal nærme sig testpersonerne med fremgår det, at 59 \% foretrækker at robotten nærmer sig fra højre, 28 \% foretrækker venstre og 13 \% foretrækker frontalt, \parencite[s. 175]{PDF:HowMayIServeYou}. Lignende tendens går igen ved hvor praktisk og komfortabelt testpersonerne vurder interaktionen, \parencite[ss. 175-176]{PDF:HowMayIServeYou}. Der skal dog tages højde for kønsforskelle, da en del af kvinderne faktisk foretrækker at robotten nærmer sig frontalt, hvilket formentlig hænger sammen med at kvinder i højere grad interagerer med hinanden ansigt-til-ansigt, sammenlignet med mænd, \parencite[s. 178]{PDF:HowMayIServeYou}. Dog argumenterer \textcite[s. 178]{PDF:HowMayIServeYou} for, at den frontale indgangsvinkel opleves som værende ukomfortabel, upraktisk, truende og konfronterende, hvorfor det bør undgåes. I det henseende kommenterer \textcite[s. 178]{PDF:HowMayIServeYou} at i situationer, hvor der er en 45$^{\circ}$s vinkel mellem to samtale partnere, vil følelser som aggression og konfrontation være reduceret. Derudover skal der tages højde for at 94 \% af testpersonerne er højrehåndede, hvilket formentlig er årsagen til at størstedelen af testpersonerne foretrækker at robotten nærmer sig fra højre, \parencite[s. 175]{PDF:HowMayIServeYou}.\blankline
%
En anden undersøgelse, der blandt andet fokuserer på afstanden mellem robot og menneske, er foretaget af \textcite{PDF:HumanRobotEmodiedInteraction}. Der differentieres mellem fire forskellige afstande, som er gældende for menneske-menneske interaktion. \textit{Intimate distance} denne afstand går direkte fra kroppen til omkring 45 cm fra kroppen, og generelt beregnet til direkte fysisk kontakt eller privat interaktion \parencite[s. 165]{PDF:HumanRobotEmodiedInteraction}. Dernæst er \textit{personal distance}, som går fra 45 cm til 1,2 m, denne afstand gør sig gældende for interaktion med familie og venner eller ved organiserede interaktion, som opstår når mennesker står i kø, \parencite[s. 165]{PDF:HumanRobotEmodiedInteraction}. Den tredje afstand er \textit{social distance}, som går fra 1,2 m til 3,5 m og som er beregnet til mere formelle og arbejdsrelateret interaktioner, interaktioner med bekendte. Ydermere fungerer \textit{social distance} også som en form for separering mellem mennesker på offentlige steder, eksempelvis på stranden, \parencite[s. 165]{PDF:HumanRobotEmodiedInteraction}. Den sidste afstand er \textit{public distance}, som starter fra 3,5 m og anvendes i situationer, hvor der er envejskommunikation, eksempelvis mellem publikum og den optrædende, \parencite[s. 165]{PDF:HumanRobotEmodiedInteraction}.

Baseret på \textcite[ss. 169-170]{PDF:HumanRobotEmodiedInteraction} anbefales det, at når en robot skal passere et mennesker bør robotten signalerer det med en afstand på 4 m til 6 m fra mennesket, formentlig er en afstand på 3,5 m også acceptabelt, da det overholder \textit{public distance}. Såfremt robotten i tide signalerer dels at den nærmer sig og dels dens intention, som er at passere mennesket, er det ikke et problem at passagen foregår indenfor \textit{personal distance} og endda kan en mindre afstand også accepteres, \parencite[s. 170]{PDF:HumanRobotEmodiedInteraction}. I tillæg kommenterer \textcite[s. 170]{PDF:HumanRobotEmodiedInteraction}, at afstanden mellem robot og menneske under passagen er mindre vigtig så længe robotten signalerer sin intention i tide, så mennesket ligeledes kan reagerer på interaktionen, eksemplvist ved at bevæge sig til siden.    

\subsubsection*{Robottens hastighed}
\label{InteraktionSocialeRobotterParametreBevaegelsesmoenstreHastighed}
%
Ifølge \textcite[s. 165]{PDF:HumanRobotEmodiedInteraction} er den normale ganghastighed for et menneske mellem 3,6 km/h og 7,2 km/h, hvorfor det må forventes at en social robot ikke bør overskride det interval ved HRI. I undersøgelsen foretaget af \textcite[s. 175]{PDF:HowMayIServeYou} varierede robottens hastighed sig mellem 0,9 km/h og 1,44 km/h, hvor 60 \% af testpersonerne angav at hastigheden var \textit{about right} og 40 \% angav at hastigheden var for langsom. Ifølge \textcite[s. 178]{PDF:HowMayIServeYou} bør en robot, efter en tilvænningsperiode eller efter behov, bevæge sig hurtigere end 1,44 km/h. 

I undersøgelsen foretaget af \textcite[s. 169]{PDF:HumanRobotEmodiedInteraction} svinger robottens gennemsnitlige hastighed mellem 0,9 km/h og 1,4 km/h. Årsagen til at robottens gennemsnitlige hastighed ikke er højere skyldes, at når robotten passerer et menneske sænkes hastigheden, \parencite[s. 169]{PDF:HumanRobotEmodiedInteraction}. Baseret på testpersonernes vurdering af robottens hastighed ville en højere hastighed være at foretrække, \parencite[s. 169]{PDF:HumanRobotEmodiedInteraction}. Ydermere tyder det på, at robottens lave hastighed blev perciperet som værende mindre sikker og tilmed irriterende, \parencite[s. 169]{PDF:HumanRobotEmodiedInteraction}, hvorfor der bør anvendes en højere hastighed. For at afgøre hvilken hastighed robotten bør bevæge sig med argumenterer \textcite[s. 167]{PDF:HumanRobotEmodiedInteraction} for, at robotten bør være i stand til at tilpasse sin hastighed afhængigt af menneskets hastighed, da dette vil medføre en mere dynamisk interaktion mellem robot og menneske. Ifølge \textcite[s. 1897]{PDF:NavigationForHRITasks} bør robotten sænke hastigheden når den nærmer sig en gruppe af mennesker, ligesom et menneske ville gøre det, hvis det nærmede sig en gruppe af mennesker.

\textcite[ss. 192-103]{PDF:PsychologicalEffects} undersøger tre forskellige hastigheder, fordelt på to måder robotten kan nærme sig et menneske på. Robotten kan enten nærme sig direkte eller indirekte, ved direkte er hastigheden enten 0,9144 km/h eller 3,6576 km/h og ved indirekte er hastigheden 1,8288 km/h, \parencite[ss. 192-103]{PDF:PsychologicalEffects}. En indirekte tilgang gengiver, at robotten først drejer mod venstre og derefter mod højre, hvilket giver en fornemmelse af at robotten bevæger sig en anelse til højre for mennesket, \parencite[s. 193]{PDF:PsychologicalEffects}. Med udgangspunkt i robottens hastighed fandt \textcite[s. 196]{PDF:PsychologicalEffects}, at den direkte hastighed på 3,6576 km/h var ubehagelig og at den indirekte hastighed på 1,8288 km/h var at foretrække, da den var mest behagelig. \blankline  
%
Baseret på de tre undersøgelser tyder det på, at hastigheder til og med 1,44 km/h i nogen grad kan accepteres, men det vil være, at foretrække hvis hastigheden var højere. Eftersom 1,44 km/h er langsommere end menneskets normale ganghastighed bør det være muligt, at øge robottens hastighed uden det påvirker interaktionen med mennesket negativt. Det lader rent faktisk til, at en øget hastighed potentielt kan forbedre interaktionen mellem robot og menneske, hvorfor der bør tages højde for dette. Dog tyder det på at robottens hastighed ikke må overstige mennesket normale ganghastighed.       
%

\subsubsection*{Robottens udseende}
\label{InteraktionSocialeRobotterParametreBevaegelsesmoenstreUdseende}
%
Ifølge \textcite[s. 226]{PDF:SocailAndCollaborative} behøver en social robot ikke at have en menneskelignende krop, da bare det at robotten bevæger sig på en ikke-forudsigelig måde kan få mennesker til at percipere robotten, som havende en intention og en personlighed, jævnfør antropomorfisering. Dette hænger formentligt sammen med hypotesen omkring \textit{Uncanny Valley} fremsat af \textcite{PDF:UncannyVally}. Hypotesen bygger på hvilken effekt robottens udseende har på det tilhørsforhold mennesket oplever, hvor tilhørsforholdet kraftig forringes hvis robotten afspejler alt for menneskelige træk, både i forhold til udseende og bevægelse, \textcite{PDF:UncannyVally}. I tillæg pointerer \textcite[s. 226]{PDF:SocailAndCollaborative}, at robottens størrelse, form, farve og bevægelsesmønstre har en indflydelse på hvordan robotten perciperes.    





  Butler and Agah [5], the speed of a robot as well as its way of approaching a human affected people’s perceptions of it. Subjects were most comfortable with a slower robot, or a robot approaching indirectly rather than directly. Moreover, their reactions were affected by the robot’s exterior shape: the level of discomfort perceived during a fast approach increased if the robot had a tall and humanoid body rather than a short, neutral cylindrical shape. Smooth movements were preferred over jerky ones, and the distance to the robot was also a factor contributing to the subjects’ perception of comfort. Adding a tall, humanoid robot body attachment caused comfort levels to drop in all cases studied.\textcite[s. 226]{PDF:SocailAndCollaborative}
 
The height of the entire robot with the body attachment is about 67 inches. \textcite[s. 189]{PDF:PsychologicalEffects} 67 inches = 170,18 cm. Den har et hovede, krop og to arme (billedet på  \textcite[s. 192]{PDF:PsychologicalEffects}. 

 two social robots whose main difference is their size: one is about 30 cm tall and other is about 120 cm. We focused on different sizes because size affects a robot’s attractiveness and the ease of initiating interaction. A normal-sized robot might attract more people than a relatively small robot. On the other hand, interaction with a small robot is probably easier. Based on these considerations, it remains unknown which size is better for advertising. Of course, loud sounds or conspicuous movements easily draw attention to a normal-sized robot, but they might also discourage people. So we focused on robot size and its effect (e.g., ease of initiating interaction) \textcite[s. 255]{PDF:RecommendationEffects}
 
  The number of people who initiated interaction with the small robot was larger than the case of the normal-sized robot.
Our results also showed partial support for our third hypothesis and support for our fourth hypothesis. When we only used the small robot, people who initiated interaction with it printed more coupons than people who were approached by it. The number of people who printed coupons with the small robot was larger than the case of normal-sized robots. These results suggest that the small robot has an ad- vantage for advertisement-use settings.  Loud sounds or conspicuous movements easily attract attention, but they are often discourage interaction. In particular, if the robot is normal-sized, people might be more afraid of it, especially young children.\textcite[s. 260]{PDF:RecommendationEffects}


 
 The results showed that the participants liked the happy robot more, but that they followed the robot’s instructions to a greater extent in the condition with the serious robot. It is con- ceivable that a robot’s personality would contribute to a user’s sense of trust in its instructions. \textcite[s. 226]{PDF:SocailAndCollaborative}
 




 
Kom ind på uncanny valley





