\section{Interaktion med sociale robotter}
\label{InteraktionSocialeRobotter}
%
Kom ind på de forskellige parametre der har indflydelse på brugerens oplevelse af hvordan det er at interagere med robotten. F.eks. online community (folk der spiller online spil hvor der er en form for kommunikation mellem én eller flere avatars) \blankline
%
Brug Harvey, det japanske studie hvor de tester to robotter i et supermarket, brug exploring influencing variables (Kan være en idé at lave en figur som vise hvordan de hænger sammen - f.eks. den "brainstorm" jeg har lavet på papir), how social distance shapes human-robot interaction og andre. \blankline
%
Kom ind på hvordan det måles, hvad har andre gjort i forhold til det (overvej om det skal være en sektion for sig selv).\blankline
%
Brug det her argument til at forsvare hvorfor det også er vigtigt at undersøge hedonic variabels og hvilken indflydelse de har: Airports were regarded as purely utilitarian infrastructures in the past. To mitigate negative experiences with prolonged security proto- cols, airports have used rebranding strategies to integrate their surroundings or to become reinvented as destinations in their own right instead of just thoroughfares (Tsui, 2014). Aside from operational quality assurance, destination-focused rebranding emphasizes the need to create enjoyable experiences at airports \textcite[s. 352]{PDF:TheImpactOfTraveler}.\blankline
%
Out of the five widely used personality dimensions, namely the extroversion, agreeableness, conscien- tiousness, neuroticism, and openness [5], the most important dimensions for social interactions are those that concern individual differences in social behavior, namely extroversion and agreeableness or their common rotations, ‘friendliness’ and ‘dominance’ [6]. Fra Personality of social robots perceived trough the appearance side. 272.



