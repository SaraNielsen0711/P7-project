\section{Problemformulering}
\label{Problemformulering}
%
Ulempen ved de førnævnte undersøgelser er dels, at HRI foregår over en kort periode, hvorfor interaktionen med robotten er kortvarig og dels at, undersøgelserne som udgangspunkt foretages i laboratorier. I ingen af tilfældene er det muligt for testpersonerne, at danne et decideret tilhørsforhold til robotten, da forholdet stort set kun bygger på førstehåndsindtrykket. Derudover præsenteres robotterne i specifikke kontekster, som ikke nødvendigvis er den kontekst robotten er designet til at indgå i. Som tidligere diskuteret er der både store kulturelleforskelle, aldersforskelle og kønsforskelle, hvorfor de parametre testpersonerne i de førnævnte undersøgelser vurderer robotten ud fra, ikke nødvendigvis er de parametre, der har betydning for danske brugere. Dertil har det ydermere ikke været muligt at finde undersøgelser med sociale robotter foretaget i lufthavne, hvorfor følgende problemstillinger kan opstilles:\blankline
%
\begin{quotation}
	\noindent
	\textit{Ud fra hvilke parametre beskriver danske rejsende interaktion med en social robot i en dansk lufthavn?\blankline
		%
		Hvordan bør en social robot bevæge sig i en lufthavn, når den skal interagere med rejsende?\blankline
		%
		Kan de fundne parametre gengives i skalaer, som efterfølgende kan bruges til at evaluere HRI?}\blankline
\end{quotation}
%
For at besvare den første problemstilling er det nødvendigt, at foretage en form for interview med danske rejsende i en dansk lufthavn, hvor de skal gengive deres subjektive oplevelse af HRI. Baseret på de rejsendes respons skal der foretages en kvalitativ analyse for at kategorisere den givne respons og dermed udlede, hvilke parametre der har indflydelse på de danske rejsendes interaktion med en social robot. Efterfølgende undersøges det hvordan en social robot bør bevæge sig i en lufthavn, når den skal interagere med rejsende. Der fokuseres særligt på robottens hastighed samt hvilken afstand den skal holde til den rejsende. Afhængigt af hvilke parametre, der udledes fra de rejsendes respons vil de ligeledes inkluderes i undersøgelsen af robottens bevægelse. For at besvare den sidste problemstilling vil der blive udarbejdet én eller flere skalaer, som har til formål at gengive de fundende parametre, hvorefter disse vil blive verificeret i en ny undersøgelse, som ligeledes udføres med danske rejsende og i en dansk lufthavn.




