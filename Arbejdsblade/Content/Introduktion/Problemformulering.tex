\section{Problemformulering}
\label{Problemformulering}
%
Ulempen ved de førnævnte undersøgelser er dels, at HRI foregår over en kort periode, hvorfor interaktionen med robotten er kortvarig og dels at, undersøgelserne som udgangspunkt foretages i laboratorier. I ingen af tilfældene er det muligt for testpersonerne, at danne et decideret tilhørsforhold til robotten, da forholdet stort set kun bygger på førstehåndsindtrykket. Derudover præsenteres robotterne i specifikke kontekster, som ikke nødvendigvis er den kontekst robotten er designet til at indgå i. Som tidligere diskuteret er der både store kulturelleforskelle, aldersforskelle og kønsforskelle, hvorfor de parametre testpersonerne i de førnævnte undersøgelser vurderer robotten ud fra, ikke nødvendigvis er de parametre, der har betydning for danske brugere. Dertil har det ydermere ikke været muligt at finde undersøgelser med sociale robotter foretaget i lufthavne, hvorfor følgende problemstillinger kan opstilles:\blankline
%
\begin{quotation}
	\noindent
	\textit{Ud fra hvilke parametre beskriver danske rejsende interaktion med en social robot i en dansk lufthavn?\blankline
		%
		Hvordan bør en social robot bevæge sig i en lufthavn, når den skal interagere med rejsende?\blankline
		%
		Kan de fundne parametre gengives i skalaer, som efterfølgende kan bruges til at evaluere HRI?}\blankline
\end{quotation}
%





Kan bruges EFTER selve formuleringen, til at give et indblik i hvordan projektet er bygget op. 

I dette projekt tages der udgangspunkt i casen om Københavns lufthavn, hvor det undersøges, hvordan folk oplever interaktionen med robotten. Potentielle brugere vil blive interviewet, med henblik på at finde generelle tendenser i deres subjektive oplevelser med den sociale robot. Disse tendenser bruges til at udvikle en eller flere skalaer til evaluering af brugeroplevelsen. Skalaerne kan herefter verificeres med andre potentielle brugere. Hvis de vigtige parametre kan generaliseres til også at gælde andre lignende situationer, som eksempelvis et shoppingcenter, sygehus eller supermarked, vil de udviklede skalaer sandsynligvis kunne bruges til at teste robottens adfærd i nævnte, lignende situationer.

Der startes med et lille studie, der forsøger at afdække hvordan folk snakker om oplevelsen med robotten og især hvilke ord folk bruger til at beskrive deres oplevelse. Det gøres for at være sikker på, at skalaerne udvikles ud fra de reelle brugeres egne forståelser. Derefter vil et forsøgsdesign blive opstillet som tester præcist de parametre, som vi har fundet ud af er vigtige for robottens sociale adfærd.