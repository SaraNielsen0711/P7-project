\section{Fremgangsmåde}
\label{ParametreFremgangsmaade}
%
For at teste hvilke parametre, der er vigtige i en social robot skal robotten præsenteres for testpersonerne, hvorefter de skal medvirke i et interview, for at få sat ord på deres oplevelse med robotten. 

Da det ønskes at observere testpersonernes umiddelbare reaktion på deres første møde med robotten, bruges den også til at rekruttere testpersoner med. Det foregår ved, at robotten kører hen til mennesker i Aalborg Lufthavn og via skærmen spørger folk, om de har lyst til at deltage i en test. Det vælges derudover at robotten skal køre hen til folk på denne måde, for at give testpersonerne en reel oplevelse af, at robotten kommer hen og henvender sig til dem på et tidspunkt, hvor de ikke er opmærksomme på at de er en del af en undersøgelse. De mennesker, der ønsker at deltage i testen, følger efter robotten over til vores testområde, hvor testpersonerne mødes med testlederen.\blankline

Det er vigtigt at der skabes en afslappet stemning, hvor testpersonerne føler sig trykke ved at udtrykke sig og fortælle om oplevelsen med robotten, samtidig med at de ikke føler de spilder dyrebar tid eller bliver stressede over at skulle nå et fly. For at skabe en god stemning, tilbydes testpersonerne en kop kaffe og det er muligt at small-talke omkring eksepelvis hvor testpersonerne skal rejse hen eller har været henne. Selvom det kan virke unødvendigt, er det vigtigt at der bruges lidt tid på denne del af testen, for at sikre at testpersonen føler sig i trygge rammer. 

Når en god stemning er skabt fortæller testlederen om formålet og fremgangsmåden med testen, og testen bedes læse og underskrive en samtykkeerklæring, vedlagt i \fullref{APP:Samtykkeerklaering}.\blankline
%
Selve interaktionen med robotten foregår med en \textit{"Wizard of Oz"} tilgang, hvor en person styrer robotten fra sin computer. Robotføreren sidder ved et bord i nærheden, så vidt muligt bag ved testpersonen, for ikke at tiltrække opmærksomhed. Hvis testpersonen spørger ind til robotføreren får de forklaret, at vedkommende er der for at notere observationer undervejs. 
\textbf{FIND UD AF HVILKEN INTERAKTION BRUGEREN SKAL HAVE MED ROBOTTEN, BASERET PÅ HVILKE BRUGSSCENARIER DER KAN VÆRE I LUFTHAVNEN} \blankline
%
Efter testpersonen er blevet præsenteret for og har interageret med robotten og den kontekst hvori robotten skal fungere, findes det relevant at interviewe testpersonen, for på den måde at forstå hvordan de tænker omkring sociale robotter. Der stilles brede og åbne spørgsmål, for at få testpersonen til at starte med at tale frit. Interviewet med testpersonen optages \textbf{(Eller filmes?)}, for til databehandlingen at kunne nedskrive udtalelser, opbygge et affinitydiagram og på den måde udlede hvilke parametre ved en social robot, der er vigtige for testpersonerne.
