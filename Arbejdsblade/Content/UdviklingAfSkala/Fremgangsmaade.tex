\section{Fremgangsmåde}
\label{ParametreFremgangsmaade}
%
Så snart en af de rejsende befinder sig i det aftalte område vil robotten henvende sig ved at køre hen til den rejsende og spørger om personen har lyst til at deltage i en undersøgelse. Spørgsmålet stilles på skærmen, hvor den rejsende trykke \textit{Ja}, vil robotten opfordre den rejsende til at følger med hen til testlederen for at deltage i undersøgelsen, hvilket ligeledes foregår på skærmen. Trykker den rejsende derimod \textit{Nej} for ikke at deltage i undersøgelsen, ønsker robotten personen god rejse og forlader stedet og venter til at en ny rejsende befinder sig i området. Når testpersonen modtages af testlederen, vil robotten køre væk og vente til at interaktion skal finde sted. 

I mellemtiden er det testlederens opgave at small-talke med testpersonen for at skabe en behagelig stemning. Testlederen kan eksempelvis spørge ind til hvor den rejsende skal hen, om de har rejst fra Aalborg Lufthavn før eller om den rejsende tidligere har besøgt destinationen. Igennem denne small-talk skal testlederen lede samtalen hen på hvad der kommer til at ske i undersøgelsen. Testlederen forklare hvem projektgruppen er, hvad formålet med undersøgelsen er og udlevere en samtykkeerklæring, som testpersonen opfordres til at læse og underskrive. Samtykkeerklæringen fremgår af HENVISNING. Derefter introducerer testlederen yderligere testpersonen i hvilken opgave, der skal løses og at der sidder to observatører og tager noter, hvilket gøres for at skjule robot styrens identitet. Testpersonens opgave er defineret som et brugsscenarie, der afspejler en mulig situation i en lufthavn, jævnfør \fullref{ParametreBrugsscenarier}. Det er også i den forbindelse at testlederne understreger, at det ikke er projektgruppen, der har udviklet robotten, hvorfor der er interesse for både positive og negative kommentarer. 

Undervejs som testpersonen løser opgaven er det muligt for testlederen at stille spørgsmål, såsom hvorfor testpersonen gør som den gør. Efter opgaven er løst vil testlederne udføre en form for semi-struktureret interview, som ikke består af decideret spørgsmål men nærmere samtaleemner, som testpersonen skal forholde sig til. Samtaleemnerne er beskrevet i \fullref{ParametreSamtaleemner}. Testlederen får derfor løbende mulighed for at stille spørgsmål, som skal få testpersonen til at reflektere over hvorfor de oplever det som de gør. Det er i den forbindelse oplagt at anvende robotten til at supplere testpersonens udtalelser. 

Når alle samtaleemner er blevet inddraget og diskuteret er det testlederens opgave at afslutte undersøgelsen ved at debriefe testpersonen og ønske dem en god rejse.       



