\section{Interview}
\label{ParametreInterview}
%
Når testpersonerne er blevet præsenteret for robotten udføres et interview. Under interviewet fokuseres der på at lade testpersonerne tale så meget som muligt, for at undgå at dreje samtalen væk fra de ting, som har optaget vedkommende. For at få uddybende information fra testpersonerne omkring de betydende parametre i en social robot, stilles spørgsmålene på samme måde som ved brug af en UX laddering, der er en tilpasset laddering metode designet til research af brugeroplevelsen, \parencite[ss. 3-4]{PDF:LadderingUX}. Ved brug af UX laddering metoden stilles testpersonen et spørgsmål om hvordan de eksempelvis oplevelede interaktionen med robotten. Når testpersonen svarer, bliver deres svar overvejet, hvorefter der stilles uddybende spørgsmål ind til dette. Sådan fortsættes det op ad stigen, indtil værdierne bag deres holdning til den sociale robot er blevet udvundet. Når UX laddering skal bruges, er kontekst vigtig, \parencite[s. 3]{PDF:LadderingUX}, hvorfor det er positivt, at testpersonerne er blevet præsenteret for robotten i en kontekst den reelt set kunne indgå i. 

Når en testperson starter med at besvare spørgsmål, er det sandsynligt at vedkommende svarer ved at opstille funktionelle konsekvenser, \parencite[s. 3]{PDF:LadderingUX}. Her er det vigtigt at få testpersonen til at træde et par skridt ned ad trappen igen, så spørgsmålene og svarene starter ved konkrete egenskaber. Dette kan gøres ved at stille spørgsmålene "Hvad skyldes det?", frem for at spørge hvorfor testpersonen mener noget. Ydermere er det vigtigt, at interviewet ikke tager for lang tid, \parencite[s. 4]{PDF:LadderingUX}. Testlederne skal naturligvis spørge videre ind til psykosociale konsekvenser og videre til de bagvedlæggende værdier, men kun hvis dette virker naturligt i samtalen. Det er ikke alle oplevelser, der trigger virkelige værdier, hvorfor det kan være nødvendigt at stoppe med at stille spørgsmål, hvis det ikke længere virker naturligt. \blankline
%
UX laddering bruges altså i denne test til at forme interviewet og forstå værdierne bag testpersonens udtalelser om robotten, hvorfra de vigtige parametre kan udvindes.

For at testpersonerne dog bliver spurgt om nogenlunde de samme ting opstilles predefinerede spørgsmål, hvorfra laddering vil blive brugt. Spørgsmålene til interviewet er som følger:\blankline
%NEDENSTÅENDE SPØRGSMÅL ER IKKE FÆRDIGE
\begin{itemize}
	\item Hvordan oplevede du den her situation med robotten?
	\item Hvad synes du om robotten?
	\item Hvordan vil du beskrive den her robot?
	\item Hvad synes du om at have en robot til at hjælpe dig med…
	\item Kunne du forestille dig at interagere med en lignende robot her i lufthavnen i fremtiden?
	\item Hvordan tror du at det vil påvirke din oplevelse af lufthavnen?
	\item Overvejede du hvad andre tænkte, når de så dig interagere med robotten? Hvordan havde du det med det?
	\item hvordan har du det med at interagere med robotten foran andre mennesker?\blankline
\end{itemize}
%

Det kan være relevant at fortælle testpersonerne om den beskrevne laddering interview-metode, så de ikke bliver irriteret over de konstant uddybende spørgsmål. Derudover kan det være relevant at fortælle testpersonerne, at det ikke er projektgruppen, der har designet Double-robotten, hvorfor de frit kan fortælle hvad der falder dem ind, når de interagerer med robotten. Hvis testpersonerne begynder at snakke om generelle oplevelser eller hvordan andre vil forstå robotten, er det vigtigt at spørge dem ind til deres egne oplevelser frem for andres. \blankline
%
Ved udførelsen af interviewet er det vigtigt, at testlederen fremstår venlig og tilstedeværende, i stedet for at følge et manuskript. Derudover kan det være en god idé at "spille dum", for at få testpersonerne til virkelig at beskrive hvad de mener, frem for selv at tolke på det de siger. Dette gør sig også gældende, selvom det de siger kan virke åbenlyst. Ydermere skal testlederen tænke over kropssprog og mimik, da det er nemt for testpersonen at se, hvornår de har sagt noget, som testlederen havde håbet på. Alle forudindtagelser og erfaringer fra tidligere tests skal derfor glemmes, så der på den måde kun fokuseres på testpersonens svar. 
