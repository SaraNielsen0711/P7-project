\section{Semi-struktureret interview}
\label{ParametreInterview}
%
Når der er skabt en god stemning og testpersonerne er blevet præsenteret for robotten
Interview scenarie:
Skab en god stemning og kom på bølgelængde med forsøgspersonerne. Tilbyd en kop kaffe mens vi snakker om hvad forsøget handler om og hvad deres opgave er. Forsøg at gøre det naturligt og få dem til at sige noget også, så det ikke er uvant i interviewet. Fortæl om hvordan vi bruger laddering, for at undgå at de bliver irriterede undervejs. Bed om tilladelse til at tage billeder og optage lyd. Fortæl at vi ikke har lavet robotten og at vi ikke bliver kede af det hvis de ikke kan lide den, men at vi gerne vil have deres ærlige mening om den.
Lad dem interagere med robotten i den opstillede kontekst.
Observer dem mens de interagerer, tag noter og billeder.
Semistruktureret interview med laddering. Snak så lidt som muligt, men sørg for at holde samtalen på sporet. Bed forsøgspersoner om at uddybe og spørg hele tiden hvorfor de siger som de gør, eller bed dem om at forklare det på en anden måde, hvis noget er uklart eller kan misforstås. For at holde samtalen på sporet, kan vi spørge ind til nogle af de her ting:
-	Hvordan oplevede du den her situation med robotten?
-	Hvad synes du om robotten?
-	Hvordan vil du beskrive den her robot?
-	Hvad synes du om at have en robot til at hjælpe dig med…
-	Kunne du forestille dig at interagere med en lignende robot her i lufthavnen i fremtiden?
-	Hvordan tror du at det vil påvirke din oplevelse af lufthavnen?
-	Overvejede du hvad andre tænkte, når de så dig interagere med robotten? Hvordan havde du det med det?
-	hvordan har du det med at interagere med robotten foran andre mennesker?

Nogle af dem er ja/nej spørgsmål, men det tror jeg er fint nok, hvis bare vi så spørger mere ind til dem bagefter og dykker ned i deres svar. 