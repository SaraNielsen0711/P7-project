\section{Samtaleemner}
\label{ParametreSamtaleemner}
%
Gennem hele undersøgelsen er det vigtigt at holde samtalen i gang for at få indsamlet data der reflekterer deres opfattelse af robotten. For at sikre at testlederen har relevante emner at spørge ind til og for for at sikre at relevante emner bliver beskrevet, er det valgt at opstille samtaleemner for undersøgelsen. Samtaleemner er inddelt i tre grupper, som er henholdsvis emner der skal fungere som indledning, emner til undervejs i undersøgelsen og afsluttende samtaleemner. Alle samtaleemner er vejledende til testlederen og vil derfor ikke nødvendigvis blive formuleret ordret til testpersonerne som de står skrevet her.
%
\subsection{Indledende samtaleemner} 
%
Samtaleemnerne til den første del af undersøgelsen, har til formål at få samtalen i gang mellem testpersonen og testlederen. Samtalen fungerer primært som en ice-breaker, hvor de to parter har mulighed for at snakke lidt løst om nogle emner, hvor det er nemt at få testpersonen til at sige noget. Spørgsmålene er udformet på baggrund af den kontekst hvor samtalen udspiller sig i, nemlig i lufthavnen, for på den måde at få spørgsmålene til at blive opfattet som naturlige spørgsmål at stille netop her. 
(Sig eventuelt til forsøgspersonerne at vi er ved at sætte noget op teknisk som vi lige venter på, så vi har lidt tid til at small-talke i, uden at det bliver mærkeligt at vi snakker om noget urelevant)
%
\begin{itemize}
\item Testpersonens rejse (Destination, formål, varighed, medrejsende)
\item Erfaringer med rejse (Første gang, lignende destinationer, kendeskab til destinationen) 
\item Testpersonens baggrund (Job, bopæl, interesser, osv.)
\end{itemize}
%
\subsection{Løbende samtaleemner} 
De løbende samtaleemner er primært emner der ønskes at sikre at få belyst, da de forventes at være relevante for beskrivelsen af testpersonernes opfattelse af robotten. 
%
\begin{itemize}
\item Førstehåndsindtryk af robot før forsøget startede (rekrutteringen)
\item Måden den henvender sig på
\item Hvad de synes om den
\item Hvad de tror andre tænker 
\item Om de kunne se et behov for den og om det giver mening at have en robot til at hjælpe sig
\item Hvordan de normalvis oplever en tur gennem lufthavnen, uden hjælp fra robot
\item Hvilke problemer de normalt har i lufthavne (hvis nogen) 
\end{itemize}
%
\subsection{Afsluttende samtaleemner} 
Disse emner har det formål at indsamle evaluerende indtryk og efterfølgende afrunde undersøgelsen.
%
\begin{itemize}
\item Spørgsmål og/eller kommentarer til undersøgelsen 
\item Spørgsmål og/eller kommentarer til vores projekt
\item Flyafgang - Tidspunkt for indtjekning ved gate. 
\end{itemize}

