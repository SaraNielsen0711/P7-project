\section{Samtaleemner}
\label{ParametreSamtaleemner}
%
Gennem hele undersøgelsen er det vigtigt, at holde samtalen i gang for at indsamle data, der gengiver testpersonernes oplevelse af interaktionen med robotten. For at sikre at testlederen har relevante emner at spørge ind til er det valgt, at opstille samtaleemner for undersøgelsen. Samtaleemnerne er inddelt i tre grupper: Indledende, løbende og afsluttende samtaleemner. Alle samtaleemner er vejledende, hvorfor de ikke nødvendigvis bliver formuleret ordret af testlederen.

\subsection{Indledende samtaleemner} 
\label{ParametreIndledendeSamtaleemner}
%
Samtaleemnerne til den første del af undersøgelsen, har til formål at få samtalen i gang mellem testperson og testleder. Samtalen fungerer primært som en ice-breaker, hvor de to parter har mulighed for at small-talke og hvor det er muligt, at få testpersonen til at involvere sig. Spørgsmålene er udformet på baggrund af den kontekst, hvor samtalen udspiller sig i, nemlig i Aalborg lufthavn, for på den måde at få spørgsmålene til, at blive opfattet som naturlige spørgsmål at stille netop i denne kontekst. Et forslag til hvordan testlederen kan starte small-talken er, at fortælle testpersonen at de andre er ved at sætte noget teknisk op, som de lige skal vente på. Samtaleemnerne er som følger:\blankline
%
\begin{itemize}
\item Testpersonens rejse: Destination, formål, varighed, medrejsende
\item Testpersonens erfaring med at rejse: Første gang, lignende destinationer, kendeskab til destinationen 
\item Testpersonens baggrund: Job, bopæl, interesser og lignende
\end{itemize}
%
\subsection{Løbende samtaleemner} 
\label{ParametreLoebendeSamtaleemner}
%
De løbende samtaleemner henvender sig primært til emner, der forventes at være relevante for testpersonernes oplevelse af interaktionen med robotten. Samtaleemnerne er som følger:\blankline
%
\begin{itemize}
\item Førstehåndsindtryk af robotten før testen starter; fra rekrutteringen
\item Måden robotten henvender sig på
\item Hvad testpersonerne synes om robotten
\item Hvad testpersonerne tror andre rejsende tænker om interaktionen 
\item Om testpersonerne kan se et behov for robotter i lufthavne og om det giver mening, at have en robot til at hjælpe sig
\item Hvordan testpersonerne normalvis oplever en tur gennem lufthavnen, uden hjælp fra robot
\item Hvilke problemer testpersonerne normalt har i lufthavne, hvis nogen 
\end{itemize}
%
\subsection{Afsluttende samtaleemner} 
\label{ParametreAfsluttendeSamtaleemner}
%
De afsluttende samtaleemner har til formål at afrunde undersøgelsen. Samtaleemnerne er som følger: \blankline
%
\begin{itemize}
\item Spørgsmål og/eller kommentarer til undersøgelsen 
\item Spørgsmål og/eller kommentarer til vores projekt
\item Afrunding og ønsk testpersonerne god rejse
\end{itemize}

