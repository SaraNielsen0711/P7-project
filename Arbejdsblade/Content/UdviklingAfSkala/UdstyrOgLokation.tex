\section{Testlokation og udstyr}
\label{ParametreTestlokationOgUdstyr}
%
Eftersom at robotten fortrinsvist skal indgå i en lufthavn og da formålet med denne undersøgelse er, at undersøge ud fra hvilke parametre danske rejsende beskriver interaktionen med en social robot i en dansk lufthavn, er det favorabelt at udføre undersøgelsen i en dansk lufthavn. Som nævnt har det ikke været muligt at udføre undersøgelsen i Københavns Lufthavn, hvor robotten er tiltænkt, hvorfor undersøgelsen foretages i Aalborg Lufthavn. Lufthavnsvalget har formentlig indflydelse på testpersonernes oplevelsen og dermed den indsamlede data, da der er stor forskel på de to lufthavne. Dette kommer særligt til udtryk ved antal rejsende, hvor Københavns Lufthavn, tiltrods for et fald på 3000 rejsende siden sidste år, servicerede 2.672.710 rejsende i september måned, \parencite{WEB:CPHStatistisk}. Dette er en del flere rejsende end hvad Aalborg Lufthavn oplevede i samme måned, hvor der blev serviceret 155.547 rejsende, \parencite{WEB:AALStatistik}. Derudover er der stor størrelses mæssigt forskel på de to lufthavne, hvor Aalborg Lufthavn har én terminal med 11 gates, \parencite{WEB:AALTerminalOversigt}, har Københavns Lufthavn to terminaler med 102 gates, \parencite{WEB:CPHTerminalOversigt}. Alene baseret på antallet af terminaler forventes det, at det tilmed bliver sværere at lokalisere sin gate dels på oversigtstavlen og dels i forhold til gatens fysiske lokalitet, hvorfor det ligeledes forventes, at der er større behov for den sociale robot i Københavns Lufthavn sammenlignet med Aalborg Lufthavn. Dette kan potentielt begrænse testpersonernes forståelse for, hvorfor det kan være en fordel at interagere med robotten, fremfor at finde information selv, da det ikke er sikkert at de oplever behov for assistance.      





Lufthavnen efter security. 
Vi markerer ikke et område, men vi har internet en idé/aftale om hvilket område vi tester i. Det er muligt for andre at gå ind i området og har lyst til at blande sig, men vi sørger for at der er en, der tager imod dem og fortæller dem at de kan for lov til selv at prøve lige om lidt. Det kan give en mere realistisk oplevelse og giver mulighed for at rekruttere folk der rent faktisk har lyst. 

Udstyr: Double robot, iPad Air2, computer med internet adgang (Karl har forslået et internt netværk f.eks. internetdeling), hvorfra Double styres, computer til at tage noter (observatør), lydoptager f.eks. egne telefoner, Interviewguide, tidsmåler til observatør (f.eks. egne telefoner). 

Borde og stole til observatør og robotfører, måske et højbord til interviewer og TP.   

Forplejning i lufthavnen



 hvorfor vi bruger Double fremfor den "rigtige" robot og hvilken effekt det har på brugerens oplevelse. Diskuter hvilken effekt det har at robotten delvist er en prototype og delvist en færdig robot.          

