\section{Udstyr og testlokation}
\label{UdstyrOgTestlokationValgAfGestikker}
%
Bare lige skriv om hvilke ting vi bruger og hvor vi tester og så tag billeder (lav figur?) af testopstilling i lufthavnen. 

Lufthavnen efter security. 
Vi markerer ikke et område, men vi har internet en idé/aftale om hvilket område vi tester i. Det er muligt for andre at gå ind i området og har lyst til at blande sig, men vi sørger for at der er en, der tager imod dem og fortæller dem at de kan for lov til selv at prøve lige om lidt. Det kan give en mere realistisk oplevelse og giver mulighed for at rekruttere folk der rent faktisk har lyst. 

Udstyr: Double robot, iPad Air2, computer med internet adgang (Karl har forslået et internt netværk f.eks. internetdeling), hvorfra Double styres, computer til at tage noter (observatør), lydoptager f.eks. egne telefoner, Interviewguide, tidsmåler til observatør (f.eks. egne telefoner). 

Borde og stole til observatør og robotfører, måske et højbord til interviewer og TP.   

Forplejning i lufthavnen


Diskuter hvilken betydning det har at vi bruger Aalborg Lufthavn fremfor Københavns Lufthavn og hvorfor vi bruger Double fremfor den "rigtige" robot og hvilken effekt det har på brugerens oplevelse. Diskuter hvilken effekt det har at robotten delvist er en prototype og delvist en færdig robot.          

