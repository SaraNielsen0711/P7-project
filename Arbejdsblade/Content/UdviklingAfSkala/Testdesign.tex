	\chapter{Parametre med indflydelse på sociale robotter}
\label{ParametreSocialeRobotter}
%
Formålet med denne undersøgelse er at få folk til at sætte deres egne ord på oplevelsen af interaktionen med en social robot. Herefter er ønsket at bruge testpersonernes egne ord til at forstå hvilke parametre, der er vigtige at tage højde for, når sociale robotter skal designes. Her er det ikke nødvendigvis vigtigt at undersøge, om en given interaktion fungerer, men nærmere hvordan den skal være for at fungere. For at finde ud af, hvordan reelle brugere snakker om robotten i den rigtige kontekst, opstilles et feltstudie i en lufthavn. Det vælges at benytte Aalborg Lufthavn til dette feltstudie, da kontakt og aftaler med Københavns Lufthavn er tidskrævende og ikke nødvendigvis kan ske inden for projektets tidsperiode. Ydermere er det nemmere at transportere robotten til Aalborg Lufthavn og der er mere ro i lufthavnen til at køre faktiske tests.\blankline
%
Feltstudier, hvor brugerne mødes ude i den virkelig verden, kan ofte give et bedre billede af hvordan brugere faktisk interagerer med produkter. Det kan også afsløre nogle af de problemer eller holdninger til produktet, som ikke bliver opdaget i mere kliniske tests. I denne test vælges det at testpersonerne skal interagere med robotten, hvorefter de vil blive bedt om at fortælle om deres oplevelse, hvilket vil blive uddybet yderligere i de følende afsnit.
