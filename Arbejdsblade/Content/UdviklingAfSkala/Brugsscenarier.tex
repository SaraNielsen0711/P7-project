\section{Brugsscenarier}
\label{ParametreBrugsscenarier}
%
Det er her vores brugsscenarier, opgaver og kontekster skal skrives\blankline
%
%FØLGENDE ER EN BLANDING FRA TO DOKUMENTER
Projektet kigger på interaktionen og det sociale aspekt i robotter tilknyttet en lufthavn, hvorfor det er relevant at opstille et scenarie baseret på en lufthavns situation og med brugere, der benytter lufthavnen.\blankline

%Skal vi ud i lufthavnen i kbh? I Aalborg? Finde parametrene med testpersoner i et supermarked?

....Præsentation af robotten kan ske på flere måder. Skal vi foruddefinere en måde at køre på eller skal vi fjernstyre den? Fordele og ulemper? ..... På denne måde kan brugeren forestille sig hvordan den færdige robot vil fungere, selvom denne ikke er færdigudviklet endnu.\blankline


Efter brugeren er blevet præsenteret for robotten og den kontekst hvori robotten skal fungere, findes det relevant at interviewe brugeren, for på den måde at forstå hvordan de tænker omkring sociale robotter.

%STOD TIDLIGERE I METODE
Da projektet er rettet mod interaktionen og det sociale aspekt i robotter tilknyttet en lufthavn, er det relevant at opstille et scenarie baseret på en lufthavnssituation og med brugere, det benytter lufthavnen. Da undersøgelsen udføres som et feltstudie er det også vigtigt at opstille en kontekst, hvori robotten potentielt vil begå sig i. Denne kontekst kan eksempelvis være hvordan robotten skal tage kontakt til og interagere med brugeren, hvis vedkommende skal have hjælp til at finde sin gate eller sin bagage.\blankline
