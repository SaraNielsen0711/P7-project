\section{Metode}
\label{ParametreMetode}
%
Følgende afsnit beskriver, hvordan robotten præsenteres for testpersonerne, hvordan interaktionen skal foregå samt hvilke metoder, der bruges for at få den relevante information fra testpersonerne.\blankline
%
Da projektet er rettet mod interaktionen og det sociale aspekt i robotter tilknyttet en lufthavn, er det relevant at opstille et scenarie baseret på en lufthavnssituation og med brugere, det benytter lufthavnen. Da undersøgelsen udføres som et feltstudie er det også vigtigt at opstille en kontekst, hvori robotten potentielt vil begå sig i. Denne kontekst kan eksempelvis være hvordan robotten skal tage kontakt til og interagere med brugeren, hvis vedkommende skal have hjælp til at finde sin gate eller sin bagage.\blankline
%
Da inspirationen til den sociale robot kommer mere eller mindre fra Double, vælges det at bruge denne til testen. Udover at inspirationen kommer herfra, så kan det påvirke folk betydeligt om de interagerer med en færdig robot eller en prototype, der hverken har det rigtig udseende eller de rigtige bevægelser. For at testpersonerne kan forstå, hvordan robotten skal opføre sig, specielt med en mere dynamisk bevægelse, vælges det at bruge Double i stedet for at sætte prototypen på en robotstøvsuger og fjernstyre denne. 

\subsection{Fremgangsmåde}
\label{ParametreFremgangsmaade}
%
For at teste hvilke parametre, der er vigtige i en social robot skal robotten præsenteres for testpersonerne, hvorefter de skal medvirke i et interview, for at få sat ord på deres oplevelse med robotten. 

Da det ønskes at observere testpersonernes umiddelbare reaktion på deres første møde med robotten, bruges den også til at rekruttere testpersoner med. Det foregår ved, at robotten kører hen til mennesker i Aalborg Lufthavn og via skærmen spørger folk, om de har lyst til at deltage i en test. Det vælges derudover at robotten skal køre hen til folk på denne måde, for at give testpersonerne en reel oplevelse af, at robotten kommer hen og henvender sig til dem på et tidspunkt, hvor de ikke er opmærksomme på at de er en del af en undersøgelse. De mennesker, der ønsker at deltage i testen, følger efter robotten over til vores testområde, hvor testpersonerne mødes med testlederen.\blankline

Det er vigtigt at der skabes en afslappet stemning, hvor testpersonerne føler sig trykke ved at udtrykke sig og fortælle om oplevelsen med robotten, samtidig med at de ikke føler de spilder dyrebar tid eller bliver stressede over at skulle nå et fly. For at skabe en god stemning, tilbydes testpersonerne en kop kaffe og det er muligt at small-talke omkring eksepelvis hvor testpersonerne skal rejse hen eller har været henne. Selvom det kan virke unødvendigt, er det vigtigt at der bruges lidt tid på denne del af testen, for at sikre at testpersonen føler sig i trygge rammer. 

Når en god stemning er skabt fortæller testlederen om formålet og fremgangsmåden med testen, og testen bedes læse og underskrive en samtykkeerklæring, vedlagt i \fullref{APP:Samtykkeerklaering}.\blankline
%
Selve interaktionen med robotten foregår med en \textit{"Wizard of Oz"} tilgang, hvor en person styrer robotten fra sin computer. Robotføreren sidder ved et bord i nærheden, så vidt muligt bag ved testpersonen, for ikke at tiltrække opmærksomhed. Hvis testpersonen spørger ind til robotføreren får de forklaret, at vedkommende er der for at notere observationer undervejs. 
\textbf{FIND UD AF HVILKEN INTERAKTION BRUGEREN SKAL HAVE MED ROBOTTEN, BASERET PÅ HVILKE BRUGSSCENARIER DER KAN VÆRE I LUFTHAVNEN} \blankline
%
Efter testpersonen er blevet præsenteret for og har interageret med robotten og den kontekst hvori robotten skal fungere, findes det relevant at interviewe testpersonen, for på den måde at forstå hvordan de tænker omkring sociale robotter. Der stilles brede og åbne spørgsmål, for at få testpersonen til at starte med at tale frit. Interviewet med testpersonen optages \textbf{(Eller filmes?)}, for til databehandlingen at kunne nedskrive udtalelser, opbygge et affinitydiagram og på den måde udlede hvilke parametre ved en social robot, der er vigtige for testpersonerne.
