\section{Præsentation af robotten}
%
Projektet kigger på interaktionen og det sociale aspekt i robotter tilknyttet en lufthavn, hvorfor det er relevant at opstille et scenarie baseret på en lufthavns situation og med brugere, der benytter lufthavnen.

Det vælges at benytte Aalborg Lufthavn til at køre test, da kontakt og aftaler med Københavns Lufthavn er tidskrævende og ikke nødvendigvis kan ske inden for projektets tidsperiode. Ydermere er det nemmere at transportere robotten til Aalborg Lufthavn og der er mere ro i lufthavnen til at køre faktiske tests.\blankline

For at teste hvilke parametre, der er vigtige i en social robot skal robotten præsenteres for testpersonerne, hvorefter de skal medvirke i et semi-struktureret interview. Præsentationen af robotten kan ske på flere måder, da det både har betydning om testpersonerne interagerer med en prototype eller en faktisk robot. Da inspirationen til den sociale robot kommer mere eller mindre fra Double, vælges det at bruge denne til undersøgelsen af de betydningsfulde parametre. Udover inspirationen kommer herfra, så kan det påvirke folk betydeligt om de interagere med en færdig robot eller en prototype, der hverken har det rigtig udseende eller de rigtige bevægelser. For at testpersonerne kan forstå, hvordan robotten skal opføre sig, specielt med en mere dynamisk bevægelse, vælges det at bruge Double i stedet for at sætte prototypen på en robotstøvsuger og fjernstyre denne. 

Det vælges at Double under testen skal fjernstyres via en computer, da testpersonen på denne måde kan være med inde over robottens bevægelser og få en dybere forståelse for, hvordan det er meningen robotten skal bevæge sig. Det forventes altså at testpersonen kan forestille sig, hvordan en færdig robot vil fungere, selvom denne ikke er færdigudviklet endnu.\blankline


Efter testpersonen er blevet præsenteret for robotten og den kontekst hvori robotten skal fungere, findes det relevant at interviewe testpersonen, for på den måde at forstå hvordan de tænker omkring sociale robotter.