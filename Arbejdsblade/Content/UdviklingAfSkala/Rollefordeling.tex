\section{Rollefordeling}
\label{ParametreRollefordeling}
%
For at udføre feltundersøgelsen er det nødvendigt at definere nogle roller, som relaterer sig til specifikke dele af undersøgelsen. Der vil i alt blive defineret fire roller, som vil rotere mellem projektgruppen.
%
\subsubsection*{Robot styre}
Da de rejsende skal interagere med en \textit{Double}-robot, som ikke kan forudprogrammeres er det nødvendigt, at have én til at styre robotten. For at de rejsende får et indtryk af at robotten er autonom og har en form for social intelligens er det favorabelt, at den rejsende ikke registrerer, at det er en person, som styrer robotten. For at efterkomme det vil personen, som styrer robotten være placeret så de rejsende ikke direkte kan se hvordan robotten styres, dog skal personen stadig have mulighed for at overvære og høre interaktionen mellem robot og den rejsende, så robotten kan styres derefter. Udover placeringen af personen, som styrer robotten, bliver de rejsende fortalt af testlederen, at personen vil tage supplerende noter. 

\subsubsection*{Testleder}
Testlederen har flere opgaver, først og fremmest er det testlederen de rejsende kommunikerer med, udover robotten, hvorfor det er vigtigt at testlederen i starten sørger for at den rejsende føler sig tryg, hvilket tilstræbes ved small-talk. Derefter gengiver testlederen formålet med undersøgelsen: At finde ud af hvordan danske rejsende beskriver interaktionen med en social robot i en lufthavn. I den forbindelse uddeler testlederen en samtykkeerklæring til den rejsende, hvis underskrevet starter testlederen lydoptagelsen. Testlederen har desuden til opgave at forklare hvilken opgave den rejsende skal løse ved hjælp af robotten og den forbindelse spørge ind til interaktionen, hvis det findes nødvendigt.

Efter opgaven er løst har testlederen til opgave at styre samtalen ind på de nævnte samtale emner beskrevet i \fullref{ParametreSamtaleemner}. Afslutningsvist har testlederen til opgave, at debriefe den rejsende omkring undersøgelsen samt at afslutte forløbet.\blankline
%
Testlederens instruktioner er som følger:
%
\begin{quotation}
\noindent
\textit{Vi er en gruppe studerende fra Aalborg Universitet, der er i gang med et projekt, hvor vi undersøger en robot, der skal være i en lufthavn. Den undersøgelse vi går i gang med om lidt, er egentlig bare for at finde ud af hvad danske rejsende (som dig) synes om robotten.\blankline
%
Din opgave er at udføre et par opgaver som jeg vil beskrive for dig lige om lidt. Jeg kommer til at spørge en del ind til hvad du siger, så det håber jeg du vil bære over med. Når du er færdig med opgaven vil vi snakke lidt mere om din oplevelse med robotten.\blankline  
%
Den her robot er ikke en vi har været med til at designe eller udvikle på nogen måde, vi bruger den bare for at finde ud af hvordan danske rejsende oplever interaktionen med den, både positivt og negativt.\blankline
%
Under hele testen sidder de to (peg på robot styrer og observatør) og tager noter, så dem skal du bare ignorere. \blankline
%
Hvis det er i orden med dig, så vil jeg gerne have lov til at optage lyden under testen, bare for at sikre at vi ikke overhører noget. Så derfor har vi lavet den her samtykkeerklæring som du gerne må læse, og underskrive hvis det er okay med dig, det der står. Du må endelig sige til hvis du har nogle spørgsmål til den.}
\end{quotation}
   

\subsubsection*{Observatør}
Observatørens opgave er at notere observationer samt tilhørende kommentarer og tidsstempel for hvornår observationen forekom. Observatørens opgave er derfor ikke at notere alt det den rejsende fortæller, da det bliver optaget. Derimod har observatøren til opgave at notere, hvis den rejsende eksempelvis kommenterer at robotten opførte sig på en bestemt måde, hvor der både skal noteres hvad robotten gjorde, den rejsendes kommentar samt et tidsstempel. Det gøres blandt andet fordi undersøgelsen ikke videooptages men kun lydoptages. For at notere tidsstemplet uden at forstyrre testlederen og den rejsende, vil observatøren starte en tidsmåler på samme tid som testlederen starter lydoptagelsen. For at undgå problemer med afbrudt internetforbindelse vil observatøren notere sine observationer i på papir.      
 
\subsubsection*{Support}
Supporten har til opgave, at sørger for at andre rejsende ikke afbryder en igangværende session. I tilfælde af at andre rejsende udviser interesse for undersøgelsen er det supportens opgave, at imødekomme dette ved at notere at den rejsende har lyst til at deltage og efterfølgende henvende sig til den rejsende om, at det er deres tur. Ydermere har supporten til opgave at gå rundt og tage noter på papir. Derudover er det supportens opgave, at træde ind som en af de førnævnte roller, hvis der opstår behov for det.  


