\chapter{Robot bevægelse}
\label{TestAfSkalaRobotBevaegelse}
%
Skriv lidt om hvordan robotten "bør" bevæge sig - brug kilder fra interaktion med sociale robotter og sæt tests op hvor vi tester stop afstand og hastighed. \blankline
%
\section{Test idé}
%
Vi tager udgangspunkt i kilderne:\textbf{ Close But Not Stuck:
Understanding Social Distance in Human-Robot Interaction Through a Computer Mediation Approach, SKRIV FLERE KILDER.} De bruger en robot der bevæger sig mellem 0.9 km/t og 1.44 km/t og har en stop afstand på 50cm $\pm$ 10 cm, den undersøgelse blev lavet i England og fokuserer på tre retninger robotten kan kommer fra: Venstre, højre og frontalt, det er alt sammen i forhold til når testpersonen sidder ned. Stop afstanden afhænger SUPER meget af kultur de 50 cm kommer fra hvor tæt folk i England (måske generelt for flere europæiske lande - nok ikke syd europa) kommer på hinanden ved menneske-menneske interaktion, hvorimod i de fleste asiatiske kulturer er den afstand altså kun mellem 20-30 cm. Da vi jo har en lufthavn hvor nærmest alle nationaliteter kan være præsenteret er det vigtigt at vi tænker over det. I forhold til hastigheden mener testpersonerne at det er \textit{about right} eller for langsomt, hvorfor jeg forestiller mig at vi kan bruge deres hastigheder som start og så øge hastigheden og få testpersonerne til at vurdere hvordan det er. Det kan gøres på flere måder - skala, mundtligrespons osv. 

Vi skal selvfølgelig give dem konteksten og jeg tænker det skal være så simpelt som muligt da dette kun er en indledende test. Det jeg forestiller mig er at vi kunne fokuserer på når robotten henvender sig til en person, i og med robotten på forhånd jo ikke nødvendigvis ved hvad den skal hjælpe personen med. 

