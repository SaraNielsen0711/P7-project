\section*{Pilottest}
\label{pilottest}
%
I forbindelse med at udvikle skalaer er det vigtigt, at sikre sig at skalaerne fungerer og at testpersonerne er i stand til anvende dem. Formålet med pilottesten er derfor ikke at undersøge hvor indbydende robotten perciperes, men derimod at undersøge om testpersonerne er i stand til at anvende den designede VAS og dertil om de forstår de angivne ankerpunkter. 


Når skalaer opstilles er det vigtigt at teste dem i en pilottest for at sikre sig at de fungerer og bliver forstået som tiltænkt. Der er før den rigtige test kørt to pilottest, hvor testpersonerne efterfølgende blev spurgt om følgende:\blankline
%
\begin{itemize}
	\item Kan du forstå hvordan du skal bruge skalaen?
	\item Hvordan synes du det var at bruge skalaen?
	\item Havde du nogen problemer med at forstå skalaen?
	\item Kunne du forstå de to labels på skalaen?
	\item Kunne du forstå spørgsmålet?
	\item Andre kommentarer?
\end{itemize} 

\subsection*{Testperson 1}
Første testperson i pilottesten gennemførte testen uden at der blev observeret nogle problemer. Ved besvarelsen af de evaluerende spørgsmål sagde han selv at han ikke havde nogle problem med at forstå og bruge skalaen. Spørgsmålet havde han heller ikke problemer med at forstå. 
Ved spørgsmålet om forståelsen af skalaens labels sagde testpersonen at han godt kunne forstå dem, men nok aldrig ville bruge udtrykket ``ekstremt indbydende'' om en robot. Denne label laves ikke om da det kan skyldes testpersonens personlige holdning til robotter, at han aldrig ville synes de var særlig indbydende. Det er dog vigtigt at være opmærksom på om det gentager sig i anden pilottest, da det så vil kræve en ændring af denne label.

\subsection*{Testperson 2}
Der blev ikke observeret nogle problemer og testpersonen mente også selv, at han ingen problemer havde med brug og forståelse af hverken skala, labels eller spørgsmål. Ved besvarelsen af de efterfølgende spørgsmål udtalte testpersonen at han \textit{``..synes den her skala er bedre end sån en 1-10 skala.''}
\\\\
Det vurderes at testsetupet og skalaen samt spørgsmål og labels fungere og anvendes som forventet, og testen kan derfor udføres. 