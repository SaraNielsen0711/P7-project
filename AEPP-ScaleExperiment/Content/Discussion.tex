\section*{Diskussion}
\label{Diskussion}
%
Foruden information omkring køn, alder og studieretning kunne det have været en fordel at måle testpersonernes højde, da den ene af testpersonerne, baseret på testledernes vurdering, var betydeligt højere end de resterende, samt en testperson, som var betydeligt lavere end de restende testpersoner. Disse højdeforskellen kan potentielt have en indflydelse på hvor indbydende testpersonerne perciperede robotten afhængigt af dens fire hovedpositioner. 

Under en samtale antages det, at samtalepartnerne generelt vil søge efter at opnå øjenkontakt og opretholde en form for øjenhøjde. Er dette tilfældet kan det ligeledes antages at robottens hovedposition positionelt har indflydelse på, hvorvidt testpersonerne betragter robotten som en form for samtalepartner og ydermere hvilke interaktionsmuligheder robotten tillader. Det vil derfor være interessant at undersøge sammenhængen mellem brugerens højde og hvor indbydende robotten perciperes afhængigt af hovedposition.\blankline  
%
Testpersonerne angav på en \textit{Visual Analogue Scale} (VAS), hvor indbydende de perciperede robotten afhængigt af dens hovedposition. Skalaen var designet med åbne endepunkter samt to ankerpunkter angivet med \textit{Slet ikke} og \textit{Ekstremt}. Som tidligere nævnt kommenterede testperson 1 i pilottesten at ankerpunktet \textit{Ekstremt} ikke var det mest passende ord at anvende som label, da det for testpersonen forbindes med at være attraktiv, hvilket testpersonen gav udtryk for robotter ikke er. Selvom det blev vurderet ikke at ændre på skalaen, kunne det have været en fordel at udføre gentagende pilottests for at undersøge om andre har en lignende holdning og på baggrund af det ændre labelen.    

%NÅET HERTIL
 
 
  Der kan også argumenteres for, at ekstremt er et negativt ladet ord i den forstand, at det oftest benyttes til at beskrive kraftige vejrforhold, naturkatastofer eller politiske holdninger langt fra, hvad der antages for normalt, \parencite{Oxford2017}. Der kunne derfor med fordel undersøges om et andet alternativ kunne have været anvendt.

Ud over dette blev der spurgt ind til hvor indbydende robotten fremstod. Dette antager en fælles forståelse for ordet "indbydende". Det kunne tænkes at lidt mere kontekst ift. hvor denne robot skulle bruges kunne have hjulpet med at enspore forsøgspersonernes forståelse af ordet "indbydende". Det kunne bl.a. have indebåret en lille opgave, som forsøgspersonerne skulle løse vha. robottens skærm (ansigtet). Hermed fås muligvis et mere økologisk estimat af, hvor indbydende robotten er, da forsøgspersonerne så har fået en fornemmelse af, at den skal anvendes til en interaktion mellem menneske og robot.


\subsubsection{Antallet af testpersoner}
I forsøget er der kun blevet anvendt 8 testpersoner, hvilket går ud over præcisionen af vores data. Dette afspejles blandt andet af den store variation, der ses hos position 2 og 4 i Figur \ref{fig:boksplot}. Havde flere testpersoner blevet kørt igennem kunne en mere pålidelig sammenhæng have været fundet.




