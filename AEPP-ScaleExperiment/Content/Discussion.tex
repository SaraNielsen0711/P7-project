\section*{Diskussion}
\label{Discussion}

Position 2 og 3 virker bedre vurderet frem for position 1 og 4. På trods af data om forsøgspersonerne, som studieretning og alder, blev der ikke målt forsøgspersonernes højde, hvilket sandsynligvis ville have en markant indflydelse på resultaterne, hvis man antager at personerne er vant til at holde øjenkontakt med personer. Det ville være interessant at se hvilke sammenhænge der eksisterer mellem vinkel på robottens hovede og brugerens højde, da en mulig interaktion da vil udspille sig forskelligt fra bruger til bruger. 

\subsubsection{Ordvalg ifb. forsøg}
Der blev valgt at anvende en \textit{Visual Analogue Scale} (VAS) til at indsamle data fra forsøgspersonerne med åbne endepunkter samt to ankerpunkter kaldet "Slet ikke" og "Ekstremt". En forsøgsperson kommenterede på, at denne personligt ikke ville have valgt ordet ekstremt i forbindelse med at beskrive indbydende. Der kan også argumenteres for, at ekstremt er et negativt ladet ord i den forstand, at det oftest benyttes til at beskrive kraftige vejrforhold, naturkatastofer eller politiske holdninger langt fra, hvad der antages for normalt \fxnote{Kilde som en link udkommenteret i latex} %\fxnote{Kilde: http://www.oxfordlearnersdictionaries.com/definition/english/extreme_1?q=extreme}.

Ud over dette blev der spurgt ind til hvor indbydende robotten fremstod. Dette antager en fælles forståelse for ordet "indbydende". Det kunne tænkes at lidt mere kontekst ift. hvor denne robot skulle bruges kunne have hjulpet med at enspore forsøgspersonernes forståelse af ordet "indbydende". Det kunne bl.a. have indebåret en lille opgave, som forsøgspersonerne skulle løse vha. robottens skærm (ansigtet). Hermed fås muligvis et mere økologisk estimat af, hvor indbydende robotten er, da forsøgspersonerne så har fået en fornemmelse af, at den anvendes til en interaktion mellem menneske og robot.



antallet af personer, statistisk analyse
