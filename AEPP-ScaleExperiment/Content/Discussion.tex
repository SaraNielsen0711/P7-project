\section*{Diskussion}
\label{Discussion}

Position 2 og 3 virker bedre vurderet frem for position 1 og 4. På trods af data om testpersonerne, som studieretning og alder, blev der ikke målt testpersonernes højde, hvilket sandsynligvis ville have en markant indflydelse på resultaterne, hvis man antager at personerne er vant til at holde øjenkontakt med personer. Det ville være interessant at se hvilke sammenhænge, der eksisterer mellem vinkel på robottens hovede og brugerens højde, da en mulig interaktion så vil udspille sig forskelligt fra bruger til bruger. 

\subsubsection{Ordvalg ifb. forsøg}
Der blev valgt at anvende en \textit{Visual Analogue Scale} (VAS) til at indsamle data fra testpersonerne med åbne endepunkter samt to ankerpunkter kaldet "Slet ikke" og "Ekstremt". En testperson kommenterede på, at denne personligt ikke ville have valgt ordet ekstremt i forbindelse med at beskrive indbydende, da det for forsøgspersonen forbindes med at være attraktiv. Der kan også argumenteres for, at ekstremt er et negativt ladet ord i den forstand, at det oftest benyttes til at beskrive kraftige vejrforhold, naturkatastofer eller politiske holdninger langt fra, hvad der antages for normalt, \parencite{Oxford2017}. Der kunne derfor med fordel undersøges om et andet alternativ kunne have været anvendt.

Ud over dette blev der spurgt ind til hvor indbydende robotten fremstod. Dette antager en fælles forståelse for ordet "indbydende". Det kunne tænkes at lidt mere kontekst ift. hvor denne robot skulle bruges kunne have hjulpet med at enspore forsøgspersonernes forståelse af ordet "indbydende". Det kunne bl.a. have indebåret en lille opgave, som forsøgspersonerne skulle løse vha. robottens skærm (ansigtet). Hermed fås muligvis et mere økologisk estimat af, hvor indbydende robotten er, da forsøgspersonerne så har fået en fornemmelse af, at den skal anvendes til en interaktion mellem menneske og robot.


\subsubsection{Antallet af testpersoner}
I forsøget er der kun blevet anvendt 8 testpersoner, hvilket går ud over præcisionen af vores data. Dette afspejles blandt andet af den store variation, der ses hos position 2 og 4 i Figur \ref{fig:boksplot}. Havde flere testpersoner blevet kørt igennem kunne en mere pålidelig sammenhæng have været fundet.
