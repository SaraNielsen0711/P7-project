\section*{Method}
\label{Method}
The subjects were shown pictures of each set of speakers as they appear in \autoref{fig:speakers}. They were then asked to rate their attributes on a VAS with 16 bipolar word-pairs. 

\begin{figure}[H]
\centering
\includegraphics[scale = 0.7]{Figure/speakers.png}
\caption{A figure of the speakers used in the data set. They are all from Bang \& Olufsen in order to avoid any bias relating to specific brands}
\label{fig:speakers}
\end{figure}

The data is analyzed using the software \textit{PanelCheck}. The analysis is quite explorative, since it is still unknown what we are looking for. 

\subsection*{PanelCheck Software}
\textit{PanelCheck} is a free software which is made to be easy-to-use. It is mainly used for visualisation of sensory data which helps to gain insight in a given assessor and panel performance. The software is shown on \autoref{fig:PanelCheck}.

\begin{figure}
\includegraphics[width = \textwidth]{Figure/PanelCheck.png}
\caption{The start interface when \textit{PanelCheck} is started and a dataset has been imported. The 16 different attributes are shown along with the five different speakers (Samples). The ten assessors are named from P01-P10. There are the main tabs Univariate, Multivariate, Consensus, and Overall.}
\label{fig:PanelCheck}
\end{figure}

The initial interface contains an overview of the subjects (assessors), the word pairs (attributes), and the different speakers (samples). At the top it is possible to choose between different tabs: Univariate, Multivariate, Consensus, or Overall. Univariate and Multivariate relating to how many variables are being compared and Consensus relates to the underlying dimensions in the data and therefore contains PCA-related options. As the name implies, Overall can be used to gain an overall impression of the data e.g. by showing multiple ANOVA-plots in the same calculation. Within each of the four tabs there exists varying sub-tabs which contains statistical plots relating to the chosen tab. In this example, within Univariate it is possible to choose Line Plots, Mean \& STD Plots Correlation Plots etc. It is also possible to exclude assessors, attributes, and samples. One just have to undo the check marks in the unwanted category as shown in \autoref{fig:PanelCheck}.