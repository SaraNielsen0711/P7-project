\section*{Discussion}
\label{discussion}
%
%SAftig er der svaret det samme til = den er lige meget. 
From the three dimensional bi-plot it looks like the attributes “Nærende”, “Mættende” and “Skorpe” are highly correlated, therefore one of these three attributes might be redundant. 
When attributes are highly correlated it can in some cases mean that they measure the same thing. To determine whether or not the three attributes is measuring the same thing, each one of them will be discussed. \blankline
%
The first attribute to be discussed will be "Skorpe''. This is a from the category Direct Perception and describes a very superficial element of the buns. It is very unlikely that this attribute is measuring the same thing as the other two, because there is a possibility that a bun can look like something on the outside and then when you see the inside it look totally different. \blankline
%
The other two attributes is “Nærende” and “Mættende”. The two words are not the same, but it is relevant to take into account that the test subjects of this study did not taste the buns, but only looked and felt the buns. Because the subjects did not taste the buns it could be argued that the attributes “Nærende” and “Mættende” are almost the same. They are both from the category "reflection'' and they are both related to what you get out of eating the bun. \blankline

The attribute "Saftig'' does not really contribute to any of the three principal components and the rating of this attribute doesn't vary that much between the different buns either. The reason for the lack of variation for the rating on this attribute is not determined. There is a possibility that it is just because the buns are equally juicy. So if a totally different bun were presented, it could possible variate from the buns used in this study. It could also mean that the participants did not know how to rate a bun on an attribute as "Saftig" when only exposed to a picture.